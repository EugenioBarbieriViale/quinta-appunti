\documentclass[12pt]{article}

\usepackage[a4paper, total={6in, 8in}]{geometry}
\usepackage{textcomp}

\begin{document}
\setlength{\parindent}{0pt}

\def \t {\textrightarrow}
\def \v {\vspace{1em}}
\def \bi {\begin{itemize}}
\def \ei {\end{itemize}}
\def \s[#1] {\section*{#1}}
\def \ss[#1] {\subsection*{#1}}
\def \sss[#1] {\subsubsection*{#1}}

\s[Marx]
"Critica ad Hegel" è la prima che fa \t ma critica tutti
\bi
    \item sinistra hegeliana
    \item economisti classici
    \item socialismo utpostico
    \item Proudon
    \item Feuerbach
\ei
Critiche sono accumunate dal fatto che lgi altri non sono pragmatici \t non indicano gli strumenti reali con cui passare dalla situazione reale a quella proposta 

\ss[Prime critiche]
Sinistra hegeliana non ha portato alle radicali conseuquenze la filosofia di hegel \t non è arrivata in fondo, quindi inefficace \t riflessione non è servita a nulla \\
Economisti classici \t non hanno scientificita \t hanno parlato di rapporti tra uomini, proprieta privata, ma non sono andati alla radice di questi fenomeni \\
Scientificita = bisogna indagare le leggi che regolano un fenomeno \t una volta conosciute queste leggi, posso modificare questo fenomeno \\
Questo deve fare il suo socialismo, che chiama scientifico \\
Economisti fanno azione descrittiva \t non arrivano alle leggi \\
Socialismo utopistico è utopistico perche è incapace di portare a una reale riforma, non propone mezzi ma solo il fine \\
Propongono falansteri, abolizione della proprieta privata \t ma come ci si arriva dalla attuale situazione? \\
Soci. ut. viene criticato anche perche loro conservano la realta che c'è \t trovano soluzioni all'interno della realta capitalista, che invece va rovesciata

\ss[Critica a Proudon]
Proudhon è il socialista utopista che critica di + \t all'inizio in realta lo apprezza, Proudon aveva affermato che la proprieta privata è un furto \\
Negli scritti del 44 Marx lo supporta \t ma poi il suo pensiero si allontana \\
Proudon diceva che la proprieta privata è un furto ed e la radice del capitalismo \t quindi va ridistribuita dallo stato \t prospettiva non è di cancellarla, ma di fare in modo che ognuno abbia qualcosa \\
Marx dice che ha sostituito analisi economica con moralismo \\
Per marx proprieta privata è un sistema economico sbagliato, che va rovesciato \t non va ridistribuita, ma cancellata \\
Per fare cio serve una rivoluzione \\
Proprieta privata è l'antitesi, è cio che permette la rivoluzione \t non va ridistribuita, ma va esaserpata per generare il cambiamento = distruzione della societa capitalista \\
Pr. è moralista secondo Marx perchè vuole fare in modo che ognuno sia un po piu contento, con la sua parte di proprieta privata \\
Marx legge la realta in modo dialettico \t realta è dinamica, cambia, e la direzione corretta è l'annullamento della proprieta \\
La storia si evolve dialetticamente, la lotta di classe è cio che permette di cambiare il sistema

\ss[Critica a Feuerbach]
All'inizio nelle "Tesi su Feuerbach" marx sostiene le idee di F. \t è antropologia \\
Ma lui non si fa la domanda fondamentale \t ovvero perche uomo è infelice, perche genera dio?, perche proietta dolore fuori di se? \\
Risposta: perche uomo è alienato, vive in una dimensione che non lo realizza
\end{document}
