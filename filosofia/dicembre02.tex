\documentclass[12pt]{article}

\usepackage[a4paper, total={6in, 8in}]{geometry}
\usepackage{textcomp}

\begin{document}
\setlength{\parindent}{0pt}

\def \t {\textrightarrow}
\def \v {\vspace{1em}}
\def \bi {\begin{itemize}}
\def \ei {\end{itemize}}
\def \s[#1] {\section*{#1}}
\def \ss[#1] {\subsection*{#1}}
\def \sss[#1] {\subsubsection*{#1}}

\s[Compte]
Fondatore della sociologia \t nuova disciplina che introduce per risolvere la conflittualita sociale e i rapporti tra gli uomini e i loro conflitti nella societa \\
Vuole rimediare ai conflitti sociali \t manca una scienza sociale \t leggi per modificare i fenomeni \\
Per lui questo è l'unico ambito che non ha ancora raggiunto stadio positivo \\
Sociologia diventa mezzo per indagare le societa, ma anche strumento della politica \t se io mi devo occupare di una societa civile, devo conoscerne i meccanismi \\
Il fallimento della politica sta nel fatto che chi governa non è sociologo, ma è un letterato, avvocato, etc. \\
Il metodo di indagine si articola in:
\bi
    \item osservazione
    \item esperimento
    \item comparazione
\ei

\ss[Osservazione]
Semplice: osservo i fatti sociali, diretta e inquadrata nella legge dei 3 stadi \t osservo cio che accade

\ss[Comparazione]
Semplice: comparo societa che si trovano allo stesso stadio (es. metafisico) e ne analizzo analogie/differenze \\
Comparazione, una volta osservata la societa, è semplice

\ss[Esperimento]
Vera difficolta \t non posso cambiare condizioni di partenza perche voglio verificare la mia ipotesi \t la condzione di partenza mi viene data, non la modifico \\
Inoltre esperimento non è repetibile \\
In realtà c'è modo per sperimentare, ovvero con l'anomalia, il caso patologico \\
Cio che altera le condizioni sociali sono cambiamenti forti (per esempio pandemie) \\
L'anomalia però non c'è sempre \t si puo sperimentare il valore di una tesi solo in quel momento

\ss[Classificazione delle scienze]
La sociologia è la scienza + importante \t si pone al vertice della gerarchia che fa
\bi
    \item astronomia
    \item fisica
    \item chimica
    \item biologia
    \item sociologia
\ei
Grado decrescente di generalizzazione e crescente di complessita \\
Scienza a grado X include gli argomenti di quelle precedenti \t sociologia ha tutto \\
Queste 5 sono le uniche vere scienze \\
Non appartengono alla scienza teologia, metafisica (non sono scienze positive), morale (che invece si risolve nella sociologia \t ha anche rapporti morali) \\
Psicologia divisa in biologia e sociologia \\
Matematica invece è la vera base di tutte le scienze \t lo scrive nel "Corso di filosofia positiva" \t nel primo libro parla solo di matematica, che è lo strumento fondamentale di tutte le scienze di natura \\
Non è una scienza a se, ma rientra in tutte le scienze \\
L'ordine che descrive delle scienze rispecchia l'ordine:
\bi
    \item logico \t si parte dall'oggetto + semplice, ovvero quello dell'astronomia
    \item storico \t momento in cui le scienze sono diventate positive \t l'astronomia è diventata positiva con Copernico, fisica con Newton, chimica con Lavoiser \t sociologia arriva al positivo con Compte
    \item pedagogico \t dovrebbero essere insegnate in questo ordine \t ogni scienza è la premessa necessaria di quella seguente
\ei
Sociologia è scienza naturale \t uomini per natura sono portati a vivere in società \t fa parte della natura umana, la societa non è una sovrastruttura imposta in un certo momento della storia \\
L'uomo è un animale sociale \t quindi sociale indaga qualcosa che pertiene alla natura umana \\
Quindi non serve un patto sociale \t uomo crea societa spontaneamente \\
La filosofia non è presente tra le scienze \t perchè va a definire il metodo di tutte le scienze \\
Matematica è lo strumento di tutte le scienze, filosofia ha il compito di determinare lo spirito di ogni scienza \t filosofia definisce metodologicamente la scienza \\
Matematica è strumento operativo, come devo farlo e perchè lo dice la filosofia \t struttura una metodologia \\
La filosofia infatti studia i nessi logici nella realta e mette in evidenza i mezzi logici

\ss[Religione dell'Umanità]
Poi scrive tra il 51 e il 54 intitolata "Sistema della politica positiva" \\
Qua parla della religione dell'Umanita \t sostituisce l'amore per dio all'umanita, che non è somma di individui \t ma comprende tutti gli individui anche non in essere \\
È la grande idea di umanita \t come se fosse di cellule che nascono e muoino continuamente \\
Ha struttura simile al cattolicesimo \t perche si rivolge direttamente all umanita \\
Ha dei dogmi (leggi scientifiche e filosofia positiva), a gerarchia come quella cattolica, ha dei sacramenti e un calendario \\
È una religione femminile \t l'angelo custode è la donna \t donna diventa la custode degli elementi fondamentali della religione \\
Questo si trova solo in Compte \t nessun altro filosofo ne positivista ha fatto questo \t stona molto in compte, che ha fondato la sociologia \\
Unica cosa interessante è la figura della donna
\end{document}
