\documentclass[12pt]{article}

\usepackage[a4paper, total={6in, 8in}]{geometry}
\usepackage{textcomp}

\begin{document}
\setlength{\parindent}{0pt}

\def \t {\textrightarrow}
\def \v {\vspace{1em}}
\def \bi {\begin{itemize}}
\def \ei {\end{itemize}}
\def \s[#1] {\section*{#1}}
\def \ss[#1] {\subsection*{#1}}
\def \sss[#1] {\subsubsection*{#1}}

\s[Schelling]
Rispetto a Fichte la sua filosofia è + romantica e + metafisica \\
Fa un tentativo di salvare la natura dal non io \t la natura è vivia, presenta delle forme di coscienza progressiva \\
Natura è spirito visibile \\
All'interno della natura agiscono le stesse forze che agiscono nello spirito \t anche nella natura agisce la stessa forza dello spirito, nella natura c'è un movimento che porta alla creazione di enti con coscienza diversa progressivamente \\
C'è un evoluzione non di forme, ma un evoluzione dello spirito sotteso alla natura che da vita a forme diverse \\
La natura si pone in essere generando forme via via piu perfette di coscienza \\
Esistono delle forme di intelligenza inconsce, cosi che la natura viene posta in essere senza essere cosciente \\
Nella scala progressiva si arrifa all'uomo \t e quando l'uomo viene posto in essere ed è cosciente, torna alla natura: si chiede cosi sia lui e cosa sia la natura \\
Sempre attraverso percorso conoscitivo torna indieto \t prima da natura a uomo, e poii ritono allo spirito attraverso via teoretica \\
Cosi l'uomo si riconosce come spirito \\
Due strade nella dimensione assoluta dello spirito:
\bi
    \item realista: autoposizione dell'essere nello spirito
    \item idealista: strada teoretia \t percorso conoscitov dell'uomo per tornare alla natura e conoscere se stesso
\ei
In schelling c'è un principio orignario che è gia pensiero ed esistenza insieme, mentre in fichte c'è prima pensiero e poi essere \t grande differenza \\
Questo significa che il fine della natura è l'uomo \t la natura è una progressione di coscienza che tende all uomo 

\v

Se pensiero ed essere coincidono, se ci sono due faccie della stessa medaglia \t bisogna inventarsi una nuova filosofia \\
Quindi scrive ??? Idealismo trascendtale \t qui fonda la nuova filosofia (che chiama id. tr.) \\
Se a questo essere è cambiato, allora anche la filosofia si deve modificare \t ne serve una nuova, perche essa indaga la realta \\
Se sprito è uno \t non si puo fare distinzione tra realismo (privilega l'oggetto) e idealismo (privilegia il soggetto), ma se oggetto e soggetto coincidono non si possono staccare \\
Quindi nasce l'idealrealismo \t bisogna tenere conto di entrambi gli apsettti della realta \t esistono l'aspetto ontologico e il momento teoretico che porta l'uomo a riconoscersi come spirito (momento soggettivo) \\
Ci sono entrambi e sono frutto dello stessa cosa \t tutto cio che è è spirito \\
Attivita realta di autoposizione nell essere e attivita ideale di conoscenza messa in atto dall uomo \\
La mia filosofia deve indagare entrambi questi aspetti \t non solo pratico, non solo teoretico, ma idealrealismo \\
Questo è l'unico modo per indagare la natura \t se no si avrebbe uno sguardo parziale \\
Questo significa anche che l'essenza è una sola \t tutto è assoluto \t qualunque ente è un ente che manifesta l'assoluto \\
Arriva problema \t all idealismo trascendentale (idealrealismo) collega la sua filosofia dell'identita (molto criticata)

\ss[Filosofia dell identita]
Se l'assoluto è un identita originaria di soggetto e oggetto, e che la filosofia è la conoscenza assoluta di questo assoluto, allora devo giustificare le differenze nella realta \\
Tutto è uguale da un punto di vista essenziale, perche tutto è assoluto \t ma allora perche l'assoluto nella sua perfetta identita ontologica si manifesta in soggetto e oggetto? \\
Perche cogliamo questa differenza, se poi tutto è assoluto? \t pensiero ed essere, soggetto e oggetto coincidono \t allora come li distinguiamo? \\
Soggetto = oggetto, oggetto = soggetto, che stanno nell assoluto che è uguale a se stesso \\
Ma in questa identita perfetta a volte prevalongo i caratteri dell'essere o del pensiero \t questo crea la differenza (es. tra uomo e oggetto) \\
Il problema è che è sempre un uguaglianza \t i due aspetti sono identici, anche se uno prevale sull'altro la differenza non si vede \t se coincidono, come fa a prevalre uno sull altro? \\
Stesso problema dell infinito che genera il finito \t il generato deve mantenere le caratteristiche del generante \\
In idealrealismo dice che oggetto e soggetto coincidono, e quindi non possono essere studiati separatamente \\
Anche l'assoluto di Schelling, come l'io puro, è una perfetta razionalità 

\ss[Ultime due fasi]
Dopo il 1804: filosofia teosofica e filosofia positiva \\
Qui lui fa una totale revisione del suo pensiero \t come fichte modifica le parole \\
Chiama questo assoluto è dio \t che quindi è la sostanza ontologica della realta \\
Schelling accetta di essere definito panteista solo se si dice che tutto è in dio, non che tutto è dio \\
Se dico che tutto è in dio, sto creando un antecedente \t dio è il principio fondante, e contiene la realta \t non sono sullo stesso piano, dio genera la realta e poi la contiene \\
Se tutto è dio, si stabilisce una perfetta coincidenza tra realta e il principio, che quindi non si puo distinguere piu \\
Ma questo dio non puo essere concepito come un dio fermo, ma è un dio che si va facendo \t tutto è in divenire, il pensiero genera l'essere e questa unita di pensiero generante la chiamo dio \t anche questo dio continua ad autoporsi in essere \\
Ma se dio è in continuo divenire ed è la massima perfezione, a che punto del divenire è massimamente perfetto? \t se cambia vuol dire che ha sempre una mancanza 

\v

Nel 1815 distingue una filosofia negativa da una positiva \\
Filosofia negativa = tutta la filosofia che la precede, ed è quella che si è interrogata sull'essenza universale delle cose \t sulle possibilità logiche della realta \\
Quella positiva si interessa effettivamente dell'esistenza degli enti nella realta \\
Non vuol negare la fil. precedente, ma ha bisogno di una integrazione \t essa si basava unicamente sul pensiero \\
Questa nuova ha bisogno anche della religione e della rivelazione \t la rivelazione non è quella cristiana, ma rappresenta un arco storico di tutte le religioni \\
La vera rivelazione di dio non è di una particolare, ma si è sempre rilevato nelle diverse religioni \t dio è in divenire, e si è rivelato in continuo divenire storico \\
In questa fase parla proprio di un dio persona \t cambiamento totale, che è un dio creatore (quindi nella sua concretezza religiosa)
\end{document}
