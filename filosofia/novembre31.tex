\documentclass[12pt]{article}

\usepackage[a4paper, total={6in, 8in}]{geometry}
\usepackage{textcomp}

\begin{document}
\setlength{\parindent}{0pt}

\def \t {\textrightarrow}
\def \v {\vspace{1em}}
\def \bi {\begin{itemize}}
\def \ei {\end{itemize}}
\def \s[#1] {\section*{#1}}
\def \ss[#1] {\subsection*{#1}}
\def \sss[#1] {\subsubsection*{#1}}

\s[Kierkegaard]
Disperazione è la malattia mortale \t malattia che prova in relazione a se stesso quando non si riconosce come finito \\
Si sente infinto, ma poi capisce di essere finito \t c'è la morte \\
Uomo si deve accettare come creatura nelle mani di dio \t ed accttare l'apertura nelle mani di dio \\
Per K. è dio ad avere la precedenza \t tutto il resto (compresa la scienza) non ha grande rilevanza \t per il cristiano la verita è l'esistenza della fede \\
La scienza come forma di vita è = all'esistenza inautentica \\
Non vuol dire rifiuto della scienza \t ma critica lo scientismo = far diventare la scienza un habitus di vita, un modello di vita \\
Scienza spiega la realta fenomenica, ma:
\bi
    \item non spiega dio e gli scienziati che hanno la pretesa di parlare di dio sono in malafeda \t la scienza spiega solo la realta di natura, sono ipocriti
    \item la ragione scientifica spiega tutto \t tutto cio che non spiegato scientificamente va rifiutato \t si vuole spiegare problemi morali, etici etc con la scienza \t ma questo è pericoloso
\ei
Pericoloso perchè se si usa un metodo scientifico per spiegare l'arazionale nell'uomo \t voglio fare di una arazionalita una razionalita \t quindi arrivo a disconoscerlo o a modificare i contenuti \\
La scienza deve stare nei sui limiti \t non averi pretesa di spiegare cose che non la competono

\v

La + pericolosa delle scienze è la teologia (e assurda) \t è una scienza che parla di dio \\
Come si fa a razionalizzare dio \t la situazione della teologia è tragicomica \t la teologia parla di dio perche è incredula ed è in malafede \\
Perche il teologo si difende da dio \t invece di assumere su di se il valore del cristianesimo, trasforma dio in un oggetto di studio \\
Quindi mette tra se e dio una barriera, ovvero quella dello studio scientifico \t non faccio entrare dio nella mia vita infatti \t cosi non c'è forma di coinvolgimento, no relazione personale con dio \\
Teologhi hanno paura \t quindi lo fanno diventare un oggetto d'indagine, e quindi asettico \t sono in malafede perche hanno paura di svelarsi davanti a dio \\
Per svelarsi a dio bisogna mettersi a nudo davanti a dio, e quindi davanti a se stesso \t si viene a conoscenza dei propri limiti e debolezze \\
Mostrarsi a dio significa guardarsi allo specchio senza filtri \t quindi le persone hanno paura, e quindi lo rifiutano oppure si mettono al riparo con la teologia \\
Quindi i teologi sono peggio uomini \t non hanno rapporto, ma fanno anche finta di averlo \t sono ipocriti \t vogliono solo evitare questo rapporto

\s[Positivismo]
In parallelo si sviluppa il positivismo \t movimento ampio e composito, che abbraccia gran parte della cultura europea \\
È un movimento filosofisco, artistico, culturale... \\
Va dal 1840 alla prima guerra mondiale \\
È un movimento ottimista \\
Si chiama cosi perche ha prospettiva positiva, ovvero empirica \\
Vivono pero una grande illusione \t ovvero che la scienza sia la chiave di volta del progresso \t quando so leggere tutti i fenomeni scientificiamente, la storia puo solo andare verso il meglio \\
Questo si scontra con la prima guerra mondiale \t c'era idea ottimistica che si va verso il meglio, ma guerra cambia completamente il paradigma \\
Nasce nella meta dell 800 perche:
\bi
    \item relativa pace internazionale
    \item sviluppo della scienza
    \item espansione coloniale
    \item rivoluzione industriale
    \item medicina progredisce
    \item urbanistica si sviluppa
\ei
Prende piede l'idea che ci sia progresso irrefrenabilie (sociale e umano) \\
La rivoluzione industriale da illusione di poter fare tutto con gli strumenti della scienza \t che permettono di dominare la realta \\
Illusione sta nel fatto che i limiti che erano presenti all'epoca sarebbero scomparsi \t tutti i limiti sociali e difficolta sarebbero stati superati con la conoscenza \\
Il negativo era transitorio \\
La scienza è lo strumento principe per indagare la realta e il metodo della scienza è il metodo delle scienze naturali \t limitato e criticato in questo, perche vuole spiegare cosi l'uomo nella sua interezza \\
Questo metodo vale per tutte le spiegazioni \t anche dal punto di vista sociale, e nasce infatti la sociologia \\
Il positivismo crede in progresso inarrestabile, definito non da provvidenza/dialettica hegeliana, ma generato dalla scienza \t è un progresso empirico perchè generato dall'uomo, no metafisico no trascendente \\
Questo è un limite \t positivismo nasce come rifiuto dell'assoluto \t la realta è positiva, ma in realta ci sono dei riferimenti assoluti e non discutibili (scienza e progresso) \\
Anche positivismo accusato di essere ideologico \t "non teme i fatti", vuole adattare la propria visione alla realtà \\
Criticano l'idealismo \t ma hanno punti di contatto \t positivismo va in continuo movimento proprio come lo spirito \t è un flusso inarrestabile verso il progresso, in cui non c'è stanzialità \\
Forte influsso marxista, che sta nel pragmatismo \t la scienza è la chiave di volta perchè è lo studio delle leggi che regolamentano i fenomeni \\
E se conosco le leggi di un fenomeno posso agire \t modifico la causa, cambia l'effetto \\
Pos. pensa che realtà è trasformabile \t si cercano le leggi dei fenomeni per poterli cambiare e farli andare verso il progresso 

\v

Dal punto di vista teoretico, confluiscono razionalismo ed empirismo (in Kant però si aveva la Ragione, non questa):
\bi
    \item razionalismo: posso leggere realta solo con ragione \t in pos. è la scienza
    \item empirismo: l'oggetto di questa lettura scientifica è il fatto empirico \t indago la realtà di fatto 
\ei
Ci sono anche filoni, uno + razionalista (francese) e uno + empirista (inglese) \\
Positivismo pero è anche tedesco, italiano, olandese \t in tutta europa ed è trasversale \\
Questa corrente di pensiero ha ricadute culturali enormi \t anche per esempio con il verismo

\ss[Compte]
Padre in generale del positivismo come movimento filosofico \\
È anche il fondatore della sociologia \\
Frequenta l'ecole politique, è bravo in matematica \t ma è attratto dal pensiero filosifico \\
È antihegeliano \t sviluppa la sua dottrina della scienza \t crea legge di sviluppo della realta, chiamata "dei 3 stadi" \t molto diversa dalla dialettica perche:
\bi
    \item non è metafisica, non è la struttura dlela realta ma un modo di vederla
    \item non è dialettica
    \item parla dello sviluppo di una realta fisica, non dello spirito
\ei
È una legge di sviluppo della realta, storia ma anche del singolo individuo \t tutti attraversiamo stadio teologico, metafisico e positivo \\
I 3 stadi corrispondo ai modi in cui gli uomini hanno realizzato la realta \\

\sss[Stadio teologico]
Prima uomini spiegavano realta ricorrendo a dio \\
Questo accadeva all'umanita, ma anche al singolo (per esempio da bambino)

\sss[Stadio metafisico]
Poi realta viene spiegata in modo metafisico e non scientifico, che si muove ancora in modo misterioso \t non c'è più una divinità, ma una dimensione nuova

\sss[Stadio positivo]
Stadio positivo è quello della scienza \t leggo i fenomeni della realta con una legge, che è valida per quel fatto in quel momento con quei caratteri \\
È il livello di consapevolezza dell'adulto anche \t spiega in modo + rigoroso \\
Nessuno usa ancora metafisica, tranne nei fenomeni sociali \t le societa quando si osservano sono un caos, e manca rigore \\
Non si puo risolvere crisi sociale o politica se non si conoscono i fatti, le dinamiche che regolano il fenomeno \\
Per risolvere questo bisogna introdurre una scienza che permetta di arrivare allo stadio psoitivo alla conoscenza dei fatti sociali \\
Serve una scienza perche mi servono le leggi \t servono le leggi per risolvere conflitti e prevedere \\
Serve una scineza che è sempre pero controllata dai fatti \t non è teorica, ma empirica \t parto dall'esperienza e torno all'esperienza (cerco legge nel fenomeno, e poi agisco sul fenomeno) \\
Ovviamente è scienza solo ciò che posso analizzare empiricamente

\v

Cosi fonda la sociologia \t che esisteva gia ma non come scienza, prima solo spiegazione metafisica \\
In realta la chiama fisica sociale \t deve risolvere, con lo studio delle leggi, i fenomeni sociali \t si divide in:

\sss[Statica sociale]
Studia le condizione di esistenza comuni a tutte le societa \t condizione necessarie perche un gruppo umano si definisca societa \\
Ci sono condizioni che fanno si che gruppi possano essere definiti sociali

\sss[Dinamica sociale]
Come si sviluppano le societa \t lo fanno con la legge dei tre stadi, e fa dei riferimenti storici \\
Al primo stadio corrisponde il feudalesimo, secondo lo stadio della rivoluzione (va da riforma protestante alla rivoluzione francese, cambiamento non ancora arrivato alla scienza), terzo societa industriale \\
Momenti di evoluzione che hanno corrispettivo come momenti storici

\v
Fisica sociale è il presupposto di ogni scienza politica \t puo essere in mano solo agli intellettuali, non a chi non sa niente di societa \t la politica deve essere in mano ai sociologi \\
Posso amministrare la societa solo se la conosco \t non posso ignorarne i meccanismi, se no caos
\end{document}
