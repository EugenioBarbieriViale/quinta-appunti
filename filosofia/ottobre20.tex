\documentclass[12pt]{article}

\usepackage[a4paper, total={6in, 8in}]{geometry}
\usepackage{textcomp}

\begin{document}
\setlength{\parindent}{0pt}

\def \t {\textrightarrow}
\def \v {\vspace{1em}}
\def \bi {\begin{itemize}}
\def \ei {\end{itemize}}
\def \s[#1] {\section*{#1}}
\def \ss[#1] {\subsection*{#1}}
\def \sss[#1] {\subsubsection*{#1}}

\s[Filosofia dello spirito]
Nella filosofia della natura spirito esce fuori di se, genera l'essere e torna in se nella filosofia dello spirito \t massima autoconoscenza di se

\ss[Spirito soggettivo]
Spirito soggettivo \t prima tappa della triade \\
Si parla di un finito \t che non è manifestazione dello spirito \t ma appare dentro lo spirito \\
Prima l'infinto aveva bisogno del finito per manifestarsi \t qua il finito ha senso solo perchè sta dentro l'infinito \t ora è il finito che ha bisogno dell'infinito \\
Prima l'infinito si serve del finito, qua c'è totale svalutazione del finito \t non vale nulla se non viene letto come un parte dell'infinito \\
Nello spirito soggettivo parla di una forma di conoscenza dello spirito legata all'uomo \t triade interna è antropologia, fenomenologia, piscologia \t modi per leggere il finito dentro l'infinito \\
Nella triade si parla dell'anima \t che è presente nell'uomo, ed è il segno della sua appartenenza all'infinito \\
Lo spirito si conosce passando dal finito, ma passando da quell'elemento spirituale/razionale che è presente nel finito è lo lega in modo indissolubile al finito \\
È come l'intelletto potenziale \t ne parla nell'antropologia \\
Nella fenomenologia riprende l'opera e ridescrive la strada che lo spirito fa per autoconoscersi \\
Nella psicologia parla dello spirito pratico e dello spirito teoretico \t si mette insieme l'aspetto logico dell'uomo, in una sintesi che culmina nello spirito libero \\
Spirito libero rappresenta  il momento in cui l'anima capisce di essere spirito \t la parte razionale/spirituale dell'uomo (anima) \t capisce di essere espressione dello spirito, a cui l'uomo ontologicamente appartiene \\
Consapevolezza si ottiene nella propria dimensione di interiorita \t non piu stando nel mondo \\
L'anima esiste perche sono parte dello spirito 

\v

Questa è la tesi \t e rivaluta il trattato sull'anima di aristotele \t che rappresenta l'opera migliore sull'anima \t dice che è l'unica speculazione corretta sull'argomento \\
Riprende da qui molte parti \t con qualche correzione \t la + evidente è quella che dice: "non c'è nulla che ci sia nell'intelletto se prima non passa nei sensi" \t hegel aggiunge "tranne l'intelletto stesso" \\
La razionalita precede il senso \t ontologia e logica coincidono \\
Fa anche riflessione sulla parola e sulla lingua \\
Spirito soggettivo si chiude con l'affermazione della liberta \t anima libera si riconosce come spirito libero, l'anima si riconosce nello spirito 

\ss[Spirito oggettivo]
Momento + significativo di tutto l'hegelismo \\
Spirito oggettivo = momento in cui lo spirito libero esce fuori di se \\
Prima interiorita \t qua lo spirito libero si realizza nelle istituzione della famiglia, dello stato, ... \t tutto il sociale e l'etico-politico \\
Lo spirito oggettivo si articola in diritto, moralita, ed eticita \\
A sua volta eticita in famiglia, societa e stato \\
E sono triadi dialettiche \t anche se siamo nell'antitesi \t ma è l'antitesi della sintesi superema, quindi va bene

\ss[Diritto]
Spirito libero non sta in una liberta astratta \t liberta si deve concretizzare e deve stare in delle regole \t se liberta si esercitano senza regole, si avrebbe anarchia \\
Liberta concreta è nelle istituzioni \t per questo libertà da vita al diritto (ovvero la legge e le pene) \\
La liberta definisce il diritto che mi garandtisce la mia posizione nel mondo \t ma questo modo di esercitare la liberta (giuridico) è esteriore \t è del fare, posso fare cose e non e vengo punito rispettivamente \\
Ma questo è troppo esteriore, troppo immediato \t serve dimesnione di liberta + interiore 

\ss[Moralita]
Io sono libero in quanto uomo \t non piu nel fare, ma nella sfera soggettiva \t quello che conta non è la mia azione sulle cose, ma una mia liberta di scelta, una volonta, che viene prima dell azione \\
Prima liberta era solo in orizzonte oggettivo \t qua liberta è totalmente nella sfera soggettiva \t si puo essere liberi anche nella prigionia \\
Liberta è come determino la mia volonta \t non come sto nel mondo \\
Riprende l etica di Kant \t che fornisce la forma universale a cui ispirare l'agire morale \t criterio fondamentale per definire volonta \\
Ma comunque la sfera soggettiva di Kant è troppo chiusa \t prima solo esterno, liberta era solo rispetto della legge \\
Ora solo dimensione interiore \t volonta si determina liberamente ma a priori \t la volonta è libera in quanto tale a prescindere dalla realta, troppo interiore \\
Diritto e moralita sono troppo unilaterali \t uno tutto a posteriori, l altro anteriore 

\ss[Eticita]
Io realizzo la mia volonta libera con dei fini concreti \t volonta si determina liberamente ad agire ma non in base a devi perche devi, ma nel mondo \t volendo delle cose concrete \\
Io definisco la mia libera volonta nel volere delle cose nel mondo \\
Questo si realizza nella triade \t volonta si realizza nella concretezza della mia storia nelle 3 istituzioni della triade:

\sss[Famiglia]
È un luogo naturale \t nasce nella forma dell'amore e del sentimento \t luogo in cui l'individuo annulla la sua individualita per l universalita \\
Uomo rinuncia alla affermazione di se, del suo egoismo \t lo fa per il bene della totalita (ovvero famiglia) \\
Famiglia è unita in cui le individualita sono in secondo piano \t prevale unita \\
Ma la famiglia puo sgretolarsi \t anche storicamente, le famiglie che diventano sempre + grandi diventano societa civili (tribu, ...) \\
Smette di esistere quando io affermo il mio volere, la mia volonta, egoisticamente \t perche voglio essere indipendente \t non riconosco + appartenenza, e quindi l'amore \\
Si passa alla societa civile \t in cui è l'utile a governare
\end{document}
