\documentclass[12pt]{article}

\usepackage[a4paper, total={6in, 8in}]{geometry}
\usepackage{textcomp}

\begin{document}
\setlength{\parindent}{0pt}

\def \t {\textrightarrow}
\def \v {\vspace{1em}}
\def \bi {\begin{itemize}}
\def \ei {\end{itemize}}
\def \s[#1] {\section*{#1}}
\def \ss[#1] {\subsection*{#1}}
\def \sss[#1] {\subsubsection*{#1}}

\s[1. Hegel]
Hegelismo \t da all'idealismo di fichte e di schelling un impronta molto personale \t ritenendo che questa impronta sia la conclusione degna, pensa di dare la conclusione all'idealismo \\
L'hegelismo diventera la filosofia tedesca dell 800 \t ci saranno pero tanti contestatori (come Kierkegaard, Schopenhauer, Marx) 

\ss[Vita]
Nasce a Stoccarda nel 1770 \t da famiglia benestante, e fa studi umanistici \\
Si appassiona al mondo classico \t in particolare la cultura greca \\
Nel 1788 studia all'universita di Tubinga \t studia filosofia e teologia (i primi suoi scritti si chiamano "Scritti teologici giovanili") \\
Non gli piace questo ambiente \t ma qua conosce Schelling \\
Nel 1789 scoppia la rivoluzione francese \t di cui condivide gli ideali della rivoluzione, insieme a Schelling \\
Poi fa il precettore e si sposta in germania \t nel frattempo studia storia politica e storia economica \t ma la sua formazione teologica è molto presente \\
Nel 1799 muore il padre \t eredita molti soldi, e quindi si dedica solo allo studio \t nel ? va a Jena \t la sede della cultura del periodo \\
Qui consegue la carriera universitaria come libero docente \t dal 1805 diventa professore straordinario \\
Nel frattempo pubblica "Differenza tra sistema filosofico di Fichte e quello di Schelling" \t punto di partenza del suo pensiero \\
Insieme a Schelling, tra 1802 e 1803 pubblica il "Giornale critico della filosofia" \\
Nel frattempo si matura la sua opera + famosa \t la "Fenomenologia dello spirito", terminata nel 1806 \\
Hegel vede Napoleone a cavallo \t ma guerre napoleoniche lo fanno cadere in disgrazia \\
Conduce il giornale di Bamberga, poi conduce il ginnasio di Norimberga fino al 1816 \\
Pubblica la "Scienza della Logica" tra il 12 e il 16 \t opera + difficile \\
Va ad Heidelberg dove pubblica l'"Enciclopedia delle scienze filosofiche in compendio" nel 17 \\
Nel 1818 si trasferisce a Berlino fino alla sua morte nel 1831 \t anni del maggiore successo \t è anche legato al potere politico \\
Ha praticamente egemonia culturale \t pubblica "Lineamenti della filosofia del diritto" nel 21 \\
Dopo la sua morte gli studenti pubblicano anche i suoi appunti che portava a lezione \\
Muore a berlino nel 1831

\v

Lui è l'antitesi del romantico \t studioso instancabile e regolare, inversato in tutti i campi del sapere \\
Studia molto i filofi passati \t inoltre matura nel corso della sua vita una visione della realtà, e ci arriva lentamente \\
Scrive tantissimo \\
Mentre fa il precettore scrive gli scritti teologici, considerati il punto di partenza di Hegel \\
Tra gli scritti:
\bi
    \item "La vita di gesu"
    \item "La positivita della fede cristiana"
    \item "Lo spirito del cristianesimo e il suo destino" \t qua pone le basi del suo sistema filosofico
\ei
Qual è l'opera è il grande capolavoro? \t no risposta, per alcuni è la fenomenologia dello spirito, ma altri dicono che è la partenza \\
Non c'è una risposta perchè il sistema di Hegel è in continuo divenire \t ci sono delle tappe successive, che vengono rispettate anche dalle sue opere \t è un percorso, non c'è mai un opera conclusiva \\
Non c'è opera che racchiude il suo sistema \t fenomenologia è la + divulgativa, ed è come una introduzion \t gli permette di entrare nell'hegelismo \\
Hegel si è sempre rifiutato di definire degli elementi fondamentali del suo pensiero \t perchè il suo pensiero sono in divenire, quindi anche i capisaldi stanno dentro questo divenire \\
Gli elementi fondamentali vanno colti nel loro divenire \t in realta i capisaldi qualche linea guida fondamentale la definisce 

\v

Chiavi di lettura, i fondamenti, sono 3:
\bi
    \item tutta la realtà è spirito e spirito infinito \t che tiene dentro di se tutto cio che era l'assoluto e l'io puro
    \item la vita dello spirito è dialettica \t si muove di movimento dialettico
    \item la peculiarita di questa dialettica è l'elemento speculativo, che coincide con la sintesi (ovvero l'elemento speculativo)
\ei
La sintesi rappresenta la novità del suo pensiero 

\ss[Scritti teologici giovanili]
All'inizio trascurati, pubblicati solo nel 900 \\
Si sofferma molto su Cristo e sul cristianesimo \\
Nello "Sprito del cristianesimo e il suo destino" \t scritto + maturo \\

\ss[Spirito del cristianesimo e il suo destino]
Contrappone religiostia greca con quella ebrea \\
Quella greca era profondamente armoniosa tra umano e divino \t non c'era scissioni \t gli dei greci erano uomini, non erano giudicatori etc. \t non si aveva il peso degli dei \\
Loro vivevano bene \t non c'era lacerazione tra uomo e divino 

\v

Gli ebrei vedono infatti un scissione completa \t il dio ebraico legifera e giudica \t l'uomo si sente sottomesso, schiavo di questo dio \\
È una forma di religiosita dove la distanza tra uomo e dio è enorme \\
All'inizio pone cristianesimo sullo stesso piano dell ebraismo \t poi modifica e dice che crist. è una sorta di mediazione tra quelle due forme \\
Mediazione avviene attraveso l'amore \t cristianesimo presenta un dio simile a quello ebraico, ma non è tiranno: è un dio che ama quello che ha creato \\
Cristiani non vivono una lacerazione come quella ebraica \t anche se dio legifera e dice come mi devo comportare, dio autosupera questa dimensione di negativita perche ama il suo creato e impone per il bene dell'uomo \\
Recupera una parte di armoniosita attraverso l'amore 

\v

Interessa lo schema \\
Parte dalla grecia: vita felice ma dio non è trascendente \\
Ebraismo: si contrappone alla vita greca, dio è trascendente ma tiranno \\
Da un lato sono felice ma dio non è dio, dall'altro sono infelice ma dio è dio \\
In ciascuna posizione c'è qualcosa di buono \t bisogna combinarle, trovare una mediazione che non sta sullo stesso piano, ma supera entrambi \\
La mediazione è il cristianesimo \t prende la felicita greca ma mantiene il dio ebraico \\
Si arriva al terzo momento, che è il cristianesimo \t dialetticamente rappresenta la sintesi \\
La sintesi è il superamento del positivo della tesi negando il negativo dell'antitesi \\
Si mette insieme il positivo della tesi con il positivo dell'antitesi, rimuovendo la parte negativa \t non è come fichte, ci si porta dietro quello che è buono degli ostacoli \\
Dialettica è un percorso in 3 tappe: 
\bi
    \item tesi (pongo una condizione, punto di partenza)
    \item antitesi (la negazione della tesi, che può anche essere il diverso dalla tesi contrapponendosi)
    \item sintesi (superamento del pos..)
\ei
La dialettica in questo scritto non esiste ancora \t ma la forma è la stessa \\
Qua Hegel aveva gia pensato quella struttura dialettica \t ma non l'aveva ancora tematizzata \\
Per lui religiosita è importante \t per lui esprime gli stessi contenuti della filosofia sotto un'altra forma

\v

Il cristianesimo di cui parla pero rifiuta la dimensione divina di cristo \t nega quindi il fondamento del cristianesimo \\
Nega infatti che cristo fosse stato sia uomo sia dio \t pensa che è stata una lettura sbagliata del cristianesimo

\v

Lui dice che l'elemento qualificante della dialettica è la sintesi \\
A Schelling e Fichte manca la sintesi \t Fichte diceva che l'io supera il non io, ma è un superamento all'infinito \t per Hegel è un cattivo infinito, perche questa opposizione deve trovare un punto di arrivo \\
Se no è un superamento senza senso, senza obbiettivo, e poi il non io è ontologicamente io puro \t alla fine l'io puro deve tornare in se, deve riappropriarsi del finito \\
Alla fine unita tra pensiero ed essere si deve ricomporre \t lo fa schelling, ma senza distinguere \\
Schelling prova a ricomporre frattura tra e \\
L'infinito di hegel è fatto di triadi, composta da tesi+antitesi+sintesi \t si parte da una condizione, contrappongo e poi medio riaffermando il positivo della tesi e negando il negativo della antitesi \\
Cosi creo una sintesi, che è il punto di arrivo, di incontro \t ma ogni sintesi si pone come la tesi di una triade successiva \\
Ed è un percorso di miglioramento \t ogni tesi è una sintesi migliorata \t vado verso il meglio con ogni superamento intermedio \\
Marx è hegeliano \t la storia del mondo è una continua lotta di classe \t da ogni scontro si è generata una nuova societa \\
Piu una classe cresce, piu la negazione si alimenta al suo interno e lo scontro è inevitabile
\end{document}
