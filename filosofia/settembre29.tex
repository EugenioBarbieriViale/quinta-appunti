\documentclass[12pt]{article}

\usepackage[a4paper, total={6in, 8in}]{geometry}
\usepackage{textcomp}

\begin{document}
\setlength{\parindent}{0pt}

\def \t {\textrightarrow}
\def \v {\vspace{1em}}
\def \bi {\begin{itemize}}
\def \ei {\end{itemize}}
\def \s[#1] {\section*{#1}}
\def \ss[#1] {\subsection*{#1}}
\def \sss[#1] {\subsubsection*{#1}}

\s[Hegel]
Senza la contraddizione (l'antitesi) non si va da nessuna parte \\
Il momento importante è la sintesi \t frattura tra finito e infinito si ricompne nella sintesi, rompe l'infinito negativo \\
Ma lo scontro si ha soltanto esasperando l'antitesi \t senza scontro non si ha crescita 

\v

Lo spirito si autopone in essere e continua ad autoporsi in essere secondo una scala di perfezione \t al vertice c'è l'uomo \\
Tutta la realtà è razionale, che poi diventa consapevole nell'uomo \\
La realta è anche razionale \t piano logico e ontologico coincido \t se il pensiero crea l'essere, allora coincidono \t ciò che esiste è reale, ciò che è razionale esiste \\
Lo spirito è la perfetta identita di pensiero ed essere \t ed è gia in partenza la totalita \t mentre in fichte l'essere era conseguente al pensiero (logicamente) \\
Qua invece sono contemporanei \\
Il movimento all'interno dello spirito è quello della dialettica, che si ha in ogni aspetto della realta \\
Ha movimento triadico \\
Diversa dalla dialettica di platone \t che era il cammino razionale che si faceva per andare dagli enti al mondo delle idee \t era quindi statico \\
Per hegel è la legge del divenire \t muove tutto, dinamica, filosofia del divenire, nulla sta fermo \t movimento circolare-spiraliforme e ogni nuova tesi è migliore della precednte \\
\bi
    \item la tesi è il momento astratto/intellettivo \t l'idea in se, la logica, che mi permette di pensare il finito ma è una conoscenza inadeguata
    \item antitesi è quello negativamente razionale \t è il momento della contraddizione, la realta della tesi viene negata
    \item sintesi è momento speculativo/positivamente razionale \t quella negazione dell'antitesi viene negata, si tiene il suo positivo, riaffermo la tesi a un livello più alto
\ei
Nella tesi sono ancora a un punto di partenza  \\
La sintesi non è un superamento in cui abbandono quello che lascio in dietro (come in fichte) \t la sintesi recupera gli aspetti precedenti \\
Positivamente razionale perche recupera tutta la positivita di tesi ed antitesi e la ripropone 

\v

Pensiero ed essere sono tesi ed antitesi \t perche non ho ancora riconosciuto la loro appartenenza allo spirito \\
Pensiero è logico, concettuale, mentre la natura è la realta che posso pensare con la logica \\
La natura è l'antitesi perche la sto contrapponendo al pensiero (di per se sono uguali) \\
Li sto contrapponendo perche non ho capito che sono entrambi spirito \\
Lo spirito quindi quando ritorna in se per se recupera il pensiero e la natura, da cui ha tolto la valenza di opposizione \\
Lo spirito ha superato la contraddizione \\
Si recuperano pensiero e realta nella dimensione piu alta della sintesi \t tra logica e filosofia della natura non c'è contrapposizione \\
Solo che la logica ha sguardo intellettivo, mentre natura dal punto di vista ontologico \t ma sono la stessa cosa \\
All'inzio non so che tutto è spirito \t penso che siano tesi ed antitesi, ma poi elimino la contr. perche riconosco che sono entrambi spirito \\
Questo lo fa l''uomo e anche lo spirito \t tutto quello che faccio io in realta lo fa lo spirito \\
Grazie a me lo spirito arriva a un autoconoscenza di se \t prima si coglie come pensiero, poi esce nella natura e poi si coglie in se per se \\
Spirito lo fa grazie all uomo \\
La logica di Hegel sara quindi anche ontologia \\
Hegel usa "Aufeben" = sintesi, superare nel mantenere ciò che si ha superato 

\s[Fenomenologia dello spirito]
È necessario che l'uomo acquisti una prospettiva corretta \t si deve innalzare sopra la coscienza particolare, di qualcosa di finito, ma deve avere una prospettiva + ampia \\
Ma come si fa? \t pero condanna il metodo, come quello di Cartesio \t non ci puo essere una filosofia che introduca la filosofia \\
Quindi racconta direttamente tutta la strada che l uomo deve fare per mettersi dalla prospettiva dell'assoluto \t non parla di metodologia \\
Attraverso triadi fa vedere come si fa ad arrivare a consapevolezza dell'assoluto \\
Ma la conoscenza dell assoluto ha due significati \t io uomo che non ho capito che tutta la realta è spirito, compio un percorso individuale che arriva a comprendere di essere spirito \\
Pero c'è alro percorso \t io uomo sono anche spirito \t io individuo faccio percorso che mi permette di capire di essere sprito, ma anche spirito fa questa strada \\
Quindi ci sono due piani: uno è la storia del singolo, altro è quello dello spirito \\
Ci sara una storia dello spirito che sara individuale, ma poi una storia dello spirito (che vive attraverso gli uomini) dell'umanita \\
Si parte da una descrizione di come il singolo puo realizzare di essere spirito, e poi vedere come spirito è diventato spirito \t quindi i due piani sono intrecciati \\
Nel singolo lo spirito ci arriva attraverso l'individuo, ma poi anche su un piano universale, costituito da tutti gli uomini

\v

Triadi si esprimo a tutti i livelli \t tradi di partenza di suddividono in sottotriadi etc... \\
Se tutta la realta è spirito, tutta la realta ha movimento dialettico \t quindi la realta si puo analizzare secondo la dialettica \\
Spirito parte dalla triade dell'uomo e la triade dell spirito in quanto tale \\
Quella dell uomo prevede coscienza (tes), autcoscienza (ant) e ragione (sin) \\
Ma all'interno di ognuna c'è un altra triade \t quindi 9 \t e cosi si ramifica in avanti \\
Quella dello spirito è: spirito (tesi), religione (ant), sapere assoluto (sin) \\
Queste due triadi non hanno relazione dialettica \t perche si sovrappongono, in una si parla dell uomo e nell altra spirito \t che pero sono la stessa cosa \\
Non sono legate dialetticamente, ma sono due punti di vista diversi \\
Fenomenologia mi permette di leggere la realta in modo corretto \t cosi posso leggere la realta dialetticamente e mi riconosco come spirito \\
Dopo essere arrivato la (con la fenom. sp.) possono cominciare a indagare la realta
\end{document}
