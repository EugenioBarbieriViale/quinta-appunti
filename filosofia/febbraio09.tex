\documentclass[12pt]{article}

\usepackage[a4paper, total={6in, 8in}]{geometry}
\usepackage{textcomp}

\begin{document}
\setlength{\parindent}{0pt}

\def \t {\textrightarrow}
\def \v {\vspace{1em}}
\def \bi {\begin{itemize}}
\def \ei {\end{itemize}}
\def \s[#1] {\section*{#1}}
\def \ss[#1] {\subsection*{#1}}
\def \sss[#1] {\subsubsection*{#1}}

\s[Heiddeger]
Parte dalla coscienza intenzionale di Usserl \t era una coscienza che era di qualche cosa, aperta ad altro \t e così è l'esserci di Heiddeger \\
L'uomo di H. è progetto, non puo fare a meno di progettarsi \t la sua è la dimensione della possibilita, e la sua coscienza è aperta al mondo \\
Non è un esistenzialista puro \t perche suo pensiero si fonda su una domanda ontologico \t e fa rfilessione ontologiche etc. \\
La domanda è qual'è il senso dell'essere \t e serve a scoprire se questo essereci, che è l'unico ente che si pone questa domanda, sia in grado di darsi una risposta \\
Per rispondere a questa domanda bisogna cercare via privilegiata all'essere \t che è l'uomo \\
La domanda diventa esistenziale perche indaga l'uomo \t nella dimensione dell'esistenza dell'esserci c'è una risposta? \t alla fine dira no, l'uomo non è in grado di arrivare al senso dell'essere \\
La rilfessione esistenzialista è una parentesi, una parte di essere e tempo \t non c'è una continuita esistenziale, ma la sua riflessione è famosa \\
Analitica esistenziale \t analizza l'esserci

\ss[L'esserci]
L'esserci ha l'esistenza, mentre tutti gli altri enti sono solo presenti \t vuol dire che stanno li, nel senso che gli manca la dimensione dlela possiblita \t sono cosi come devono essere, e realizzano in forma necessaria la loro esistenza (kierk.) \\
L'esistenza è possibilita, e l uomo è l unico che ha la progettualita \t l'esistenza è sola dell uomo \\
Distingue anche la dimensione della presenza da quella dell'uomo \t uomo esiste su un piano ontologico, mentre enti su un piano ontico (piano dell essere ma della presenza) ed esistintivo

\v

L'esserci è l'unico ente che vive la dimensione della trascendenza \t l'esistenza è essenzialmente trascendenza, vado sempre oltre cio che sono, supero sempre la posizione in cui mi trovo \t non in un senso metafisico \\
L'esserci come prima notazione è caratterizzato dagli esistenziali \t sono i caratteri connaturati alla dimensione dell'esserci, non degli attributi contingenti, ma appartengono alla natura dell'esserci \\
I caratteri caratterizzano il rapportarsi con il mondo e con gli altri e sono:
\item l'essere-nel-mondo
\item l'essere-con-gli-altri
\item l'essere-per-la-morte \t è l'esistenziale che definisce il tono dell'esistenza dell'uomo, e senga lo spartiacque tra un esistenza autentica e una inautentica

\ss[Essere- nel mondo]
Uomo non sta nel mondo in modo contemplativo \t uomo è trascendenza, è sempre in divenire \t guarda sempre oltreo \\
L'essere nel mondo si traduce nello stare nel mondo stando nel mondo come una serie di strumenti che uso per realizzare il mio progetto \\
Uso il mondo per realizzare il mio progetto \t il mondo mi si presenta come un insieme di utensili, e lo sguardo dell uomo non è uno sguardo conoscitivo o contemplativo, ma è legato all'utilizzo \\
Io uso il mondo per realizzare il mio progetto, ma cosi mi do anche un senso \t il mondo ha un senso perche è il mezzo per realizzarmi, se no non avrebbe senso \\
L'essre delle cose equivale all'essere utilizzato da me \t se non ho un progetto, il mondo non ha un senso \t se vivo in modo contemplativo, non ha senso il monodo \t perche non lo uso, non fa parte del mio progetto \\
In questo modo realizzo anche l'esssenza degli enti \t e do un senso al mondo \t l'ente si realizza quando io lo uso, gli do un senso \t realizzo la sua finalita, essenza utilizzandolo per il mio progetto \\
Se il mondo fosse fatto solo di esseri presenti, mondo non avrebbe senso \t nessuno realizzerebbe la loro essenza, e starebbero li senza un senso \\
Sta quindi dicendo che gli enti hanno un finalita \t e se non la realizzo, non hanno senso \\
L'esserci da senso al suo progetto dando senso al mondo

\ss[Essere-con-gli-altri]
Come l'ess. nel mondo \t si posson orealizzare in una esistenza autentica e non, definita dal mio rapporto con la morte \\
Oltre all'essere nel mondo, uomo si trova in un contesto sociale \t l'uomo è gettato nel mondo, ovvero viene lanciato nel mondo e si trova in mezzo a degli enti e ad altri esserci \t io mi colgo nel mondo in mezzo agli altri \\
Essere con gli altri è una conzioine originaria, come essere nel mondo \\
L'essere con gli altri realizza l'intenzionalita della coscienza di Usserl \t io sono aperto ad altri come gli altri sono aperti al mondo \t siamo tutti rivolti verso un mondo che condividiamo, verso cui abbiamo apertura comune \\
Essere con gli altri significa prendersi cura degli altri \t prendersi cura è molto presente, ed ha la accezione di permettere agli altri/realta di realizzare se stessi \\
Bisogna anche prendersi cura del mondo \t e significa curarsi degli oggetti che mi servono per realizzare il mio progetto, e allo stesso tempo permettere loro di realizzarsi \\
Lo stesso faccio con gli altri \t faccio in modo che gli altri realizzano la loro essenza, ovvero essere per la morte \\
Stare con gli altri vuol dire aiutare gli altri ad assumersi le loro preoccupazioni, le loro cure \t ovvero la loro dimensione di morte \\
Stare con gli altri significa aiutare gli altri ad avere consapevolezza di cio che sono, e di realizzare il loro essere per la morte \t l'uomo è un ente finito, e deve fare i conti con questo \t l'uomo disperato non accetta di essere fintio (kierk.) \t in H. pero l uomo non si dispera

\v

La modalita inautentica di stare con gli altri è quella del diveritmento fine a se stesso \t io sto con gli altri per sottrarli alle loro cure, alle loro preoccupazioni \t sto con gli altri per allontanarci dalla nostra natura \\
Diventa quindi un modo di coesistere \t nessuno si preoccupa \\
Il modo inautentico di stare nel mondo è di non avere un progetto \t ma come faccio a non averne uno se esistenza è porgetto ? \t non ne ho uno se vivio nel mondo del si (si impersonale) \\
Non ho un progetto quando vivo come si vive, penso come si pensa, ovvero mi adeguo \t mi calo in un mondo in cui tutti vivono in modo, e io mi perdo in questa unita \\
Io vivo cosi perche lo fanno tutti \t quando io guardo la mia vita, sembra che la mia vita abbia realizzato un progetto \t ma poteva non essere il mio \t possono anche dissentire, andare in un altra direzione \\
Il mondo dell'esistenza inautentica è il mondo dell impersonalita \t vivo cosi come un ente, cado sul piano della semplice presenza \t ed è la deiezione, al caduta dell'uomo sul piano delle cose del mdono, sul piano esistentivo, ontico \\
L'esistenza inautentica ha caratteristiche riconoscibli:
\bi
    \item chiacchera \t io chiaccero ma non dico niente \t io dico quello che si dice, parlo quello di cui si parla \t la chiacchera ha pero bisogno di alimentarsi \t e si alimenta con la curiosita  \t che non è la curiositas, ma la banalizzazione della informazione
    \item curiostia \t è come un mare in cui si annega, cerco tutte le storielle che possono diventare chiacchera \t e di fatto non sto parlando di niente
    \item equivoco \t la chiacchera genera l'equivoco \t che è l'alimento della esist. inautentica, perche rende annebbiate tutte le informazioni \t se non parlo di conoscenze, della mia esistenza, alla fine la chiacchera cade nell'ecquivoco che alimenta la curiosita e poi la chiacchera
\ei
Cosi anche il linguaggio si trasforma \t io parlo di niente, quindi il lingauggio è vuoto  \\
Perche l'uomo puo cadere in questa esistenza inautentica?

\ss[Essere-per-la-morte]
Perhce l'uomo ha paura di morire \\
La morte è la possibilita che tutte le possibilita diventano impossibili \t è la fine, è una possibilita certa \\
È cio che chiude ogni forma di progettualita \t smette di poter essere, ed è l'unica certezza \t è una possibilita che non so quando accadra \\
L'esistenza contempla in modo necessario la morte \t e non posso sfuggire, mi appartiene ontoglicamente \t posso fare mille scelte, ma non possono non morire \\
L'uomo quindi cosa puo fare? puo viverlo in due modi \\
Lo vivo in modo autentico assumendomi la consapevolezza dell'essere per morte con la decisione anticipatrice, accetto la morte nella mia vita \\
Vuol dire che tutti i miei progetti sono vani, finiranno \\
Allo stesso tempo pero, questo essere per la morte che mi mostra una fine, relativizza \t con la presenza consapevole della morte nella mia vita, io mi oriento \t ci sara una scala di priorita nella mia vita \\
Tutto finira, quindi ci sono cose a cui non posso rinunciare \t mentre altre non son rilevnati \\
È una prospettiva di una vita impegnativa, con la consapevolezza della morte \t la morte appartiene alla dimensione del finito \\
Sai della vanita del tuo progetto, e sai che alcuni progetti sono + importanti \t questo orienta nella vita \\
È una vita impegnativa anche perche pressupone un essere pensante \t chi non si fa domande, vivo piu leggero \t non ho prospettiva di senso \\
Lui aveva rinnegato la forma storica del nazismo, ma non la parte ideologica

\v

L'esistenza autentica non è quindi felice, ma è un esistenza angosciata \t è il sentimento dell'indefinito \\
In kierk. nasce dal sentimento del possibile \t sono davanti a infinite possibilita, quindi provo angoscia \t non ho scelta orientata \\
In H. angoscia nasce davanti all'essere per la moerte \t la possibilita non mi angoscia perche è conntaturata all'esistenza, ma sono angosciato davanti al mio essere finito \t sono angosciato davanti alla accettazione del mio limite e di portarlo nella mia vita \\
L'angoscia è l'angoscia per l'essere per la morte \t è sempre un angoscia che provo davanti al nulla, davanti all'indeterminato \\
Ma in H. è il nulla della morte, ed il nulla del senso dei miei progetti \t non hanno senso per via della morte \\
In k. era sempre l'indefinito ma nel senso delle infinite possibilita \\
In sartre l'angoscia è di fronte alla liberta, al nulla della liberta \t tutto è euqamente possibile, e mio devo assumermi la responsabilita della mia scelta 

\v

In esistenza inaut. c'è allontanamento della morte \t non parla della morte, e si parla sempre della morte degli altri \t non della mia \\
Io parlo della morte come un fenomeno che riguarda le altre persone \t l'esistenza inautentica allontana in tutti i modi dalla morte \\
È l'esistenza di chi non è neanche in grado di pensare che io sono finito \t e quindi l'angoscia viene banalizzata in paura \\
Io non posso non essere angosciato \t se sono progetto, posso vivere inautenticamente, ma guardo sempre avanti \t e alla fine c'è sempre la morte \\
Ho quindi una forma ridimensionata dell angoscia, che è la paura \t è un sentimento dell'ente, che è controllabile \t è un sentimenot che lego a un ente finito, in assenza del quale non lo provo piu \\
Lego la mia angoscia a un ente, cosi la posso gestire meglio \t banalizzo l angoscia alla paura cosi la posso controllare \\
Mentre l'esistenza autentica è angosciata
\end{document}
