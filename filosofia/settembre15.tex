\documentclass[12pt]{article}

\usepackage[a4paper, total={6in, 8in}]{geometry}
\usepackage{textcomp}

\begin{document}
\setlength{\parindent}{0pt}

\def \t {\textrightarrow}
\def \v {\vspace{1em}}
\def \bi {\begin{itemize}}
\def \ei {\end{itemize}}
\def \s[#1] {\section*{#1}}
\def \ss[#1] {\subsection*{#1}}
\def \sss[#1] {\subsubsection*{#1}}

\s[Idealismo - 15 settembre]
Kant mette insieme empirismo e ragione \t il mezzo per conoscere è la ragione e posso conoscere solo il fenomeno \\
Kant si colloca tra illuminismo e romanticismo \t esprime la ragione illuminsta \\
Ma si intreccia con il romanticismo \t infinito, sentimento, sturm und drang \t categorie che nell' illuminismo non ci sono \\
Alcune cose si possono cogliere con altro oltre la ragione \\
La filosofia kantiana viene riletta secondo questa nuova visione romantica \t si vogliono trarre le conclusioni \\
Fichte infatti vuole trarre le conclusioni \t le da lui, pero in un clima culturale diversa \\
Da una rilettura che stravolge il kantismo \t da vita all' idealismo \t a cui Kant si oppone \\
Riconosce che ci sono delle fallacie, ma la sua filosofia viene accolta come quella di Fichte \\
Nasce cosi l'idealismo tedesco \t Fichte ne è il fondatore \\
Idealismo interpreta Kant, si pone come la lettura corretta \t non crea una nuova filosofia basandosi sul criticismo \\
Infatti l'idealismo non è Kant

\s[Fichte]
Elementi che rivede (quello che l idealismo trattiene del criticismo):

\sss[Lettura ontologica del trascendentale]
Trascendentale \t tutto cio che ha che fare con la conoscenza a priori \\
In san tommasso era legge dell oggetto, in kant del soggetto \t io do le condizioni di conoscibilita \\
Fichte dice che da anche le condizioni di esistenza \t il soggetto fa essere l oggetto \\
Passa dalla dimensione gnoseologica a una ontologica \t questo vuol dire stravolgere il pensiero di Kant \\
Quello che in kant era conoscitivo qua viene letto ontologicamente

\sss[Distinzione tra fenomeno e noumeno]
Questa distinzione non era secondaria in kant \t sanciva una distinzione tra cio che posso conoscere (la realta che mi appare) \\
Distinzione era completamente netta \t limite dell intelletto \\
Idealismo fa cadere questa distinzione \t idealismo fa coincidere fenomeno e noumeno \t la sostanza del noumeno si manifesta nel fenomeno, sono la stessa cosa \\
Noumeno si puo conoscere \t questo significa anche ammettere l'intuizione intellettiva (che per kant solo il creatore della realta poteva avere \t non aveva bisogno dei sensi, perche la realta prima era nel suo pensiero)

\v

Idealismo presuppone un Soggetto che da le condizioni di conoscenza e di esistenza \t lui è la causa dell' essere della realta \\
E anche un intelletto creatore \t ma questo Soggetto non sono io \\
Questo Soggetto pone in essere la realta \\
Se lui fa essere la realta e lui è pensiero \t ed è la stessa sostanza della realta, non è un dio creatore \t è una formazione dalla mia materia \\
Quindi il principio che da vita alla realta è l'intuizione intellettiva che kant aveva negato

\v

Fichte dice queste cose nella "Dottrina della Scienza" \t 1794 \\
Poi la modifica per tutta la sua vita ed è testo fondante dell' idealismo \\
Fa anche cambiamenti sostanziali 

\ss[Vita]
Nasce nel 1762 da una vita contadina molto povera \\
Un barone ricco gli paga gli studi \t era stato colpito quando aveva sentito fichte da ragazzo che ripeteva a memoria una predica \\
Studia teologia a Jena, poi Lipsia \t anni difficili preche i soldi del barone smettono di arrivare \t da lezioni private \\
Fa il precettore a Zurigo dove conosce la moglie \\
Poi nell anno 1790 c'è anno decisivo \t studia kant perche un suo studente gli chiede delle lezioni su kant \\
Quindi deve leggerlo \t questa è la sua grande luce \t cambia la sua visione del mondo \\
Fichte poi scrive "Saggio sulla critica di ogni rivelazione" \t applica perfettamente i principi del criticismo \\
Va a Konigsberg dove lo presenta a Kant, che lo apprezza \t viene pubblicato con l'intermezzo di Fichte \\
La pubblica pero anonima \t e viene confusa con un opera di Kant \t poi rivelera dopo il nome dell autore \\
Nel 94 Fichte viene chiamato all universita di Jena \\
Anni dal 91 al 99 è il periodo dello splendore della filosofia di Fichte \\
Poi nel 99 entra nella polemica dell'ateismo \t lui sostiene che dio coincide con l ordine morale del mondo \\
Non si puo negare dio perche da ordine razionale alla realta \t accademia non è d'accordo, quindi va a Berlino \\
Qua conosce anche Schegel \\
Nel 1808 scrive "I discorsi alla nazione tedesca" \t inneggia alla riscossa della prussia dopo essere stata sconfitta da napoleone \\
Afferma anche primato del popolo prussiano \t questo lo riporta in auge \\
Diventa rettore dell universita di Berlino \t nel 14 muore di Colera \\

\ss[Dottrina della scienza]
La filosofia critica deve diventare una scienza rigorosa che ha origine in un principio primo \\
Il discorso di Kant non è conclusivo \t bisogna evidenziare il fondamento del suo discorso \\
A kant manca un principio primo \\
Elenca quindi 3 punti fondamentali che sono i capisaldi dell'idealismo (segnano il passaggio dal criticismo all idealismo)

\sss[Primo principio]
Aristotele aveva detto che il primo principio della scienza e della realta è quello di non contraddizione \\
Kant aggiunge anche il principio di identita \t è ancora + originario di quello di non contraddizione \\
Fichte dice che manca un pezzo \t A = A, ma vale solo se A esiste \t legame logico A = A presuppone l'esistenza di A \\
Principio di identita parte da qualcosa che gia c'è \t quindi non puo essere originario \\
Il principio supremo deve essere incondizionato e che si autopone in essere \t ha la garanzia della sua esistenza in se stesso \\
A = A si traduce in se A esiste allora A = A \t ma non si puo mettere a fondamento un ipotesi \\
Non deve avere condizioni antecedenti \t deve essere incondizionato \\
Principio incondizionato non è causato da qualcosa \t si autopone nell essere \\
Se A esiste \t non è un esistenza necessaria, perchè A potrebbe non esistere \t identita vale solo se è posto in essere

\v

Principio primo deve avere le condizioni di esistenza in se stesso \t deve essere incondizionato e autoporsi nell'essere \\
Questo principio lo prendo in Kant \t è l' Io Penso \\
Kant aveva dato unita alla conoscenza \t Io Penso era l'unita trascendentale delle categorie \\
In Fichte diventa Io Puro \t fichte legge l'io penso (un trascendentale kantiano) in senso ontologico \t non è quindi la percezione fonamentale della conoscenza, ma dell esistenza \\
Io Penso diventa l'Io Puro che permette alla realta di esistere \\
In kant io penso non ha nulla di antecedente \t la conoscenza parte da li \t anche in fichte, diventa il principio incondizionato della realta \\
\textbf{Primo fondamento}: C'è un principio primo che si autopone in essere \t ed è l' Io Puro \\
Io penso era un attivita \t adesso è una realta \\
Fichte sta dicendo che il pensiero viene prima dell essere (non come diceva aristotele) \t il pensiero genera l'essere = intuizione inellettiva \\
Io puro è anche condizione di esistenza per cui posso dire A = A
\end{document}
