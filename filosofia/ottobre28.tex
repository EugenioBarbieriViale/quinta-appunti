\documentclass[12pt]{article}

\usepackage[a4paper, total={6in, 8in}]{geometry}
\usepackage{textcomp}

\begin{document}
\setlength{\parindent}{0pt}

\def \t {\textrightarrow}
\def \v {\vspace{1em}}
\def \bi {\begin{itemize}}
\def \ei {\end{itemize}}
\def \s[#1] {\section*{#1}}
\def \ss[#1] {\subsection*{#1}}
\def \sss[#1] {\subsubsection*{#1}}

\s[Marx]
L'aggancio tra la sua ciritica e riflessione sua autonoma è la riflessione sulla religione di Feuerbach \\
È d'accordo con Feuerbach, ma lui non è andato alla radice del problema \t la radice permette di modificare l'infelicita dell uomo e permette di eliminare dio \\
Una societa positiva è senza dio \t perchè vuol dire che l'uomo trova le sue soddisfazioni nella societa in cui vive \\
"Religione è oppio dei popoli" \t perche stordisce gli uomini, che non capiscono le cause della loro infelicita

\v

Causa è l'alienazione del lavoro \t Marx passa dalla cirtica del cielo alla ciritica della terra \\
Nella societa capitalista uomo viveva condizione di alienazione \t vengono espropriati del loro lavoro (lavoro viene strappato all'uomo nella sua dimensione realizzante) e porta alla non realizzazione dell uomo \\
Uomo cosi e estraniato e alienato \t e uomo non si puo realizzare altrove \\
Per marx uomo è cio che produce \t non ha dimesnione metafisica, non ha finalita \\
Cio che distingue l uomo da tutti gli altri enti è la capacita dell uomo di produrre i suoi mezzi di sussistenza \t uomo è in grado di produrre, mentre animale per esempio sfruttano semplicemente la natura \\
Quindi se uomo vive dimensione alienata di lavoro, non potra mai realizzarsi ed essere felice \\
Uomini si distinguno dagli animali quando prodcuno i propri mezzi di sussistenza

\v

Uomo dovrebbre realizzare la sua progettualita di uomo \t lavoro realizzante dovreebbe essere:
\bi
    \item deve progettare e pensare cio che vuole creare
    \item poi deve possedere gli strumenti del suo progetto
    \item poi realizza il suo progetto (che pero prima è pensato)
    \item alla fine oggetto realizzato deve appartenere totalemente all uomo, perche è frutto della sua progettualita
\ei
Uomo possiede l'oggetto totalmente \t anche economicamente \t puo farne quello che vuole \\
Uomo possiede l oggetto perche lo ha pensato \t una cosa è mia perche l ho pensata io, appartiene al mio orizzonte di progettualita \\
Questo fa parte della proprieta personale \t ha la sua origine nella mia creatività \\
Questo è un lavoro realizzante \t come quello delle api 

\v

In capitalismo non è cosi \t lavoro è originato dal bisogno \\
Nella societa capitalista non c e dimensione della progettualita \t uomo viene pagato perche deve sopravvivere, e operaio non possiede ne progetto, ne mezzi per produrlo, ne oggetto finale \\
L'oggetto finale non appartiene all operaio, ma non se lo potra neanche permetterlo \t ha un valore molto piu superiore \\
Tutto viene strappato \t operaio è solo una merce in mano alla societa capitalista \\
Alienaizone del lavoro è l'origine di tutte le altre alinazioni vissute dall uomo \t come quella politica (stato è superiore all uomo) e quella religiosa \\
Uomo deve essere quindi tolto da questa condizione di bruto \t per uscirne serve lotta di classe \\
Uomo quindi è solo libero nelle funzioni basiche \t mentre per la sua attivita lavorativa, che e quella propriamente umana, uomo è parte di un ingranaggio \\
È libero nelle funzioni animali (bere mangiare ...) mentre animale nelle funzioni che dovebbero realizzarlo \\
Per uscierne serve lotta di classe \t ma prima due tesi teoretiche:

\ss[Materialsimo dialettico]
Realta come materia, diviene secondo un processo dialettico \t struttura della realta è dialettica, la materia si evolve in modo triadico \\
Ma hegel parla di spirito, marx invece parla di materia \t prospettiva materialista \\
Marx critica hegel in quando è il finito che genera l infinito, che pero alla fine viene nefato \t è un invenzione umana \\
È tutto finito, tutto materia \t non infinito, che viene svalutato \\
Dialettica dovrebbe continuare all'infinito \t contraddizione perche comunismo è arrivo \\
In realta pero la materia è finita \t prima o poi materia arriva a una fine \t nello spirito dimensione infinita è assoluta, mentre per la materia, che è finita, si puo considerare un punto di arrivo \\
Anche se noi non siamo arrivati mai alla societa comunista \t tutte le attuazioni si sono fermate alla dittatura del proletariato \\
Realta che è tutta materia e si evolve dialetticamente \t Marx dice di aver riportato in terra la dialettica hegeliana \\
Hegel ha trasformato il pensiero in soggetto indipendente, che pero esiste solo in correlazione con l'esistenza \\
L'idale è l'elemento materiale tradotto in quello che c'è ??? \\

\ss[Materialsimo storico]
Leggere la storia come un susseguirsi di fatti e rapporti economici \t storia ha struttura esclusivamente economica \\
Arte, letteratura, etc \t è sovrastruttura \\
Storia è fatta da rapporti di produzione che si sussegunono nel tempo \t l'essenza della storia è la produzione, perche uomo è ente che produce \\
Il pensiero economico del perido è cio che genera la sovrastruttura \t struttura economca definisce la sovrastruttura, ovvero tutto cio che non è economia \\
Ribalta \t storia classica legge economia di un periodo come un prodotto dell'ambiente storico \\
Sovrastruttura è ideologica \t struttura della realta invece è economica \\
La storia vera è fatta dagli uomini che hanno agito nella realta, che hanno modificato la natura \t che hanno prodotto qualcosa \\
Tutto il resto (morale, religione, etc) non hanno storia autonoma \t non possono prescindere dal rapporto economico \\
La base economica muta, muta anche tutto il resto 

\v

Questi tre aspetti sono quelli speculativi \t poi passa alla parte pragmatica: la lotta di classe

\ss[Lotta di classe]
Condizione di frustrazione e alienazione si risolve con una rivoluzione, ovvero la lotta diclase \t è il mezzo \\
Tutta la storia sociale è storia di lotta tra classi \t liberi/schiavi, patrizi/plebei, etc \\
Storia è uno continuo scontro tra oppressi e opressori \t a volte è stato anche uno scontro latente, non violento \t ma ha sempre portato a un cambiamento (che poteva essera la rivoluzione con nuovo assetto sociale, o la rovina di entrambe le classi, nuovo modello sociale) \\
La sua epoca presenta un antagonimso tra classi molto semplificato, ma evidente \t le classi sono due: borghesia e proletariato \\
Borghesia viene definita da Hengels come: la classe dei morderni capitalisti, ovvero colo che possiedono i mezzi di produzione, e sono in grado di assumere dei salariati \\
Il proletariato: è la classe dei salariati, che non hanno niente (unica ricchezza è la prole), e il proletario deve vendere la sua forza lavoro, che è l'unica cio che ha \\
Scontro è sempre dialettico \t borghesia possiede i mezzi e proprieta privata, esaspera la sua relazione con il proletariato (antitesi) che porta a uno scontro con sintesi superiore \\
Scontro è inevitabile \t ogni tesi genera necessariamente al suo interno l'amplificarsi della sua contraddizione, per poi esplodere \\
In borghesia devo avere dei salaraiti \t piu diventa potente e si allarga, piu proletari ci saranno \\
Piu societa si irrigidisce nella sua contraddizione, piu alimenta il proletariato \t che sara cosi numeroso ed esasperato che si arrivera a sintesi \\
Tesi alimenta al suo interno negazione \\
Societa capitalista è sintesi della societa feudale \t borghesia, che era nata (commercianti artigiani) distrugge societa feudale \\
Questo perche era produttrice di beni \t e quindi possono attuare una rivoluzione e attura nuova societa \t in questo caso nasce societa capitalista \\
Societa capitalista sta alimentando al suo interno il proletariato, che fara la rivoluzione \t sta alimentando i suoi distruttori \\
Lotta di classe non avra come punto di arrivo la societa comunista \t prima passaggio intermedio: dittatura del proletariato, che è necessaria per transizione
\end{document}
