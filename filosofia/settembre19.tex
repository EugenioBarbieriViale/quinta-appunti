\documentclass[12pt]{article}

\usepackage[a4paper, total={6in, 8in}]{geometry}
\usepackage{textcomp}

\begin{document}
\setlength{\parindent}{0pt}

\def \t {\textrightarrow}
\def \v {\vspace{1em}}
\def \bi {\begin{itemize}}
\def \ei {\end{itemize}}
\def \s[#1] {\section*{#1}}
\def \ss[#1] {\subsection*{#1}}
\def \sss[#1] {\subsubsection*{#1}}

\s[Fichte]
Scopo dell'attivita morale ??? \\
Percorso etico non arriva mai a un punto fermo \t c'è sempre un ostacolo da superare \\
Anche la realizzazione morale non puo essere raggiunta \t contraddice la natura della realta \\
La mia realizzazione tende a, non raggiunge mai uno status \t io mi realizzo nel mio farmi virtuoso, non nell'essere virtuoso \\
Io non mi colgo mai in uno stadio definitivo \t infinito confronto con la realta \t infinto andare verso la liberta, perfezione, virtu \t si tende a qualcosa \\
Liberta ??? \\
L'uomo realizza se stesso in un progressivo avvicinamento alla virtu \\
Nel mio confronto con la realta c'è sempre una progressione \t è un andare sempre verso un perfezionamento continuo \\
Il miglioramento sta solo nel superamento del non io, che ogni volta supero in modo migliore 

\v

\ss[Seconda fase della filosofia di Fichte]
Dopo il 1800 la filosofia cambia \t cambia in modo sostanziale, ma non se ne rende conto \t cerca di dire le stesse cose di prima con parole diverse \\
Ora vuole approfondire la dottrina della scienza in via metafisica e mistico-religiosa \\
Se Io Puro, al posto di essere un principio di razionalita, lo chiamo dio \t allora si raggiunge una visione panteista \\
La realta è dio \t Spinoza: perfetta coincidenza tra la sostanza dio e realta \\
Se Io puro coincide con dio \t dio rappresenta la massima perfezione \t si autopone in essere e si manifesta nel finito \t ma allora questa massima perfezione è perfezione quando si autopone in essere o quando genera il finito? ?? \\
Vuol dire anche che la realta è perfetta \t ma è perfetta singolarmente o nel suo complesso? \\
Io puro non era perfetto \t non poteva non essere, infatti genera il non io \t ma dio è perfetto, perche si deve esprimere nel non Io? È gia perfetto di se stesso \\
Ora la realta è attraversato dalla perfezione (dio), prima solo dalla razionalita (io puro) \\
Nel 1806 opera "Introduzione alla vita beata" \t afferma una visione panteista

\ss[Discorso alla nazione tedesca]
Sono stati letti come un opera nazionalistica, che genera delle premesse teoriche del nazismo \\
Non e proprio cosi \t contengono affermazioni nazionalistiche e di esaltazione della popolazione germanica, che abbia la missione di risollevarsi \\
Sono stati scritto dopo le guerre napoleoniche \t riguradano la prussia \\
Sta dicendo che bisogna riprendersi, non conquistare il mondo \\
Inoltre avevano un obbiettivo pedagogico \t voleva mostrare come recuperare i propri valori \t non era un inento veramente politico \\
Pero un inneggiare alla potenza germanica è sicuramente presente \t ma poi questo testo viene strumentalizzato \\

\s[Schelling]
Parte da Fichte (che era partito da Kant) \\
Era amico di Fichte \t lo affianca all universita \t poi scrive "Sistema dell'idealismo trascendentale" \\
Trova in Fichte delle mancanze \\
Era inoltre l'idealista più romantico di tutti \t di Fichte intravede le difficolta del suo pensiero, vede delle lacune \t in paticolare una, da cui parte \\
Il limite di Fichte sta proprio nell' idea di natura \t fichte riduce la natura tutta in un io \t non fa alcuna differenza nel non io, che rappresenta tutta la realta non cosciente \\
Ma per schelling la natura rappresenta un valore, è viva \t non c'è una consapevolezza come nell uomo, ma lo stesso non è completamente morta \\
Inaugura il suo pensiero con la "Filosofia della natura" \t anche qui ci sono 6 fasi della sua filosofia (in realta esiste un filo conduttore nelle prime 4, poi c'è una svolta dove accenta taglio metafisico)

\ss[Filosofia della natura]
% In realta la prima fase è chiamata "primi fichtiani" \\
La natura non puo essere considerata non io, perche ha una forza viva \t la natura è lo spirito visibile, e l'assoluto è la natura invisibile \\
L'assoluto si autopone in essere, in realta non genera qualcosa privo di consapevolezza \t genera gradi di perfezione \\
Non c'è un io puro che genera gli ostacoli \t lo spirito di schelling è una scala, e si sale di grado di perfezione \\
L'uomo è il punto di arrivo della natura \\
Toglio la natura dal non io, ma cosi io puro non esiste più \\
Se io puro genera tutta la realta, come mai nel non io non c'è nulla di questa coscienza? \t per schelling c'è anche nella natura \\
Razionalita non puo generare qualcosa dove di razionale non c'è \\
C'è una scala ascendente \t va dallo spirito che si autopone in essere, e poi sale i gradi di perfezione fino all'uomo \\
Qua natura partecipa di questa presenza dello spirito 

\v

Natura e spirito sono una faccia della stessa medaglia \t natura è la faccia fenomenica dello spirito, che invece è l'aspetto metafisico \\
Allora le stesse forze che agiscono nello spirito agiranno nella natura \t lo spirito agisce con diversi gradi di perfezione, cosi la natura \\
La crescita della natura in gradi di perfezione, cio che si evolve secondo i gradi, è lo spirito, non la natura \\
Lo spirito genera il sasso, che poi non diventa il cane \t spirito genera sasso, poi cane, poi uomo
\end{document}
