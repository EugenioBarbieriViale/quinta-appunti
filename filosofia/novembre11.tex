\documentclass[12pt]{article}

\usepackage[a4paper, total={6in, 8in}]{geometry}
\usepackage{textcomp}

\begin{document}
\setlength{\parindent}{0pt}

\def \t {\textrightarrow}
\def \v {\vspace{1em}}
\def \bi {\begin{itemize}}
\def \ei {\end{itemize}}
\def \s[#1] {\section*{#1}}
\def \ss[#1] {\subsection*{#1}}
\def \sss[#1] {\subsubsection*{#1}}

\s[Schopenhauer - Il mondo come volontà e rappresentazione]
Oggetto e soggetto non prevalgono l'uno sull'altro \t il soggetto legge l'oggetto secondo una sua rappresetnazione, e l'oggetto non puo uscire dalla sua rappresentazione \\
Pero fenomeno rimanda al noumeno (Kant) \t questa è la realta che mi appare, ma quindi ci deve essere una realta vera \t è la volonta \\
Volonta la scopro in me come particolare, e poi in tutta la natura \t c'è una tendenza a mantenersi nell'essere, una volonta di vivere in tutti gli enti \\
Volonta è essenza metafisica dell uomo, ma anche di tutta la realta \\
Uomo \t è la manifestazione consapevole della volonta, che pero è presente in tutti gli enti di natura (non oggetti inanimati) \\
Volonta è unica, e poi si manifesta negli enti di natura in modo particolare

\v

Questa volontà si definisce come una volonta insaziabile, eternamente insoddisfatta \t cieca e libera, senza scopo \t è una volontà irrazionale \\
Uomo è attraversato dal continuo desiderio di volere qualcosa \t senza pero essere appagato da nulla \\
Uomo è attraversato da questa volonta, e questo lo rende massimanete infelice \t vita oscilla tra dolore e noia \\
Uomo è razionale, e tanto piu è consapevole tanto piu l'uomo è infelice e cresce questa insoddisfazione \\
Vita è come un pendolo \t il volere qualcosa è una mancanza, io desidero quello che mi manca \t se io voglio qualcosa, sto vivendo una condizione di bisogno, mancanza \\
Se mi trovo in una mancanza, non sono felice \t non sono compiuto \t cosi vado alla ricerca di ciò che mi manca \\
Se io raggiungo l'obbiettivo del mio desiderio, anche qua non saro felice \t potro anche essere appagato, ma subito cadro nella noia \t e cosi inizio a desiderare qualcosa d'altro \\
Cosi dalla noia passo di nuovo alla condizione di mancanza, che porta nuovo dolore \t infatti vita è un pendolo che oscilla \\
Questo perche la volonta è insaziabile, non ha uno scopo che una volta raggiunto la appaga \t è irrazionale \\
Questo accade anche perchè l'uomo non è un animale mansueto \t è nella sua natura essere insoddisfatto, l'uomo è feroce \t la civilta è solo una condizione di addomesticamento dell'uomo \\
Si sono generate delle condizioni per cui l'uomo vive civilmente \t ma basta un po di anarchia perche basti che l uomo si manifesti in tutta la sua violenza \\
Fa esempio che l'uomo è l'unico animale che fa soffrire un altro per gusto di farlo, ed è alla continua ricerca di prede, piu deboli, da sopraffare \\
L'uomo è condannato alla vita, che è dolore-noia-aggressività \t l'unica cosa reale, positiva, è il dolore \\
Ciò che invece è illusorio è la felicita \t nessun oggetto della volonta puo portare a un appagamento durevole, perche poi svanisce 

\v

Antipodo di Hegel \t non c'è fine (no progresso, no crescita), no razionalita nella realta \t filosofia di Hegel è ottimista: ha un finalismo, razionalita, ... \\
Per S. la storia è destino \t riprende sempre le stesse condizioni cambiando il contenuto \t dolore e noia si ripetono di continuo, ma con bisogni diversi \\
La vita vive una condizione di dolore nel momento in cui nasce \t l'uomo è condannato a vivere \\
Schopenhauer dice che si puo essere felici solo se si annulla la propria volonta \t e per lui è possibile \t ci sono due strade per allontanre dolore:

\ss[Arte - contemplazione estetica]
È una via momentanea \t permette di sollevarsi da questo dolore per un momento \\
Nel momento che contemplo una opera d'arte, mi immergo nell'oggetto che sto contemplando \t mi distacco dalla mia esistenza ed entro nell'opera \\
Vengo catturato da cio che c'è oltre l'oggetto fisico dell'opera \t la volonta viene sospesa perche intravedo qualcosa che va oltre, entro dentro cio che l opera d'arte mi vuole dire \\
Mi stacco dalla condizione fisica e divento occhio del mondo \t perdo la mia individualita, mi immergo perche anniento il mio desiderio \\
Uomo si stacca dalla volonta \t perche si coglie qualcosa che appartiene a tutti, qualcosa di universale \t ma è momentanea \\
Alla fine torno nella mia realta \t e cosi la vita sara ancora piu dolorosa \t l'uomo ha provato per un breve istante la vita senza volonta \\
Quando poi ripiomba in uno stato di volonta, questo ritorno sara ancora + doloroso della condizone di partenza \\
Volonta non è stata cancellata dalla vita \t perche poi si ritorna \t ci deve essere strada + definitiva, che è l'ascesi

\ss[Vita ascetica]
Scelta definitiva di isolamento \t scelta di vita con degli step \\
È un percorso interiore che si compie \t percorso porta ad una vita veramente libera dalla volonta, perche viene allontanato ogni desiderio \\
Ma per annullare la volonta, che è la mia essenza, devo totalemente me stesso e la volonta di vivere \t è l'unico modo per annullarla definitivamente \\
Devo vivere senza volonta, e quindi ci si deve allontanare dalla vita \t allontanarsi dal contesto in cui si vive

\sss[Giustizia]
Prima devo riconoscere che tutti gli uomini sono uguali \t e hanno uguali diritti e doveri \t equita \\
Giustizia pero parte da me \t supero il mio egoismo \t ma sto ancora cogliendo gli altri come altro da me \t non c'è ancora legame con gli altri

\sss[Bonta]
Giustizia permette di guardare gli altri in un nuovo modo \t e cosi riconosco la mia equita con gli altri anche nella condizione di dolore che è la vita \\
Riconosco questo dolore dato dalla volonta in tutti \t quindi giust. sfoacia in bonta = amore disinteressato per gli altri, che vivono il mio stesso destino \\
Bonta porta a guardare gli altri in modo benevolo \t mi muovo in una dimensione di amore \t è superiore a giustizia

\sss[Compassione]
Io sento il dolore degli altri \t comprendo il mio dolore, e sono in grado cosi di sentire anche il dolore degli altri \t che diventa il mio \\
Non piu solo riconoscimento razionale \t ma è irrazionale \\
Compassione è fondamento dell'etica \t non ci deve essere una legge morale, perche mi impedisce di stabilire un legame con gli altri \\
La tipologia di rapporti con gli altri si deve fondare sul fatto che io mi faccio carico del dolore degli altri \t non è relazione razionale, ma emotiva \\
Compassione sorvola sul fatto che molti uomini hanno intelletto limitato, che molti cuori sono malvagi, che la loro razionalita non si esprime come dovrebbe \\
Compassione non è un analisi razionale degli altri \t altrimenti non ci sarebbe nessun tipo di relazione con gli altri, ma l'etica non si puo fondare su questo \\
Compassione permette di cogliere affinita con gli altri, per la condizione dolorosa in cui tutti viviamo \\
Il compatire pero ha limite \t perchè è sempre un patire, un provare dolore \t io soffro ancora di piu perche mi carico anhe della sofferenza degli altri \\
Questo porta l'uomo ha provare orrore per la vita \t una vita la cui essenza sia soffrire 

\v

Cosi decido di non volere più soffire \t quindi scelgo l'ascesi, la via dei saggi indiani (e anche cristiani) \\
Faccio un altra scelta, di allontanarmi dal mondo \t per farlo:

\sss[Castita]
Libera e perfetta, perche rinuncio consapevolmente alla volonta fondamentale dell uomo \t ovvero di perpetuarsi in eterno

\sss[Poverta]
Deve pero essere volontaria \t bisogna diventare come san francesco \t bisogna rinunicare agli agi, e di non desiderare le cose del mondo, le cose terrene \\
Poverta volontaria e intenzionale

\sss[Rassegnazione]
Parola che verra molto criticata di Nietzsche \t decido che vivo senza nulla, mi rassegno a una vita priva di desiderio \t questa vita cosi mi rende libero \\
Entro nello stato di grazia cristiano \t totale disinteresse per le cose del mondo \\
Cosi facendo pero ci si allontana da tutto \t ma cosi si raggiunge una condizione di pace \t per farlo pero bisogna strapparsi dalla volonta della vita \\
Essere privi di volonta significa allontanarsi dall'essenza umana 

\v

Nietzsche all'inizio apprezza Schopenhauer, e condivide la sua visione del mondo \\
Non condivide pero il pessimismo della rassegnazione \t Schopenhauer è rassegnato, l'unica scelta è la fuga \\
L'arte e l'ascesi sono fuga \t la vita è dolore, quindi mi allontano \t e lo faccio perche sono rassegnato

\s[Kierkegaard]
Pensatore isolato \t è uno dei grandi contestatori di Hegel \\
S. pero si affianca a Kant \t mentre in K. non c'è niente di tutto questo \t non c'è riflessione speculativa \\
Inizia il percorso che fa: Kierkegaard -> Nietzsche -> Esistenzialismo \\
K. ha filosofia esistenziale, ma non esistenzialista \t esistenzialismo è proprio del 900 \\
K. riflette sull' esistenza \t ed esistenzialismo ha le sue radici in K., perche riprendono le sue categorie
\end{document}
