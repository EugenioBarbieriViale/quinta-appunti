\documentclass[12pt]{article}

\usepackage[a4paper, total={6in, 8in}]{geometry}
\usepackage{textcomp}

\begin{document}
\setlength{\parindent}{0pt}

\def \t {\textrightarrow}
\def \v {\vspace{1em}}
\def \bi {\begin{itemize}}
\def \ei {\end{itemize}}
\def \s[#1] {\section*{#1}}
\def \ss[#1] {\subsection*{#1}}
\def \sss[#1] {\subsubsection*{#1}}

\s[Sistema della Logica - Mill]
Rappresenta l'ala + empirista del positivismo \t movimento erede dell'empirismo inglese \\
Utilitarismo di Bentan \t il benessere di ciascuno garantisce il benessere di tutti \t prospettiva che punta sull'individuo, ma non individualista perchè ricade su tutti \\
Si punta però alla realizzazione di ognuno

\v

Mill è molto influenzato dalla sua famiglia \t e da giovane crede di aver trovato lo scopo della sua vita, ovvero di essere un riformatore del mondo \t ma poi dice di essersi risvegliato da questo sogno e nl 26 cade in depressione \\
La depressione lo porta a non avere + interessi e a essere apatico \t e in questo stato si fa la domanda: se io potessi realizzare tutti gli obbiettivi della mia vita e i cambiamenti politici, io sarei felice? \t lui si dà la rispota di no \\
Quello che pensava garantirgli la felicita in realta non contava \t va in crisi spirituale e ne esce con obbiettivi diversi, legati all'umanita in quanto tale \\
Nessuno puo essere felice da solo \t anche se io mi realizzo, non mi renderebbe felice \t perche gli altri non sono inclusi \t sarei felice solo se si realizza il progresso dell umanita \\
Bisogna quindi uscire dalla prospettiva individuale ed è necessario aspirare alla felicità di tutti \t strada facendo si raggiunge anche la propria \\
Metto prima la totalità, e strada facendo questo garantisce la mia felicità \t non rinuncio ad essere felice \\
Tutto cio che caratterizza la mia vita personale sono dei piaceri transitori, di cui l'individuo si puo beare, ma sempre orientando la vita a un ideale + alto \\
Mill lavorera quindi dentro la tradizione empirista, e cerchera di costruire delle teorie etico-politiche ed etiche che mirano a questo \\
La logica cosa centra? \t Mill è un empirista, e quindi vuole capire se è possibile per l'uomo ragionare prescindendo dall'esperienza \t l'individuo può allontanarsi dall'orizzonte empirico, e che valore ha un ragionamento senza fondamento empirico? \\
Letta cosi, la sua filosofia sembra metodologica \t ma non lo è: la ricerca non è come quella di Cartesio, che cerca metodo fine a se stesso \t lui riflette sull'esperienza e il suo valore, e si chiede come l'uomo si rapporta con questa esperienza \\
Uomo si confronta con la realta con l'esperienza o con l'intelleto (quindi con la logica) \t sua logica non vuole ricercare un metodo conoscitivo e capire come funziona inteletto \t ma vuole mostrare come conoscenza sia imprescind. legata all'oggetto empirico \\
Per questo rimane empirista e non razionalista \t e questa indagine permette all'uomo di essere felice ??

\v

Scrive la "Logica" \t e qui si chiede quale sia il valore del sillogismo \t vuole capire se rappresenta una struttura del pensiero valida oppure aritificiosa \\
"Tutti gli uomini sono mortali \t socrate è un uomo \t socrate è mortale" \t noi ricaviamo che sia mortale basandoci sull'uomo che è mortale \t ma come lo sappiamo? \\
Tutti gli uomini sono mortali \t la basiamo sulla nostra esperienza e generalizzandola \t non passiamo dall'universale al particolare \\
Si passa da osservazione di casi particolari (gli uomini che ho visto io) alla generalizzazione \\
Socrate è mortale lo basiamo anche sull'esperienza \t lo possiamo dire solo quando abbiamo visto Socrate morire \t quindi sillogismo non è una deduzione del particolare nell'universale \t ma si usa sempre esperienza \\
La proposizione universale, non è altro che un modo utile per ricordare tanti casi particolari \t noi abbiamo solo conoscenze empiriche \t è solo un espediente logico per far memoria di eventi singoli che ho visto nel passato \\
Sillogismo in realta non da nessuna nuova informazione \t se la proposizione iniziale è vera, allora il fatto che Socrate sia mortale è tautologico \t se tutti sono mortali, non ho bisogno di dire che scorate è mortale \t lo so già \\
O sillogismo continua a ripetere le stesse cose, oppure puo diventare assurdo ed arrivare a conclusione non verificate dall'esperienza \\
Noi eravamo certi che l'intelletto funzionava come aveva detto aristotele, e che poteva funzionare da solo \t ed usare procedimenti logici che presciendono dall'esperienza \t per lui no \\
Sillogismo non puo essere fondato su se stesso \t ogni conoscenza si basa sull'esperienza 

\v

L'induzione è nient'altro che una generalizzazione dell'esperienza \t è cio che permette di passare da alcuni casi ad altri casi simili, che crediamo che si verifichino perche sono simili \\
Perche non siamo autorizzati a compiere questa operazione di generalizzazione dell'esperienza? \t perche natura ha un corso uniforme: le cose sono sempre andate in quel modo \\
Il corso della natura è uniforme: questa cosa si è verificata per millenni, quindi siamo autorizzati a generalizzare la nostra esperienza \\
L'induzione si fonda quindi sul principio di uniformita della natura \t ma esso su cosa si fonda se non sull'induzione? \t io come faccio a dire che quello che si è verificato y volte, si verifichera sempre? \\
L'induzione serve per formulare quindi delle induzioni \t l'induzione è garantito dal principio di unif., che pero si basa su delle induzioni \t circolo vizioso \\
Mill dice che tutto è generale e non universale, poi introduce criterio che generalizza esperienza ma si basa esso stesso sull'esperienza \\
Quindi da risposta: non è cosi, circolo viz. non esiste \t esperienza garantisce se stessa \t obbiezione avrebbe senso nell'ottica del sillogismo (a è b, b è c, a è c) \\
Ma bisogna uscire da questa logica di proposizioni che garantiscono altre prop. \t princ. e ind. sono entrambi garantiti dall'esperienza, non si garantiscono a vicenda \\
Sta quindi smontando il pensiero \\
Lui sta mettendo in discussione il valore della consocenza universale \t non si puo avere conoscenze universali da cui si deducono quelle paritcolari \t il valore della conoscenza con dei fondamenti certi e universali, non esiste \\
Anche Locke \t la mia conoscenza non va oltre l'esperienza \t non esiste induzione e neanche deduzione \t non si può dedurre il particolare dal generale \\
Esiste solo il criterio dell'esperienza, che garantisce se stessa \\
Tutto questo discorso perchè vuole mostrare che l'unica cosa che gli uomini possono conoscere è l'oggetto d'esperienza

\ss[Logica delle scienze morali]
È vero però che la natura procede per casi simili \t l'evoluzione della natura non è mai data da certezze \\
È positista perchè è empirista \\
Sempre qua parla della liberta della volonta dell'uomo \t se conoscessimo tutte le ragioni degli atteggiamenti di una persona, si possono prevedere tutte le sue azioni come se fosse una legge fisica \\
Questo non significa che questa persona sia indotta dalla necessita \t ma perche valuto sempre la propbabilita della natura, e come mi baso sulla particolarita che diventa generale nel caso fisico, posso farlo nel campo morale

\ss[Principi di economia politica - 1848]
Sempre stesso anno pubblicato il manifesto comunsita \t ma è opposto a Marx \\
Ripresenta qua tutte le riflessione degli economisti classici (Smith, Maltus, ...) \\
Difende la teoria dell'indipendenza = ciascuno deve prendere provvedimenti che lo riguardano \\
Il benessere del popolo deve risultare dalla giustizia e dall'autogovenro \t i lavoratori stessi devono prende i provvedimenti utili alla lo posizione \t ma come? \t non con rivoluzione (no marx), ma con mezzi pacifici (associazionismo, collaborazione) \\
Con rivoluzione no giustiza \t perche con riv. si mette qualcuno in condizione di non esercitare la propria liberta \t valore fondamentale da preservare \\
Dittatura del proletariato nega la liberta di tutti \t bisogna tenere insieme cio che è giusto (diritti uguali, etc) con la liberta \\
Idee simili ma non aderisce al socialismo \t che mette in discussione la liberta individuale \\
Il mezzo del cambiamento è l'associazionismo e la collaborazione etc. \\
È dalla parte del popolo, ma cambiamento deve essere di questa forma

\ss[Considerazioni sul governo rappresentativo - 1861]
Il problema della democrazia rappresentativa è di impedire alla classe che ha la maggioranza di costringere le altre classi di vivere ai margini della politica e di agire secondo il suo interesse \t va impedito \\
Bisogna garantire governo democratico ed impedire qualsiasi abuso \t quando io vengo eletto e ho la maggioranza, smetto di governare i miei elettori ma governo lo stato \\
Quando divento il governo, governo per quel paese \t bisogna tenere il valore della democrazia e gli interessi di tutte le classi sociali \\
Una democrazia deve avere:
\bi
    \item uguaglianza
    \item imparzialita
    \item governare per tutti
\ei

\end{document}
