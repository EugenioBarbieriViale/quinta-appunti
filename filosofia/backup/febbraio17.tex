\documentclass[12pt]{article}

\usepackage[a4paper, total={6in, 8in}]{geometry}
\usepackage{textcomp}

\begin{document}
\setlength{\parindent}{0pt}

\def \t {\textrightarrow}
\def \v {\vspace{1em}}
\def \bi {\begin{itemize}}
\def \ei {\end{itemize}}
\def \s[#1] {\section*{#1}}
\def \ss[#1] {\subsection*{#1}}
\def \sss[#1] {\subsubsection*{#1}}

\s[Esistenzialismo]
È la corrente di pensiero che ha le sue radici remote in Kierkegaard \\
Ne H. ne K. sono esistenzialisti (H a meta strada) \\
Si afferma in europa dopo WWI \t ed è pensiero dominante tra le due guerre, e diventa una moda culturale nel secondo dopo guerra \\
Questi decenni sono infatti di profonda crisi, delle guerre mondiali e delle crisi economiche \t momenti di grande destabilizzazione \\
L'esistenzialismo è la filosofia della crisi, di un europa dilaniata \t fisicamente e moralmente, da due guerre mondiali \\
La crisi dal punto di vista filosofico è la crisi dell'ottimismo ottocentesco \t l'ottimismo romantico, c'era l idea che si poteva andare solo verso il meglio \\
Entra in crisi l'idea che la storia sia razionale \t che nella storia, nel modo idealista o positivista, ci sia una razionalita intrinseca \t ma in questi decenni si vede che l'irrazionale entra nella storia \\
La storia non va secondo ragione, ma puo anche deviare in vie prive di senso, come l olocausto \t storia non si fonda sulla ragione e non segue percorso fondato sulla razionalita \\
Uomo esistenzialista ha perso il senso \t della storia, e il senso dell'assoluto \t capisce di essere allo sbando, ha perso il senso di marcia, di direzione, oltre che dell'esistenza \\
Le due guerre, i totalitarismi segnano crisi profonda \\
L'esistenzialismo è la filosofia dell'uomo finito, di un uomo gettato nel mondo \t che non ha scelto chi essere e dove essere \t di un uomo costantemente lacerato, che vive dimensioni assurde o altamente problematiche \\
Esistenzialismo esprime questa condizione di crisi \t quindi esistenzialismo mette al centro l uomo, ed è l uomo che ha preso tutti i riferimenti 

\v

L'esistenzialismo non si interessa dell uomo come categoria pero \t l uomo di cui parla l esistenzialismo, è l esistenza particolare, la singolarita dell uomo, come individuo \\
L uomo singolo in quanto tale, che vive una dimensione di esistenza (categoria unicamente umana come in H.) in una realta che non si identifica con la razionalita, e non è riconducibile ad alcuna razionalita \\
Caratteri fondamentali:
\bi
    \item la razionalita non si idenifica con la realta
    \item centralita dell esistenza come carattere tipico umano \t è un modo di essere che è solo dell'uomo, perche esistenza è apertura alla possibilita, mentre tutti gli altri enti vivono necessariamente
    \item trascendenza dell'essere \t non c'è identita tra ente ed essere: l'uomo che è ente, è finito perche possiede l'essere \t ma non è essere \t c'è qualcosa che va al di la dell'esistenza \t tutto cio che non è finito va oltre, non c'è alcuna coincidenza ontologica \t l'essere trascende
    \item mette al centro la categoria della possibilita, che diventa il modo costitutivo dell'esistenza \t è categoria fondamentale \t radice remota è Kierkegaard, mentre la radice prossima è la fenomenologia (perche introduce la coscienza intenzionale, che è sempre apertura ad altro) e non H.
\ei
L'uomo si progetta oltre se stesso \t l'esistenza è una trascendenza continua \t ma non guardo orizzonte metafisico, ma nel momento che esisto guardo sempre avanti e trascendo la mia condizione presente \\
Ma trascendo me stesso verso cosa? \t esist. si declina in modo diverso rispondendo a questa domanda \t assume tante forme \\
Mi progetto verso dove? \t alcuni dicono verso dio, altri verso il mondo, altri verso me stesso, altri verso la liberta o il nulla (sartre) \\
Differenziazione sta in questo

\v

Esistenzialismo è stato letto anche come una crisi dell'hegelismo \t che è stato ridiscusso da Marx, Feuerbach, Nietzsche, etc. \\
All'interno di questo si colloca anche l'esistenzialimo \t riaffermazione del finito contro la visione hegeliana \\
Esist. ha delle ripercussioni ha tutti i livelli \t c'è letteratura esistenziale (Camus, letteratura russa) \\
La francia è dominante con Sartre, Camus, e altri \t e qui c'è anche una rinascita esistenzialistica di Hegel \t infatti Hegel affronta anche l'esistenza (dialettica servo padrone, morte, finitudine umana) \\
Cougev per esempio ripropone la filosofia di hegel alla luce di queste catergorie esist. \t lui identifica l'assoluto di Hegel come uomo-nel-mondo

\s[Sartre]
Figura centrale dell'esistenzialismo per quello che dice, e lui con la compagna Simone de Bouvar diventano simbolo di cambiamento \t diventano riferimento del pensiero del 900 \\
Diventano di moda \t loro vivono amore libero, non si sposano mai ma vivono sullo stesso pianerottolo \t hanno anche diverse orientazioni, ed avevano anche presitgio accademico \\
Bouvar scrive "il secondo sesso", dove la categoria della donnna non è una categoria di natura \t pero la societa ha costuito intorno alla femmina il ruolo di donna con delle categorie che pero non le appartengono alla natura \t che sono quindi cultura e non natura

\ss[Vita]
Nasce a Parigi nel 1905 e studia alla scuola normale superiore \t ed insegna filosfia nei licei \\
Tra il 33 e 34 va a berlino \t qua studia la fenomenologia \\
Poi scoppia la guerra, viene fatto prigioniero in germania e viene deportato \t e quando torna fonda un gruppo di resistenza intellettuale, chiamato "Socialismo e liberta" \t è un gruppo che si oppone al totalitarismo \\
Sartre è marxista, anche se poi lo critichera \\
Scrive anche delle opere teatrali \t e nel secondo dopo guerra suo pensiero si impone grazie a queste opere teatrali \\
A partire dagli anni 60, lui scrive in modo frenetico romanzi, saggi, fa viaggi e conferenze \\
Fa viaggi a contenuto politico \t conosce Fidel castro, Kruschov a mosca \t in un momento della guerra fredda \\
Muore a parigi nel 1980

\v

Buovar dice di essere stata colpita dalla sua passione tranquilla e forsennata \t lui guardava la sua vita non in modo rivoluzionario, ma era anche animato dalla volonta di cambiare le cose \\
Lui detestava la routine \\
Si sentiva in realta uno scrittore \t ha scritto diversi romanzi, come la "Nausea", e opere teatrali come "Mani sporche", "A porte chiuse" \\
Poi scrive dei pamphlet politici \t come "L'antisemitismo" e "I comuisti e la pace" \\
Opera filosofica + importante è "L'essere e il nulla" è del 43 (durante guerra), ma ne scrive molte \\
Nel 46 scrive "L'esistenzialismo è un umanismo" e nel 60 scrive la "Critica della ragione dialettica"

\ss[La nausea]
È la prima opera, il primo romanzo \t è la sua prima opera famosa \\
È un romanzo che si occupa di psicologia fenomenologica \t indaga le emozioni, la coscienza, etc. \\
Analizza l uomo partendo da Usserl \t ma poi dice che usserl era caduto in una prospettiva idealista \t lui tende a fare della coscienza un ente diverso da tutti gli altri enti, la ha allontanata dal mondo \\
La coscienza che è intenzionale è un ente come gli altri, sta nel mondo \\
Dice che usserl è caduto in una prospettiva solipsista \t mente lui ha perso l'aspettoe sistenziale della coscienza (ma non era veramente l oggetto della sua riflessione) \\
Uomo sta nel mondo se la coscienza è apertura al mondo \t è un ente tra altri enti \\
Nella nausea sviluppa quindi lo sguardo che la coscienza ha nel mondo \t e nella nausea parla dell'esperienza di Antoine \t che riflette sulle ragioni della sua esistenza \\
Lui si ritrova su una panchina in un parco e si guarda intorno, e qui fa l esperienza della nauesa \t comprende che tutto cio che lo circonda è contingente, anzi è superfluo \t è di troppo \t è rindondante, anche se non ci fosse tutto questo albero, questo parchetto, non cambierebbe nulla \\
Questa riflessione parte dall esterno, ma poi ricade su di se \t capisce anche lui di essere superfluo, e questo sentimento della contingenza che provoca nausea \\
Proviamo nausea quando realizziamo il non sesno del reale \t è assolutamente contingente, potrebbe benissimo non esistere \\
Non c'è una vera ragione per cui la realta è cosi, non c'è quindi un senso \t potrebbe essere completamente diversa e rimanere lo stesso indifferente \\
Con la nausea proietto il senso della contingenza su di me \t non ho in me la condizione della mia esistenza \t non ha una ragione, quindi si cade nell assurdo e non c'è un senso \\
Questo sentimento di nauesea mozza il fiato \t antoine è in una estasi orribile \\
Quindi se tutta la realta è contingente, l'essenziale è la contingenza \t non c'è necessita nell esistenza dell uomo, e quindi io sto li perche mi trovo li \\
Non sono il risultato di un ragionamento \t l'esistenza degli enti non si puo dedurre \t realta non è risultato di un discorso razionale, ma è cosi e io mi trovo li \\
Appaio nel mondo, e anche se non apparissi non cambierebbe nulla \\
La contingenza non è un carattere della realta, ma è la realta stessa \t è l'assoluto \\
Tutto è contingente, e questa consapevolezza rivolta lo stomaco \t questo significa scoprire che la vita non ha un senso, non ha uno scopo da realizzare \t non c'è un fine che orienta la mia vita \\
Io esisto come una cosa \t tra me e gli enti non c'è alcuna differenza

\ss[L'essere e il nulla]
Questa cambia relazione con il mondo totalmente \t nell'Essere e il nulla approfondisce \\
Se io non ho un senso, non ho neanche un progetto \t sto nel mondo perche sono gettato \t e del mondo non so cosa farmene, quindi il mondo non ha senso \\
È tutta una caduta di senso a partire da me \\
Qua fa una distinzione tra per se e in se \t il per se è la coscienza, l'in se è il mondo \\
La coscienza è sempre coscienza di qualche cosa \t ma è coscienza di qualcosa che è altro \t è massima apertura a un oggetto di coscienza che è diverso dalla coscienza \\
QUini la coscienza è un nulla di essere \t non è piena di qualcosa, è un per se \t sta da sola, non viene definita in partenza \t non è qualcosa che sa cosa deve essere in forma necessaria, è un contenitore vuoto che si apre al mondo \\
Se l'esserci è progetto, si coglie il positivo \t qua prospettiva + negativa rispetto a H. \\
Qua invece la coscienza è un nulla di essere \\
La coscienza si apre su un mondo in se, ovvero pieno \t il mondo è impastato di se stesso, di essere, di cio che deve essere \t il mondo è opaco, non è aperto ed è gia tutto pieno \t mentre cosicenza si apre su questo mondo pieno di essere \\
La coscienza si apre su questo monod in una totale incondizionata liberta \t il nulla di essere non ha direzione, e di questo mondo posso farne quello che voglio \\
La coscienza è libertà \t lui accentua sul discorso della liberta, perche vuole arrivare al discorso della responsabilita \t infatti è anche vero che coscienza è possibilita davanti al mondo
\end{document}
