\documentclass[12pt]{article}

\usepackage[a4paper, total={6in, 8in}]{geometry}
\usepackage{textcomp}

\begin{document}
\setlength{\parindent}{0pt}

\def \t {\textrightarrow}
\def \v {\vspace{1em}}
\def \bi {\begin{itemize}}
\def \ei {\end{itemize}}
\def \s[#1] {\section*{#1}}
\def \ss[#1] {\subsection*{#1}}
\def \sss[#1] {\subsubsection*{#1}}

\s[Marx - Capitale]
Capitale è il suo capolavoro \t tre volumi (primo nel 67), gli altri due pubblicati da Hengels dopo la sua morte (lui muore nell 83) \\
Capitale spiega come è nata la proprieta privata \t analizza la nascita del capitale e la societa capistalista \\
Si apre con l'analisi della merce \t che ha duplice valore
\bi
    \item valore d'uso \t corrisponde al bisogno che la merce va a soddisfare (es. sapone per lavare le mani) \t ma merci diverse venogno scambiate in diverse quantita nel mercato
    \item valore di scambio \t è cio che permette di scambiare merci diverse in date proporzioni (es. 100 patate per un tavolo)
\ei
Valore di scambio corrisponde alla quantita di lavoro socialmente necessaria per produrre una merce \\
Questo valore è dato dal lavoro che c'è dietro \t differenza tra artigianato e produzione industriale, in cui valore diminuisce \\
Valore di scambio permette di scambiare merci diverse in diverse proporzioni, perche hanno qualcosa di identico: la quantita di lavoro per produrla \\
In un contesto di baratto questo funziona 

\v

Poi pero compare una merce diversa sul mercato \t la forza lavoro \\
Nella societa capitalista l'operaio vende al capitalista non un prodotto ma la sua forza lavoro \t riceve un salario in cambio del suo lavoro \\
Il capitalista paga giustamente questa merce \t la paga secondo il suo valore di scambio \t che è la quantita di lavoro necessaria per mantenermi in forza, in essere \\
Con lo stipendio mi sfamo e mi metto in condizione di continuare a portare la forza lavoro \\
Il vero problema di questa nuova merce è che non è come le altre \t forza lavoro ha la capacita di produrre valore \t quella forza lavoro comprata produce del lavoro per l'imprenditore, lavora \\
Viene pagato il valore di scambio, ma non viene pagato il valore che si viene a creare attraverso la forza lavoro \\
Non è uno scambio equo quindi \t perche non viene comprata una merce con un valore fisso, il valore prodotto/creato non viene considerato nello stipendio \\
Stipendio serve solo per mantenere in essere la forza lavoro \t ma il valore prodotto non viene considerato dal capitalista, la forza lavoro viene considerata come un altra merce \\
Lo stipendio è giusto per il valore di scambio della forza lavoro, ma poi non viene considerato il valore prodotto \t non viene coperto \\
Il valore prodotto non viene rispettato \t il capitalista ne usufruisce 

\v

Questo valore non coperto dallo stipendio è il "plusvalore" \t valore che non viene pagato quindi, ed è legato a questa nuova merce (la forza lavoro) \\
L'accumulo di tanti plusvalori porta alla creazione della proprieta privata e di un capitale \t il proletario continua a prendere 1000 euro, ma ne produce 10000 \t il capitalista tiene il plusvalore \\
C'è obiezione pero \t il capitalista reinveste \t tutti questi soldi che tiene per se in realta li usa anche per mantenere l'azienda e per migliorare le condizioni di lavoro dell'operaio \\
Marx dice che in realta è sempre un danno per l operaio \t il capitalista fa investimenti per guadagnare di piu, ma soprattuto ammodernando l'industria la forza lavoro umana serversa sempre meno \\
A un certo punto l'operaio non servira piu \t non solo non verra riconosciuto il plusvalore, ma non avra neanche il lavoro \\
Quindi proprieta privata è un furto \t il plusvalore non viene infatti pagato \t quindi va abolita, e non ridistribuita \\
Come si fa inoltre a quantificare la ridistribuzione? anche volendo, è utopico pensare di ridistribuire la proprieta privata in parti uguali \t perche dovrei dare un valore a ogni prodotto di ogni azienda, e quantificare il plusvalore che crea \\
Per esempio la mole della Ferrari ridistribuita ai suoi operai e quella di un azienda di pasta ai suoi è comunque diversa \\
Inoltre il capitalista si trova in un regime di concorrenza rispetto agli altri capitalisti \\
Tutto questo è nel primo volume del capitale

\v

Dopo Marx, pone fine alla sinistra hegeliana \t è contestatore di Hegel, ma ha le radici nella sua filosofia \\
Ci sono invece dei pensatori in completa opposizione a hegel, che rifiutano completamente \t come Schopenauer e Kierkegaard \\
Hanno tratto comune \t nessuno dei due entra nel merito del sistema di Hegel \t che viene ritenuto come una cialtroneria \t il suo sistema viene definito "ridicolo" da Kierkegaard perchè sbagliato in toto \\
Sbagliato perche ha voluto guardare il finito con gli occhi dell'infinito, essendo pero un uomo \t e ha trascurato la vera dimensione di realta, ovvero quella del finito \\
Hegel è partito dall idea che tutta la realta è spirito e si muove dialetticamente \t e poi ha costruito una teoria in cui ha costretto la realta a starci \\
La realta è fatta di singolarita, di uomini singoli \t in K. la vera filosofia parte da una categoria specificia, che è quella del Singolo \t questo uomo particolare \\
Singolo è la categoria fondamentale della realta \t la storia passa per il Singolo \\
Entrambi non appartangono a nessuna scuola di pensiero \\
C'è pero differenza \t Schopenuer è filosofo kantiano \t rientra nella ripresa del criticismo  \\
K. invece non riprende nulla \t è una filosofia esistenziale, c'è il problema dell'esistenza, come l'uomo vive \t non ha base ne teoretica ne gneseologica

\s[Schopenhauer]
Per lui il sistema di Hegel è una grande buffonata \t e inoltre ha conseguenza politiche pericolose \\
Hegel giustifica lo stato assoluto \t inoltre lo accusa di farlo per soldi \\

\ss[Vita]
Scrive nel 1819 "Il mondo come volonta e rappresentazione" \\
Studia Kant e Platone \\
Nasce nel 1779 \t da un ricco commerciante \t poi padre si suicida e S. abbandona commercio per studiare \\
Nel 12 va a Berlino e qui ascolta Fichte \t S. è disgustato da Fichte \t nel 13 si lauerea a Jena, mentre a Weimar frequenta il salotto della madre, che scriveva romanzi \\
In questo salotto conosce Goethe, e anche Meier (un orientalista) che lo influenza molto sul pensiero indiano \\
Ma la madre era mondana \t quando lei introduce un fidanzato a S., si trasferisce a Dresta dove completa la sua opera \\
Quest'opera ha una fortuna scarssima \t la sua prima edizione va al macero \\
Nel 20 va di nuovo a Berlino \t vuole intraprendere carriera accademica \t fa lezioni di prova, ma durante la discussione per farlo rimanere si scontra con Hegel, che era il pensiero dominante \\
Questo gli blocca ogni possibilita di carriera accademica \t fara anche lezioni in concorrenza con Hegel, ma non avra studenti \\
Nel 31 abbandona Berlino perche arriva la peste \t va a Francoforte, dove muore nel 60 \\
Lui pubblica nel frattempo diverse opere, minori, come "Aforismi per una vita saggia" \\
Fa vita infelice \t ma influenza molto Wittgenstein per esempio, e influenza anche i romanzi (Thomas Mann, ...)

\ss[Il mondo come volonta e rappresentazione]
Il suo capolavoro \t in cui divide il mondo in volonta e rappresentazione \t in termini kantiani sono il fenomeno e il noumeno

\sss[Rappresentazione]
Aspetto fenomenico: il mondo è una mia rappresentazione \t l'oggetto esiste solo in relazione al soggetto \t il mondo non ha valore di oggettivita dal punto di vista conoscitivo \\
Il mondo esiste quando io lo incontro conoscitivamente \t io non posso vedere il mondo con degli occhi diversi, tutto quello che vedo posso solo con i miei occhi \\
Il soggetto sono io che conosco, senza essere conosciuto da alcun oggetto \t mentre l'oggetto è cio che viene conosciuto \\
Questa conoscenza è determinata a priori da spazio e tempo, ed esiste una sola categoria: quella della causalita \t la realta nella sua struttura essenziale è definita da un rapporto di causa/effetto \\
Spazio e tempo \t condizioni di conoscibilita \t siamo nella rivoluzione copernicana \\
Quindi la realta esiste solo per me che la conosco \t non vuol dire che io la faccio esistere, ma vuol dire che la posso conoscere solo con un soggetto \\
La rappresentazione è il risultato dell'incotro tra soggetto e oggetto \t rappresentazione non è una verita, è come io guardo il mondo

\v

Sia il materialismo sia l'idalismo sbagliano \t materialismo nega il soggetto (tutto è materia), mentre idealismo nega l'oggetto (sogg e ogg coincidono) \\
Tuttavia, tra le due è meglio l'idealismo (depurato di tutte le sciocchezze) \t perche io dico che la realta è una mia rappresentazione, perche c e il soggetto \t privilega il soggetto sull oggetto \\
Dire che il mondo è una mia rappresentazione significa che è il fenomencio \t questo presuppone un noumeno, che in Schopnauer è conoscibile \\
È conoscibile perche io devo squarciare il "velo di maia" (filo indiana) che vela l'aspetto noumenico della realta, per conoscere l'essenza della realta \\
Velo di maia = l'apparenza fenomenica della realta \\
Per squarciarlo devo ritrovarmi nella mia interiorita (filo. indiana) \t cosi facendo mi scopro come volonta, che è la radice del mio essere \\
Volonta di vivere, conoscenza, desiderio \t è cio che caratterizza l'uomo \t è quindi la mia essenza, ma puo essere anche l'essenza della realta? \\
Si \t l'universalita dei fenomeni hanno stessa essenza \t la natura vuole preservare se stessa, gli animali vogliono preservare la specie, forza di gravita, etc... \\
Questa brama di essere, la volonta, è dappertutto \t ed è essenza della realta \t l uomo ne contiene una parzialita \\
La volonta umana è una traccia consapevole della volonta universale di tutta la realta
\end{document}
