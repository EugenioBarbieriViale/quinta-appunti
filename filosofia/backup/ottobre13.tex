\documentclass[12pt]{article}

\usepackage[a4paper, total={6in, 8in}]{geometry}
\usepackage{textcomp}

\begin{document}
\setlength{\parindent}{0pt}

\def \t {\textrightarrow}
\def \v {\vspace{1em}}
\def \bi {\begin{itemize}}
\def \ei {\end{itemize}}
\def \s[#1] {\section*{#1}}
\def \ss[#1] {\subsection*{#1}}
\def \sss[#1] {\subsubsection*{#1}}

\s[Logica della scienza]
Idea in se, fuori di se, in se per se ?? \\
Sistema della scienza \\
Lo spirito si fa vedere all inizio come pensiero \t si pone nella logica come oggetto della logica \t poi mostra l altro suo aspetto dell essere, nella ? della natura, per poi mostrare di essere entrambi \\
Ma quindi l assoluto dove sta \t hegel parla di dio prima della creazione del mondo \\
La logica è il momento in cui lo spirito si autopone in essere \t è il momento della tesi, a cui manca il percorso la filosofia della natura e poi dello spirito ? \\
Logica descrice dio prima della natura, prima della creazione dell essere \\
Obbiezioine \t dove c e l amssima completezza di dio \t quando esce di se e va nella natura e solo alla fine si puo parla di assoluto \t o gia all inizio è assoluto \\
Se dio all inizio è manchevole \t cade il sistema \\
Ma in realta lo spirito si pone in atto in momenti diversi \t mostrando facce diverse \t nella filosofia dello spirito ricompone e si mostra come spirito \\
Si mostra nei diversi aspetti, ma è gia assoluto ?? \\
Nella logica l'idea si pone in atto 

\v

La logica si apre con una considerazione ovvia \t la logica è anche ontologia (pensiero ed essere coincidono) \\
Vuol dire che la logica di hegel parla di concetti (princi, sillogismi, ..) \t ma tutto questo ha un immediata ricaduta sul piano ontologico \\
Il concetto è il nome piu perfetto che posso dare all assouluto \t e quindi è unico \t il concetto esprime l'essenza, ma se ttuto cio che è è spirito allora c e un solo concetto che è spirito \\
Concetto è un modo per definire lo spirito \t è il nome dello spirito quando lo guardo dal punto di vista logico \\
La logica si divide in una triade
\bi
    \item logica dell essere
    \item logica dell essenza
    \item logica del concetto
\ei

\v

All inizio il pensiero è incosapevole \t e lo spirito si mostra come pensiero \t cerca di capire quale sia la sua natura \\
Nella logica dell essere lo spirito cerca di autoconoscersi muovendosi su un piano orizzontale \\
Nella logica dell essenza lo spirito approfonidsce \t movimento orrizontale non è sufficiente \t approfondire = riflette su se stesso (prima si "era guardato in torno") \t procede verticalmente \\
Nella logica del concetto scopre in radice di essere unita di pensiero ed essere \t il pensiero raggiunge la sua completezza\\

\ss[Logica dell'essere]
Parte dalla triade qualita, quantita e misura \t svolge riflessioni sul rapporto tra finito e infinito, che finito non ha natura propria ma appartiene all infinito \t finito è non reale \\
Ripete le cose dette nella fenomelogia \\

\ss[Logica dell'essenza]
Spirito vuole capire le radici dell'essere \t prima l aveva solo guardato \\
Ma in realta l essere è sempre lui \t si ripiega su di se \\
Hegel si sofferma sui principi fondamentali della logica 

\v

Identita viene criticato (cosi come formulato da aristotele) non va bene \t presuppone una realta statica \t ma in dialettica è impossibile \\
Non ci puo essere identita che annulla la differenza \t io posso dire che A = A' \t che la tesi sara ugugale alla sintesi \t in un movimento di divenire torna a se stesso ma con qualche differenza \\
Principio di identita va rivisto \t non puo essere fermo

\v

Principio di non contr. viene rifiutato \t ma per hegel, se tolgo la contraddizione tolgo l'antitesti, che è la forza della dialettica \\
L unico elemento non contraddittorio è l infinito nella sua totalita \t ma al suo interno la contraddizione esiste ed è necessaria \t se no dialetica 

\v

Poi difende la prova ontologica di sant ansemlo \t è l unica prova valida di dio \t essere e pensiero coincidono \\
La critica di kant non regge \t lui parte dall idea che pensiero e realta siano diversi \t ma per hegel è sbagliato \\
Tutto cio che è razionale è reale, cio che è reale è razionale 

\ss[Logica del concetto]
Spirito ha cpaito che piano logico e ontologico coincidono \t ma non ha capito perche \t motivo è che c'è unico soggetto della realta, che è lo spirito \\
La realta è fatto da un unico principio, che è la perfetta coincidenza di pensiero ed essere \\
Concetto diventa il nome piu perfetto con cui posso definire l assoluto \t perche il concetto è l espressione dell essenza, che è unica \\
Usare altri concetti (penna, cane, ..) è la stessa cosa \t tutti questi concetti universali sono solo dei modi utili per definire la realta, ma non esprimono l essenza del reale (che è lo spirito) \\
Dal punto di vista teoretico non hanno un fondamento \t frase reale razionale la posso girare \\
Ma questo cambia anche il valore del sillogismo, del giudizio \t cambia le strutture logiche che si basano sui concetti 

\v

Siamo arrivati al concetto, spirito si autoriconosce come unico concetto della realta \t logica del concetto mette un punto al percorso che lo spirito ha fatto \\
Prima forma di riconoscimento di se \t manca pero la faccia della natura \t finora solo aspetto logico

\s[Filosofia della natura]
Spirito quindi deve uscire fuori di se, mostrare l altra faccia dell essere, e questa faccia la mostra nella natura \\
Ma nella scienza della logica ha gia detto tutto in realta \t la filosofia della natura è la parte + debole della sua riflessione \t non era necessaria \\
Doveva essere solo rispettata la dialettica \t l antitesi è la filosofia della natura \\
Qui Hegel si rifa ad una analisi scientifica della realta che è vecchia \\
È influenzato qui dalla filosofia neoplatonica \t parla della natura come uscita dal principio \\
È influenzato dal dogma teologico della creazione \\
Ma cosa + interessante sono 2 osservazioni:
\bi
    \item la natura non si muove \t scala di perfezione uomo al vertice, ma non si evolvono le forme della natura, ma lo spirito che si pone in essere in tante forme, che coesistono \t no passaggio dall una all altra \\
    \item la totale avversione nei confronti di Newton \t hegel dice che le sue idee sono barbare \t avversione totale
\ei
Si divide in:
\bi
    \item meccanica \t studia la corporeita universale
    \item la fisica \t scioglie la meccanica, rompe la sua rigidita nei processi elettrici e magnetici
    \item l'organica \t dove si interiorizza tutto questo e nasce la vita
\ei

\s[Filosofia dello spirito]
La parte + famosa della filosofia del hegelismo \t parte in cui lo spirito torna in se, parte conclusiva \t arriva ad avere massima consapevolezza \\
È dove fa riflessioni sullo stato e sulla storia \t che vennero anche strumentalizzate \\
Sue conclusioni giustificano uno stato forte \\
Lo spirito quindi torna in se \t si è mostrato come pensiero, è uscito nella natura e poi recupra nel momento di sintesi \\
È il momento in cui l'idea torna in se dalla sua alterita \t questa è la piu alta manifestazione dell assoluto \\
È definibile anche come l'autorealizzazione di dio \t questo spirito è il corrispettivo di dio nel cristianesimo \t ma filosofia deve chiarirlo dal punto di vista del concetto \t deve fare riflessione teoretica, non teologica \\
Questa dimesnione assoluta è analoga al dio del cristianesimo \t ma il suo compito è di rendere chiaro questo assoluto in modo razionale, non teologico \\
Ha tre tappe: lo spirito soggettivo, lo spirito oggettivo, lo spirito assoluto
\end{document}
