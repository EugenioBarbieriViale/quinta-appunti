\documentclass[12pt]{article}

\usepackage[a4paper, total={6in, 8in}]{geometry}
\usepackage{textcomp}

\begin{document}
\setlength{\parindent}{0pt}

\def \t {\textrightarrow}
\def \v {\vspace{1em}}
\def \bi {\begin{itemize}}
\def \ei {\end{itemize}}
\def \s[#1] {\section*{#1}}
\def \ss[#1] {\subsection*{#1}}
\def \sss[#1] {\subsubsection*{#1}}

\s[Nietzsche]
\ss[Considerazioni inattuali]
L'opera viene criticata \t quindi scrive le "Considerazioni inattuali" \t fa riferimento alla storia e l'atteggiamento che uomini devono avere \t quello giusto è quello critico \\
Bisogna analizzare criticamente il passato e di tenere solo ciò che serve nel presente \\
La saturazione della storia porta l'uomo a sentirsi il punto di arrivo del passato \t questo è sbagliato perchè è rassegnato e ci impedisce di affrontare il futuro \\
Nel frattempo matura il suo distacco da S. e poi da W. \t testimoniato da "Umano troppo umano"

\ss[Umano troppo umano]
Esistono due tipi di pessimismo
\bi
    \item pessimismo romantico \t quello dei rassegnati, che scappano dalla realta  \t di S.
    \item pessimismo + consapevole \t di chi affronta la vita, conoscendone pero la tragicita \t del coraggio, della fortezza
\ei
Il primo è da rifiutare, il secondo invece impedisce la fuga e porta l'uomo a calarsi nella realta e nella storia come si presentano \\
S. è l'erede della tradizione cristiana secondo N. \t lui pone la rassegnazione di fronte a qualcosa che non si puo cambiare, e questo viene dal cristianesimo \\
Il distaccio da S. avviene in modo indolore

\v

Quello con W. è invece + sentito \t era amico con lui e si sente tradito da W. \\
N. dice di essersi illuso di aver trovato un generatore della cultura, come si sentiva lui \t e dice che W. è una malattia, e ha ammalato la musica \t lo definisce un genio istrionico \\
Lui scrive "Il caso Wagner"

\v

L'allontanamento dai suoi maestri segna l'allontanamento dal pensiero del romanticismo, e rappresenta il distacco dalla filosofia metafisica (chiamati camuffamenti metafisici, come l'idealismo e il positivismo) che hanno letto la realta partendo da degli apriori, partendo da un ideologia (una struttura predefinita) \\
Quini è grande momento di transizione che culmina con la "Gaia scienza"

\ss[Gaia scienza]
L'uomo pazzo qua annuncia la morte di dio \t ed è il fatto che non c'è ne è stato di più grande \t ed è l'evento centrale della storia umana, perche segna l'inizio di una nuova era \\
Morte di tutti quei valore e quella morale legati a Dio, e la concezione stessa di uomo in relazione a dio \\
Valori cambiano, diventano dei disvalori \t tutto cio che la religione riteneva buono è da contestare \\
Il nuovo valore sta nell accettazione del non valore \t nichilismo di N. \t esiste il superuomo, che ha i valori della terra (e non dello spirito) \t ma questi valori non sono paragonabili \\
Infatti il superuomo N. non ha legami, e questi valori sono assolutamente singoli \t ognuno ha i suoi punti di riferimento, e sono i miei \t non c'è morale condivisa, perche una morale come quella cristiana non esiste piu \\
Davanti alla morte di dio, si apre una nuova era \t in cui io sono al centro della mia esistenza \t sono consapevole che non ci sono valori assoluti e che la vita è dolore \t ma non scappo \\
In realta non sono valori quelli del superuomo, ma un modo di stare nel mondo \t esempio io sono temerario, mentre un altro è molto carnale \\
I valori del superuomo sono quelli della terra, ovvero quelli dionisiaci \t ma questi valori non sono trasversali, non c'è morale \\
In questa opera, la morte di dio viene annunciata ma non si è ancora realizzata \t la gaia scienza è di passaggio \\
Uomo pazzo dice che dio è morto e voi l'avete ucciso, noi l'abbiamo ucciso = la societa occidentale \t si è allontanata da dio \\
Ma non è ancora tempo dell'uomo nuovo dice l'uomo pazzo \t è scomparso l'uomo vecchio, ma quello nuovo non è ancora apparso \\
La morte di dio è quella che annunciera anche Zaratustra, ma qua c'è anche l'uomo nuovo \t che volge lo sguardo al cielo e accetta la sanita della terra \\
Non c'e la riprosota di un valore, nel senso di un valore come si intende fino a quel momento \\
Dio e il superuomo non possono convivere, e dio è morto perche il superuomo viva \t le loro prospettive sono antitetiche \\
Il superuomo, per il cristianesimo, si pone come la perfezione e quindi si contrappone a dio ??? \t dio deve morire per far continuare la storia \\
Non c'è una morale superiore, che è un obbiettivo, ma una morale umana \t incarna tutti i caratteri fondamentali dell'umanita \\
Dio è morto e noi l abbiamo ucciso \t si sottointende che dio è una nullita ontologica \t come lo abbiamo generato, cosi l'abbiamo ucciso \t uccidere dio = eliminare ogni riferimento all'assoluto

\ss[L'anticristo e al di la del bene del male e genealogia della morale]
Dio è cosi negativo perche crisitanesimo ha pervertito l'uomo \t un animale perverso = preferisce cio che gli è nocivo \t volge la sua istintualita verso cio che gli fa male \\
Il crist. ha fatto dinventare peccati tutti i caratteri della terra che sono fondamentali \t e ha fatto diventare valore tutto cio che è debole e miserevole, che è abbietto \\
Il cristianesimo è la religione dei malriusciti \t uomini che non incarnano l'ideale di uomo \\
Questo è succsso perche cristianesimo si fonda sulla compassione, che è un sentimento dannosissimo \\
Perche la comp. è un sentimento debole, e mi fa perdere forza \t la mia dimensione di autoreferenzialita viene meno \\
Inoltre comp. si contrappone alla selezione naturale \t i deboli sopperiscono per i forti \\
Ma se provo comp., questo sentimento mi trattiene \t con comp. storia delgi uomini non potrebbe andare avanti \\
Il dio cristiano è un dio degenerato e degli infermi, nemico della natura perche contraddice la natura dell'uomo e la vita \\
La compassione fa parte dell animalita umana, ma non devo seguirla per orientarmi nella vita come dice il crisitanesimo \\
Cristianesimo è religione della decandenza \t come il buddismo, che pero è piu realistico del crist. \\
Quello che è insensato è che il crist. parla di peccato \t che quindi c'è una condanna, che dio giudica \t mentre buddismo parla di dolore \t se io accetto di lottare contro questo dolore \\
Anche budd. è religione della decadenza perche è un modo di scappare dal dolore in ottica religiosa \\
Lui pero salva cristo \t perche gesu è morto per far vedere come si deve vivere (è un superuomo) \\
Infatti sapeva che sarebbe morto, ma ha mostrato forza e coerenza \t ma tra cristianesimo e gesu c'è un abisso \\
Gesu è stato questo \t mentre il cristianesimo ha tradito cio che gesu ha mostrato \t crist. è diventato la religione dei deboli, che è diversa dalla figura di gesu \\
I valori di gesu, che ha posto in essere con la sua stessa esistenza, sono stati traditi \\
Gesu era forte, e ha imposto se stesso \t è diventato il centro dell umanita \\
Un altro uomo forte è stato Ponzio Pilato \t che ha disprezzato la verita \t ha disprezzato il vero, che non è un assoluto (non esiste, presuppone dei valori assoluti) \t ha applicato un totale disinteresse, che è ammirevole \\
Dopo questo è successo un disastro nella storia \t rinascimento ha provato ad andare verso i valori della terra e l'uomo \\
Borgia era figlio di Alessandro II \t ambiva al soglio pontifico, il padre voleva farlo diventare un principe \\
Nel rinascimento la chiesa andava verso l'affermazione dei valori della terra \t borgia aveva accettato i valori della terra \\
Finche non arriva lutero \t che rovina tutto quello che il rinascimento stava creando, ovvero il crollo del valore finto della compassione \\
Lutero è un prete malriuscito \\
Chiesa ha costruito una grande menzogna, ovvero dell uomo debole che non sa riscattarsi

\v

Bisogna quindi transvalutare questi valori e ribaltarli \t in particolare critica la morale cristiana \\
È una morale dei vinti e degli schiavi \t c'è la morale dei forti (aristocratica) e dei vinti \\
Fa un analisi psicologica: ci sono due morali \t quella cristiana è la morale degli schiavi (della vita) \\
Loro hanno fatto diventare disvalore tutto cio che non sono in grado di fare \t e hanno fatto diventare valore cio che è alla loro portata \\
Non essendo la loro natura in grado di comportarsi in un modo, hanno fatto diventare peccato quello che gli altri non riuscivano a farlo \t quindi hanno detto che non si comportavano in quel modo perche non volevano, non perche potevano \\
La base della morale cristiana è quindi il risentimento \t e faccio tutte le cose che non appartengono alla terra \t essendo io un vinto, li condanno \\
Questi valori sono il coraggio, la forza, l'amore per la vita \t quindi i miei valori diventano compassione, disinteresse \t che non appartengono alla natura umana \\
L'ascesi, che mostra apparentemente un disinteresse per gli altri, manifesta una volonta di dominio \t il fatto di staccarmi dal mondo e generare un nuovo stile di vita, voglio essere superiore e mi pongo come un modello \\
Non è altro che un tentativo per ergermi sopra agli altri e soggiogarli \t l'asceta vuole diventare un modello 

\v

La morale aristocratica è quella dell'individualismo è quella dell'affermazione del proprio ego, della superbia, del coraggio \\
È la morale + antidemocratica di tutte \t io affermo me stesso \t la morale degli schiavi è quella della democrazia e del socialismo, che contesta \\
Il socialismo crea legame tra gli uomini che sono tutti uguali \t ma non è vero \t uomini non sono uguali per natura, e quindi i migliori devono comandare sui deboli \\
Crist. non esiste perche i forti devono schiacciare i deboli \t la compassione non puo essere
\end{document}
