\documentclass[12pt]{article}

\usepackage[a4paper, total={6in, 8in}]{geometry}
\usepackage{textcomp}

\begin{document}
\setlength{\parindent}{0pt}

\def \t {\textrightarrow}
\def \v {\vspace{1em}}
\def \bi {\begin{itemize}}
\def \ei {\end{itemize}}
\def \s[#1] {\section*{#1}}
\def \ss[#1] {\subsection*{#1}}
\def \sss[#1] {\subsubsection*{#1}}

\s[Marx]
Alienazione del lavoro puo essere modicifcata sol con la lotta di classe \t abolizione proprieta privata e la societa borghese viene aboita \\
Lotta di classe non approda direttamente nella societa comunismo \t c'è prima dittatura del proletariato

\ss[Dittatura del proletariato]
Fase autoritaria in cui il proletariato assume il potere assoluto \\
Nella societa successiva non ci saranno + classi \t se esistono ci sara sicuramente lo sfruttamento \\
Per passare da una societa all altra serve la forza \t quando proletariato raggiunge potere, serve espropriazione forzata \\
C'è un momento in cui la proprieta privata diventa dello stato (nazionalizzata) \t fase del comunismo rozzo, o dittatura del proletariato \\
Nel comunismo rozzo la proprieta privata è ancora presente, ma è nelle mani dello stato \t è rozzo perche c'è un soggetto che detiene tutte le proprieta \\
Le classi non esistono +, ma proprieta si \\
Poi il potere politico si dovra ritare progressivamente \t per poi estinguersi \\
Lo stato è un potere borghese che tutela la classe borghese \t quando non ci saranno piu classi sociale, non serve stato \t era solo uno strumento borghese per mantenere classi sociali \\
Dopo espropriazione avviene ridistribuzione

\v

Prima pero serve dittatura del proletariato \t che vuol dire dispotismo \\
Fara:
\bi
    \item espropraizione
    \item imposta sul reddito prograssiva
    \item confisca dei beni
    \item accentramento del credito nella banca dello stato, che assume il monopolio sul capitale
    \item accentrazione nelle mani del prol. dei mezzi di trasporto, statalizzati
    \item abolit il diritto di successione delle terre 
    \item fabbriche nazionali \t aziende agricole sono in mano allo stato e danno lavoro, e dovranno migliorare i terreni non coltivati
    \item obbligo di lavoro uguale per tutti
    \item istruzione pubblica \t che però viene combinata con il lavoro manuale
    \item abolizione del lavoro minorile
\ei
Poi avverra il "salto nella liberta" \t alla societa borghese si sostituisce un associazione in cui il libero sviluppo di ciascuno sara condizione del libero sviluppo di tutti \\
Questo è la base della societa comunista \t ma questo salto, come avviene effettivamente? \\
Come arrivare alla libera associazione non viene descritto da Marx \t come si passa dalla dittatura del proletariato, i proletari che hanno potere poi abbandonano il controllo?
\end{document}
