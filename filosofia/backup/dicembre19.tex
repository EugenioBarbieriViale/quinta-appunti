\documentclass[12pt]{article}

\usepackage[a4paper, total={6in, 8in}]{geometry}
\usepackage{textcomp}

\begin{document}
\setlength{\parindent}{0pt}

\def \t {\textrightarrow}
\def \v {\vspace{1em}}
\def \bi {\begin{itemize}}
\def \ei {\end{itemize}}
\def \s[#1] {\section*{#1}}
\def \ss[#1] {\subsection*{#1}}
\def \sss[#1] {\subsubsection*{#1}}

\s[Genealogia della morale - Nietzsche]
Cristiani fanno diventare peccato tutto cio che non è alla loro portata \\
La morale dei forti è invece morale dell'individualismo estremo e dei valori della terra \t che schiaccia il debole, e non prova compassione \t inoltre è antidemocratica e antisocialista \\
N. critica tutte le forme politiche che dicono che tutti gli uomini sono uguali \t non è vero ci sono superuomini e deboli \\
N. toglie anche tutte le costruzioni metafisiche, che vogliono dare una lettura oggettiva alla morale \\
Tutte le speculazioni che permetto alla morale di rimanere in essere \\
Da qui nasce l'anima \t i cristiani sono dei frustrati e infelici, che non trovano soddisfazione dei loro istinti nel mondo \t quindo sono ritornati in se (al posto di lanciarsi nella realta) \\
Hanno fatto in modo di creare un mondo di bellezza nel loro interno \t cosi è nata l'anima = istinto dell'uomo che si rivolge contro l'uomo, non trovando soddisfazione al di fuori \\
Mondo integro all'interno dell'uomo \t che è un invenzione umana \t ed è malata, perche nasce dal mancato confronto e realizzazione dell'uomo col mondo \\
Dire che fare una certa cosa è peccato, non mi sto confrontando con la realta \\
Dimensione dell'anima non ha sussistenza ontologica per N. perche è metafisica

\v

Smantellando tutto questo (anche la verita, anima, ..) subenta l'abisso del nulla \t nichilismo è uno stato psicologico che subentra di necessita \\
Nichilismo = avevo creduto ci fosse un senso ma non c'è, cercavo risposte ma non ci sono risposte vere \\
Quindi l'uomo che si libera della morale è un uomo che si libera delle illusioni \t ma questo non deve generare una fuga e l'uomo deve essere coragggioso \\
Nulla ha valore, e lo accetto \t superuomo è consapevole della mancanza di senso, ma ci vive dentro \\
Non ci sono valori ma solo disvalori, etc.

\v

Nella realta qualcosa c'è pero ancora \t ovvero una neccessita nella realta e nella storia \t la necessita della volonta, riprendendo schopenhauer \\
Nel mondo c'è l'eterna volonta a ripetersi \t la realta non ha un senso (ovvero direzione), non va da nessuna parte, non ha uno scopo \\
Quindi scopro anche che il mondo procede in modo circolare, che prevede la ripetizione eterna di tutto cio che è accaduto = mito dell'eterno ritorno (lo prende dalla grecia e filosofie indiane) \\
Tutti i dettagli che caratterizzano la nostra storia si ripeteranno cosi come sono \t percorso storico non è rettilineo \t quindi non c'è neanche progresso \\
Rispetto a questo mito il superuomo come si deve comportare? \t lui sa di non poter cambiare le cose, e deve accettare \t ma qua deve fare un passo in piu: deve amare \\
Dottrina della amor fati viene legata al mito dell eterno ritorno \t io accetto quesot mondo e lo amo, quindi mi immergo in questa necessita \t no accettazione passiva \\
C'è una riconciliazione tra l'uomo e la realta negativa, che torna sempre uguale a se stessa \t ma l'uomo la desidera e la ama \\
L'uomo ama il ripetersi della realta perche riconosce questa volonta anche dentro di se (Schopenhauer) \\
Per entrare in sintonia con questo mondo che si riptere, uomo capisce che ha la stessa volonta che governa il mondo \\
L'uomo vuole stare nel mondo, il mondo vuole ripteresi \t quindi nasce l'amor fati \\
E l'uomo sta in questo mondo \t lo dice in Zaratustra, che ha un messaggio da dare agli uomini, ovvero insegnare il superuomo \\
Nella gaia scienza invece si annuncia la morte di dio ma non è ancora tempo per uomo nuovo \\
Uomo nuovo (in Z.) deve creare un nuovo senso della terra \t deve realizzare una nuova realta \\
L'uomo nuovo è l'oltre uomo \t l'uomo di N. va oltre l'uomo, realizza un umanita che supera l'umanita dei deboli \t ne realizza una nuova legata alla terra, allo spirito dionisiaco, alla salute \\
Uomo nuovo si lascia alle spalle l'uomo vecchio e i vecchi doveri, a cui sostituisce la sua volonta \t non ci sono dei valori imposti, ma ci sono io e basta \\
Z. dice che ci sono dei predicatori di morte che predicano mondi sovrannaturali \t Z: dice di essere la voce del corpo ? \\
In Z. manda un messaggio simile a quello di Protagora \t l'uomo è la misura di tutte le cose \\
Il superuomo non avra mai la schiena piegata \t chi piega la schiena obbedisce ai valori e al vecchio mondo \t i valori li genera lui, per questo protagora \\
Generare il senso pero significa accettare il non senso \t è il superuomo che crea un nuovo senso della realta, è la necessita \t uomo non puo cambiare mito dell eterno ritorno \\
Si compie la riflessione sull'uomo che c'era in Kierkegaard e Schopenhauer

\v

Hitler riprendera la completa esaltazione del singolo sugli altri \t il fuhrer comanda \\
N. ha fornito gli strumenti teorici al nazismo? \t non consapevolmente, ma la cultura tedesca si è ispirata a questa idea 
\end{document}
