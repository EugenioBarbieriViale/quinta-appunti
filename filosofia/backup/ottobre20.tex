\documentclass[12pt]{article}

\usepackage[a4paper, total={6in, 8in}]{geometry}
\usepackage{textcomp}

\begin{document}
\setlength{\parindent}{0pt}

\def \t {\textrightarrow}
\def \v {\vspace{1em}}
\def \bi {\begin{itemize}}
\def \ei {\end{itemize}}
\def \s[#1] {\section*{#1}}
\def \ss[#1] {\subsection*{#1}}
\def \sss[#1] {\subsubsection*{#1}}

\s[Filosofia dello spirito]
Nella filosofia della natura spirito esce fuori di se, genera l'essere e torna in se nella filosofia dello spirito \t massima autoconoscenza di se

\ss[Spirito soggettivo]
Spirito soggettivo \t prima tappa della triade \\
Si parla di un finito \t che non è manifestazione dello spirito \t ma appare dentro lo spirito \\
Prima l'infinto aveva bisogno del finito per manifestarsi \t qua il finito ha senso solo perchè sta dentro l'infinito \t ora è il finito che ha bisogno dell'infinito \\
Prima l'infinito si serve del finito, qua c'è totale svalutazione del finito \t non vale nulla se non viene letto come un parte dell'infinito \\
Nello spirito soggettivo parla di una forma di conoscenza dello spirito legata all'uomo \t triade interna è antropologia, fenomenologia, piscologia \t modi per leggere il finito dentro l'infinito \\
Nella triade si parla dell'anima \t che è presente nell'uomo, ed è il segno della sua appartenenza all'infinito \\
Lo spirito si conosce passando dal finito, ma passando da quell'elemento spirituale/razionale che è presente nel finito è lo lega in modo indissolubile al finito \\
È come l'intelletto potenziale \t ne parla nell'antropologia \\
Nella fenomenologia riprende l'opera e ridescrive la strada che lo spirito fa per autoconoscersi \\
Nella psicologia parla dello spirito pratico e dello spirito teoretico \t si mette insieme l'aspetto logico dell'uomo, in una sintesi che culmina nello spirito libero \\
Spirito libero rappresenta  il momento in cui l'anima capisce di essere spirito \t la parte razionale/spirituale dell'uomo (anima) \t capisce di essere espressione dello spirito, a cui l'uomo ontologicamente appartiene \\
Consapevolezza si ottiene nella propria dimensione di interiorita \t non piu stando nel mondo \\
L'anima esiste perche sono parte dello spirito 

\v

Questa è la tesi \t e rivaluta il trattato sull'anima di aristotele \t che rappresenta l'opera migliore sull'anima \t dice che è l'unica speculazione corretta sull'argomento \\
Riprende da qui molte parti \t con qualche correzione \t la + evidente è quella che dice: "non c'è nulla che ci sia nell'intelletto se prima non passa nei sensi" \t hegel aggiunge "tranne l'intelletto stesso" \\
La razionalita precede il senso \t ontologia e logica coincidono \\
Fa anche riflessione sulla parola e sulla lingua \\
Spirito soggettivo si chiude con l'affermazione della liberta \t anima libera si riconosce come spirito libero, l'anima si riconosce nello spirito 

\ss[Spirito oggettivo]
Momento + significativo di tutto l'hegelismo \\
Spirito oggettivo = momento in cui lo spirito libero esce fuori di se \\
Prima interiorita \t qua lo spirito libero si realizza nelle istituzione della famiglia, dello stato, ... \t tutto il sociale e l'etico-politico \\
Lo spirito oggettivo si articola in diritto, moralita, ed eticita \\
A sua volta eticita in famiglia, societa e stato \\
E sono triadi dialettiche \t anche se siamo nell'antitesi \t ma è l'antitesi della sintesi superema, quindi va bene

\ss[Diritto]
Spirito libero non sta in una liberta astratta \t liberta si deve concretizzare e deve stare in delle regole \t se liberta si esercitano senza regole, si avrebbe anarchia \\
Liberta concreta è nelle istituzioni \t per questo libertà da vita al diritto (ovvero la legge e le pene) \\
La liberta definisce il diritto che mi garandtisce la mia posizione nel mondo \t ma questo modo di esercitare la liberta (giuridico) è esteriore \t è del fare, posso fare cose e non e vengo punito rispettivamente \\
Ma questo è troppo esteriore, troppo immediato \t serve dimesnione di liberta + interiore 

\ss[Moralita]
Io sono libero in quanto uomo \t non piu nel fare, ma nella sfera soggettiva \t quello che conta non è la mia azione sulle cose, ma una mia liberta di scelta, una volonta, che viene prima dell azione \\
Prima liberta era solo in orizzonte oggettivo \t qua liberta è totalmente nella sfera soggettiva \t si puo essere liberi anche nella prigionia \\
Liberta è come determino la mia volonta \t non come sto nel mondo \\
Riprende l etica di Kant \t che fornisce la forma universale a cui ispirare l'agire morale \t criterio fondamentale per definire volonta \\
Ma comunque la sfera soggettiva di Kant è troppo chiusa \t prima solo esterno, liberta era solo rispetto della legge \\
Ora solo dimensione interiore \t volonta si determina liberamente ma a priori \t la volonta è libera in quanto tale a prescindere dalla realta, troppo interiore \\
Diritto e moralita sono troppo unilaterali \t uno tutto a posteriori, l altro anteriore 

\ss[Eticita]
Io realizzo la mia volonta libera con dei fini concreti \t volonta si determina liberamente ad agire ma non in base a devi perche devi, ma nel mondo \t volendo delle cose concrete \\
Io definisco la mia libera volonta nel volere delle cose nel mondo \\
Volontà si realizza nella azione concreta \\
Questo si realizza nella triade \t volonta si realizza nella concretezza della mia storia nelle 3 istituzioni della triade:

\sss[Famiglia]
È un luogo naturale \t nasce nella forma dell'amore e del sentimento \t luogo in cui l'individuo annulla la sua individualita per l universalita \\
Uomo rinuncia alla affermazione di se, del suo egoismo \t lo fa per il bene della totalita (ovvero famiglia) \\
Famiglia è unita in cui le individualita sono in secondo piano \t prevale unita \\
Ma la famiglia puo sgretolarsi \t storicamente, le famiglie che diventano sempre + grandi diventano societa civili (tribu, ...) \\
Cosi il legame familiare si sfalda \\
Smette di esistere quando io affermo il mio volere, la mia volonta, egoisticamente \t perche voglio essere indipendente \t non riconosco + appartenenza, e quindi l'amore

\sss[Societa civile]
Quando legame d'amore si spezza \t prevale di nuovo la societa \\
Si passa alla societa civile \t in cui è l'utile a governare \\
Mentre nella famiglia c'era un mettere da parte l individualita per il bene della famiglia, nella societa affermo la mia individualita e intrattegono relazioni perche ne ho bisogno \\
Unico legame tra individui è definito dalla necessita come bisogno = l'utile \\
Spesso la società è stata spesso confusa con lo stato \t hanno confuso questo livello di condivisione con lo stato

\sss[Stato]
È la sintesi di entrambe le triadi in realtà \t è il punto di arrivo dello spirito oggettivo \\
È la massima manifestazione dell assoluto \t mette insieme la famiglia e la societa \t riafferma l'unità della famiglia (è l ingresso di dio nel mondo) ma il legame è quello sociale, di bisogno e non di amore \\
L individuo ha bisogno di stare nello stato \t e lo stato realizza massimamente lo spirito, che quindi si impone sulla societa \\
Non è lo stato per il cittadino, ma il contrario \t lo stato è superiore, unita che si impone sulla particolarita \\
Stato è la massima espressione della razionalita dello spirito \t è nettamente superiore alla razionalita dell uomo \t che quindi è particolare \\
Stato è la razionalita dello spirito \\
Stato è l'espressione + vera di una volonta che si determina ad agire \t lo stato vuole necessariamente cio che deve essere \t nello stato l'essere e il dover essere coincidono \\
Lo stato realizza nella storia cio che deve essere realizzato \t lo stato non è da migliorare, perche sta realizzando cio che deve essere storicamente \\
Anche lo stato + difettoso non elimina il positivo dello stato stesso, in quanto è uno stato e quindi esprime la razionalita dello spirito \\
Si parla quindi dell'isitiuzione stato \t che rappresenta l'unita assoluta e la risposta al bisgno della societa civile \t non parla della natura dello stato

\v

Lo stato si impone quindi dall'alto \t e non può essere messo in discussione  \\
L'uomo è libero solo se fa partre di una sostanza etica \t io sono libero nella misura in cui mi trovo nello stato \t che non puo essere messo in discussione \t lo stato è lo spirito nella storia \\
Stato è l'espressione massima di perfezione e razionalita, è l'ingresso di dio nel mondo \\
Questa visione è stata manipolata e strumentalizzata nella storia \\
Lo stato è legittimato nell essere razionale \t se e max spirito, allora e max razionalita \t quindi la sua azione nel mondo è fondata necessariamente \\
Stato quindi schiaccia l'uomo \t la sua liberta è soltatno stare nella sostanza etica dello stato \\
Inoltre lo stato sopra di se non ha niente \t statalismo giuridico = lo stato è la fonte del diritto e disprezza tutto cio che precede \\
Ma la sua visione dello stato è coerente \t finito non ha essenza propria \t lo stato è max sprito, ha valore universale che quindi riassorbe il contingente 

\v

La storia è quindi la storia della relazione tra gli stati \t e quindi è il dispiegarsi della razionalita degli stati \\
La storia è lo spirito che si mostra nella relazione tra gli stati \t storia è il dispiegamento dello spirito nel tempo, dove lo spirito si realizza nell istituzione dello stato \\
La filosofia della storia è quindi la massima forma di conoscenza \t prima era solo una conoscenza dei fatti storici (dal punto di vista dell intelletto), sua interpretazione è dal pov. della ragione \\
Lui chiama la razionalita quella che la religione chiama provvidenza \t c'è un progetto nella storia \\
Nella storia quindi tutto è giustificato \t il momento significativo della storia (la guerra) è l'antitesi, il momento che garantisce il movimento dello spirito \\
Senza guerre la storia avrebbe realizzato solo pagine bianche \t quindi non esiste neanche il giudizio storico \t deve essere cosi, perche e max raz dello spirito \t doveva accadere cosi \\
Individui cosmici sono gli individui che governano \t non hanno alcun merito (per esempio napoleone) \t solo quello di aver compreso la direzione in cui lo spirito andava \t si sono adeguati massimamente (massima eticita) e hanno realizzato quello che lo spirito voleva realizzare attraverso loro \\
Anche il negativo è giustificato \t legittima la tesi per poi passare alla sintesi \\
"Tutto cio che è reale è razionale, e viceversa" \t lo stato storico è l'espressione della massima razionalità \\
Non sta parlando pero di un organismo politico \t sta parlando della idea di stato, che è una realtà immateriale ma ontologica 

\v

L'interesse particolare del singolo appartiene comunque allo spirito \t anche le passioni del singolo individuo sono funzionali alla storia \\
Anche la storia passa per tappe dialettiche: il mondo orientale, il mondo greco/romano, il mondo cristiano-germanico \t con una progessione di razionalita \\
Lo stato si realizza nell ultima tappa \t e hegel si ferma qua, non dice che ci sarà un progressivo incremento di razionalita successiva \\
È un po' un'aporia \t che poi si ripercuote anche in marx, dove la societa comunista è il punto di arrivo

\ss[Spirito assoluto]
Nello spirito assoluto si rimette insieme esteriorita ed interiorita \t è il punto di arrivo, l'autocoscenza perfetta dell'infinto \\
E lo fa sempre attraverso l'uomo \t che è la massima manifestazione finita dello spirito \\
Parla quindi di 3 tappe: l'arte, la religione, la filosofia \t progressiva autoconoscenza dell'assoluto \\
Ancora si parla di uomo \t che pero è diventato funzionale all'assoluto \t ribaltamento \t prima uomo era manifestazione empirica dell infinito \t qua invece uomo sta dentro l'infinto, è tornato dentro \\
Ha senso tutto quello che fa perchè è funzionale allo spirito \\
Sono le tappe attraverso le quali l'assoluto (che chiama dio) si autoconsce

\sss[Arte]
Si cogli l'infinto attraverso l'intuizione sensibile \t l'opera d arte si presenta con una forma empirica \t attraverso la sua dimensione sensibile permette di cogliere qualcosa che va oltre \\
Intuizione estetica \t si coglie con i sensi, ma la forma ci esprime qualcosa che va oltre la forma stessa \t è un modo di intravedere l'infinito, ma non di definirlo 

\sss[Religione]
Si avvicina all infinito \t attraverso la rappresentazione della fede \t siamo nell orizzonte dell intellegibile (prima del sensibile) \\
Fede permette di cogliere l'assoluto, perche ci credo \t è svincolata dalla forma sensibile, si avvicina a una forma teoretica di conoscenza \\
Religione pone nell interiorita quello che nell arte era esteriore 

\sss[Filosofia]
La filosofia media tra esteriorita e interiorita \t filosofia si interroga sulla realta, conosce quello che è fuori (dimensione esteriore) \\
Poi pero la realta viene pensata da me \t la conosco perche sono dotato di un libero pensiero \\
Filosofia mette insieme l oggetto, con lo strumento attraverso il quale avviene la conoscenza \t che sta dentro l'uomo \\
Mette insieme oggettivita dell arte e soggettivita della religione (che qui diventa quindi del pensiero) \\
Filosofia è quindi massima autorealizzazione dello spirito \t che si conosce perfettamente \\
Tutte hanno tappe dialettiche:
\bi
    \item arte orientale, greca, romantica
    \item religion orientale, greca, cristiana
    \item filosofia (che coincide con la storia della filosofia) \t antichita greca, medioevo cristiano, modernita germanica
\ei
Racconta la filosofia da talete ad hegel \t e spiega ogni passaggio che porta all hegelismo, che è il massimo

\s[Dopo hegel]
Hegel è il grande riferimento della filosofia dell 800 \t quello che succede dopo è un relazionarsi con l'hegelismo (non idealismo) \\
Non si potra prescindere da questa visione \\
Gli stessi hegeliani si dividono in:
\bi
    \item destra hegeliana \t prospettiva + radicale del sistema di hegel \t sostiene il sistema cosi come e
    \item destra hegeliana \t + critico, a cui appartiene marx \t sinistra ha ridimensionato e ridiscusso il sistema di hegel \t + interessante dal punto di vista speculativo
\ei
Schopenhauer e Kierkegaard saranno grandi oppositori \t contestano dell hegelismo la radice \t ovvero l'assorbimento totale del finito nell infinito \\
Hegel lo accusano di una dottrina ideologica \t hegel è partito da un idea e ci ha creato il suo sistema \t ma per loro bisogna partire dalla concretezza finita, che è dove viviamo \t da li bisogna partire \\
Viene anche ridiscussa la razionalita \t che poi viene distrutta \t chiedere ??

\v

L'autoritarismo ha anche sfruttato l'hegelismo \t ma lui non parla mai di stati autoritari \t è una lettura facile \\
Totalitarismo hanno desunto le basi ideologiche da hegel \\
Cosa vive ancora di hegel? \t molte analisi in vari campi del sapere (antigone, servo padrone, riv. francese, ...) \t molte analisi sono interessanti \\
È morta invece la pretesa di dare all uomo la totalizzante conoscenza dell assoluto \t che l uomo si puo confrontare con l assoluto \t al finito viene ridata una dignita
\end{document}
