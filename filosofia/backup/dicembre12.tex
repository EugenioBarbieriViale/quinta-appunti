\documentclass[12pt]{article}

\usepackage[a4paper, total={6in, 8in}]{geometry}
\usepackage{textcomp}

\begin{document}
\setlength{\parindent}{0pt}

\def \t {\textrightarrow}
\def \v {\vspace{1em}}
\def \bi {\begin{itemize}}
\def \ei {\end{itemize}}
\def \s[#1] {\section*{#1}}
\def \ss[#1] {\subsection*{#1}}
\def \sss[#1] {\subsubsection*{#1}}

\s[Erder Spencer]
Si colloca nel positivismo britannico, ma fonda prospettiva teoretica diversa da Mill \t prospettiva evoluzionista, infatti parte dalla teoria di Darwin \\
Elabora teoria evoluzionistica legata a tutta la realta \t un evoluzione dall omogeneita all eterogeneita, quindi applica evoluzione biologica \\
La sua nota importante è l'evoluzionismo legato alla totalita della realta \\
Afferma che non c'è nulla di apriorristico, ma tutto si forma in divenire \t e si va da una realta omogenea (e instabile) a una eterogenea (+ stabile proprio perche ogni ente ha un suo scopo, e si risolve il disordine dell'instabilita) \\
Tutto è sempre in divenire \t e in questo divenire c'è un continuo miglioramento (prospettiva ottimistica) \\
Anche la morale è in divenire \t e ritiene che il momento migliore è sempre quello che deve arrivare \\
Non è pero teoreticamente molto rilevante pero \\

\s[Nietzsche]
Nasce nel 1844 e muore nel 1900 \\
N. non appariene ad alcuna corrente ed è un pensatore isolato \t si puo collocare un po nell'esistenzialismo, perche riflette sull'esistenza \t ma come K. non è collocabile in un pensiero \\
È un pensatore molto critico \\
Si scaglia contro il positivismo \t perche ha fatto dei fatti degli idoli \t il fatto empirico è l unico punto di riferimento \t ma N. dice che il fatto è stupido, di per se non significa nulla e ha bisogno interpretazione \\
Atteggiamento positivista è quindi da allontanare \t e considera i positivisti degli esaltati, che non volevano vedere altro che la loro prospettiva empirica, di cui hanno fatto il loro assoluto \\
N. è anti assoluto \t svaluta l'infinito \\
Critica anche la concezione del progresso \t è una falsa idea che non trova riscontro nella realta, e secondo lui non è vero che si va verso il meglio \\
L'altro idolo positivista è il progresso, e non riscontra nella realta \\
Neanche la scienza è la verita \t si basa sui fatti empirici \\
Idealismo e cristianesimo, che si basano su una prospettiva metafisica, vengono ripiudati \\
In N. c'è l accettazione del finito in quanto tale \t e quindi si accetta il suo limite \t sono consapevole di essere finito e quindi limitato \t tutto il mio stare nel mondo ha un margine di negatività \\
L'infinito è un'illusione che ha costruito l'uomo \t e che porta l'uomo a voler uscire dalla vita dell'uomo \\
Ogni filosofia metafisica, che propone una via di fuga, va rifiutata \t si riallaccia quindi al discorso di Feuerbach \\
Qua c'è una totale sostituzione dell'infinito con la prospettiva finita \\
Il suo è un messaggio esistenziale \t no riflessione gnoseolgogica, no ontologica \\
Fa un discorso abbastanza nichilista \t il mondo del super uomo, nasce con la morte di dio \t la morte di dio annulla ogni valore e ogni riferimento \t nichilismo perche viene parte dalla negativita 

\v

"La volonta di potenza" \t ha fatto nascere l idea che N. sia un precursore del nazismo \t parla di rinascita, destino della nazione tedesca, superuomo etc. \\
In realta la sorella aveva manomesso questa sua opera, che era stata pubblicata postuma \t lei era nazionalsocialista \\
Ma lui non era \t alcuni concetti vengono presi nel nazismo, ma lui non poteva esserlo \t pero a partire dai discorsi alla nazione tedesca di fichte, ci sono degli spunti che preparano il terreno per un fenomeno nazista \\
N. non giustifica a priori il nazismo \\
Il superuomo è infatti il filosofo \t annuncia una umanita che ora è libera, che supera tutti i valori del passato \\
Inoltre non è nazista perche il superuomo deve anche rompere le catene dello stato \t lo stato è un freddo mostro, ed è un idolo che puzza \t si pone sulla terra come la realtà + grande, ma non è vero \\
Solo dove lo stato cessa di esistere, l'uomo inizia a smettere di non essere inutile \\
Di sicuro non ha teorizzato quindi lo stato nazista \t la sorella si \\
Inoltre dice che lo stato e la cultura sono antagonisti 

\ss[Vita]
Nasce nel 44 \t e nella sua vita e produzione letteraria si possono distinguere dei momenti \\
Studia filologia classica a Bonn e a Lipsia \\
All'inizio è appassionate di Schopenhauer \t legge il mondo conme volonta \t e condivide la prospettiva che la vita è una tragedia \\
Poi viene chiamato a Basilea \t qui conosce Burkard, ma soprattuto Wagner \t apprezza la sua musica, che dice che traduce la in musica le sue idee \t inoltre collabora con lui \\
Nel 72 esce la Nascita della tragedia \t e tra il 73 e 76 scrive Considerazioni inattuali \\
Prima era di grande vicinanza con S. e W. \t ma poi matura distacco:
\bi
    \item per motivi personali da Wagner (e anche per ragioni teoretiche) \t infatti Wagner aveva grande successo e N. no \t inoltre N. riteneva che W. era assetato solo di fama e ricchezza, e quindi non poteva essere un rigeneratore della cultura \t si rompe l'amicizia in "Umano troppo umano"
    \item nell'opera si distanzai anche da S. \t la sua filosofia è pessimista come quella di N, ma è un pessimismo della rassegnazione \t uomo di S. cerca delle vie di fuga: l'arte e l'ascesi \t non è quindi un uomo forte, è un rassegnato \t non è quindi il pessimismo del superuomo
\ei
Nel 69 abbandona Basilea \t si dimette dall'insegnamento e peregrina in Svizzera, Italia e Francia \\
Nel 81 pubblica l'Aurora e poi nel ? la gaia scienza \t qui prende forma il suo pensiero \t entrambe vengono scritte a Genova \\
Qui vuole sposare una ragazza russa \t ma lei no, e si mette con il suo amico \t dolore forte, e centrava anche la sorella \\
Nel 83 a Rapallo scirve "Cosi parlo Zaratustra" \t viene finita tra Roma e Nizza \\
Nel 86 pubblicato "Al di la del bene e del male" e nel ? "Genealogia della morale" \\
Primo periodo: vicinanza ai due \t poi umano troppo umano \t allontanamento \t poi pensiero centrale nelle ultime 3 \\
Poi nell 88 scrive "Il caso Wagner", poi "L'anticristo" e "Crepuscolo degli idoli", "Ecce homo" \\
Poi si sistema a Torino \t sente di appartenere qua e inizia l'opera che poi finira Elisabeth (sorella) \\
Nel 99 inizia la pazzia \t e muore a Weimar, senza neanche realizzare di essere diventato famoso

\ss[Primo momento: nascita della tragedia]
La vita è dolore, irrazionalita, \t solo l'arte puo dare sollievo all uomo \t è l unico mezzo per fronteggiare la vita, senza scappare pero \\
Parla quindi di tragedia classica (presocratica) \t la tragedia in grecia aveva rappresentato la massima espr. aristica \\
I presocratici rappresentavano nella tragedia un accettazione della vita che non era subire, ma un accettazione ebbra di vita \t si va incontro alla vita tragica, e il destino non schiaccia l uomo \t si va con coraggio contro \\
Tragedia presocratica era un modo per esaltare la dimensione della vita \\
La grecia presocratica era per lui un modo disordinato \t in cui si celebrava la vita \\
Era anche quindi pre-euripide \t era la grecia di sofocle ed eschilo \\
La tragedia pres. in particolare perche emergeva lo spirito dionisiaco \t lo spirito di una umanita in pieno accordo con la natura, uno spirito di passione, salute, istinto = arazionale \\
Lo spirito dionisiaco era nel coro della tragedia, che portava il significato etico \\
A fianco c'è lo spirito apollinio \t che è lo spirito della misura, dell equilibrio, della ragione \t ma non sono in esclusione, sono una dicotomia \\
Poi pero Euripide ha tentato di eliminare lo spirito dionisiaco \t ha voluto inserire nella tragedia elementi morali \t assegna valore pedagogico, ma non si puo \\
Socrate è disprezzato da N. \t è lo strumento della dissoluzione greca per lui \t socrate ha insegnato equilibrio, controllo di se etc. \t ha eliminato il dionisiaco dalla vita (mentre eu. dalla tragedia) \\
Socrate è quindi stato ostile alla vita \t ha voluto eliminare cio che è proprio umano \t ed ha aperto un periodo di decadenza \\
Lo spirito dionisiaco è di base migliore perche è in sintonia con la natura, tutto il resto è un tetnativo di controllare la natura dell uomo \t quindi sp. dion. realizza la natura umana \\
Questo nell influenza di S. e di W. \t dice che W. è l'artista moderno (inoltre dedica l'opera a W.) \t infatti parla qua di arte \\
Quest'opera suscita pero molte critiche iniziali \t anche se viene difesa da Wagner \\
Viene attaccata perche viene accusato di ignoranza e uno scarso senso di verita \t lui viene richiamato all'ordine della scienza, della storia \t che N. sembra disconoscere

\ss[Considerazioni inattuali]
Risponde con le "Considerazioni inattuali" \t ancora esaltazione di S. \t ma afferma che Compte è autore di un vangelo da birreria \\
Qua soprattutto combatte la saturazione della storia \t era stato attaccatto perche antistorico, quindi risponde che la storia è sicuramente importante (non viene negata) \\
Ma non bisogna idolatrare il fatto, come si bisogna liberare dalle illsuioni storiciste che sono politiche \t bisogna liberarsi da queste illusioni \\
Bisogna sempre considerare che i fatti sono stupidi \t la teoria no, perche la elaboro \\
Se si pensa che la storia sono dei fatti da rispettare in quanto tali, è sbagliato \t i fatti sono accadimenti \\
Chi crede inoltre nella potenza della storia, è una persona insicura e non crede in se \t alla luca di quel passato, non esprime una opinione del presente \\
Saturazione della storia è riproporre quello che è stato in continuazione \t senza guardare il presente, allora non si va avanti \t si diventa succubi e schiavi di un opinione \\
Non ci si puo continuare a riferire al passato \\
Quindi individua 3 atteggiamenti davanti alla storia:
\bi
    \item atteggiamento della storia monumentale: chi cerca nella storia dei modelli e dei maestri che soddisfino tutte le sue aspriazoni \t il meglio sta nel passato
    \item storia antiquaria: chi capisce che il passato è il fondamento del presente \t chi vive questo atteggiamento, cerca e conserva il valori fondamentali su cui si radica la vita presente \t si cercano le radici del presente
    \item storia critica: atteggiamento di chi guarda il passato con l'atteggiamento del giudice \t abbandono tutto cio che è di ostacolo per la realizzazione dei miei valori nel presente (e condanno), e mi porto nel presente solo cio che mi permette di realizzarmi \t ed è atteggiamento giusto
\ei
Un eccesso di storia (saturazione) non va bene \t impedisce al singolo e al gruppo di maturare \t si genera l'idea che chi vive il presente, non è altro che il frutto del passato \\
Che non si è un nuovo momento, ma la fine di qualcosa
\end{document}
