\documentclass[12pt]{article}

\usepackage[a4paper, total={6in, 8in}]{geometry}
\usepackage{textcomp}

\begin{document}
\setlength{\parindent}{0pt}

\def \t {\textrightarrow}
\def \v {\vspace{1em}}
\def \bi {\begin{itemize}}
\def \ei {\end{itemize}}
\def \s[#1] {\section*{#1}}
\def \ss[#1] {\subsection*{#1}}
\def \sss[#1] {\subsubsection*{#1}}

\s[Destra e sinistra hegeliana]
Ale che riprendono il sistema hegeliano \\
Destra + conservatrice \t e quindi meno innovativa \\
La destra e sinistra assumono posizioni diverse su due temi

\ss[Politico]
Allo stato aveva attribuito i caratteri dello stato prussiano del suo tempo \t il suo stato ha connotati storici \\
Lo stato prussiano è il punto di arrivo della dialettica \t destra hegeliana sostiene questo \\
Stato è massima razionalità dello spirito \t quindi stato non puo essere contestato \\
Sinistra invece dice che dialettica con è statica \t una configurazione politica dello stato non puo rappresentare la fine della dialettica \t la dialettica superera lo stato prussiano e andra avanti \\
Sinistra constesta hegel \t aveva detto che stato prussiano rappresentaza max. perfezione di quello che uno stato puo essere storicamente \\
La destra dice che l hegelismo giustifica lo stato in cui si vive \t mentre la sinistra contesta questa idea, lo stato prussiano non è perfetto e va superato

\ss[Religioso]
Controversia + forte \\
Nasce da ambiguita \t Hegel dice che filosofia e religione hanno lo stesso contenuto, ma lo esprimono in forme diverse \\
La filosofia esprime l'assoluto nella forma del concetto (conoscenza), mentre religione sotto rappresentazione (fede) \\
Quello che viene espresso dalla religione viene ripreso dalla filosofia a livello razionale e + alto \\
Essendo la filosofia la sintesi, quando la filosofia supera l'arte e la religione e li deve recuperare \t ma quindi la religione deve scomparire, è l'antitesi \\
Dovrei quindi non avere + religione \t filosofia è + perfetta, quindi dovrebbe scomparire religione (e anche arte) \t si tende a maggiore perfezione nello spirito \\
Pero hegel dice che religione e filosofia coestino \t quindi domanda è: cristianesimo e hegelismo possono stare insieme? \\
Destra dice si, anzi hegelismo è lo sforzo + adeguato dal punto di vista razionale di chiarire i dogmi della fede cristiana \\
Hegelismo è come una giustificazione razionale della fede \t e quindi anche compatibile nei contenuti \\
Sinistra invece dice no \t in prospettiva dialettica, filosofia assorbe completamente la religione \t o si ha cristianemismo o si ha hegelismo, non sono conciliabili \\
Religione viene assorbita nella filosofia, cosi non ha + connotati da scomparire \\
Del hegelismo si tiene l'identita tra finito e infinito \t ma cambia la loro relazione \t non è l'infinito che genera il finito, ma il contrario \\
Coincidono nella forma della negazione della trascendenza, del valore ontologico dell'infinito \t Hegel invece nega il valore del finito \\
Nella sinistra c'è ancora questa coincidenza, ma è il finito che genere l'infinito \\
Per Feuerbach è l'uomo che genera dio \\ 
La sinistra estremizza la posizione dialettica \t nega che in presenza di una filosofia la religione abbia ancora un senso \t perchè l'oggetto della religione non esiste più, è riassorbito nel finito \\
La destra hegeliana viene definita come la scolastica hegeliana \t pensatori si appoggiano a Hegel come scolastici ad Aristotele

\s[Feuerbach]
Studia teologia ad Heidelberg \t ascolta le lezioni di Hegel a berlino, si entusiasta, ma poi prende le distanze \\
Perche il singolo individuo in Hegel non esiste \t Feuerbach riflette sull'immortalita dell annima dell uomo \t ma questa categoria, della singolarita, in hegel non esiste \\
Scrive "Pensieri sulla morte e sulla immortalita" (1830) \t e questo blocca la sua carriera universitaria perche assume prospettiva critica nei confronti della destra \t hegel è ancora dominante \\
Nel 48 però clima è + progressista \t tiene un corso ad Heidelber e tiene "Lezioni sull'essenza della religione" \t pubblicate nel 51 \\
Rappresentano il suo momento mondano \t poi morira nel 72 solo e dimenticato \\
Fino al 27 è ancora un hegeliano, anche se ha preso le distanze \t nel 39 scrive "Per la critica della filosofia Hegeliana" \t critica che hegel ha perso il senso della realta e della singolarita \\
Opera + famosa, in cui si vede la rottura con hegel \t è del 41 e si chiama "L'essenza del cristianesimo" \\
Qui F. opera la riduzione della teologia ad antropologia \\
Vuol dire che partendo da hegel, che ha descritto un infinito (categoria) senza considerare che al suo interno ci sono singoli che vivono vita difficili, che si interrogano \\
Questa singolarita non puo essere ridotta a un infinito, senza considerare i suoi aspetti \t si pone sul piano della concretezza

\v

L'aspetto caratterizzante del singolo è la frustrazione \t uomo ha dei desideri, delle speranze, bisogni che non si realizzano \\
La coscienza frustrata (uomo) proietta fuori di se tutto questo \t tutto cio che non trova risposta nel finito \\
Prospetta tutto questo dolore in una figura chiamata dio \t cosi lo puo vedere oggettivamente, come se appartenesse a qualcun altro \\
Da dio uomo si aspetta anche delle risposte \t è un conforto, dove ci sono le risposte che non si trovano nella realta \\
F. ha una passione per la teologia \t l'ha studiata \\
Se si studia dio, si scopre che in realta li c'è l'uomo \\
Studiano dio studio tutto cio che l uomo non è in grado di esprimere nel mondo fisico \\
Teologia diventa antropologia perche io credo di studiare una dimensione ontolgica chiamata dio, ma in realta studio tutte le sofferenze e i dolori dell uomo \\
Il nucleo segreto della teologia è l'antropologia 

\v

Il presupposto di questo discorso è che la filosofia non ha il compito di negare la religione \t ma di indagarla, e facendo cio scopre questa cosa \\
Non è ideologico \t quando la filosofia indaga sulla religione, scopre questa cosa, che dio non ha la sua essenza ontologica ma è un proiezione dell uomo \\
Quindi dio è uomo \t finito e infinito coincidono, ma l'infinito viene annullato dal finito (in hegel al contrario) \\
Al posto di dio noi sostituiamo un altra divinita, che è l uomo materiale, fisico

\v

Questo modifica anche la morale \t non invita + all amore di dio, ma diventa una morale umanista \\
Amare l'uomo, a cui ci si deve rivolgere \t uomo è al centro \t si parla di umanesimo di Feuerbach \\
Uomini non devono essere amici di dio, ma degli uomini \\
Lui sta combattendo la fede cristiana in nome di una dimensione immanentistica, + concreta, che però è ancora metafisica \\
Non sta dicendo che dio non esiste (o almeno non esiste nella propsettiva cristiana) ma c'è un dio che è la proieizone dell'umanita, dio è immanente nella dimensione reale \\
Cio che c e prima deve essere superato, ma non viene svalutato \t la cosa importante è che da li sono parite le domande, da una fede rigorosa e dogmatica \\
Quelle persone si sono poste delle domande, che poi hanno portato a questa dimensione \t non c'è atteggiamento di superiorita in Feuerbach \\
Prima non avevano capito il punto di vista corretto \t che è identitario tra finito e infinito, ma senza dogmatico e trascendente \\
La realta umana non deve essere definita come spirito ma come uomo concreto \t questo è la realta, e si scopre anche che è l'uomo a generare l'infinito \\
Critica a hegel = hai trascurato l'uomo

\s[Marx]
Egli condivide la critica ad Hegel \t anzi fa ciritca + radicale \t lo accusa di ideologia = una presa di posizione che costringe la realta a stare dentro quella visione \\
Hegel ha costruitio lo spirito infinto \t e poi ha fatto in modo che la realta si adattasse a questa teoria \\
Hegel è parito dalla sua visione di realta, ideologica, e ha adatto la realta \t e la ha capovolta \\
L'uomo non puo essere studiato fuori dalla sua realta finita \t io sono quell uomo li in quella storia li, che mi appartengono, perche sto dentro la realta \\
Uomo non è punto di partenza astratto in un percorso infinito \t ma è da considerasi nella sua posizione concreta e storica \\
Altra critica è quella politica \t hegel ha subordinato la societa civile allo stato \t ma è la societa il punto di partenza per lo stato \t prospettiva capovolta \\
Lo stato per h. è una forza superiore che non prende in considerazione gli uomini \\
Inoltre hegel ha descritto lo stato come lo stato prussiano \t errore teoretico, ha unito veritia storica, empirica, non puo diventare verita filosofia \t salto da particolare a unvirsale \\
Cosi come è l'uomo a creare la relgione, è l uomo a generare lo stato \t hegel ha invertito il soggetto particolare e il predicato universale \\
Questo nasce contro hegel, ma a partire da hegel \t si forma a contatto con la filosofia hegeliana \\
Opera + critica \t del 44, "Critica della filosofia del diritto di Hegel" (pubblicato nel 1932)

\v

Marx in realta critica anche i socialisti utopisti, economisti classici, ... \t li critica in un atteggiamento + pragmatico \\
Finora i filosofi hanno descritto il mondo, ora bisogna cambiarlo \t filsofi hanno sempre indicato il punto di arrivo, ma non il mezzo \\
Filosofia ha il compito di agire adesso \\
La sua riflessione filosofico ha aspetto si teoretico, ma anche molto pragmatico (lotta classe, ...)

\ss[Vita]
Nasce nel 18 da una famiglia ebraica \t fra il 16 e il 17 il padre decide di abbandonare l ebraismo per persecuzioni \\
Marx studia legge a Bonn \t ma li non  studia, quindi padre lo manda a berlino \\
Si findanza con Jenny (sua moglie) e parente di Jenny sara anche ministro di Prussia \\
A berlino diventa assiguo frequentatore di "Doctor club" \t di giovani hegeliani, dove conosce esponenti della sinistra \\
Poi si laurea in filosofia con una tesi su democrito ed epicureo \\
Marx poi vuole insegnare a Bonn \t dove insegnava Bruno Bauer, che pero viene allontatno dall universita \t quindi finisce la sua carriera \\
Diventa quindi giornalista \t redattore della "Gazzetta renana", rivista borghese \t ma poi nel 43 viene interdetto (perche le pubblicazioni non piacevano) \\
Nel 44 pubblica critica ad hegel e studia feuerbach \\
A parigi conosce Bakunin, Proudhon, e anche Hengels \t sara il suo grande collaboratore \\
A parigi scrivi sugli annali Franco-tedeschi \t ma poco \t viene finanziato solo dagli amici \\
44: "Manoscritti economico-filosofici" \t poi collabora con giornali comunisti, e viene espulso dalla francia \\
Scappa a Bruxxelles, dove scrive "L'ideologia tedesca" e nel 45 le "Tesi su feuerbach" \\
Nel 47 scrive "Miseria della filosofia: risposta alla filosofia della miseria di Proudhon" \t attacca il socialismo utopistico con quello scientifco che propone lui \\
Rimanein belgio fino al 48, quando pubblica il manifesto \\
Poi torna in germania, poi parigi e alla fine nel 49 arriva a Londra \t viene aiutato da Hengels \\
Qua compie tutte le sue ricerche che vengono pubblicate nel "Capitale" primo volume pubblicato nel 67, poi il secondo nel 85 e terzo nel 94 \\
Nel 59 pubblicato anche "Critica della economia politica" \\
A londra è anche impegnato nell organizzazione del movimento operaio \\
Organizza anche la prima internazionale, che nasce a londra e sciolta nel 70 \\
Pubblica poi "Critica al programma di Gota" \t critca partitio social democratico in germania, nato nel 75 \\
Muore a londra nel 83
\end{document}
