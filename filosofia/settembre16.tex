\documentclass[12pt]{article}

\usepackage[a4paper, total={6in, 8in}]{geometry}
\usepackage{textcomp}

\begin{document}
\setlength{\parindent}{0pt}

\def \t {\textrightarrow}
\def \v {\vspace{1em}}
\def \bi {\begin{itemize}}
\def \ei {\end{itemize}}
\def \s[#1] {\section*{#1}}
\def \ss[#1] {\subsection*{#1}}
\def \sss[#1] {\subsubsection*{#1}}

\s[Fichte]
È caratterizzato da un dinamismo esistenzaiale \t non puo smettere di essere \\
È un attivita dinamica perche è pensiero \t io puro è una realta, una sostanza (non piu una attivita) ma mantiene il dinamismo del pensiero \t non puo smettere di essere perche ha una natura di movimento \\
Quando viene posto in essere non puo fermarsi \\
Si va da piano logico a piano ontologico, ma le caratteristiche le mantiene \\
Anche solo l'autoporsi in essere è un movimento, un' azione

\ss[Primo fondamento: Io puro si autopone in essere]

\ss[Secondo fondamento: l'io puro oppone a se il non io]
Con questo fluire incessante della sua sostanza genera il non io \t l'altro dall' io, ma non in forma sostanziale \\
Io si manifesta in forme diverse, che sono delle determinazioni dell io \t sono fenomenicamente diverse dall'io \\
Ontologicamente sono la stessa cosa, ma fenomenicamente diverse \\
Non io esiste perche si autolimita \\
Tutto cio che esiste sta dento l io puro \\
Non io è una determinazione fenomenica dell io puro \t io puro si ferma, diventa una negazione dell io, gli si oppone fenomenicamente \\
Non io rappresenta anche un assenza di coscenza \\
Io puro mantiene kant, è ragione pura \t si dovrebbe manifestare in tutti gli enti \t ma in realta la ragione si manifesta secondo diversi livelli di coscienza \\
Non io è dove la razionalita non si manifesta (ma è presente), non è consapevole \t nell uomo la razionalita che c e nell io puro si manifesta \\
Nella natura non c e consapevolezza razionale \t ma c'è ragione, uomo e natura sono della stessa sostanza \\
Tutto cio che non è uomo è non io, solo uomo è io puro \t consapevolezza razionale si manifesta solo nell'uomo \\
Enti di natura assumono forme di razionalita diverse a seconda del loro grado di perfezione \t uomo si manifesta sia come essere sia come pensiero \\
Non io non compre l uomo

\ss[Terzo fondamento: all'interno dell io puro si oppongono e si limitano reciprocamente un io empirico e un non io empirico]
Io puro è un pricipio primo incondizionato infinito \t non io è un ente \\
Il non io viene superato, e una volta superato diviene privo di senso \\
La realta non ha un essenza proprio \t tutto è io puro, tutto ha la stessa essenza \\
La finalita del non io è soltanto quella di opporsi all io puro \\
Io puro genera non io, poi lo supera e ne crea altro \\
La funzione del non io è di opposizione all io puro \\
Ma puo esistere una realta priva di senso, la cui ultima funzione è quella di opporsi all io puro, e che una volta generataviene superata? \\
Ma non puo essere, perche tutta la realta è razionale \t non puo generare qualcosa che poi diventa privo di senso \\
Cosi la realta non è autonomo \t la finalita unica della realta è di manifestare l io puro 

\v

Non puo essere cosi \t gli enti generati e superati diventano privi di senso ma mantengono razionalita? \t impossibile \\
Quindi ci deve essere l io empirico \t ha le stesse caratteristiche dell io puro, ma è finito \\
Confronto tra non io e io empirico è sullo stesso piano \t mentre confronto tra non io e io puro non è sullo stesso piano \\
Uomo si confronta all'infinito con la realta \t non c'è mai un superamento definitivo \\
Unico senso della realta è di opporsi all io \t ma se questo io è l io empirico, e quindi l uomo, allora la realta avra senso \t perche si confronta con un io finito \\
Uomo permette alla realta di realizzare la sua essenza \t che è quella di opporsi 

\v

Uomo ed enti si confrontano all infinito in due modi (la modalita in cui l uomo e la realta puo essere di due tipi):
\bi
    \item conoscitivo \t realta è oggetto di conoscenza
    \item morale  \t realta è il mezzo in cui io uomo mi realizzo moralmente
\ei
Quando un uomo muore, ci sara un altro uomo che si confronta con la realta \t all'infinito

\sss[Attivita conoscitiva]
È l'oggetto che definisce il soggetto \\
Quando conosco la realta non scelgo io cosa conoscere, mi adeguo alla realta \\
La conoscenza parte dai sensi e quindi da cosa ho davanti \t sono in un qualche modo definita dall oggetto della sensazione, e quindi della conoscenza \\
Lo scopo della conoscenza è diventare consapevole di diventare io puro \t non lo sa \t ma perche non lo sa, se è razionale? \\
Uomo è la massima manifestazione dell io puro \t come fa a non sapere di essere io puro? \t perche deve farsi la domanda che cosa è, dovrebbe gia saperlo \\
Fichte risponde che l io puro continua a porre in essere se stesso, ma non in modo consapevole \t quindi non c e consapevolezza nell attivita generante \\
E nel generato non c'è consapevolezza, ma c'è razionalita \t nella sostanza c e la ragione \\
L immaginazione produttiva è la ??? \\
Uomo coglie gli enti come altro da se \t attraverso la conoscenza gli permette di emerge di superare questa attivita inconscia \\
Il percorso della conoscenza porta a capire la nostra natura \t alla fine l uomo si riconosce come io puro \\
Quindi il non io è un elemento fondamentale \t il confronto con non io è necessario perche l uomo capisca cosa sia, ovvero io puro
\end{document}
