\documentclass[12pt]{article}

\usepackage[a4paper, total={6in, 8in}]{geometry}
\usepackage{textcomp}

\begin{document}
\setlength{\parindent}{0pt}

\def \t {\textrightarrow}
\def \v {\vspace{1em}}
\def \bi {\begin{itemize}}
\def \ei {\end{itemize}}
\def \s[#1] {\section*{#1}}
\def \ss[#1] {\subsection*{#1}}
\def \sss[#1] {\subsubsection*{#1}}

\s[Kierkegaard]
Filosofo esistenziale \t non esistenzialista \t riflette sull'esistenza \t due elementi:
\bi
    \item dio \t che nasce dal secondo elemento
    \item ovvero la sua biografia
\ei

\ss[Vita]
Nasce dal secondo matrimonio del padre \t morta prima madre, sposa la domestica \t è il secondo di molti fratelli \\
Il padre vive pero una forma di intransigenza religiosa \t che gravera sui figli e sul padre, infatti da ragazzo K. si rende conto che il padre nasconde un segreto \\
La sua anima è attraversata da qualcosa di misterioso \t che il padre vuole espiare con una via molto religiosa \\
Una volta lo trova ubrico che bestemmia \\
Forse vive come una colpa il fatto di essersi rispostato subito \t che grava pero su tutta la famiglia

\v

Quindi K. ha visione opprimente di dio \t che diventa elemento importante del suo pensiero \t dio è l'interlocutore dell'uomo e viceversa \\
Questa è la relazione fondamentale \t e definisce la sua vita \t non è una religiostia serena \\
K. introduce la categoria del singolo \t per stremare l'irrepitibilita dell'uomo, che è individuo \\
Il rapporto con dio è infatti esclusivo, solo mio \t non esiste quindi la categoria teoretica "uomo" in generale, ma sinoglo \\
Il rapporto con dio puo essere solo personale \\
Nel recupero dell individualita \t c'è la critica dell hegelismo \t non entra in merito del suo sistema, perche immanenza (singolo) e trascendenza (dio) sono mischiati \t mentre per K. sono distinti \\
Dio non può coincidere con il singolo \t non si puo guardare il finito con l'occhio dell'infinito e svalutarlo come ha fatto hegel \\
In K. non c'è una visione filosofica a cui si ispira (è isolato) e non fa riflesssione teoretica/gnoseologica \t ma solo esistenziale \\
Inoltre il suo fratello diventa un pastore protestante \\
Poi K. chiede in sposa la regina Holsen \t il grande amore della sua vita, ma in alla fine non la sposa \t questo travaglio influenzera la sua vita

\s[Aut-Aut]
Kierkegaard nel 43 scrivere Aut-Aut \t opera in cui K. da chiave di lettura del rapporto con regina \\
Dice che uomini sono progetto, la loro esistenza trascende il presente \t a partire da cio che sono, sono posto davanti alla scelta di cio che voglio essere \\
Il singolo si trova davanti a scelte esclusive, e questa possibilita dlela scelta è cio che qualifica l'esistenza \t la scelta è pura possibilita di essere, è una categoria umana \\
Uomo è l'unico ente che puo progettare se stesso nelle infinte possibilita, che pero sono esclusive \\
Qui descrive 3 possibilita di vita \t che sono assolute e si escludono \\
Parla di queste 3 scelte, ma in realta la scelta è presenta in ogni istante della vita dell'uomo \t la scelta implica scartare le altre possibilita \\
3 vite:
\bi
    \item vita estetica, del don giovanni
    \item vita etica
    \item vita religiosa
\ei
Sono caratterizzate tutte dalla tensione che l uomo ha verso l'infinito \t uomo è influenzato da questa ricerca, anche se incosapevole, come nel seduttore

\ss[Don giovanni]
È il seduttore \t ha tante donne, di cui non è mai appagato \t il don giovanni non sa che una somma di finiti non puo appagare la sua sete di infinito \\
Nessuna donna sara definitiva \t vuole riempire quella insoddisfazione, tensione, all'infinito che pensa sia data dalla mancanza di infinita di donne \\
Qualitativamente si parla di due ambiti diversi \t molteplicita infinta non da infinito \t infatti il seduttore non è mail felice, cerca qualcosa che non sa cosa è e pensa di poterla colmare cosi \\
Vita estetica è la vita pessima \\

\ss[Vita etica]
Vita dell'uomo sposato \t l'uomo della vita etica sceglie una donna per sempre: a questa donna finita, attribuisce un valore di infinito (ma non la confonde con l'infinito) \\
Non è quindi una ricerca di altre donne \t ma anche questo modello è imperfetto, perche comporta alla rinuncia al vero infinito \\
Migliore della vita estetica \t pero so che si rinuncia al vero infinito, ovvero dio

\ss[Vita religiosa]
È la scelta dell'infinito \t uomo non si sposa, e rinuncia a tutte le donne \t scelgo dio

\v

Queste vite sono aut aut \t e aver sposato regina Holsen significava fare vita etica \\
Ma allo stesso tempo non fa vita religiosa \t perche è innamorato \t quindi non puo vedere il vero infinito \\
Quindi rimane tra le due \\
La sua biografia pesa molto sulla sua vita e filosofia

\ss[Angoscia]
Uomo è unico soggetto della scelta \t ma questo genera il sentimento dell'angoscia \t che poi sarà fondamentale nell'esistenzailismo \\
L'angoscia è il sentimento del possibile \t è ciò che prova l'uomo quando l'uomo si trova davanti a qualcosa di massimamente indefinito \\
La possibilita non è inatto, è solo possibile \t la scelta è mia, ma non è orientata da nulla \t e uno dei rischi è la non scelta, perche il timore della scelta paralizza \\
In heidegger questo indefinito è la morte, mentre in sartre è la liberta/responsabilita \\
Quindi uomo è angosciato, non è felice \t ed angoscia è il sentimento dell'esistenza, non si puo non provare \\
La natura umana fa incontrare l'angoscia \t Heidegger dirà: "l'angoscia in un'esistenza inautentica viene banalizzata in paura" \\
Paura è il sentimento del finito, ed è controllabile \t mentre angoscia non ha un oggetto, non si puo controllare \t non si puo non provare piu, non so identificare qualcosa che genera angoscia \\
Questo perche uomo è progetto, è possibilita di scelta

\v

Esiste però un altro sentimento \t che non provano tutti (mentre angoscia si) ed è la disperazione \\
Disperazione = rifiuto dell'uomo che non si riconosce come dio \t il finito che si allontana dall infinito \\
Uomo disperato è l'uomo che ha creduto di essere dio \t e ha vissuto rifiutando i limiti del finito \t come se vivesse in un eterno presente \\
Sono si è riconosciuto come un ente finito, generato da dio \t pero alla fine uomo muore, è contingente \\
La disperazione nasce dal fatto che l uomo non riesce a dare un senso a qeusto \t perche io devo finire \\
E questa è la malattia mortale \t infatti non guarisce \\
L'uomo di fede vive la speranza, tende verso un altra dimensione \\
Chi rifiuta dio invece non ha questa speranza \t quindi malattia sancisce veramente la fine \\
La disperazione pero non è un sentimento esistenziale \t non è di tutti \t lo prova quando si rapporta con se stesso \\
"Non bisogna giocare al cristianesimo" \t il cristianesimo è La scelta, non una tra le scelte \\
L'essere cristiano qualifica la mia esistenza e il mio rapporto con dio \t gioca al crisitianesimo chi non fa del cristianesimo un attributo essenziale della propria vita \\
Questo atteggiamento di gioco puo sfociare in disperazione \t ma non necessariamente
\end{document}
