\documentclass[12pt]{article}

\usepackage[a4paper, total={6in, 8in}]{geometry}
\usepackage{textcomp}

\begin{document}
\setlength{\parindent}{0pt}

\def \t {\textrightarrow}
\def \v {\vspace{1em}}
\def \bi {\begin{itemize}}
\def \ei {\end{itemize}}
\def \s[#1] {\section*{#1}}
\def \ss[#1] {\subsection*{#1}}
\def \sss[#1] {\subsubsection*{#1}}

\s[Freud]
\ss[Associazionel libera]
Scrive una simbologia di natura sessuale per svelare il significato vero dei sogni \\
L'ipnosi quindi non funziona, e quindi cambia metodo: adotta l'associazione libera \\
Lo psicanalista fa sdraiare il paziente, e lo psicanalista si mette alle spalle \t e il paziente deve dire tutto quello che gli viene in mente, senza intenzionalmente formulare un pensiero \\
Psicanalist aprende appunti di tutto \t anche delle pause, dei momenti di incertezza, etc. \t tutto questo per mettere a nudo quella resistenza che impedisce al paziente di farsi curare \\
Tutto questo dipende dalla predisposizione del paziente, sia dall'abilità del psicanalista, che deve essere un bravo interprete \\
Il paziente racconta anche i sogni \t tutto cio che gli viene in mente \\
Si guarisce quindi rendendo la dimensione inconscia conscia \t si fa emergere a livello di consapevolezza cio che è la causa della malattia \\
A volte pero questo non fa sempre bene \t se l emersione dell'inconscio perggiora lo stato psicologico del paziente, allora bisogna fermarsi \\
Freud pero nota che si instaura tra il paziente e lo psicanalista una connesione \t che chiama transfert \\
Puo essere buono affinche il paziente usa questo legame affettivo (che deve essere misurato) per migliorare la teropia, ma se questo legame va oltre e diventa per il paziente un legame affettivo e di vicinanza amorosa, questo compromette il percorso \\
Ci deve essere una giusta vicinanza tra paziente e psicanalista, ma nella giusta misura \t pero senza transfert non è possibile alcuna analisi \t se io non mi fido, se non mi apro, se no stabilisco relazione \t qualsiasi lavoro è inutile \\
Il transfert funziona quando lo psic. lo rileva, lo fa presente al paziente e il paziente riesce ad elaborare questa relazione nel modo corretto \t ed arriva attraverso questa relazione a rielaborare relazioni passate \\
Transfert è un fenomeno normale ma va guidato \t e cosi permette di risolvere alcune patologie \t va sfruttato nel suo limite, che è la parte + difficile del lavoro dello psicanalista

\ss[Complesso di Edipo]
L'origine di molte malattie secondo Freud puo essere il non superamento del complesso di Edipo \\
Prima analisi: tre parti:
\bi
    \item es \t è l'inconscio, tutto quello che non so di avere dentro di me ed è la sorgente di ogni forma di energia vitale e sessuale
    \item ego \t è la faccia consapevole dell'es
    \item il superego è il censore morale \t si forma nel 5 anno d'età ed è cio che rende l umano diverso dall animale \t è la sede del senso morale, e del senso di colpa \t è qui che si stabilisce cosa è lectio far passare nel conscio
\ei
La dimensione conscia si trova tra l'es e il superego \t cerca di mediare tra queste due dimensioni, che da una parte sono pulsioni egoiste e arazionali, mentre il superego impone le regole morali e impone anche le regole della civilta \t e l'ego deve mediare tra questi due estremi \\
Queste due funzoni psichiche devono essere contrllate e bisogna trovare un equilibrio tra es e superego \t sono guidate da due principi:
\bi
    \item principio di piacere \t è la libido ed è il desiderio di soddisfare in modo totale e immediato un istinto \t insieme di quelle pulsioni che voglino essere soddisfatte in modo completo e immediato
    \item principio di realta \t costringe queste puslioni ad incanalarsi per altre strade, e lo fa nelle vie della civilta, ovvero tutte quelle occupazioni umane (arte, scienza, etc.) \t e cosi la libido trova soddisfazione in altro modo, e in modi che sono socialmente accettabili
\ei
La libido non puo essere cancellata \t ma l'uomo le rende civili, accettabili e le sublima in altri modi \t ma non esmpre queste strade soddsfano il principio di piacere \\
Es. uomo che si soddisfa nell'arte, nella scienza, nella letteratura \t a volte non basta, e quindi quando non trova soddisfacimento trova la nevrosi \t e questo succede anche in persone perfettamente equilibrate

\v

Il superego nasce con il superamento del complesso di Edipo \t si chiama cosi per Edipo che uccide suo padre e sposa sua madre incosapevolmente \\
Nel complesso di Edipo, il bambino entra in competizione con il gentiori dello stesso sesso per l'affetto con il genitore dell'altro sesso \t c'è un attrattiva esercitata dal genitore di sesso opposto, per quell'amore che invece è rivolto verso l'altro genitore \\
Ma il bambino capisce che quell'amore non potra mai essere ricevuto \t è una lotta impari perche non potra mai ricevere anche l'amore sensuale, anche se ha sessualita \t quindi il bambino diventa alleato del genitore dello stesso sesso, perche cosi vuole diventare meritevole anche lui di quell amore \\
In questo modo il bambino interiorizza il modello morale del genitore stesso sesso, e cosi nasce il superego \t interiorizzo anche il suo stare nel mondo, il sui comportamenti morali etc. \\
Il superego nasce come interiorizzazione di quella autorita, ma poi faccio miei altri modelli morali \t che possono essere altri modelli di autorita, che apprezzo e voglio essere come loro \\
Quel modello morale e di autorita che quello propone, diventa anche il mio \t lo "imito" \t autorevole nel senso che ha un modello morale che ritengo valido e mi piace come persona

\v

Il supergo pero puo nascere distorto \t e il complesso d'edipo puo non essere superato \t e da qui nascono diverse malattie: il bambino, nella sua lettura, non sviluppa una sessualita sana \\
Questo puo succedere per esempio quando vivo con un solo genitore, oppure quando il modello del genitore è deplorevole \\
Se non supero il complesso mi ammalo, perche non ho un modello morale di riferimento \t e non ho un censore \\
Freud dice che la sessualita non è solo adulta \t ma nella psicanalisi si fa riferimento a delgi impulsi che devono trovare soddisfacimento \t con sessualita si riconoscono degli impulsi \\
Ci sono 3 fasi nella sessualita del bambino (che hanno a che fare con il principio di piacere):
\bi
    \item fase orale \t percepiscono la realta mettendo le cose in bocca, perche nel farlo provano piacere \t trovano soddisfacimento, e la pulsione al piacere si verifica nel succhiare \t e questo dura il primo anno di vita
    \item fase anale \t dal secondo al terzo anno di vita, in cui il piacere è dato dal defecare 
    \item fase genitale \t nei 4/5 anni il bambino capisce di avere dei genitali, e si scoprono nel loro genere e nasce la consapevolezza della diversita \t e i bambini sono incuriositi da questa cosa
\ei
Nel frattempo c'è il complesso d'edipo \\
Poi c'è un lunghissimo periodo di latenza \t in cui la sessualita si assopisce, fino ad arrivare all'adolescenza \t in cui si entra in una fase di sessualita adulta \\
I sogni degli audlti rimandono spesso a desideri inesauriti della sessualita infantile

\ss[Il disagio della civilta]
Anche la civilta crea una censura \t e infatti scrive questo libro \\
L'uomo è in realta una bestia selvaggia \t c'è una aggressivita di base (homo homini lupus) e si interssa solo di se \t a causa di questa aggressvita la comunita è continuamente minacciata, da questo "zanatos", principio di distruzione/morte \\
Per vivere in comunita, gli uomini devono reprimere questo loro desiderio di distruzione \t e la civilta addomestica questo istinto \\
Freud condanna della civilta l'esasperazione, che ha avuto delle repressioni inutili sull uomo \t la civilta ha costretto l uomo ha repressioni che sono al di la del necessario, e tormento la vita dei singoli \\
Il controllo di questi principi è anche una autotutela \t metto da parte la mia felicita per la comunita \t ma questo è repressione: il principio di piacere (eros, spirito dioniaco) trovano strade nella civita, che possono essere malate \\
Quindi civilta ammala l'uomo, perche non è felice e reprime se stesso \t Freud si riconosce in diverse teorie dei Schopenahuer, ma dice di non essere stato influenzato da nessuno perche lo legge dopo \\
Riconosce delle coincidenza con altri pensatori (anche con marx) ma sono coincidenze, lui non si ispira
\end{document}
