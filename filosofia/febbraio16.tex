\documentclass[12pt]{article}

\usepackage[a4paper, total={6in, 8in}]{geometry}
\usepackage{textcomp}

\begin{document}
\setlength{\parindent}{0pt}

\def \t {\textrightarrow}
\def \v {\vspace{1em}}
\def \bi {\begin{itemize}}
\def \ei {\end{itemize}}
\def \s[#1] {\section*{#1}}
\def \ss[#1] {\subsection*{#1}}
\def \sss[#1] {\subsubsection*{#1}}

\s[Heiddeger]
L'angoscia è il sentimento dell'esistenza autentica \t è il sentimento dell'indefinito, ma in Kierk è davanti alla possibilita, qua davanti alla morte, che è indefinita \\
La morte tradisce la mia essenza, che è progettualita \t mentre in vita autentica l angoscia si banalizza in paura, che diventa il sentimento del definito

\v

Uomo pero puo tornare all'esistenza autentica \t con voce della coscienza, che ci richiama al senso della morte e alla nullita del nostro progetto \\
Questo accade davanti a un trauma, a un lutto \t ci sono dei momenti in cui possono permettere di ritorare all esistenza autentica, con tutta la negativita che porta con se \\
Si chiude con il discorso dell'angoscia la riflessione esistenzaile \t finora abbiamo parlato di essere (opera si chiama "Essere e tempo") \\

\ss[La temporalita]
Gli interessa in quanto carattere costitutivo dell essere \t tempo non è inserito nel tempo, ma è una dimensione che appartiene all uomo \t è una dimensione originaria, gli appartiene da sempre \\
Uomo infatti è essenzialmente futuro \t uomo è progetto, quindi uomo è orientato verso il suo futuro \t uomo è apertura verso cio che puo accadere \\
Il tempo per l uomo è l'originario fuori di se \t ovvero il proiettarsi verso situaizone di partenza \\
Pero io mi progetto a partire da cio che sono stato \t le radici della progettualita stanno nel mio passato, in tutto cio che mi ha condotto all'individuo che sono \\
Il mio progetto inoltre si elabora ora, nel presente \t uomo è fondamentalmente futuro, ma il progetto nasce dal mio passato e viene pensato nel presente \\
Il futuro è la categoria che meglio mi definisce, ma ingloba anche altri due \t uomo è tempo perche fa parte della sua essenza, non si trova nel tempo \\
Lui chiama i tre momenti come "estasi" ovvero l'essere fuori

\v

Anche il tempo ha due dimensioni, una di autenticita e una di inautenticita \\
Il tempo inautentico è pensato dalla preoccupazione per il successo \t progettualita attraversata dal pensiero che devo avere successo, la mia progettualita è tutta orientata verso il mio succesos \\
Mentre in quella autentica c'è la morte \t il mio futuro si qualifica nel mio essere per la morte, che è un freno che impedisce di essere travolto dalle esigenze mondane, dal successo etc. \\
Il passato inautentico è un passato in cui mi uniformo alla tradizione, il mondo del sì \t che è anche il mondo del passato \t io seguo le mie tradizioni, che sono sempre state cosi, allora le seguo perche sono sempre andate bene \\
Il passo autentico invece è una grande occasione \t offre possibilita di attuare potenzialita che non sono state attuate, di far rivivere cio che nella tradizione c'è di buono, e di capire cosa era sbagliato \\
C'è quindi uno sguardo + aperto al passato \t passato mi ha proposto occasioni che non si sono realizzte \\
Presente inautentico è travolto dalle cose del mondo, mentre il presente atutentico è l'istante in cui l uomo decide il suo destino \t è l'istante in cui mi faccio carico di cio che voglio essere 

\v

Quindi pero tutte le determinazioni di tempo della societa e della scienza sono inautentiche \t le date, la misurazione del tempo \t sono tempo che appartengono agli enti, alle cose presenti \t non alla possibilita dell uomo \\
La temporalita che esce dalla dimensione di coscienza è sempre inautentica \\
Inoltre il tempo, se vissuto autenticamente, mi proietta verso la morte \t l'esistenza autentica vede il non valore di tutti i progetti, e quindi la loro equivalenza \\
È un esistenza angosciata, che pero mi fa stare nel mio tempo \t la dimensione dello stare è importante \\
Stare nel mio tempo significa accettare come mio il tempo che sto vivendo, come in una sorta di amor fati \t accetto il tempo che appartiene alla comunita degli uomini, quindi faccio diventare mio un tempo finito \\
Angoscia porta a interiorizzare, a fare propria la mia finitezza, a far proprio il tempo che mi è concesso che è finito \t e porta anche a capire l'equivalenza dei miei progetti

\v

Detto tutto cio, H. si richiede se tutto questo ci ha portato ha capire il senso dell'esere? \t no, non si puo coggliere neanche attraverso l'esserci \\
Questo perche l uomo è ente, e per quantos ia privilegiato, rimane sempre un ente \\
Questa indagine ha rilevato la mancanza di senso di tutti i progetti \t ma non il senso, perche cio che manca all uomo è il linguaggio \\
L uomo parla il linguaggio dell ente perche lui è un ente \\
Nons i puo solo analizzare esserci, perche questo ci porta ad altre riflessioni? \t il senso dell essere non puo essere svelato dall esserci \\
Manca il linguaggio \\
L uomo non è in grado di attingere il senso dell essere perche non ha il linguaggio 

\v

Nel 1930 c'è la chere di H., la svolta, un secondo heidegger che cambia il punto di vista \\
Non sono io esserci che ricerca il senso dell essere, ma cerca dove l'essere si manifesta da solo \t non devo indagare dove si trova, ma devo vedere dov è e custodirlo \\
Non lo cerco io ma è lui che arriva \\
Queste considerazioni le fa in "Introduzione alla metafisica" \t qua critica metafisica classica da Aristotele a Hegel \t perche hanno cercato il senso delle essere indagando gli enti \\
Hanno sbagliato perche la metafisica ha confuso l essere con gli enti \t ha identificato l essere con l oggettivita, con la semplice presenza degli enti \t non è stata una metafisica, ma una fisica mascherata che ha abbaondonato l essere e lo ha travestito \\
L ha talmente allontanato che ci siamo dimenticati di averlo dimenticato \\
Il primo di aver commesso questo è stato platone, responsabile della degradazione di metafisica a fisica \t mentre i presocratici (eraclito, anassimandro, etc.) avevano capito \t loro parlavano di alezeia, ovvero verita, che vuol dire pero svelamento \\
Loro avevano fondato l'essere nella verita e non negli enti \\
Poi platone ha cambiato

\v

Per arrivare al senso metafisico dell essere dobbiamo andare a vedere dove lui ci parla \\
E l essere parla nella poesia \t è come se l'iniziativa passasse all essere, che finora era dell ente (dell uomo) \t ci ha provato facendo molti errori \\
Ora uomo deve essere il pastore dell essere \t deve fare la guardia dell essere \\
Occorre quindi risollevare la filosofia dalla sua dimensione umanistica (uomo al centro, per cui metafisica è diventata fisica) alla sua dimensione misterica \t il mistero dell essere si trova nella poesia \\
Nella poesia il linguaggio si trasforma, perche il linguaggio nella poesia diventa un dono dell'essere \\
Un linguaggio che usa un linguaggio dell'ente, ma quando entra in un discorso poetico dice qualcos altro \t poeta fa emergere nella poesia, il linguaggio dell essere NO \\
L'atteggiamento giusto nei confronti dell essere è il silenzio \t deve recuperare l abbandono \t uomo non deve forzare l essere a svelarsi, ma deve permettergli di riverlarsi in modo libero \\
Il silenzio e l ascolto sono le uniche strade per cogliere il senso dell essere \\
Le parole che uso in una poesia, perdono il loro significato contingente e vanno a dire qualcosa d altro

\v

La metafisica occidentale è un grande fallimento \t i pensatori essenziali sono i presocratici, loro hanno ascoltato la voce delle ssere \\
La metafisica occ. ha voluto essere padrone dell'essere \t ha voluto far dire alla realta cio che voleva lui \\
Questo è stato aggravato dal primato della tecnica

\ss[Primato della tecnica]
La tecnica non è uno strumento neutrale \t uomo puo usarla per il bene o per il mael \t e la tecnica è l esito scontato dello sviluppo dell umanita che, dimenticando l essere, si è lasciata travolgere dalle cose \\
La realta è diventa qualcosa da dominare, non da ascoltare \t c'è una fede pazza nella tecnica, ma questo si ritorcera nella vita stessa dell uomo \t uomo ha perso il senso dell essere, il fondamento della realta \\
Se vivo una vita che dimentica l essere, minaccero la mia stessa vita \t pensero di poter dominare l essere, e il mistero della perdita dell essere è che nessuono si è piu chiesto il senso profondo della realta \\
Se si perde il senso dell essere, si perde anche il senso dell ente \t perche è contigente, e se una cosa puo essere e non essere non ha senso \t a meno che io non riconduco il fatto che alla radice dell ente c'è l essere
\end{document}
