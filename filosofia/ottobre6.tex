\documentclass[12pt]{article}

\usepackage[a4paper, total={6in, 8in}]{geometry}
\usepackage{textcomp}

\begin{document}
\setlength{\parindent}{0pt}

\def \t {\textrightarrow}
\def \v {\vspace{1em}}
\def \bi {\begin{itemize}}
\def \ei {\end{itemize}}
\def \s[#1] {\section*{#1}}
\def \ss[#1] {\subsection*{#1}}
\def \sss[#1] {\subsubsection*{#1}}

\s[Fenomenologia]
Si parte da una non conosco la realta \\
La rivoluzione copernicana non c'è \t è l'intelleto che astrae il concetto universale, nel senso aristoteliano \t realta dipende dal conoscenza in queto senso \\
L'autoconoscenza acquisisce la conoscenza dell'unita tra finito e infinito, ma poi la va a verificare \\
L auto. capsice anche l unita tra pensiero ed essere \t non c'è differenza tra soggetto e oggetto, ... \t nella ragione pero vado a verificare \\
La fine della tappa dell'autoconoscneza \t si passa alla tappa della ragione \t confermo l'intuizione che ho avuto nella autoconoscenza \\
La ragione è la sintesi \t recupera quindi, e ha 3 tappe (con onguna sotto triadi):

\ss[La ragione che osserva la natura]
Prima di tutto auto. osserva la natura (come tappa della coscienza, uomo osserva la nautra per confermare identita ontologica tra me e la realta \t ma non la trovo) \\
La nuova consapevolezza, rispetto alla coscienza, è che la natura è razionale e puo essere conosciuta da lei \\
Tappa necessaria, ma non sufficiente \t non ho conferma

\ss[La ragione che agisce moralmente]
Momento + attivo \t antitesi perche nega la passivita della tesi \\
Ripropone la tappa dell autocoscienza (a livello piu alto) \t la ragione va alla ricerca della conferma della sua idea \\
Va alla ricerca agendo nel mondo come individuo \t tento di affermare la mia individualita \\
Poi cerco di superare dimensione di individualita per raggiungere una "pseudo-universalita" \\
Quando uomo agisce cerca la sua felicita, il godimento nello stare nel mondo \t ma uomo che agisce cosi non si trova \t il piacere è transitorio, non mi garantisce una certezza definitiva \\
Provo ad affermare me stesso perche voglio essere felice \t ma non funziona \\
Quindi cambio punto di vista \t non sto + nel mondo volendo affermare me, ma seguo la legge del cuore \\
Legge del cuore = non voglio cercare il mio bene, ma quello dell universalita \t mi apro al mondo, legge universale nel singolare \\
Ma è contradditoria \t non so cosa è il bene dell umanita \t penso io cosa sia in modo soggettivo, non mi aspiro a una legge universale condivisa \\
Voglio render quello che è per me il bene dell umanita universale in modo immediato \t passagio fallimentare: passo dal particolare (io) all universale (legge del cuore) senza mediazione \t passaggio troppo immediato, serve mediazione \\
Se voglio bene dell umanita entro in conflitto con altre autocoscienze

\v

Emerge quindi l'idea di virtù \t gli uomini cercano legge universale che possa stabilire legame e regola tra di loro \\
Nasce quindi l'idea di virtu, l'idea che esistano delle azioni virtuose e visiziose \t nasce solo il concetto \\
Bisogna pero dare a questa virtu un contenuto \t non si puo dire che ci si deve orientare secondo la virtu, ma se questa idea non viene calata nella realta virtu non vuol dire niente \\
Virtu cerca di mediare tra individualita a universalita \t ma non risolve conflitto \\
Sto sempre cercando di verificare che io sono spirito \\

\ss[La ragione che acquisisce la certezza di essere spirito]
Cosa vuol dire virtu quindi? \t uomo ha provato a dare dei contenuti
\bi
    \item primo contenuto: virtu è una regola (ragione legislativa) \t ma è ancora individuale (Kant, regola con contenuto è ancora particolare)
    \item Kant: devi perche devi, non posso definire con degli imperativi, regola deve essere solo formale senza contenuto \t ma rimane astratto per Hegel, non risolve nulla \t Kant non era arrivato alle sue estreme conseguenze del suo sistema \t è caduto in contraddizione perchè non era idealista
    \item hegel: perche non si riesce a trovare un concetto di virtu universale? \t uomo ha sempre sbagliato perche ci aspettiamo che il contenuto venga dall esterno, e poi uomo si deve uniformare a questo contentuo \t soluzione è stare nello spirito \t la regola che devo seguire è quella in cui sono immerso \t l'ethos della societa in cui vivo, se cerco delle leggi morali che vengono da qualche parte non funziona \t devo cercare le regole nel contesto dove vivo, perchè sono spirito \t devo solo seguire il percorso dello spirito (perche io individuo non sono nulla)
\ei
Se sei immerso nel contesto con cui vivi e assecondi questo percorso, stai assecondando lo spirito \t non devo cercare niente che sia altro, la morale è gia stare dove devi stare e fare quello che devi fare \\
Si ha quindi la certezza di essere spirito \t l'unico modo per avere un concetto di virtu e di arrivare alla conclusione che io, uomo, appartengo allo spirito \\
Ottengo la conferma di essere spirito quando capisco di essere immerso nella sostanza etica dello spirito \t la virtu è gia qui

\v

Questo percorso ovviamente l'ha fatto anche lo spirito \t anche se coincide con l uomo \t triade dal punto di vista dello spirito:

\ss[Spirito]
Lo spirito sa di essere spirito \t ma riflette su di se, partendo dalla sua struttura morale \\
Spirito si è gia conosciuto come unita di pensiero ed essere \t sa gia chi è, quindi segue strada morale \\
Inaugura questa strada come eticita \t ed arriva alla perfetta autoconsapevolezza di se, attraverso l'oggetto intermedio che è l'uomo \\

\sss[Spirito in se come eticita]
Nella polis greca si realizzava la bella vita etica \t la vita di armonia in cui gli uomini stavano dentro \\
Non esisteva una dimensione metafisica, di un dio esterno che comandava \\
Ma la loro vita è immediata (e quindi instabile) \t non avevano elaborato una riflessione sulla loro vita, non avevano capito perche si doveva vivere cosi \\
Non avevano dato un fondamento a questa vita \\
La bve viene spezzata per esempio nell Antigone \t antigone secondo la legge umana (di creonte) non doveva seppellire il fratello, perche era stato ucciso fuori dalla citta (more fuori citta no sepolutra per creonte) \\
Antigone pero seppellisce il fratello, seguendo la legge divina \t afferma legge divina contro quella umana \t ma entrambi vengono puniti \\
Antigone viene uccisa, ma anche creonte viene punito, perche il figlio si uccide \\
Entrambi sono stati puniti perche avevano creato una scissione nell unita dello spirito \t sono entrati in conflitto perche volevano seguire delle leggi che non stavano dentro la vita dello spirito \\
Antigone rompe l'armonia \t contrappone le due leggi, ma contrapposizione fallisce perche vengono puniti entrambi

\sss[Spirito che esce fuori di se]
Ma antigone rappresenta questa messa in discussione \t che porta a un nuovo nella sostanza etica dello spirito ??? \\
L individuo si allontana dalla sostanza etica \t e allontanamento raggiunge il culime con la persona guiridica (durante l'impero romano) \\
Ogni persona che vive nell impero viene riconosciuto come persona giuridica \t ma lacerazione: io che vivo dentro l'imperoma sono persiano, ma vengo riconosciuto dal punto di vista giuridico al popolo romano \\
La nascita della persona giuridica rapprensenta l'esaltazione dell individuo \t io non sto dentro lo spirito, posso essere cittadino di qualsiasi stato \t posso vivere in qualsiare realta, anche diversa dalla mia \\
L'uomo si è staccato dal suo contesto etico \\
Il passaggio successivo è del dittatore (del cesare) \t individuo si stacca talmente da sostanza etica che vive in una condizione in cui ottiene diritti che non gli appartengono \\
Tuttti hanno gli stessi \t cosi nessuno ne ha piu \t caos e quindi sorge un cesare che esercita la sua individualita sugli altri \\
Il tiranno diventa colui che impone la regola, l'eticita?? \t massima scissione tra sostanza etica dello spirito e individuo

\v

Culime della scissione pero in europa moderna \t uomo si è rivolto principalemtne alla conquista del potere \\
Europa moderna è antitesi \t tre tappe \\
Momento della cultura \t cultura fatua, che in nome di dei valori artificiali allontana l uomo dall eticita dello spirito, cultura che non veicola veri valori \t a cavallo dell eta moderna c'era contrapposzione tra cultura fatura e fede \\
Illuminismo segna punto di non ritorno \t non rinnega illuminsmo perche fa parte dello spirito (fa rivalutare la ragione) ma ha commesso l errore che ha appittito tutte le cose \\
Ill. ha sostituito il criterio dell utile per ogni altro valore \t illumismo parla della propria ragione di io uomo singolo, rappresenta quindi la presunzione dell uomo \t esaltazione del singolo \\
Illuminismo degenera nel terrore \t che è il fallimento del percorso illuminista \\
Rivoluzione francese è importante per hefel \t ma ne vede il limite, nel terrore \t riv. distrugge valori sbalgiati, ma poi quando ha finito distrugge anche i propri valori \\
Allontanamento dallo spirito crea sempre piu problemi

\sss[Spirito che ritorna in se nella moralita]
Poi spirito rientra in se \t cerca di darsi dei valori, nel momento della sintesi \\
Eticita è la tesi (i valori che regolano le relazioni), mentre moralita è una legge \t l'ethos (la cultura, le tradizioni) di un popolo non è la legge morale \\
Ritorna in se nella moralita (dandosi una regola, morale) \\
Ritorna Kant \t ha avuto il merito di essere stato il primo a cercare regola universale \t ma mancano i contenuti, troppo astratto \\
Poi reintroduce i contenuti nella tipica del giudizio \t entra in contraddizione, alla fine un contenuto è necessario \\
Hegel elabora la figura della coscienziosità \t che risolve conflitto tra legge formale e contenuto \\
Legge morale mi comanda di fare concretamente ciò che è giusto \t conscienziosita è conoscenza effettuale del mio dovere \\
Nella concretezza il mio dovere si introduce in delle azioni \t l'uomo capisce che c'è un dovere (kant), ma non è astratto: è un dovere che porta dentro il suo ethos, nella realta in cui vive \\
Grecia \t non mi ero posto il problema \t passo attraverso illuminismo \t trovo la legge morale che è devi perche devi, ma mi riporta nella condizione di eticita \\
Ognuno deve ritrovare il devi perche devi nella sua vita etica \t il devi perche devi deve guidare lo spirito \\
Il criterio morale unico è il dovere \\
Greci vivevano quello che c'era \t non avevano un elaborazione consapevole, neanche un dio che imponesse una morale \\
Come sto in questa sostanza etica? \t c'è una regola: il devi perche devi, che pero calo dentro la mia storia (ovvero la sostanza etica dello spirito in cui si è immersi) \\
Recupera il positivo della tesi (che sto dentro la mia realta) ma seguendo una legge morale universale (se non la ho, succede la lacerazione \t individuo si separa dallo spirito è conseguenze storiche, come terrore) \\
I contenuti sono gia nella sostanza etica \t manca solo di dare una regola a questi contenuti \t c'è identita tra pensiero ed essere \t il devi perche devi funziona solo nell idealismo, altrimenti contraddizione \\
Individuo è gia spirito \t ed è manifestazione necessaria dello psirito \t come rispetti questa tua manfiestazione? vivi con la consapevolezza di avere un dovere, ovvero realizzare la vita dello spirito \\
Come la realizzi? \t stando dentro e vivendo lo spirito

\v

La coscienziosita è questa mediazione tra la legge morale e la sostanza etica dello spirito \\
Lo spirito ha capito che deve stare dentro e l individuo anche \\
Manacno pero ancora due tappe: religione e sapere assoluto \t lo spirito ha acquisito consapevolezza morale, ma si deve autoriconoscenre anche dal punto di vista teoretico

\ss[Religione]
Tappa intermedia \t è una forma di conoscenza dello spirito \\
L'uomo che è spirito è lo strumento attraverso il quale lo spirito si autoconosce \\
Si conosce da un punto di vista imperfetto \t mi fa vedere l'assoluto, ma ci credo \\
Ho un aggancio con l assoluto che non è concettuale \t ma di fede \\
Per arrivare a vera autoconoscenza di se ha bisogno di ragione \\
Religione orientale, greca, cristiana (che è il vertice) \\
Descrive il punto di arrivo come il contesto in cui vive \t il culmine dello spirito è l'hegelismo \t pero non dice che ci sono tappe successive

\ss[Sapere assoluto]
Attraverso la ragione che si esprime in logica, filosfia della natura, filosofia dello spirito ??? \\
La conoscenza dello spirito deve essere costruita \t lo faccio razionalmente attraverso un sistema della scienza che si articole in (triade detta sopra) 

\v

Fenomenologia \t uomo parte da nulla, realta sembra essera altro \\
Poi capisce che realta dipende da me per conoscenza e capisco di essere sipirito \\
Spirito si conosce con la storia umana \t la legge morale deve gia essere dentro lo spirito \t religione c'è pseudoconoscenza e poi nel sapere assoluto perfetta autoconoscenza
\end{document}
