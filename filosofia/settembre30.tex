\documentclass[12pt]{article}

\usepackage[a4paper, total={6in, 8in}]{geometry}
\usepackage{textcomp}

\begin{document}
\setlength{\parindent}{0pt}

\def \t {\textrightarrow}
\def \v {\vspace{1em}}
\def \bi {\begin{itemize}}
\def \ei {\end{itemize}}
\def \s[#1] {\section*{#1}}
\def \ss[#1] {\subsection*{#1}}
\def \sss[#1] {\subsubsection*{#1}}

\s[Fenomenologia]
Due percorsi \t io mi devo porre dal punto di vista dell'assoluto \t ma così anche lo spirito arriva a una perfetta autocooscienza di se \\
Percorso individuale dell uomo, ma anche universale dello spirito attraverso l'umanità \\
Singolo uomo \t prima triade, storia dell'umanità \t seconda triade \\

\ss[Prima triade]
Si parte da nessuna consapevolezza di essere spritio \t coscienza = io \\
Primo approccio con la realta è conoscitivo \t la realta è tutt'altro rispetto a me \\
La coscienza si avvicina gnoseologicamente alla realta \t dalla coscienza si arriva alla nuova triade:
\bi
    \item certezza sensibile
    \item percezione
    \item intelletto
\ei
Rappresenta la triade della coscienza, che è la tesi della grande triade dell'uomo 

\sss[Certezza sensibile]
L'uomo si avvicina all oggetto attraverso le sensazioni \t colgo aspetti particolari dell oggetto con sensazioni \\
Colgo una molteplicita di aspetti \t che non da la conoscenza dell ente pero

\sss[Percezione]
Passo dal particolare al generale \t nel momento della percezione percepisco l'oggetto nella sua unita \\
Percepisco l'unità di fondo dell'ente \\
Ma colgo anche una contraddizione \t non so definire l'essenza dell oggetto, ma questo oggetto si presenta ai miei sensi come uno e molteplice = contraddittorio \\
Io colgo ente come unitario, ma io colgo la molteplicita dei suoi aspetti contingenti 

\sss[Intelletto]
Contraddizione viene risolta quando interviene l'intelletto \t l'oggetto viene colto nella sua unita dall intelletto, mentre i sensi continuano a fornirmi senzazioni molteplici \\
Intelletto e sensazioni possono stare insieme perche appartengono a piani diversi di conoscenza \t mentre percezione e sensazione stanno sullo stesso livello \\
Per risolvere contraddizione della percezione serve salto di qualità \t rispondo alla domanda che cosa è

\v

Il punto di arrivo della coscienza è l'intelletto \t così ha acquisito una prima consapevolezza \t colgie un prima relazione con l'oggetto, di caratter gnoseologico \\
La relazione è che quel banco per essere conosciuto ha bisogno del mio intelletto \\
Cosi io acquisto una sorta di consapevolezza \t si diventa autocoscienza \t la coscienza impara qualcosa di se stessa \\
La coscienza diventa supponente \t si sente superiore \t se realta ha bisogno di me per essere consociuta, l autocoscienza è superiore, dominante \\
La realta ha bisogno di me, a me non serve \\
Autocoscienza va nel mondo con un atteggiamento altero \t ma autocoscienze ce ne sono molte nella realta \t che hanno questo atteggiamento nei miei confronti \\
Si genera uno scontro \t che porta a capire che ogni autocoscienza ha bisogno delle altre per autorealizzarsi \\
Lo scontro permette di riconoscersi in una relazione con le altre autocoscienze \t il mio interesse non sara quindi di elimarle, ma di sottometterle \t io ho bisogno delle altre \\
Mantengo condizione di alterigia

\ss[Dialettica servo padrone]
Nella tappa dell'autocoscienza non esiste triade \t perchè è l'antitesi, ci sono solo tappe \\
Dialettica servo-padrone \t ma in questo caso vuol dire rapporto, ed è un racconto a se stante \t è una figura della fenomenologia dello spirito \\
Racconta come gli uomini si relazionano tra uomini \\
Servo e padrone sono diventare tali perche:
\bi
    \item padrone ha rischiato la vita nello scontro delle autoconoscienze, e  ha vinto
    \item pur di aver salva la vita ha accettato la condizione di schiavitu
\ei
Servo ha svilito la sua soggettivita e si è messo a disposizione \t ma padrone gli fa fare tutte cose che lui non vuole fare \\
Questo pero porta a rovesciamento delle parti \t servo impara cio che padrone sa fare, ma lui smette di saperle \t padrone dipendera dal servo \\
Ma servo sara indipendente \\
Ma padrone non si potra realizzarsi come autocoscienza \t perche non riconosce il servo come un altra autocoscienza, ma solo come un ogetto \\
Ma il servo continua a vedere nel padrone un uomo \t sara quindi indipendente, ma si puo anche realizzare come autocoscienza \\
Autocoscienza si realizza nella sua relazione con le altro autoscienze \t uomo si realizza con altri uomini con cui avere uno scambio, non con altri enti \\
È diventato famoso \t mette alla luce la potenza assoluta del lavoro \t il servo lavora, all'inizio sembra schiacciato ma poi ne esce vincente \\

\v

Autocoscienze sono superbe e si scontrano \t entrano in relazione di sottomissione/padronanza \\
Ma qual'è la finalita della tappa della autocoscienza? \t serve a farle capire cosa essa sia \\
Si chiede cosa vuol dire essere una autocoscienza \t tende a autoconsapevolezza \\
Per fare ciò attraversa 3 tappe:
\bi
    \item stoicismo \t scuola ellenistica
    \item scetticismo \t scuola ellenistica
    \item coscienza infelice \t coscienza del cristianesimo medievale
\ei
Sono tappe dell uomo singolo, ma anche passaggio storici

\ss[Stoicismo]
Io mi pongo al di sopra della realta \t atteggiamento di totale superiorita che disprezzo la realta \\
Pero questo att. allontana l uomo dalla vita \t la liberta dello stoicismo è astratta \t non esercito la mia liberta in un confronto con gli altri \t mi ritiro \\
Piu sono indifferente verso la realta \t piu mi isolo e mi allontano dalla realta \t questo mi porta ad un atteggiamento di negazione verso la realta 

\ss[Scetticismo]
È un passaggio necessario \t lo stoicismo non puo che degenerare nello scettiscismo, in cui nego la realta \\
Passo da atteggiamento di indifferenza verso la realta a negazione \\
Metto in discussione tutto \t ma questo genera contraddizione: nego cio che faccio \\
La coscienza non puo smettere di essere cio che è \t essa pensa, ma nega tutto, quindi anche il pensiero \t contraddizione \\
Att. scettico genera la negazione di tutto cio che mi fa stare nel mondo \t vivere nel mondo vuol dire pensare, avere atteggiamenti, che dovrei negare secondo la mia posizione scettica \\
Sono scettico ma continuo a vivere nella realta \t contraddizione tra cio che vivo e l'atteggiamento che assumo nei confronti del mondo \\

\ss[Coscienza infelice]
Si genera lacerazione tra io che sto nel mondo e i valori che vado a negare, che guidano il mio stare nel mondo \\
Nello scetticismo lacerazione non è ancora esplosa \t nasce qua \\
La coscienza infelice è il carattere dominante del cristianesimo medievale \t nasce da scissione tra realta mutevole (che non ha certezze) e l aspetto di infinito (che è dio) \\
È infelice perche la coscienza non vive piu nessuna dimensione \t il finito è negativo, quindi devo tendere all'infinito \t non vivo la dimensione che mi appartiene ma neache quella cui tendo, perche e irraggiungibile \\
La coscienza è massimamente infelice perche non vive piu da nessuna parte \\
Progressiva accentuazione della lacerazione

\ss[Ragione]
Scissione, distacco viene superata dall autocoscienza con la ragione \t in realta scissione tra finito e infinito non c'è \\
Perche quell infinito sta dentro di me \t colgo l'identita ontologica con lo spirito \\
Cosi intuisco la possibilita di questa identita \\
Devo pero verificare questo con la ragione \t vado a capire che oggetto+soggetto=spirito \\
Nella tappa dell autocoscienza non ho ancora sviluppato un argomento razionale \t nella ragione comprendo con strumenti logico/razionali
\end{document}
