\documentclass[12pt]{article}

\usepackage[a4paper, total={6in, 8in}]{geometry}
\usepackage{textcomp}

\begin{document}
\setlength{\parindent}{0pt}

\def \t {\textrightarrow}
\def \v {\vspace{1em}}
\def \bi {\begin{itemize}}
\def \ei {\end{itemize}}
\def \s[#1] {\section*{#1}}
\def \ss[#1] {\subsection*{#1}}
\def \sss[#1] {\subsubsection*{#1}}

\s[Freud]
Non è un filosofo ma un medico \t pero sue scoperte hanno ricaduta sul pensiero \\
La psicanalisi è una scienza nuova, che viene creata da lui (cosi dice) ed è una scienza che è stata molto osteggiata \\
Nonostante questi ostacoli la psicanalisi si è affermata \t perchè tocca tutti i fatti umani \\
La psicanalisi utilizza ciò che la tradizione vede come peccaminoso per spiegare la psiche \\
La psicanalisi spiega non solo la patologia, ma anche i comportamenti umani \t ricorrendo a una visione per lo più di natura sessuale \\
La psicanalisi non è solo freudiana, ma c'è anche di Jung, di Adler \t ci sono diverse letture, questa è quella di Freud

\v

Freud nasce  da una famiglia ebrea e si laurea a Vienna in medicina \\
Sutia anatomia celebrale, e per guadagnare si interessa alle malattie nervose \t questo perche nel 25 scrive "La mia vita e la psicanalisi" \\
Studiano le malattie nervose conosce Charcue, il quale era famoso e stava studiano le forme isteriche \t Freud si trasferisce a Parigi apposta per conoscerlo \\
Qua si avviciana alla usa idea \t il malato di isteria puo tornare a uno stato di normalita, perche l'isteria si puo risolvere con l'ipnosi \\
Charcue riteneva che fosse possibile anche il contrario \t con l'ipnosi si potesse creare una cirsi isterica \\
Freud si appassiona allo studio dell'ipnsoi \t e nl 89 lascia Parigi e va a Nancy, perchè qui c'era Bernhein \t medico che stava facendo degli esperimenti su soggetti malati \\
Stava ipnotizzando soggetti malati \t e sotto ipnosi gli diceva di fare una certa cosa, che poi veniva effettivamente fatta \\
Alla domanda perche hai compiuto questa azione, il soggetto non sapeva rispondere \t perche non veniva dalla sua volonta \\
Poi torna a vienna, e scrive su un caso di isteria che aveva curato qualche anno prima \t pero si rivela che queta ipnosi è molto limitante \\
Inizialmente era a favore di questa pratica \t ma poi si rende conto che l'ipnosi è limitata \t il soggetto sotto ipnosi rivela quello che ha dentro di se e abbandona i freni inibitori, ma poi si incontra una resistenza \\
Esiste una resistenza evidente, oltre la quale l'ipnosi non puo andare \t per risolvere veramente la malattia bisogna far emergere cio che sta dietro \\
Serve un altro metodo, e teorizza la teoria della rimozione \t segna l'inizio della psicanalisi, con una nuova indagine

\ss[La teoria della rimozione]
All'interno dell'individuo c'è un blocco \t l'io cosciente blocca una serie di impulsi, ovvero nega l'accesso alla coscienza, che vengono rimossi nella parte cosciente della psiche \\
Non viene represso (sono consapevole di impulso, che reprimo) \t rimozione vuol dire cancellazione e io non ho consapevolezza che questi impulsi sono nella mia psiche \\
Questi impulsi sono solo di natura sessuale \t all'inizio non se ne accorge, ma poi con un analisi di casi concreti \\
Vengono rimossi perche ho una censura interna, ma anche esterna che io subisco \\
Questi impulsi che mando via pero stanno li e cercano di riemergere \t per esempio nei sogni o in alcuni atteggiamenti/azioni quotidiane in cui allento il controllo \\
Laddove questi impulsi nascono da traumi etc. \t questi impusli sono cosi rimossi e cosi gravi, nasce la malattia \t se non sono cosi importanti riemergono solo nella nostra vita quotidiana \\
Lui scirve "Psicolopatologia nella vita quotidiana" \t qui abbandona la tecnica ipnotica e individua altri metodi

\ss[Gli atti mancati]
Questi impulsi rimossi si manifestano nella vita quotidiana dove? \t per esempio negli atti mancati \\
Sono tutti atti inconsapevoli che pero possono essere letti psicanaliticamente \t es. rompo una cosa che sia molto preziosa \t magari l'aveva regalato una persona che non mi piaceva, ma l'avevo cancellato \\
Appena ho potuto l'ho rotto incosapevolmente \\
L'inconscio cerca di venire fuori in questo modo \t anche per esempio con i lapsus: per un istante, senza averne percezione, la mia dimensione di controllo viene a meno \t e dico qualcosa che in quel momento non doveva uscire \\
Lapsus non è mi sono confuso, ma un momento in cui l'inconscio viene fuori

\ss[Interpetazione dei sogni]
L'inconsco si manifesta nei sogni \t scrive nel 99 "L'interpretazione dei sogni" \t e fa diventare il sogno uno strumento della scienza \\
Nel sogno il controllo è inferiore \t il controllo della dimensione conscia si allenta \\
Due contenuti:
\bi
    \item latente: il vero significato del sogno \t quello che il sogno vuol dire \t che pero magari non è accettabile moralmente o socialmente
    \item manifesto: il sogno subisce una censura onirica \t che rappresenta quello che mi ricordo
\ei
Censura avviene con delle trasformazioni \t per cui sogno delle cose strane, che non accadono nella realta \\
Censura avviene perche il contenuto del sogno non è accettabile \t spesso è sessula e perverso
\end{document}
