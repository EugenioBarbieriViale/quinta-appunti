\documentclass[12pt]{article}

\usepackage[a4paper, total={6in, 8in}]{geometry}
\usepackage{textcomp}

\begin{document}
\setlength{\parindent}{0pt}

\def \t {\textrightarrow}
\def \v {\vspace{1em}}
\def \bi {\begin{itemize}}
\def \ei {\end{itemize}}
\def \s[#1] {\section*{#1}}
\def \ss[#1] {\subsection*{#1}}
\def \sss[#1] {\subsubsection*{#1}}

\s[Lipidi]
Alifatico = catena di carboni legati da legami singoli \\
Triglicerdi sono composti da tre acidi grassi e un glicerolo \t e le catene di acidi grassi possono essere saturi o insaturi \\
Servono come riserva energetica

\ss[Fosfogliceride]
Un fosfogliceride \t glicerolo + 2 acidi grassi + fruppo fosfato \t il gruppo fosfato ha sempre una funzione energetica \\
ATP \t prodotta dai mitocondri ed è il risultato della respirazione cellulare \\
Il glicerolo esterifica \t ovvero forma un estere (gruppo funzionale O=C-O) \\
Testa idrofila \\
I fosfogliceridi formano le membrane cellulari \t è una molecola anfipatica, con una parte idrosolubile e una parte liposolubile ? \t ha una parte idrofoba e una idrofila

\ss[Terpeni]
Butadiene \t molecola con 4 carboni con due doppi legami \\
Terpeni sono isopropeni (2-metil-1,3 butadiene) \\
Ha una spiccata tendenza alla polimerizzazione \\
Non è saponificabile \t per saponificazione si ha bisogno di una base forte, e lui non ha una natura acida \t non puo reagire con una base forte \\

\ss[Colesterolo]
Ha 27 atomi di carbonio \t ed è una catena alchilica (ovvero senza doppi legami, alchili sono i gruppi funzionali che derivano dagli alcani) \\
Il colesterolo viene prodotto dal fegato \t e ne produce la quantita necessaria \\
Il colesterolo pero si introduce anche con la dieta \t e se ne assumo in eccesso, si creano problemi \\
Il colesterolo sballato puo esserci anche per disfunzioni epatiche \\
È importante perche se si mettono tra le code fosfolipidche cambiano la permeabilita della membrana \t influenzano quindi la fluidita delle membrane cellulari \\
Ne regolano questo aspetto \t membrane hanno diversi compiti: in alcune cellule devono essere + forti, in altre devono far passare molecole \t colesterolo regola la fluidita \\
Regola anche la temperatura corporea \t a temperatura ambiente (è stabile e ha cuore ciclico, è solido quindi) garantisce compattezza \\
A basse temperatura impedisce alla membrana di cristallizzarsi \t a basse temp. c'è il rischio che si irrigidisca, ma colesterolo previene questo \\
Colesterolo è importante anche per gli steroidi \t la vitaima D (composto che non puo essere sintentizzato) si produce con il sole \\
La bile \t nel fegato permette depurazione

\v

Se voglio digerire un grasso, lo emulsiono e cosi aumento superifcie di contatto \t cosi enizimi agiscono + efficientemente \t i sali biliari fanno questo
\end{document}
