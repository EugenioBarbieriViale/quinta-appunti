\documentclass[12pt]{article}

\usepackage[a4paper, total={6in, 8in}]{geometry}
\usepackage{textcomp}

\begin{document}
\setlength{\parindent}{0pt}

\def \t {\textrightarrow}
\def \v {\vspace{1em}}
\def \bi {\begin{itemize}}
\def \ei {\end{itemize}}
\def \s[#1] {\section*{#1}}
\def \ss[#1] {\subsection*{#1}}
\def \sss[#1] {\subsubsection*{#1}}

\s[Bo]
Isomeria ottica \t la molecola viene deviata in maniera diversa, perchè c'è un centro nella molecola \t un carbonio \\
Chiralita \t simmetria ma non sovrapponibilita \t puo accadere solo quando ho un carbonio legato a 4 costituenti diversi \t se solo due, allora si ha simmetria \\
La proiezione di Fischer \t quando si parla della enantiomeria si usano per indicare i 4 sostituenti diversi \\
Carbonio C1 è il carbonio + ossidato (ovvero ha carica parziale + positiva) \t quindi quello con il gruppo carbonilico e si mette in alto

\v

Reazione di ciclizzazione \t prevale la parte ciclica in soluzione acquosa, ed è la + stabile \\
Per metabolizzare il glucosio ho bisogno però la forma aperta \t e questa è la caratteristica importante del glucosio, e perche gli zuccheri sono usate per l'energia \\
Il D-glucosio è quello che metabolizziamo \t il L-glucosio invece no \t i nostri enzimi si attaccano al D

\ss[Formule di proiezione di Haworth]
Maggiore spessor eai legami che sono frontali \t si vuole far vedere la molecola ciclica inclinata \\
C1 e C5 etc. \t sono i legami che formano la catena, dei carboni ???

\v

Il carbonio anomerico è il carbonio che prima era aldedico mentre ora con la ciclizzazione si chiude e diventa stereogenico \t genera una stereoisomeria \\
I due isomeri si chiamano anomeri: alpha-d-glucosio e beta-d-glucosio

\v

I legami tra i monomeri sono da sapere \t nei casi dei monosaccaridi studiamo carboni aldosi e chetosi \\
Quando abbiamo polimeri, parliamo di oligosaccaridi (hanno pochi monomeri, sotto 10 circa) \\
Lattosio è un disaccaride \t composto da glucosio e galattosio \t vengono separati in due, e ci deve essere una catena che mi permette di tramutare ?? \\
Uomo si è evoluto per digerire il latte \t in realta grazie alla simbiosi con una flora batterica \\
Un gruppo ossidrilico reagisce con un altro gruppo ossidrilico \t si forma acqua (reazione di condensazione) e si forma il legame alpha1,4-glicosidico \t disaccaridi sono acetali, quindi reazione si chiama acetalizzazione (che è una reazione di condensazione)

\v

I polisaccaridi fanno da riserva energetica, sono strutturali (chitina e cellulosa) \\
Amido serve come riserva energetica, cellulosa è vegetale per struttura mentre chitina struttura ma animale \\
L'amidosio è elicoidale ma è una catena lineare fondamentalmente, mentre la amilopectina è ramificata \t e l'insieme dei due fornisce la struttura dell'amido \\
Il glicogeno è riserva energetica ma è animale \t nel fegato e nei muscoli, deve essere mobilitato velocemente \\
Il glucosio viene sintetizzato quando è in piu \t quando siamo in devito di energia, lo anabolizziamo (lo degradiamo) \\
Insulina e glucagone \t sono importanti per il diabete: senza insulina non riesco a sotrattre glucosio dall'organismo \\
Il pancreas produce e rilascia gli ormoni

\v

La cellulosa è un omopolisaccaride \t ma ha bisogno di un altro polisaccaride \\
Ha una forma lineare \t perche deve servire una funzione strutturale \\
La lignina ha una struttura a reticolo \t le fibre di cellulosa si incastrano nella lignina, che ha una forma a reticolo \t la cellulosa in se è flessibile, ma serve rigidita \\
Alcuni alberi sono + flessibili \t quelli che crescono in ambienti climatici forti, che devono proteggersi dal vento e da altri agenti atmosferici \\
La lignina permette alla cellulosa di organizzarsi in fasci e questo permette di avere rigidita \t diverse percentuali dei due determinano + o - flessibilita
\end{document}
