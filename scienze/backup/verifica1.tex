\documentclass[12pt]{article}

\usepackage{graphicx}
\usepackage[a4paper, total={6in, 8in}]{geometry}
\usepackage{textcomp}

\begin{document}
\setlength{\parindent}{0pt}

\def \t {\textrightarrow}
\def \v {\vspace{1em}}
\def \bi {\begin{itemize}}
\def \ei {\end{itemize}}
\def \s[#1] {\section*{#1}}
\def \ss[#1] {\subsection*{#1}}
\def \sss[#1] {\subsubsection*{#1}}

\s[Chimica organica 1]
\ss[Composti organici]
Si suddividono in:
\bi
    \item idrocarburi: carbonio + idrogeno
        \bi
            \item alifatici
            \item aromatici
        \ei
    \item derivati degli idrocarburi: idrocarburo + (ossigeno, azoto, fosforo, zolfo)
        \bi
            \item alogenati
            \item azotati
            \item ossigenati
        \ei
    \item biomolecole: polimeri (tranne i lipidi) con (carbonio, idrogeno, ossigeno, azoto)
        \bi
            \item carboidrati
            \item lipidi
            \item proteine
            \item acidi nucleici
        \ei
\ei

\ss[Proprietà del carbonio]
\bi
    \item forma sempre 4 legami covalenti \t ibridazione degli orbitali 
    \item ha tutti i numeri di ossidazione compresi tra -4 e 4
    \item tende a condividere gli elettroni (elettronegatività media) e quindi crea legami covalenti poco polari, ma forti
\ei

\ss[Catene]
\bi 
    \item aperta lineare
    \item aperta ramificata 
    \item chiusa
\ei

\v

Se il carbonio è legato a:
\bi
    \item 3 H \t primario
    \item 2 H \t secondario
    \item 1 H \t terziario
\ei

\ss[Isomerie]
Definizione: \textit{composti con la stessa formula molecolare, ma differente formula di struttura, e quindi con diverse proprietà fisiche e chimiche}

\bi
    \item isomeria di struttura
        \bi
            \item di catena
            \item di posizione
            \item di gruppo funzionale
        \ei
    \item stereoisomeria
        \bi
            \item di conformazione
            \item di configurazione
        \ei
\ei

\ss[Isomeria di struttura]
Definizione: \textit{gli atomi di due composti con la stessa formula molecolare sono legati tra loro in sequenze differenti}

\bi
    \item di catena: modo diverso in cui gli atomi di carbonio sono legati nella catena carboniosa
    \item di posizione: differenza nella posizione di legami multipli, atomi o gruppi atomici
    \item di gruppo funzionale: gruppi funzionali diversi
\ei

Definizione: \textit{un gruppo funzionale è un legame multiplo, uno specifico atomo o un gruppo atomico presente nella catena carboniosa}


\ss[Stereoisomeria]
Definizione: \textit{atomi o gruppi atomici di due composti sono nella stessa sequenza ma con differente disposizione spaziale} \\
Può essere di due tipi:

\ss[Di conformazione]
Rotazione di atomi o gruppi atomici intorno a un legame semplice carbonio-carbonio. \\
I due composti (conformeri) hanno le stesse proprietà fisiche e chimiche \\
Si hanno due possibili configurazioni:
\bi
    \item conformazione sfalsata: atomi sono distanti il più possibile
    \item conformazione eclissata: atomi sono allineati
\ei

\ss[Di configurazione]
I composti differiscono per l'orientazione nello spazio di atomi o gruppi, ma non si possono interconvertire per rotazione \\
Si suddivide in:

\sss[Isomeria geometrica]
Differisce la disposizione di atomi o gruppi legati a due atomi di carbonio uniti da un legame semplice (cicloalcani) o doppio (alcheni)
\bi
    \item \textit{cis}: gli atomi sono disposti dalla stessa parte rispetto all'anello carbonioso o al doppio legame
    \item \textit{trans}: gli atomi sono disposti dalla parte opposta 
\ei

\sss[Isomeria ottica]
Gli isomeri ottici sono molecole che sono l'una l'immagine speculare dell'altra, ma non sono sovrapponibili \t vengono chiamate chirali \\
Perchè la molecola si chirale, deve mancare un piano di simmetria all'interno della molecola

\ss[Legami intermolecolari]
\bi
    \item gruppi idrofili: $-OH$, $-NH_2$, $-COOH$ 
    \item gruppi idrofobici: $-CH_3$, $-(CH_2)_n-CH_3$, $-C_6H_5$ \t solventi apolari
\ei

\includegraphics[width=\linewidth]{tabella.png}

\s[Nomenclatura]
\ss[Alcani non ramificati]
\bi
    \item 1C \t metano
    \item 2C \t etano
    \item 3C \t propano
    \item 4C \t butano
    \item ... \t pentano, esano, eptano, ...
\ei

\ss[Alcani con ramificazioni]
Le ramificazioni vengono considerate come nuove catene di alcani, ma con un idrogeno in meno \t diventano quindi alchili, che si chiamano come gli alcani ma con \textit{-ile} al posto di \textit{-ano} \\
Per esempio: metano \t metile, etano \t etile \\
Catena dell alcano + numero della posizione delle ramificazioni + nome delle ramificazioni alchine, in ordine alfabetico \\
Se composto è chiuso \t prendere non catena + lunga, ma quella chiusa

\ss[Alcheni]
\bi
    \item gruppo funzionale: doppio legame con carbonio
    \item suffisso: è \textit{-ene}
    \item nota: bisogna segnalare la posizione del doppio legame
\ei

\ss[Alchini]
\bi
    \item gruppo funzionale: triplo legame con carbonio
    \item suffisso: è \textit{-ino}
    \item nota: bisogna segnalare la posizione del triplo legame
\ei

\ss[Alogenuri alchilici]
\bi
    \item gruppo funzionale: un alogeno
    \item suffisso: è \textit{-ano}
    \item nota: bisogna segnalare la posizione dell'alogeno, con la quantità (\textit{di, tri, ...}), e il nome dell'alogeno
\ei

\ss[Alcoli]
\bi
    \item gruppo funzionale: gruppo ossidrile ($-OH$)
    \item suffisso: è \textit{-olo}
    \item nota: bisogna segnalare la posizione del gruppo ossidrile
\ei

\ss[Eteri]
\bi
    \item gruppo funzionale: atomo di ossigeno ($-O-$)
    \item suffisso: è \textit{-etere}
    \item nota: catena di sinistra + catena di destra (rispetto all'ossigeno) + \textit{etere} \t usare \textit{di-} se catene sono uguali
\ei

\ss[Aldeidi]
\bi
    \item gruppo funzionale: gruppo carbossile con idrogeno ($-CHO$)
    \item suffisso: è \textit{-ale}
\ei

\ss[Chetoni]
\bi
    \item gruppo funzionale: gruppo carbossile ($-CO$)
    \item suffisso: è \textit{-one}
    \item nota: bisogna segnalare la posizione del gruppo carbossile
\ei

\ss[Acidi carbossilici]
\bi
    \item gruppo funzionale: gruppo carbossilico ($-COOH$)
    \item acido + suffisso: \textit{-oico}
\ei

\ss[Esteri]
\bi
    \item gruppo funzionale: gruppo carbossilico ($-COO-$)
    \item suffisso: \textit{-ato} + di + alchile
\ei
\end{document}
