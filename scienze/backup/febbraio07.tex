\documentclass[12pt]{article}

\usepackage[a4paper, total={6in, 8in}]{geometry}
\usepackage{textcomp}

\begin{document}
\setlength{\parindent}{0pt}

\def \t {\textrightarrow}
\def \v {\vspace{1em}}
\def \bi {\begin{itemize}}
\def \ei {\end{itemize}}
\def \s[#1] {\section*{#1}}
\def \ss[#1] {\subsection*{#1}}
\def \sss[#1] {\subsubsection*{#1}}

\s[Polisaccaridi]
L'amido si trova nei polisaccaridi \\
Il glicogeno è l'equivalente per gli organismi eterotrofi \\
La differenza tra i lipidi e il glicogeno \t il glicogeno non è legato all'adipe, ma si accumula nel fegato che è l'oragano + reattivo \t è il centro di purificazione delle attivita metaboliche \\
Infatti è quello + soggetto alle malattie \\
Anche nei muscoli si accumula \t hanno bisogno di energia da impiegare \\
L'energia viene presa dalle cellule \t quindi il glicogeno deve essere vicnio alle cellule che ne hanno bisogno \t è quindi una riserva

\v

La chitina \t nell'esosceletro degli insetti/crostacei \t diversa dalla cheratina, che è nei capelli e nelle unghie \\
La cellulosa è una molecola lineare \t ma da sola non basta \t se ci fosse solo lei, sarebbe elastica \t ha bisogno di un altro polisaccaride, che fa da reticolo e permette una struttura solida

\v

I monosaccaridi sono semplici \\
Noi non ci nutriamo di glucosio in realta \t non lo trovo infatti come monosaccaride, ma lo trovo associato ad altri saccaridi \\
I disaccaridi sono i + importanti \t hanno gruppi OH che a secondo della posizione (levo/destro) definiscono la funzione?? \\
Il legame O-glicosidico \t OH reagisce con l'altro ossidrile, c'è una condensazione \\
Ma cos'è un anomero? \t nei monosaccaridi c'è una forma aperta o una forma chiusa \t nel passaggio dalla forma aperta alla forma chiusa si determinano le caratteristiche del glucosio \\
C'è una coesistenza tra le due forme \t e l'ossidrile anomerico determina questo \t nella reazione di condensazione (ovvero la reazione che produce dell'acqua) reagisce l'OH e l'O con un ponte di ossgeno \\
Il legame si forma tra il gruppo caratterisitco e l'HO \t cosi avviene la ciclizzazione e l'OH è anomerico \\
Con $\alpha$ si indica il legame (carbonio $\alpha$ è il carbone vicino al legame aldedico/chetoso)

\v

Il lattosio è il disaccaride nel latto \t ha un glucosio e un galattosio \t il galattosio per essere digerito deve essere modificato in glucosio, mentre il glucosio viene subito metabolizato \\
Bisogna quindi prima idrolizzare \t enzima dell'elattomateri ?? permette di digerire il latte

\v

Omopolisaccaride = è un polisaccaride costituito dalla ripetizione di un singolo monosaccaride, e la catena può essere lineare o ramificata \\
Mentre eteropolisaccaride ha diversi monosaccaridi \\
L'amido è un omopolisaccaride \\
L'amido è costituito dall'amidosio e amilopectina \t non lo digeriamo direttamente perche non abbiamo l'enzima che accellera la separazione \\
Gli enzimi sono catalizzatori e sono sempre proteine qua \t il catalizzatore velocizza la reazione, e la rende efficace \t la reazione magari avviene, ma molto lentamente (come nel caso della digestione dell'amido) \\
Ma l'enzima non interviene nella reazione in se \t è un substrato che permette il verificarsi della reazione, ma non è presente nell'equazione chimica \t non interagisce chimicamente \\
Rendono + efficace l'urto \t e permettono anche di far avvenire la reazione a temperature più basse (abbassando la soglia energetica)
\end{document}
