\documentclass[12pt]{article}

\usepackage[a4paper, total={6in, 8in}]{geometry}
\usepackage{textcomp}

\begin{document}
\setlength{\parindent}{0pt}

\def \t {\textrightarrow}
\def \v {\vspace{1em}}
\def \bi {\begin{itemize}}
\def \ei {\end{itemize}}
\def \s[#1] {\section*{#1}}
\def \ss[#1] {\subsection*{#1}}
\def \sss[#1] {\subsubsection*{#1}}

\s[Lipidi]
Il glicogeno viene convertito in adipe \t che fa da riserva di energia, ma anche da isolante termico \\
La cellulosa è un omopolisaccaride \\
Vitamine non vengono sintetizzate dagli umani \\
I lipidi si dividono in saponificabili e non-saponificabili \t se tratto lipidi con una base forte, diventano sapone \\
I lipidi sono solubili in solventi apolari (etere, cloroformio etc.) e quindi non in acqua, che è polare \\
Saponific. e non sono molto diversi \\
Saturo o insaturo per il numero di doppi legami \\
I semplici \t sono esteri di acidi grassi con alcoli \\
Il fatto che sia grasso dipende dalla presenza del glicerolo \\
Gli eteri hanno un ponte ossigeno che lega i due carboni, e nasce dalla reazione di due gruppi ossidrilici \t ma se questa reazione si realizza con un cruppo carbossilico, porta alla creazione di un estere (ovvero -COO-) ?? \\
Mentre i lipidi complessi sono esteri di acidi grassi con alcoli che contengono gruppi atomici \t per esempio i fosfolipidi, ovvero quelli che costituiscono le membrane \\
Ci sono anche gli amminolipidi (gruppo NH_2, ovvero il gruppo amminico e ha carattere basico) \\
Infine, i "precursori e derivati lipidici" non sono nessuno dei due e si convertono in altre molecole \t es. sali biliari etc. \\
Acidita degli alcoli è bassa \\
Sono acidi grassi gli acidi carbossilici che hanno 4 o + carboni \t quelli con catena + lunga non sono solubili, perche la catena carboniosa + è lunga meno interagisce con l'acqua \t anche con gruppi elettronegativi???

\ss[Denominazione degli acidi grassi]
Il doppio legame ha una sua reattivita (è un addensamento di elettroni, quindi ha natura nucleofila, ed è forte e corto) \t e nel doppio legmane non è flessibile la catena, assume un angolazione precisa \\
È importante sapere la distanza tra il doppio legame e COOH \t se sono vicini si influenzano, se lontani no \\
Estremita metilica è quela senza COOH \t e chiamo omega il carbonio dell'estermita metilica (con CH_3, omega sta per ultimo perche è la ultima lettera dell alf.) \\
Non li possiamo produrre \t si chimano quindi grassi essenziali, non li possiamo produrre ma li dobbiamo assumere con la dieta 

\ss[Trigliceridi]
Sono dannosi ma sono anche necessari \t fanno da riserva energetica e negli animali formano il tessuto adiposo \t è il grasso che accumulo \\
Si chiamano invece oli nelle piante \t l'olio lo estraiamo e sono riserve energetiche \t all'oliva l'olio serve perche fa da riserva per il germoglio \\
Il trigliceride bisogna prima spezzarlo e ricavare i carboidrati \t per ricavare l'energia \\
1,2,3-propantriolo è un trigliceride \\
Il trigliceride si forma con il legame del glicerolo con tre acidi grassi \\
Una esterificazione \t reazione fra on OH e un COH porta a un composto che contiene O-C=O, ovvero un estere \\
Tre acidi grassi + glicerolo mi da trigliceride + acqua, perche estereficazione produce acqua \\
Si hanno catene lineari, emntre al centro si ha un doppio legame = geometria irregolare della molecola, quindi ha un certo ingombro e comportamenti diversi \\
I grassi animali sono ricchi di grassi saturi (senza doppi legami, e a temperatura ambiente sono solidi \t ma a temperatura elevata si sciolgono) \\
Invece l'olio a temperatura ambiente è liquido, è grasso insaturo

\ss[Saponificazione dei trigliceridi]
NaOH e KOH sono basi forti \t se si immerge in questi grassi caldi si formano i saponi \t si elimina il glicerolo e si ottiene sale \\
La molecola di sapone è una molecola anfipatica \t ovvero sia polare sia apolare \t infatti i saponi reagiscono con'acqua in cui si sciolgono, ma reagiscono con i grassi e infatti puliscono \\

\ss[Idrogenazione dei oli vegetali]
Gli oli sono insaturi \t e l'insaturazione porta a maggiore solubilita ??
\end{document}
