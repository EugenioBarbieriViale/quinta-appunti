\documentclass[12pt]{article}

\usepackage[a4paper, total={6in, 8in}]{geometry}
\usepackage{textcomp}

\begin{document}
\setlength{\parindent}{0pt}

\def \t {\textrightarrow}
\def \v {\vspace{1em}}
\def \bi {\begin{itemize}}
\def \ei {\end{itemize}}
\def \s[#1] {\section*{#1}}
\def \ss[#1] {\subsection*{#1}}
\def \sss[#1] {\subsubsection*{#1}}

\s[Lipidi]
Un lipide è caratterizzato da: (principale caratteristica) è insolubile in acqua \t quindi mancano legami che permettono legami a idrogeno con acqua \t quindi sono catene lunghe di carbonio \t sono composti apolari \\
Quelli grassi hanno COOH \\
Glicerolo si puo esterificare con tre diversi acidi grassi \\
Un acido grasso è caratterizzato dalla saturazione (grassi saturi non hanno doppi legami) \t e perchè è importante la presenza o meno dei doppi legami? \t il doppio legame è una zona nucleofila, e ha un impatto sulla forma \t un doppio legame irrigidisce \\
Questo provoca una precisa forma, che non è quella di una catena lineare che si può adattare \\
Il doppio legame cambia la forma perchè modifica una particolare configurazione: determina una geometria planare o una angolare, perchè cambia l'ibridazione (con il doppio legame è sp2)

\v

L'ibridazione è il fenomeno in cui avviene un rimodellamento della struttura esterna di un atomo, per cui ottegono degli orbitali intermedi tra gli atomi \\
Carbono ha numero atomico 6 \t quindi le scrittura degli orbitali è ?? \\
Mi aspetto nei diversi carboni con i diversi idrogeni degli orbitali diversi \t ma scopro che sono tutti uguali, quindi devono essere ibridati \\
Devo avere 75 per cento di p e 25 per cento di s \t sovrapposizione che permette di ottenere 4 orbitali che per 3/4 hanno natura p, 1/4 s \\
Uno di questo non è ibridato, un p rimane solo \t ho un orbitale p e due sp2

\v

Grassi animali sono saturi e solidi a temperatura ambiente \t perche? \t liquido o solido è determinato dalla forza attrativa delle molecole \\
Le forze di Van Der Waals \t prevedono che ci sono dei momenti in cui l'elettrone è spostato rispetto a dove ci si aspetta che sia, e questo cambia la configurazione \t sorge quindi una forza elettrica, e l'elettrone della molecola vicina si sposta etc. \\
Forze di Van Der Waals tengono insieme molecole che apparentemente sono neture \\
Quelle insature invece hanno gruppi negativi, che fanno pero anche da forze repulsiva \t insaturi provocano minori legami, e questo rende liquido a temperatura ambiente

\v

\ss[Saponificazione]
Reazione che avviene tra acidi grassi e una soluzione di base forte \t un ancido grasso ha gruppo COH, e con la base forte si forma un sale \\
I lipidi si dividono in saponificabili e non-saponificabili \\
Trigliceride con NaOH \t si ottiene uno ione ossalato che ?? \\
È una molecola anfipatica \t sia polare sia apolare \\
La margarina è solida ma è vegetale \t perché? \t è l'effetto dell'idrogenazione, ovvero l'aggiunta idrogeni a un doppio legame \t con l'idrogenazione posso ottenere la margarina, che è un olio che viene idrogenato \t cosi i doppi legami scompaiono e diventa solido \\
Si usa così perchè è meno ossidato rispetto ai doppi legami (quindi stabilizzato) e sostituisce il grasso animale del burro
\end{document}
