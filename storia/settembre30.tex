\documentclass[12pt]{article}

\usepackage[a4paper, total={6in, 8in}]{geometry}
\usepackage{textcomp}

\begin{document}
\setlength{\parindent}{0pt}

\def \t {\textrightarrow}
\def \v {\vspace{1em}}
\def \bi {\begin{itemize}}
\def \ei {\end{itemize}}
\def \s[#1] {\section*{#1}}
\def \ss[#1] {\subsection*{#1}}
\def \sss[#1] {\subsubsection*{#1}}

\s[Riprendere appunti del 30]
\s[Prima guerra mondiale]
Le vicende del 1916 sono favorevoli agli imperi centrali (3A) \\
I tedeschi occupane zone industriali della francia e del belgio \\
Sul fronte orientale la russia subisce sconfitta ai laghi Masuri \\
Entra in guerra la Bulgaria \t che segna il crollo della serbia, occupata da truppe austriache \\
Tedeschi fanno assedio alla fortezza di Verdan \t luglio del 1916 \t assedio lungo, alleati anglo-francesi rispondono con la battaglia de la Somme \\
Questa battaglia causa mezzo milione di morti \\
Sul fronte russo \t aveva fatto arretrare le truppe austriache ??? \\
Gran bretagna stava facendo blocco navale contro germania \t impediva rifornimenti nei porti tedeschi \t gia dall inizio della guerra \\
Dopo due anni effetti si fanno sentire \t per questo germania affronta marina britannica nel mare del nord \\
Si svolge la battaglia dello Iutland \t nel maggio del 1916 \t non vince nessuno: dominio dei mari rimane inglese, ma germania infligge grandi perdite \\
3A si impadronisce anche della romania \t forniva petrolio e provvigioni alimentari \\
Turchia è in difficolta \t affronta rivolta delle tribu arabe, fomentata dagli inglesi \t Lorentz d'Arabia \\
1916 muore giuseppe d austria?? e sale al trono Carlo I \\
Guerra sembra a favore degli imperi centrali (3A) 

\v

1917 cambia tutto \t due grandi eventi + disfatta di caporetto:
\bi
    \item vivoluzione bolscevia \t Russia esce dalla guerra \t imperi centrali possono concentrare le truppe ad occidente
    \item ingresso americano \t ma entrano usa \t sono ben armati, non affaticati da 3 anni di guerra \t svolta per triplice intesa
\ei

\ss[Ingresso americano]
Per rompere blocco navale inglese \t tedeschi fanno lotta sotto marina \t ma volevano anche bloccare l inghilterra ??? \\
Questa guerra viene intensificata \t tedeschi affondavano navi mercantili e navi civili (es. transatlantico Lusitania, gia nel 1915, morti 124 cittadini americani) \\
Nel 17, quando germania intensifica la battaglia navale, entrano in guerra \t erano danneggiati da questo \\
Colonnello Wilson perpetuava politica isolazionista \t non voleva entrare, ma poi decide di entrare \t è pressato dai gruppi industriali \\
Interessi economici di 2 tipi:
\bi
    \item non volevano perdere contratti con l europa in atto
    \item america aveva fatto prestiti all intesa \t se perdevano, soldi adavano persi
\ei
Cosi si rompe politica isolazionista

\ss[Rivoluzioni del 17]
C'è rivoluzione a frebbraio e poi ottobre \t tra questi mesi c'è la crisi \\
A febbraio regime repubblicano \t che si affiancava al governo dei soviet, riconosciuto a livello internazionale \\
I soviet ritenevano il vero potere in realta \t operai si appoggiavano a loro \\
Lenin fa poi rivoluzione irruenta \t la prima cosa che fa è uscire dalla guerra e poi fa riforma agraria \\
In realta russia non combatteva piu da molto \t l esercito russo spesso si era rifiutato di combattere, per instailita politica e condizioni terribili \\
Trattato di Brest-Litosvk \t russia esce dalla guerra e viene penalizzata territorialmente
\bi
    \item germania ottiene paesi baltici e polonia
    \item ucraina diventa indipendente \t che era il granaio dell impero
\ei

\ss[Caporetto]
Conseguenza della rivoluzione russa \t si chiude il fronte orientale, e austria puo richiamare le forze sul fronte occidentale \\
Le linee italiane vengono sfondate a Caporetto \t oggi è vicino all'Istria \\
Le truppe italiane si ritirano \t truppe austriache penetrano in italia per 150 km \\
Ritirata è una disfatta \t soldati fuggono \\
Viene formato un nuovo governo \t Vittorio Emanuele orlando (non il re) \\
Cadorna viene sostituito da generale Diaz \t è meno rigido e si preoccupa di addestrare i soldati \\
Crea linea di difesa sul Piave \t viene bloccata l avanzata austriaca \\
Ma in quel momento l italia era molto sfiduciata, i soldati erano demotivati, la guerra di trincea li aveva resi passivi rispetto alla guerra (non genrava spirito di combattimento) \\
Nel 17 si genera anche il rifiuto della guerra tra i soldati, che dissertano e scappano \t anche forme di autolesionismo \\
Quindi diverse motivazioni per disfatta di caporetto \t non solo forza austriaca 

\v

Poi arrivano soldati statunitensi \t ma austria e germania vogliono chiudere conflitto \t non hanno piu risorse \t blocco economico aveva messo in difficolta \\
Nel 18 l esercito tedesco porta una ???? \\
Amiens \t tra luglio ed agosto del 1918 \t truppe tedesche venogono sconfitte \t inizio del declino \\
Progressivamente si arrendono alleati degli imperi centrali \t bulgaria si arrende, ungheria, cecoslovacchia si dichiarono indipendenti \\
Austria viene sconfitta da controffensiva italiana nel 1918 \t nella battaglia di Vittorio Veneto \\
Il 3 novembre del 18 \t viene firmato l'armistizio a Villagiusti, in cui vince l italia \\
11 novmebre Carlo I abdica \t probalamata la repubblica \t anche turchia si arrende \\
In germania c'è difficolta \t Guglielmo II vuole dettare le sue condizioni e non vuole lasciare il potere \t viene invitato ad abdicare \\
Germania pensa di poter ottenere condizioni di pace migliori senza di lui \t viene fatto abdicare, da un governo provvisorio che firma armistizio \\
Armistizio di Retondes??? 11 novembre \\
Nuovo governo repubblicano, presieduto da un social democratico di nome Erbert ???? 

\v

In totatale 8 milioni e mezzo di morti \t 21 milioni di feriti gravi \t inoltre ci sono tantissime vedove e tantissimi morti \\
Pace viene firmata a parigi nel gennaio del 19 \t si riuniscono i paesi vincitori, mentre i vinti vengono convocati solo per firmare \\
Francia, Inghilterra, Italia e USA \t sono i vincitori \\
Il presidente americano Wilson si presenta con 14 punti \t che dovevano essere gli accordi di pace e il futuro assetto internazionale statunitense \\
Due punti:
\bi
    \item punto 1: principio di autodeterminazione dei popoli e liberta dei mari \t ogni nazione ha il diritto di autodeterminare il suo futuro politico ed economico \t sara disatteso per i trattati di pace
    \item punto 14: societa delle nazioni \t doveva essere una societa internazionale super partes, in grado di garantire l integrita politica di tutti gli stati \t risoluzione delle controversie 
\ei
Questi punti vengono disattesi perche:
\bi
    \item francia voleva danneggiare germania
    \item ing. voleva evitare la distruzione della germania per non rendere francia dominante
    \item l italia chiedeva il rispetto del trattato di Londra \t non verra rispettato, perche viene viene considerato contro il punto 1
    \item Wilson voleva ottenere un libero mercato e voleva garantire la superiorita economica degli stati uniti
\ei

\v

Al congresso di Parigi prevale la linea della francia \t punitiva \\
I 14 punti non vengono rispettati \\
Trattative durano 1.5 anni \t tra 19 e 20 \t prevale alla fine la visione francese \\
Trattato conclusivo è quello di Versaille 

\ss[Trattato Versaille]
Vincitori firmano con germania \t considerato il trattato conclusivo della WWI \\
Il trattato di versaille non è l unico pero \t italia per esempio firma Saint Germain con austria \\
Viene firmato nel giugno del 19 \t viene firmato mentre germania sta elaborando la costituzione di Weimar
\bi
    \item germania deve pagare 132 miliardi di marchi d'oro \t cifra impensabile \t per ripagare danni di guerra
    \item similitarizzazione totale \t flotta viene ridotta
\ei

\v

Germania perde:
\bi
    \item l alsazia e lorena
    \item lo schlesswig
    \item distretti minori vanno al belgio (abitati pero da tedeschi)
    \item polonia ritorna a essere riconosciuto e le cede la slesia, la posmania, il corddoio di danzica, ma danzica rimane libera, ceduti i Sudeti
\ei
Abolito trattato di Litvosk??? \t lettonia, lituania e filandia e estonia divengono indipendenti \t paesi cuscinetto con russia \\
Inoltre parte della renania??? sotto controllo francese per 15 anni, per verificare che germania rispettasse trattati \\
Inoltre la francia puo sfruttare temporaneamente le miniere di Saars (dopo 15 anni la popolazione poteva decidere se diventare tedesca o francese) \\
Poi germania perde colonie, che vengono divise dai vincitori \t ma questo significava andare contro il punto 1 di Wilson \\
Viene escogitato il "mandato" = affidamento internazionale della colonia temporaneo, simile al protettorato (che pero ricordava imperialismo) \\
Nasce anche la Societa delle Nazioni \t nasce nel 19, ha sede a Ginevra, e doveva risolvere diplomaticamente le questioni internazionali \t senza avere successo 

\v

Italia firma trattato di saint germain \t austria e ungheria sono separati \t ungheria è indipendente \t lei firma il trattato del Triadon???? \t che riconoscono anche la Jugoslavia \\
S.G. vieta all'austria di unirsi alla germania a meno della approvazione di tutta la societa delle nazioni \t austria era rimasta piccola \t si voleva unire alla germania
\end{document}
