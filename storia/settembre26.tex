\documentclass[12pt]{article}

\usepackage[a4paper, total={6in, 8in}]{geometry}
\usepackage{textcomp}

\begin{document}
\setlength{\parindent}{0pt}

\def \t {\textrightarrow}
\def \v {\vspace{1em}}
\def \bi {\begin{itemize}}
\def \ei {\end{itemize}}
\def \s[#1] {\section*{#1}}
\def \ss[#1] {\subsection*{#1}}
\def \sss[#1] {\subsubsection*{#1}}

\s[Prima guerra mondiale]
Scoppia nel contesto imperialistico \t le motivazioni stanno nelle volonta imperialistiche delle potenze europee \\
Tutte queste tensioni portano all'omicidio di Sarajevo \\
Ci sono anche delle motivazioni economiche \t per esempio le rivalita coloniali (tra germania e inghilterra) \\
La germania era cresciuta molto industrialmente \t si sentiva forte abbastanza da dominare tutto il centro europa \\
Altra causa economica era che le nazioni europee volevano espandere il proprio mercato, ma anche ottenere materie prime attraverso l'espansione \\
Si vuole scardinare il'impero coloniale altrui per ragioni di dominio, ma anche economiche \\
Anche ragione militare \t gli stati avevano preparato degli esereciti di offesa, pronti alla guerra \t non di difesa \\
C'era stata una corsa agli armamenti \t c'è una causa militare, per cui tutti volevamo mostrare la loro industria bellica \\
Tutti volevano mostrare la loro forza \\
Anche culturale \t il nazionalismo era dominante, che si basava su tesi razziste \t si pensa che per preservare una nazione bisogna avere un unica razza in quella nazione \\
Inoltre si pensava che c'era una stanzialita nella situazione politica \t i giovani vedono nella guerra un modo per creare del cambiamento \t la guerra viene vista come un mezzo 

\v

28 giugno del 14 il casus belli \t Cavilli Princip uccide Francisco Ferdinando e la moglie a Sarajevo \t l'attentato era stato preparato a Belgrado, quindi la serbia viene ritenuta responsabile \\
L'austria approfitta dell'omicidio per accusare la serbia e fa ultimatum nel 23 luglio. La serbia doveva:
\bi
    \item a serbia doveva sopprimere le organizzazioni irrdedenstiste slave
    \item vietare ogni forma di propaganda anti austriaca
    \item doveva creare una commisione serba-austriaca per indagare sull omicidio
\ei
Serbia rifiuta \t il 28 luglio austria dichiara la guerra \\
Immediatamente scoppia il gioco delle alleanze \t lo Zar ordina la mobilitazione generale del suo esercito \\
La germania interviene: o la russia smobilita l'esercito o dichiara guerra \\
Il 1 agosto la germania dichiara guerra alla russia e alla francia (era scontato che intervenisse con la russia) \\
Secondo una visione, la germania è quella che ha generato la guerra mondiale \\
Piano Schliffen \t prevede un attacco massiccio alla francia, in modo da potersi dedicare alla russia \t si pensava che la russia sarebbe stata + lenta nella mobilitazione \\
Germania voleva neutralizzare subito la francia, che era vicina \\
Invade la francia passando per belgio e lux. \t che erano neutrali, a quel punto inghilterra entra in guerra contro germ. \\
Il 4 agosto gli schieramenti sono definiti \t mentre l'italia si dichiara neutrale \t perche triplice alleanza era difensiva, l'alleanza non la obbligava \\
Si crea pero grande dibatti pubblico \t Salandra dichiara la neutralita, poi valuta le condizioni

\v

L'austria non voleva discutere di cessione dei territori se non dopo la guerra \\
La triplice intesa invece regalava dei territori (trentino, dalmazia, addirituttra territori in turchia), all italia \t voleva sottrarre alleanza \\
Si crea dibattito, italia è divisa: \\
Neutralisti: maggioranza della popolazione, del parlamento, dei cattolici (per papa) e dei socialisti (guerra è sempre imperialista, fa gli interessi dei potenti), liberali (si ispirano a giolitti) \\
Giolitti fa proposta ad austria \t in cambio di neutralita l'austria cede le terre irredente, come se l'italia fosse di peso per la guerra \\
Interventisti (nazionalisti, irredentisti):
\bi
    \item destra: volevano terre irredente e sono la corte, gli ufficiali, il re, gli industriali \t volevano accresce il prestigio con guerra e vincere l'asutria
    \item sinistra: repubbliani, socialisti, Mussolini \t inneggiava alla guerra attraverso l'Avanti, e lo cacciano dal partito socialista \t vogliono combattere in nome delle minoranze oppresse (come una lotta di classe globale)
\ei
Ci sono pero anche interventisti cattolici, liberali che vogliono andare contro Francia, ... (considerate potenze laiciste) e combattere a fianco degli alleati cattolici (germania, austria) \\
Decide il governo, che cerca di convincere i neutralisti della necessita della guerra \t incoraggia manifestazioni aggressive e disordini per far vedere che la guerra è necessaria \\
Alla fine riesce \t ottiene l'appoggio del parlamento e della popolazione \\
Sonnino nel 15, dopo tutte queste promesse, firma il patto di Londra ed entra in guerra contro 3alleanza \\
Nel 15 esce dall 3A e il 24 maggio 15 italia dichiara guerra all'austria 

\v

I primi 6 mesi era una guerra di movimento \t le truppe si spostano, mentre i restanti 4 anni diventa di trincea \\
Diventa una guerra di logoramento presto \\
Germania invade la francia, che blocca l avanzata sul fiume Marna \t muoino 500.000 soldati, e nessuno riesce a vincere \\
Nasce il primo lungo fronte di combattimento, che era lungo 800 km (dal mare del nord alla svizzera) \\
Si costruiscono trincee \\
Sul fronte orientale i tedeschi affrontano i russi \t battaglie dei Laghi Masuri e Tannenberg nel settembre del 14 \t anche qua situazione di stallo \\
Il 31 ottobre del 14 entra in guerra anche la turchia, con la triplice alleanza (che non esiste piu, solo germania e austria) \\
Guerra è mondiale perche tutti coloro che entrano in guerra aprono fronti \t russo-turco in armenia, anglo-turco in egitto \\
I fronti di combattimento sono ovunque le nazioni non alleate si possono confrontare
\end{document}
