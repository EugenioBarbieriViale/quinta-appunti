\documentclass[12pt]{article}

\usepackage[a4paper, total={6in, 8in}]{geometry}
\usepackage{textcomp}

\begin{document}
\setlength{\parindent}{0pt}

\def \t {\textrightarrow}
\def \v {\vspace{1em}}
\def \bi {\begin{itemize}}
\def \ei {\end{itemize}}
\def \s[#1] {\section*{#1}}
\def \ss[#1] {\subsection*{#1}}
\def \sss[#1] {\subsubsection*{#1}}

\s[Stalin]
Stalin e Trozky si confrontano per succedere a Lenin su diversi argomenti \t prevale la linea di Stalin, anche se per quanto riguarda la nep reucperera la proposta di Trozky \\
Stalin è il segretario del parito \\
Poi russia entra in crisi economica \t colpa viene data alla nep \t anche se non era vero \\
Nep, che gia non piaceva, viene accantonata \t stalin inaugura piano economico di Trozky = collettivizzazione e industrializzazione forzata \\
Cosi si va verso la direzione di un vero socialimo \t il fine è fare della russia una potenza industriale che possa confrontarsi con potenze \\
Prevale infatti l'industrializzazione forzata \t l'ambito agrario verra quindi trascurato e penalizzato \t generando situazione critica \\
Russia diventa potenza industriale e competitiva, ma i russi moriranno di fame \t tutti i renditi della produzione agricola venivano usati per l industrializzazione \\
Stalin fa della russia una potenza, ai danni della popolazione

\ss[Industrializzazione]
Stalin vara i piani quinquennali \t progettazioni economiche che durano 5 anni, considerato un tempo ragionevole per incrementare il risultato sia per correggere i piani in caso di necessita \\
Piani vengono proposti dallo stato \t che si fa carico del rilancio dell industria \t e anche la verifica dell'attuazione \\
Tutto questo viene realizzato senza cosiderare aspetti sociali e umani \\
Primo nel 28 \t si vuole incrementare carbonio, petrolio, acciaio \t quindi viene sostenuta l'industria pesante, che è uno dei punti forti di stalin \\
Per piano servono soldi \t si razionano quindi i beni di consumo e cibo, operai hanno ritmi di lavoro pensantissimi e disciplina ferrea \\
Inoltre in russia no personale qualificato \t chiamati dall'esterno (america e germania) molto pagati \\
Loro formano i tecnici russi (per generare un personale russo) e conducono \\
Stalin inoltre deporta i contadini \t gli operai vengono reclutati forzatamente nelle campagne, e il contadino non poteva rifiutarsi \\
Questo genera incremento della popolazione cittadina, e svuotamente delle campagne \\
Agricoltura è subordinata all industria e viene trascurata \\
Tutto questo viene giustificato con una propaganda imponente \t stalin fa ricorso all'idea di una nazione russa forte \t motiva gli operai dicenco che lavorano per la grande madre russia \\
Operai migliori vengono anche premiati \t nasce lo Stackanovismo \t Stackanov aveva lavorato 3 volte tanto, quindi era stato ricompensato \\
Tra 20 e 32 industria cresce del 40 percento

\v

Secondo piano quinquennale del ?? \t produce incremento del 120 percento \\
Qua si privilegia anche la media industria (prima solo la grande)

\v

Terzo \t doveveva essere il piano che portava al socialismo vero \t ma interrotto dalla prima guerra mondiale \\
Fine anni 30 \t si vede lo sviluppo della industria bellica \\
Tra il 32 e 33 c'è stata un altra carestia \t che venne nascosta e non resa pubblica, neanche a livello internazionale \t stalin voleva mostrare che la russia era forte 

\ss[Colettivizzazione]
Collettivizzazione forzata \t agricoltura viene piegata totalemnte all industria \\
I pulaki (medi propetari) si rifiutano di cedere le terre \t Stalin li elimina tutti, e rimuove una classe sociale \\
Repressione sara una modalita di attualizzazione della collettivizzazione \t qua si tocca la proprieta \\
Stato assume controllo totale delle campange \t cittadini devono entrare in due aziende agricole:
\bi
    \item kolkozi \t aziende agricole nelle quali contadini usano terra, che rimane dello stato
    \item sovkozi \t completamente statali, contadini non gestiscono \t stato prende quota dal raccolto dei contadini, e quello che rimane va ai componenti del sovkozi (coloro che lavorano nella azienda)
\ei
Cosi stalin controlla tutta la campagna \t ed erano vietate attivita commerciali indipendenti \t tutto dello stato \\
Repressione è in realta un tratto tipico di stalin \t negli anni 30 Stalin da vita alle purghe \t tutti quelli che si oppongono vengono eliminati \\
Tutti i dirigenti bolscevichi vengono accusati di tradimento \t e vengono condannati \t le prove erano le loro confessioni sotto tortura \\
Inoltre nel 39 Stalin di 139 membri del comitato centrale, ne rimangono solo 29 \t Bukarin era stato fucilato nel 38 \\
C'erano 5 marescialli, 3 fucilati \t anche purga all'interno dell'esercito \\
Anche intellettuali, tecnici, professionisti \t vengono o fucilati o deporati

\v

Questo si affianca a una propganda forte \\
Stato totalitario
\bi
    \item partito solo
    \item le strutture del partito sono quelle dello stato
    \item controllo statale su ogni ambito della vita \t dall'economia alla cultura ai mezzi di informazione
\ei
Stato assoluto è divero \t stato totalitario ha sovrapposizione di parito e stato \\
Stato totalitario + culto del capo \t lui si presentava come il continuatore della rivoluzione \t lui era infallibile, e aveva il merito di aver portato a compimento il lavoro di Lenin, che aveva solo iniziato \\
Lui fa anche riscrivere la storia del bolscevismo \t al cui centro c'è la sua figura \t manipola la storia, e tutti gli insuccessi dei bolscevichi vengono imputati ai nemici \\
Lui viene amato pero dalla popolazione \t la russia diventa forte, e stato offre dei serivizi \t condizioni di vita alla fine sono migliori
\end{document}
