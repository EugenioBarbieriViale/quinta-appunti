\documentclass[12pt]{article}

\usepackage[a4paper, total={6in, 8in}]{geometry}
\usepackage{textcomp}

\begin{document}
\setlength{\parindent}{0pt}

\def \t {\textrightarrow}
\def \v {\vspace{1em}}
\def \bi {\begin{itemize}}
\def \ei {\end{itemize}}
\def \s[#1] {\section*{#1}}
\def \ss[#1] {\subsection*{#1}}
\def \sss[#1] {\subsubsection*{#1}}

\s[Seconda guerra mondiale]
Con l'aggressione della polonia scatta il meccanismo delle alleanze \t francia e inghilterra dichiarano guerra \\
Russia invade polonia verso est, mentre tra francia e germania si crea linea Vagineau in cui si crea situzione di stallo \\
Poi germania invade daminarca e norvegia \\
Si illude di poter fare un blitzkrieg

\v

Il suo alleato sovietico si prende anche finlandia e i paesi baltici \t indispensabili per la sicurezza sovietica \\
Mentre invade la norvegia, attacca Hitler la francia a sorpresa con il piano Manstein \\
Le truppe invadono il beglio e l olanda, e puntano a parigi \t ma esercito francese si aspettava l'attacco + a sud \t le truppe francesi ed inglesi si trovano accerchiati \\
Infatti le truppe tedesche penetrano \t e le truppe anglofrancesi devono scappare da Dunkirk \t con l'aiuto dell'inghilterra riescono a scappare \\
Nel 40 germania occupa la francia, che viene divisa \t e c'erano due opzioni: o poteva continuare a combattere o scendere a patti, che era quella voluta da Peten e dai conservatori \\
Si firma armistizio il 22 giugno che prevede la divisione della francia in due:
\bi
    \item centro-settentrionale è sotto diretto controllo dei tedeschi
    \item centro-meridionale da vita a un governo collaborazionsita, Vichy, condotto da Peten
\ei
Il maresciallo ?? Goulin ?? dice che non bisogna piegarsi e sostiene francesi liberi \\
Nasce a Vichy un governo francese ma disposto a collaborare con invasore \t questi governi di solito sono governi fantoccio, sono in realta nelle mani dell invasore \\
La francia non sara l unico esempio \t ci sara un governo collab. in norvegia e ci sara la repubblica di Salo, che puo essere considerata un governo collaboraz.

\v

Allo scoppio della guerra mussolini è in situzione difficile \t italia deve intervenire per patto, ma in questo momento non puo sopportare conflitto come questo \t aveva gia combattuto in etiopia e spagna \\
Il governo italiano proclama quindi la non belligeranza \t non neutralita, perche il patto d'acciaio è vincolante \t significa che per il momento italia non entra in conflitto, ma è alleata con la germania \t questo nel 39 \\
Poi Hitler stravince \t quindi mussolini si convice che questa sara effettiamente una guerra lampo \t la guerra sta per finire, e bisogna sedersi al tavolo dei vincitori \\
Muss. dice che ha bisogno di qualche migliaio di morti per poter stare con i vincitori \\
Nel 10 giugno 40 quindi italia dichiara guerra a francia e ingh. \t e questo viene letto come una vigliaccata a livello internazionale, perche la francia era gia in agonia \t anche critiche interne \\
Quando l italia entra in guerra, la reatla è pero diversa \t mussolini aveva sottovalutato la forza dell'ingh (la francia in questo momento è schiacciata) \\
Italia aveva poche armi e arretrare, non aveva bombardieri (solo tedeschi), flotta abbastanza valida ma con poco carburante \t italia è inadeguata a guerra di questo tipo \\
Italia prova a strappare malta agli inglesi, ma fallisce \t qua si capisce gia inadeguatezza \\
Poi nell'agosto italia attaca somalia e sudan (che sono inglesi) ma anche qua c'è fallimento \t tale che la germania dovra sostenere italia \\
Fallimento totale con l'invasione della grecia \t l'esercito italiano subisce perdite enormi, e la conquista si attuera con l'aiuto tedesco \\
L'italia attaca in queste zone perche il nemico da domare è l'ingh. \t francia è gia domata

\v

Nel ? era diventato ministro Churcill, che disdegna la politica di concessione dei precedenti (Chamberlain etc.) \\
Hitler propone a Churcill trattativa di pace, di firmare la pace e riconoscere cosi i paesi invasi sotto la germania \\
Churcill, con il parlamento in parte contrario, che l'ing. combattera fino alla vittoria \\
L'opinione pubblica è con Churcill, ma il parlamento lo contesta \t voleva politica di concessioni con Hitler \\
Hitler quindi vuole sbarca in ingh. \t tra agosto e settembre del 40 con l'operazione leone marino, i tedeschi tentano lo sbarco su coste britanniche \\
Per sbarcare bisognava prima distruggere l'avizione inglese \t quindi la RAF e la Luftwaffe per due mesi si scontrano nella battaglia di Inghilterra nei cieli britannici \\
Londra verra bombardata duramente, e la RAf pero infliggera perdite importanti \t Hitler il 17 settembre rinuncia \\
Con questo finisce anche l'illusione della guerra lampo \t Hitler capisce che sara guerra impegnativa \\
Anche questa diventa ora una guerra di logoramento \t le forze in campo non riescono ad avere la meglio una sull altra \\
Il 40 si chiude pero a vantaggio delle potenze dell asse \t Hitler ha zone strategiche importanti e Italia è con Hitler \t inotlre Germ. aveva rafforzato rapporti con Giappone \\
Il ? del 40 frimato Pattto Tripartito \t che assomoglia a un alleanza militare e rafforza legame tra germ it e giap

\v

Nella primavera del 41 la germania deve aiutare l italia, che rischia disfatta militare pericolosa (pericolosa per l'andamento della guerra in generale) \\
Vengono mandate truppe in nord africa e paesi balcani \t Rommel guida le truppe naz. che arrivano fino al canale di Suez \\
Germ. conquiesta nel frattempo Grecia, Jugoslavia e Creta \t e vegono imposte condizioni a Romania e Bulgaria \\
Italia ora capisce che nessuna operazione militare autonoma puo essere svolta dall'italia \\
Hitler quindi nel 41 capisce di aver sconfitto tutti i nemici (tranne ingh.) e quindi si puo dedicare all'estensione del lebensraum \t quindi si estende in URSS \\
Infatti il patto molotv-ribb era un patto di cinismo politico che vedeva alleati due paesi storicamente nemici, e inoltre Hitler considerava URSS il male: slavi sono inferiori e comunisti \\
Va verso la urss perche era anche ricca di materie prima, di cui la germaniza aveva bisogno \\
Quindi il 22 giugno del 41 la germania invade la russia, con il piano Barbarossa che prevedeva:
\bi
    \item un attacco lampo
    \item avanzamento veloce con rapido annientamento della resistenza
    \item vengono impiegati 13 milioni di uomini
    \item 10 mila carri armati
\ei
È un operazione grandiosa \\
A seguito di questo esercito, lui porta dei drappelli di SS tutti gli ebrei slavi che incontrano \t quindi invasione corrisponde anche a progetto razziale \\
Anche l'italia partecipa e manda il CSIR \t corpo di speidzione italiano in russia \t poi mandera l'ARMIR \\
Cosi sovietici perdono 
\bi
    \item paesi baltici
    \item bielorussia
    \item crimea settentrionale
    \item gran parte dell ucraina 
\ei
I sovietici resistono pero, e arrestano l avanzata alla fine dell'autunno \t arriva infatti l'inverno, e cosi diventa guerra di logoramento \\
Le truppe italo-tedesche vengono rafforzate \t l'italia rafforza la sua presenza con 230 mila soldati italiani (ARMIR, armata italiana in russia) \t quando la germania lacera offensiva forte contro resistenza sovietica

\v

Nel 41 pero entra in guerra anche il giappone, che aveva intrapreso politica espansionista aggressiva ai danni della cina \\
Questa politica aveva contrapposto giappone a USA e ingh. per il controllo del pacifico \t uno dei movitivi che porta ad alleanza con germania ed italia, che erano esclusi da questa zona \\
Ci sono pero altri fattori ideolgici e delle tangeze ideolgiche con germania \\
Nel 41 il giappone firma patto di neutralita con URSS, per coprirsi le spalle \\
L'espansionismo del giappone ha due caratteri: 
\bi
\item è sicuramente imperialista
\item ma è anche espansionismo nazionalista \t giappone si espandeva in regioni controllate dagli USA \t è una nazione in lotta contro paesi occidentali che non ci dovrebbero essere
\ei
Il giappone da fiato al nazionalismo asiatico \t si fa portavoce dell'esigenza di allontanamento dell occidente dall asia, condivisa anche da altri paesi \\
Nel 41 quindi giappone occupa l'indocina francese (francia in difficolta, non poteva intervenire) \t usa risponde con embargo (blocco di esportazioni) con Giappone, che pero è duro \t giappone è poberi di materie prime \\
Giappone allora decide di entrare in guerra \t bombara Perl Harbour (postazione + grande americana sul pacifico) e viene distrutta completamente, senza una dichiarazione di guerra \\
L'8 dicembre (giorno dopo) USA e ingh. dichiarano guerra a giappone \t quindi con Perl Harbour diventa effettivamente guerra globale \t infatti italia e germ. subito sostengono giappone

\v

Questo è un cambiamento radicale di orientamento \t gli USA negli anni 30 continuavano ad avere politica isolazionista, derivata da WW1 e accentuata da Wall street \\
Roosevelt cambia pero questo assetto \t nel 40 viene rieletto per la 3 volta, e si muove in una direzione diversa \\
Gia nel marzo 41 aveva fatto approvare la legge affitti e prestiti, legge che forniva materiale bellico ai paesi antifascisti a condizioni vantaggiose (in particolare a ingh.) \\
Per quattro anni legge garantisce quindi materiale bellico conveniente \t cosi USA diventa "arsenale della democrazia" \t fornisce armi senza entrare nel conflitto \\
Questa legge sancisce avvicinamento tra usa e ingh. \t al punto che nell agosto del 41 viene firmata la Carta Atlantica \\
Essa è un documento firmato tra usa e ing. che dicono che collaboreranno nella linea della tutela dell'autodeterminazione dei popoli alla fine del conflitto, e nella linea di garantire la pace a livello internazionale dopo dominio nazista \t danno per scontato che vinceranno, che finira il dominio nazista e si impegnano a stare dalla stessa parte

\v

È l'attacco di P. Harbour che fa entrare in guerra \t questo fa decollare la produzione industriale \\
In 4 anni gli occupati passeranno da 4 milioni a 17 \\
Tutto questo in un europa che è dominata dalla germania nazista \t nel 42, il nazismo è alla sua massima espansione: controlla francia ed est, anche baltici e grecia e ha alleata l'italia \\
Hitler vuole quindi costruire una nuova europa, basata sulla supremazia della germania e la subordinazione di popoli sottomessi ma anche alleati (anche l italia sara sottomessa, comandera solo la germania) \\
La supremazia doveva essere della razza ariana, mentre gli slavi dovevano fornire solo mano d'opera \t questo spiega brutalita del dominio nazista in polonia e URSS \\
Tra i civili sono morti 6 milioni di sovietici e 2 milioni di polacchi \t e nel ghetto di Varsavia, su 450 mila persone i sopravvissuti sono 50 mila \\
Nell aprile del 43 il ghetto di Varsavia insorge e Hitler reprime

\v

In tutti i paesi si sviluppano movimenti di opposizione e liberazione \t la resistenza al nazismo avra forme diverse da paese a paese \\
In francia DeGoul guida la resistenza \t si trova a londra e parla con la radio \\
In Jugoslavia ci saranno i partigiani comunisti, guidati da Tito \t e Tito liberera la Jugo. da solo, tedeschi verrano scacciati senza aiuto e garantira indipendenza alla jugo. \\
Anche in Germania c'è opposizione, con l'operazione valchiria \t opposizione all'interno dell esercito \\
Quando hitler nazifica l'esercito, gli alti gradi dell esercito non sono contenti \t vedono sottomissione \t ma Hitler controlla perche vince, esercito è esaltato \\ 
Pero nell esercito cova opposizione \t che sfocia in una serie di attentati \t nel luglio del 44 il piu clamoroso fallisce, Von Staufenten era il generale che mette la bomba vicino a Hitler ma non esplode \\
In italia la resistenza è un fenomeno politico e complesso, che si caratterizza per contrasti politici interni \\
La resistenza si manifesta anche con collaborazione con eserciti alleati \t a partire dal 43 ci sara collegamento tra partigiani e alleati \\
La resistenza in italia sara anche un laboratorio politico \t usciranno dalla resistenza i grandi partiti di massa del dopo guerra \t il parito socialista e comunista esistevano gia \\
La resistenza pero aveva schierato tutte le forze di opposizione \t le brigate garibaldi erano le + numerose, ma anche forze liberali, e democristiani \t la democrazia cristiana nasce qua \\
La resistenza ha un doppio valore, tutti i grandi politici sono stati paritgiani (es. Pertini)

\v

Entrano in guerra gli stati uniti dopo Perl Harbour, ed era gia stat firmata carta atlantica \\
Quando arrivano usa le guerra svolta \t hanno potenziale di uomini e mezzi enormi \t quindi i paesi del patto tripartito iniziano a perdere \\
Il 42 si apre con grandi successi del giappone in realta \t che occupa grandi territori sul pacifico, ma usa rovesciano situazione \\
Nel maggio 42 \t vittoria nel mare dei coralli, poi vittoria di Guadal Canal nelle isole salomone (nel 43??) \t e vittoria nelle isole?? \\
Conquista di Guadal canal l'offensiva giapponese viene fermata, e giappone assumera posizione difensiva
\end{document}
