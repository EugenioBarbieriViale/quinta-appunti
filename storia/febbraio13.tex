\documentclass[12pt]{article}

\usepackage[a4paper, total={6in, 8in}]{geometry}
\usepackage{textcomp}

\begin{document}
\setlength{\parindent}{0pt}

\def \t {\textrightarrow}
\def \v {\vspace{1em}}
\def \bi {\begin{itemize}}
\def \ei {\end{itemize}}
\def \s[#1] {\section*{#1}}
\def \ss[#1] {\subsection*{#1}}
\def \sss[#1] {\subsubsection*{#1}}

\s[Il crollo di Wall Street]
Politica protezionistica e xenofoba dell'america \t periodo di benessere, tutti investono \t ovvero anche i piccoli risparmiatori investono tutto quello che hanno \t chiedono mutui alle banche, investono e guadagnano \\
Le persone si espongono finanziariamente \t nella fiducia di questa cresita inarrestabile \\
Quando i repubblicano salgono al potere, i repubblicani agevolano questo clima ottimistico:
\bi
    \item riducono le imposte dirette \t ma cosi salgono quelle indirette, e i + poveri sono colpiti e i + ricchi favoritri
    \item riducono la spesa pubblica \t rinunciare a finanziare programmi sociali
    \item riducono il tasso di interesse \t quindi permettono maggiore accesso al credito
\ei
Quindi ricaduta negativa sociale \\
La vendita dei titoli in borsa raddopia in un anno \t max crescita è tra il 27 e il 29 \\
Poi pero si genera una bolla speculativa \t è uno scollamento tra il piano economico e il piano finanziario \t quando vado a quotare un azienda in borsa, gli viene attribuito un valore a seconda del suo profitto \\
Il valore delle azioni che compro è garantito dalla validita economica dell'azienda \t ma ora il piano finanziario (la crescita in borsa) diverge piano reale \\
Valore in borsa non corrisponde al valore reale \t alcune aziende si fermano, ma il loro valore in borsa continua ad aumentare \t e quindi porta ad investire di piu \\
Nel momento in cui qualcuno si rende conto di questo, uno rivende le azioni \t e se lo fanno in molti, le azioni valgono meno \t meno valgono le azioni, piu ci si rende conto che il valore in borsa non corrisponde al valore reale dell'azienda \t e quindi diminuisce ancora il valore delle azioni \\
Questo significa che quindi le aziende non hanno piu soci, e quindi crolla \t inoltre chi aveva comprato un azione a mille, ora si trova che la rivende a 1

\v

L'economia si era gia cominciata a fermare \t infatti c'è una saturazione del mercato: i beni di consumo durevoli non vengono cambiati tutti gli anni (es. auto, lavatrice etc.) \\
Appena la lavatrice viene immessa nel mercato, ne vendo milioni \t poi continuo a produrne milioni, ma nessuno le compra perche sono beni che dureranno molto \\
Questo genera una sovrapproduzione \t ci sono prodotti in eccedenza, che non vengono venduti \t ma hanno un costo \\
Tutte queste novita saturano il mercato \t tutti hanno comprato tutto, quindi non andranno a comprare altre cose \\
Nel frattempo ci sono ancora fasce povere anche se benessere generale \t es. agricoltori dell'est devono affrontare una crisi agricola data dal calo del prezzo dei prodotti agricoli ?? \\
Anche lo stipendio degli operai viene calmierato \t e il potere dei sindacati è ridotto, perche lo stato vuole difendere l'ordine sociale e le manifestaizoni venogno represse \t il sindacato del lavoro viene indebolito e le rivendicazioni sociali non vengono portate avanti

\v

Quindi il pinao finanziario continua a crescere in modo illusorio, mentre il piano economico effetivo si è fermato \t le aziende sono in crisi \\
C'erano pero segnali di crisi \t gia all'inizio del 28 \t agricoltura era in crisi e salari erano abbassati \\
Arriviamo quindi all'estate 29, in cui la produzione industriale viene rallentata molto \t ma i titoli salgono \\
In ottobre si percepisce che le quotazioni in borsa sono gonfiate e si prevede che le azioni cadranno \t quindi alcuni iniziano a liquidare i titoli \\
Quindi si genera il panico \t grandi investitori iniziano a vendere, quindi si capisce che ci sara un disastro \\
Il giovedi nero \t vengono cedute 13 milioni di azioni, e il 29 ottobre 16 milioni \\
Quindi grande fortune finanziare vengono polverizzate in qualche giorno 

\v

Ci sono conseguenza devstanti quindi \t i piccoli risparmiatori non possono piu pagare debiti e non hanno neanche soldi, perche erano tutti in borsa \\
Gli agenti di borsa, devono dichiarare il fallimento \\
Le banche invece crollano \t avevano fatto credito, ma non ottengono piu nulla \t quindi molte chiudono, e molti perdono i loro soldi \\
Riduzione della liquidità \t molti soldi non rircolano piu \t e quindi ora lincenziano, mentre prima avevano decuratato gli stipendi \\
La produzione industriale scende molto, la disoccupazione è enorme \t la gente non compra piu niente, quindi la produzione si contrae, e quindi la disoccupazione cresce ancora \t si genera un circolo vizioso

\v

Anche il governo aveva fatto degli errori \t il governo avrebbe dovuto diminuire il tasso di interesse per agevolare chi era in crisi ancora di piu, inoltre svalutare la moneta e mettendo in circolazione + soldi \\
Questo pero non viene fatto, perche Hoover, non vuole l'inflazione con la paura che avrebbe accresciuto il divario sociale ?? \\
Inoltre si fa portavoce di una tesi economica che queste crisi erano normali \t i mercati ogni 10 anni subiscono un rallentamento economico, come in una sorta di ciclo economico \t quindi lui non agisce \\
Nel 1930 fa provvedimento protezionista, che voleva tutelare il mercato interno con tariffe \t e governo approva con l'opposizione di molti economsti statunitensi

\v

Questo ha una ricaduta alta \t prima di tutti in germania: fine triangolazione finanziaria \\
Falliscono banche tedesche, non regge concorrenza straniera e nel 32 alla conferenza di Losanna, viene ratificata dall'europa l impossibilita della germania di rispettare le sanzioni di guerra \\
La germania di Bruning è impreparata a gestire questa crisi \t la germania non si aspettava questo crollo americano \t quindi quando wall street crolla, anche la germania cade

\v

Questo crollo ha ripercussioni anche in italia \t il crollo del 29 accentua il protezionismo nello stato, e si va verso autarchia economica \\
Gran bretagna decide di abbandonare il rapporto oro-sterlina \t quindi svaluta la sterlina, cosi da rendere le sue merci piu conpetitive sul mercato internazionale \\
Inoltre abbandona la tradizione secolare liberale attuando un sistema di prefernze imperiali \t nelle colonie vengono preferiti i prodotti inglese \\
La francia invece non vuole svalutare (probabilmente per motivi di prestigio) \t quindi la sua ripresa sara rallentata ed esportazioni francesi verrano rallentate \t si riprendera solo nel 37/38 \\
Germania pagava debiti di guerra, ma non riesce piu \t quindi francia e ing. vengono indeboliti

\v

Per wall street vengon accusati i grandi figure dell imprenditoria e finanza \t che hanno creato la bolla speculativa \\
Alle presidenziali Hoover era considerato troppo vicino a questa realta \t e nasce Roosvelt \\
Lui propone una politica meno schiacciata dalla grande industria, + attenta ai benestanti normali \t lui è considerato l uomo della provvidenza, perche aveva avuto una poliomelite ma cammina lo stesso \t incarna la forza di volonta \\
Gia in campagna elettorale lui parla del New Deal \t e stra vince nel novembre del 32

\ss[New deal]
Vuol dire nuovo corso della politica statunitense \\
Costituisce un brain trust \t ovvero mette insieme specialisti di vari settori per creare un programma politico che faccia uscire usa dalla crisi \\
Voleva rilanciare gli investimenti delle aziende e incentivare il consumo dei cittadini \\
Assume atteggiamento pragmatico \t non ha progetto finale, ma davanti alle diverse situzioni e problemi li risolve volta per volta \t agisce nel concreto \\
Sceglie politica intervenzionista \t liberalimsmo va abbandonato \\
Il new deal prevede interventi diretti e indirettti \t e si ispira a Kanes, economista inglese, che è l'antitesi del liberismo \t Roosevelt viene influenzato, troppa liberta nel mercato genera crisi

\sss[Interventi indiretti]
Vanno a genrare condizioni indirette per migliorare le attivita produttive e la vita delle famiglie \\
Svaluta la moneta, favorsice le esportazioni (perche cosi riduce la sovrapproduzione \t manda nel mercato estero i prodotti che sono in eccesso) \\
Poi vara una legge che concede dei premi in denaro agli agricoltori che avrebbero limitato i loro raccolti \t vuole contrastare la sovrapproduzione agricola \\
Lo stato copre cio che l'azienda agricola non produce \t compensa quindi \t in modo che una minore quantita di prodotti nel mercato, faccia salire il valore e il prezzo \\
Anche legge per aziende \t che è un codice di disciplina della produzione \t gli imprenditori vengono sottoposti a dei vincoli per limitare produzione e per rinunciare al lavoro infantile, al lavoro nero \\
Gli imprenditori devono accetare un salario minimo, e devono definire orario di lavoro comune \t si chiama IRA?? \\
Viene varata una riforma fiscale che prevede tassazione progressiva \t aliquote + alte per redditti + alti \\
INoltre promulga il Partner Act, che sancisce il diritto di sciopero, all organizzazione sindacale e a una contrattazione collettiva (dei vari lavori, es. contratti uguali per i metalmeccanici, altri per insegnanti etc.) \\
Questi sono interventi indiretti perche sono attraverso leggi \t stato non è proativvo

\sss[Interventi diretti]
Stato agisce come agente nel mercato \t e fornisce lavoro \\
Per esempio si vuole sfruttare il bacino del fiume Tennesee \t che aveva potenzialita idrolettrica alta \\
Viene create agenzia, che deve sfruttare le risorse di questa regione \t vuol dire costruire dighe, quindi si creano posti di lavoro \\
Viene varato un piano di enormi cantieri pubblici in tutta america \t e questo genera lavoro e diminuisce disoccupazione \\
Cosi l'industria puo anche disporre di energia elettrica con prezzi + bassi \\
Viene varato anche un sistema pensionisto ed assistenziale \t che prevede sussidi, protezioni sociali per i lavoratori \t tutto finanziato con i soldi dello stato e le tasse \\
Inoltre Roosevelt è una perosna molto credibile \t mantiene promesse dalla campagna elettorale \\
Anche se new deal non risolve la crisi, per cui ci vorranno anni \t ma lo stesso lui dara grande fiducia, e terra le conversaioni del gabinetto \\
Una volta alla settimana, lui parlava con la radio alla nazione e raccontava cio che lo stato faceva, e la situzione \t il presidente si relazionava direttamente con la popolazione \\
Questo rida fiducia al popolo \\
New deal non tocca pero donne e neri \t e ci sono categorie quindi escluse \t ma ha successo perche lui aveva carisma

\v

Alle elezioni del 36, lui ottiene il 60 per cento dei voti \t mentre il candidato americano molto sotto \\
Ha pero degli oppositori \t si dovre scontrare con la corte suprema, perche lui applica politica interventista \t nega ogni forma di liberismo con il new deal \\
Mentre l'eocnomia americana era da sempre e si vantava del suo liberismo \\
Questo ha dato fastidio alle grandi lobby (es. bancarie), che erano infastidite dalla presenza dello stato \t l'elite finanziaria e imprenditoriale è contro questi provvedimenti \\
Infatti, con alta disoccupazione, permette di avere bassi salari e tanto mano d'opera \t il fatto che stato impedisse questo non piaceva \\
Tutti loro si riferiscono alla corte suprema, che aveva il diritto di valutare la costituzionalita delle leggi varate dal congresso \\
La corte suprema è un organo giudiziario molto conservatore, e respinge come incostituzionali questi provvedimenti contrari all'elite finanziaria \t dicono che c'è eccessivo interveno nella liberta dei cittadini \\
Lo scontro è molto accesso, e viene minacciato di impeachment \\
Quindi Roosevelt si appella al popolo \t la corte suprema rappresenta i ceti abbienti, che non vuole ridistribuzione \t c'è quindi scontro molto duro che si risolve nel 37, in cui Roosevelt sostituisce giudici con giudici + favorevoli al suo assetto

\v

New deal funziona quindi alla lunga \t pero c'è ripresa della fiducia, quindi fuaniona \\
Il new deal cambia totalmente il rapporto tra politica ed economia, e il rapporto tra societa civile e lo stato \\
Infatti dagli anno 30 nascono le basi del Welfare State (stato di benessere) = sistema in cui lo stato assicura protezioni fondamentali al popolo \t prima del new deal questo non c'era \\
Col new deal cresce anche la burocrazia \t c'è un espansione della amministrazione pubblica, che in america non esisteva \t c'è un cambiamento della faccia dello stato \\
Cambia rapporto tra stato ed economia anche perche il new deal non nega la proprieta privata, ma lo stato deve essere un regolatore del sistema economico \t stato regola per evitare tensioni sociali \\
Non è intervento forte, ma regolazione dei rapporti \t stato si fa solamente garante della correttezza dei rapporti, per evitare crisi \\
Infine, i sindacati vengono riammessi da Roosevelt \t non sono i nemici, ma devono essere degli importanti interlocutori politici e non solo economici \t i lavoratori sono ampie fasce della popolazione e possono essere utili per canalizzare conflitto sociale \\
Per stabili raccordi con popolo uso sindacato \t esso puo fare da mediatore per far rientrare conflitto sociale per vie istituzionali \\
Nel 40 i disoccupati sono milioni, quindi questione non è completamente risolta, ma migliorata \t e la guerra fara il suo effetto \\
Con l'industria bellica a pieno regime, risolve disoccupazione \t viene riassorbita

\s[Seconda guerra mondiale]
1 settembre del 39, Hitler invade la polonia \t 3 settembre inghilterra e francia dichiarano guerra alla germania \\
Il 17 sett. anche URSS entra in guerra con il patto Molotov-Ribb. \t invade la polonia \\
Inizia per la polonia un regime di occupazione feroce \t eccidi e violenze, e nella foresta di Katin in russia, i sovietici uccidono migliaia di ufficiali polacchi e li buttano in fosse comuni \\
La germania deve fronteggiare sul fronte occidentale e si mobilita anche ad occidente \t prova a invadere la francia, ma viene fermata a Vagineu \t e c'è una situazione di stallo perche non si combatte \t "chiamata la strana guerra" \\
La linea Vagineu è molto lunga, e gli eserciti si fronteggiano in modo statico \\
Sul fronte occ. gli eserciti si studiano, mentre a oriente l'armata rossa avanza \t invade i paesi baltici, perche sono indispensabili per i confini \t e invade anche in Finlandia \\
Stalin va quindi ben oltre il patto molotov-etc. \\
Hitler cosi nell'aprile del 40 attacca danimarca e norvegia \t danimarca easy, mentre norvegia ci vogliono due mesi \\
Fa eusto perche vuole:
\bi
    \item miniere di ferro e basi navali
    \item vuole accerchiare ing.
\ei
Crede nella guerra lampo (bismark) \t hitler mette in campo tutte le forze, e vuole concludere velocemente \t perche non ha soldi per portare avanti guerra di logoramento
\end{document}
