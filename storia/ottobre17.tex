\documentclass[12pt]{article}

\usepackage[a4paper, total={6in, 8in}]{geometry}
\usepackage{textcomp}

\begin{document}
\setlength{\parindent}{0pt}

\def \t {\textrightarrow}
\def \v {\vspace{1em}}
\def \bi {\begin{itemize}}
\def \ei {\end{itemize}}
\def \s[#1] {\section*{#1}}
\def \ss[#1] {\subsection*{#1}}
\def \sss[#1] {\subsubsection*{#1}}

\s[Recuperare oggi]
\s[Rivoluzione d'ottobre]
Nel giugno del 17 si svolge il primo congresso panrusso dei soviet \t bolscevichi in minoranza \\
Nell estate del 17 la russia vive un momento difficile \t il governo provvisorio decide di lanciare un offensiva contro le forze austru-tedesche \t offensiva fallisce subito: soldati ammutinano \\
Perche erano stanchi, non allenati, poche armi \\
Nel luglio del 17 operai e soldati manifestano a Pietrogrado in piazza \t vogliono impedire la partenza per il fronte di altri soldati \t governo voleva rifare un offensiva \\
Proteste sedate con truppe fedeli al governo \t bolscevichi partecipano \t anche lenin c era, che viene arrestato e poi scappa in finlandia \\
Poi il generale Cornilov \t conservatore e capo dell'esercito \t tenta colpo di stato e marcia su ? \t vuole abbattere il governo \\
Vuolge scioglere i soviet e imporre la legge marziale \t ripropone un regime zarista \t ma suo piano fallisce, perchè i soviet di pietrogrado organizza una milizia operaia \\
Inoltre molte unita militari dell esercito si schierano con il governo provvisorio \t e anche i ferrovieri stanno con il governo, quindi truppe non possono arrivare a pietrogrado \\
Inoltre i bolscevichi si schierano a difesa del governo (governato da Kierensky) \t non volevano regime conservatore \\
I bolscevichi escono rafforzarti \t appaiono come i difensori della democrazia \\
Si rafforza la loro posizione come consenso e nel soviet \t arrivano anche nei soviet di Mosca e ?? \\
Inoltre governo provvisorio aveva dimostrato che non aveva delle risorse per sventare un colpo di stato \t non è forte \\
Nasce quindi l'idea che gli unici che possono assumere il controllo (mantenendo liberta, pace, democrazia) sono i bolscevichi 

\v

Nel frattempo Lenin scrive "Stato e rivoluzione" \t parla di dittura democratica del proletariato e abbattimento dello stato \\
Dice che è democratica perche dittatura rappresenta la maggioranza della popolazione \\
Trozky è l ideologo della rivoluzione \t e organizza al Guardia Rossa, che supporta \\
La rivoluzione viene anche organizzata pubblicamente \t i giornali e la pop. parlavano di questa rivoluzione, che si sapeva che stava arrivando \t mancava solo una data \\
Questo mostra come il governo era impotente \t non poteva fare nulla \\
Lenin sceglie la notte del 24 per la riv. \t voleva presentarsi al congresso dei soviet cosi \\
Le guardie rosse invadono ?? \t riv. non violenta pero \t la popolazione non aveva neanche la percezione della rivoluzione \\
Guardie rosse occupano solo le sedi del potere, e neanche violentemente \\
Inoltre l'esercito si era dichiarato neutrale \t non interviene \\
La sera del 25 ottobre viene occupato il Palazzo d Inverno \t che è il simbolo della rivoluzione

\v

Si organizza poi il congresso panrusso \t pietrogrado è nei bolscevichi, a mosca ancora qualche resistenza \\
Lenin è quindi al potere \t ma c'è convinzione diffusa che i bolscevichi non avrebbero retto \t si pensava che governo sarebbe durato due settimane e poi sarbbe arrivata la controrivolzuzione \\
Il 26 ottobre Lenin al soviet da inizio al potere sovietico 

\v

Fa approvare:

\ss[Decreto sulla pace]
\bi
    \item dichiara la volonta di uscire dalla guerra
    \item pace immediata senza annesione territoriali \t non piace a tutti 
\ei

\ss[Decreto sulla terra]
Abolisce la proprieta privata e le proprieta terriere vengono confiscate \t lenin cerca l appoggio delle masse contadine

\v

Crea consiglio dei commissari del popolo \t composto solo da bolscevichi \t nel che fare era previsto un partito monolitico \\
Questo governo nazionalizza le banche e consegna provvisoriamente la gestione delle fabbriche agli operai \t prima di nazionalizzare anche le fabbriche \\
Questo consiglio deve guidare il paese a un assemblea costituente \t ci si aspetta questo, una nuova costituzione democratica \\
Ma le prime azioni dei bol. vengono lette come un tentativo totalitario \t i menschevici e social rivolzionari abbandonano i soviet \\
Rimangono solo i social riv. + radicali \\
Nel novembre vengono fissate le elezioini per assemblea \t i bol. pero non vincono e ottengono il 25 per. dei voti \t 58 per. dei social rivoluzionari (i populisti)\\
C'è anche spacatura geografica \t campagne con i social riv. \t mentre esercito e le grandi citta con i bol. \\
Lenin è deluso da questi risultati \t voleva attraverso l assembla cost. un organo ?? \\
Si riunisce una sola volta e per un giono \t i bol. capiscono che l assemblea è contraria a dare potere assoluto ai bol. \\
Quindi Lenin sciogli l assemblea \t capisce che non la puo manipolare \\
Giustifica lo scioglimento \t perche rispetta la dittatura del proletariato \t necessaria per comunismo \\
Nel passaggio da stato a societa democratica ci deve essere un momento di dittatura \t se no come si fa a confiscare la proprieta privata, a ridistribuire la terra \t serve atto di autorita \\
Marx prevede questo \\
Scioglimento dell assemblea corrisponde a questo atto di prepotenza necessario pero

\v

I bol. cosi si alieano le simpatie delle altre forze socialiste e rivoluzionarie \t e anche del popolo \t i soviet verrano ridimensionati per il loro potere \\
Soviet diventano solo organi esecutivi degli ordini di Lenin

\ss[In europa]
I governi dell intesa si preoccupano con la chiusura dell assemblea \t i governi dell intesa vedono la pericolosita del nuovo governo \\
Intellettuali e aristocratici scappano \t gli esuli volontari saranno + di un milione 

\v

Nel ? Lenin firma la pace di Brest-Litvosk \t trattato che segna uscita dalla guerra con condizioni dure
\bi
    \item cede alla germania regioini tra caucaso e bielorussia
    \item riconosce indipendenti ? e ? 
    \item rinuncia a paesi baltici e polonia
\ei
Social riv. polemizzano per questa pace \\
RECUPERARE

\v


Lenin doveva affrontare ?, ? e ? \\
Lenin ha vinto perche i contadini l'hanno appoggiato \t perche era male minore \t alternativa era zar \\
Il popolo sta dalla sua parte \t la rivoluzione si fa col popolo \\
Consolidamento del potere interno molto difficile \\
A partire dall aprile \t le potenze occidentali volevano eliminare il potere bolscevico \t intesa voleva di nuovo la russia nella guerra \\
Volevano ricostituire governo democratico \t e volevano elminare un pericolo precedente \t se li riv. avesse avuto successo, questo poteva alimentare gli operai negli stati occidentali \\
Truppe anglo-francesi e usa \t sbarcano nel mar nero e russia occidentale \\
Anche giapponesi vanno a Vladimostok \t per espansionismo \\
Truppe estere supportano forze antirivoluzionarie \t supportano l armata bianca (zarista) \\
Guardia rossa diventa armata rossa con Trozky
\end{document}
