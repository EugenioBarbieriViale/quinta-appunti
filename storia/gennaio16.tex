\documentclass[12pt]{article}

\usepackage[a4paper, total={6in, 8in}]{geometry}
\usepackage{textcomp}

\begin{document}
\setlength{\parindent}{0pt}

\def \t {\textrightarrow}
\def \v {\vspace{1em}}
\def \bi {\begin{itemize}}
\def \ei {\end{itemize}}
\def \s[#1] {\section*{#1}}
\def \ss[#1] {\subsection*{#1}}
\def \sss[#1] {\subsubsection*{#1}}

\s[La repubblica di Weimar - Locarno]
Stresermann è il grande fautore della stabilizzazione della repubblica \t al punto che la germania partecipa alla conferenza di Locarno, dove vengono firmati gli accordi con la francia (sottoscritit poi da tutti i paesi europei) \\
Cosi si pensa di andare verso un periodo di distensione delle relazioni internazionali \t nasce lo spirito di locarno, in cui si pensa che ci sara periodo di pace \\
La germania riconosce che l'alsazia e la lorena appartengono alla francia, accetta la smilitarizzazione della zona della renania: quindi accetta i confini occidentali del trattato di versaille \\
Ma il trattato di Locarno non prevede alcuna definizione dei confini orientali \t "la germania si regolera per le frontiere orientali trattando con i paesi interessati" \t infatti Hitler andra ad oriente \\
Hitler ha giocato su cio che questi trattati internazionali permettevano \\
Nel 26 la germania entra nella societa delle nazioni (che aveva sottoscritto il trattato) \t vuol dire che germania si rinserisce tra potenze \\
Inoltre germania fa anche accordi con URSS \t in questo trattato, quindi, la germania smette di essere vista come la sconfitta e si riinserisce

\v

Nuovo accordo di Parigi del 28 (patto Brian(ministro francese)-Kellog(? americano)) \t entrera in vigore nel 29, sottoscritto poi anche dal Giappone e altri paesi \\
Poi viene ratificato da 62 Paesei (anche URSS) \t nel patto si rinuncia alla guerra per risolvere questioni internazionali 

\v

Tra il 24 e il 28 c'è quindi stabilita economica, e le cose sembrano funzionare \t ci sono alcuni segni di fragilita pero \\
Nel 25 per esempio muore Ebert (social democratico), quindi nuove elezioni per eleggere presidente \t in cui vince Hindenburg, che era un maresciallo ed era popolare perche aveva combattuto nella WW1 \t pero era monarchico \\
Non ha senso: un presidente monarchico non ha senso, perche non crede nella repubblica stessa \\
Questo succede perche la sinistra decide di sostenere dei candidati separatamente, e non di confluire tutti i voti in una figura \\
Nel 28, poi, si eleggono le elezioni politiche \t sinistra rafforzata, ma non ha maggioranza \t quindi governo è di coalizione, con social democrazia, cattolici \t e Muller governa ?? \\
Queste forze hanno per alcuni aspetti visioni comuni, ma su altri no \t e ora si mostra questa divisione: lo stato non ha soldi \\
Quindi i popolari e i democratici chiedono di ridurre le spese pubbliche (es. sussidi etc.) ma social democratici (la maggioranza nella coalizione) non vogliono toccare queste spese \\
Si crea quindi frattura, che pero era invetabile

\v

Nel 29 arriva il crollo di Wall street \t non arrivano + soldi americani, quindi economia tedesca precipita \\
L'industria funzionava, pero i capitali che circolavano erano sempre americani \t economia tedesca non era ancora autonoma \\
Disoccupazione aumenta (4 milioni e mezzo nel 31), crollo della produzione industriale del 50 per cento e piccole industrie crollano \\
Muller non riesce a gestire la situazione \t quindi si fa strada l'idea che serva un governo + deciso \t quindi l'opposizione si irrigidisce (sia a destra sia a sinistra, e si radicalizzano) \\
Su fronti diversi, le opposizioni hanno lo stesso obbiettivo: far crollare la repubblica di Weimar \t destra voleva potere forte, mentre sinistra rivoluzione comunista \\
Nel 30 Muller si dimette \t questo segna la rottura tra i social democratici e le forze di centro \\
Governo passa a Bruning, che era molto vicino a Hindenburg \t e causera la distruzione della repubb.

\v

Estrema destra e sinistra vogliono affossare la repubblica, mentre social dem. si sono staccati dal centro \\
Bruning governa fino al 32 \t vorra risanare le casse pubbliche diminuendo le spese pubbliche \t per fare questo governa per decreti legge \\
Progressivamente togli potere al parlamento \t anche lui si appella all'articolo 48 della costituzione di Weimar \t in crisi, presidente ha poteri eccezionali \\
Bruning motiva questi decreti legge con lo stato di emergenza \t quindi governa in modo antidemocratico \\
Questo provoca nel parlamento una forte opposizione \t parlamento chiede le dimissioni \\
Nel 30, quando era stato eletto, aveva sciolto il parlamento \t e indice le elezioni per rafforzarsi \t ma elezioni sono tese e nazisti e comunisti si scontrano \\
Ma il partito nazional socialista ottiene 6 milioni e mezzo e i comunisti 4 milioni \t che entra in governo con un grande numero di seggi \t quindi i nemici (anche i comunisti) della repubblica di weimar sono nel governo \\
La social democrazia vuole difendere la rep. di weimar, quindi sostengono bruning \t ma lui con Hindenburg continua a procedere con decreti legge \\
La destra conservatrice pensa di poter manovrare hitler per togliere bruning \\
Quindi nel 32, si tengono le elezioni presidenziali \t e uno dei candidati è Hindenburg, ma il candidato della destra è Hitler \\
Hitler perde, ma ottiene lo stesso 13 milioni di voti \t vince Hind. perchè aveva la coalizione (cattolici, democratici, etc) \\
Hitler pero acqusisce consenso popolare \\
Hind. licenzia bruning, e chiama dei governatori di destra + duri ?? \t intanto la crisi economica continua \\
Dal punto di vista sociale, è praticamente di guerra civile

\v

Nel 32, la popo. vota due volte \t ma governo non esce rafforzato, mentre partito nazista si \t nel ? ottiene il 37 percento dei voti \t mentre social dem. il 21 \\
Quindi Hitler rivendica il ruolo di cancelliere \t con l appoggio di esercito ed industriali \t solo un potere forte come il suo puo riportare l'ordine \\
Quindi Hind. cede, e nel 33 affida a Hitler il compito di creare un nuovo governo = fine della rep. di Weimar (non formalmente) \\
Hind., che è ancora presidente, muore nel 34 \t e a quel punto Hitler, che era cancelliere, diventa anche presidente \\
Diff. con Mussolini: lui non perde tempo e subito incendia il parlamento nel 27 febbraio del 33 \t non ha una fase legalitaria 

\ss[Mein Kampf]
Quando lui va in carcere, scrive il programma del partito e la summa ideologica del nazismo \\
È un documento soprattutto contro (tutti) e critica tutti \t non va bene nessuon perche l'idea di stato che ha Hitler e quella praticamente di Hegel \\
Lo stato per lui è un unita suprema che non puo essere messa in discussione da altre forze \t tutto cio che va contro non puo esserci \\
In questo documento c'è:
\bi
    \item lotta al liberalismo \t perche parla di liberta individuale, ma nello stato che ha in mente non c'è \t individualita divide lo stato  \t lotta al lib. vuol dire lotta anche alla democrazia, parlamentarismo, etc. che sono il segno del decadimento di una societa
    \item lotta al marxismo \t anche il marxismo frammenta, perche la storia politica viene letta come una lotta tra classi \t divisione nello stato, in cui societa è divisa in classi
    \item odio personale per la classe operaia, senza ragione politica 
    \item lotta agli ebrei \t vengono accusati di aver sfruttato la situzione critica della germania per arricchirsi \t loro controllano la finanza internazionali, e hanno guadagnato sulla pelle dei tedeschi
\ei
L'elemento centrale è la razza \t la germania era fatta solo dalla razza ariana, superiore e perfetta, che ha il merito di aver fatto tutte le cose positive fatte fino ad ora \\
Questa razza ha bisogno di un capo, che è unico, è supremo e assoluto, è la fonte dell'autorita (non è soggetto ad alcun altro potere), e il capo condurra la razza alla conquiesta del Lebensraum \\
Per fare questo, ha bisogno di un spazio vitale \t necessario alla propserita dell impero \\
Questo spazio vitale va verso est \t perche gli slavi sono una razza inferiore, e c'è anche l URSS, che è l'incarnazione nel marxismo e ha dato potere agli operai \\
Per costruire una nuova germania, bisogna inoltre costruire una forza militare \\
Il razzismo è la forza di coesione del nazismo \t l'esaltazione della razza tedesca è cio che unisce \t il culto della razza è fondamentale \\
L'ideologo del nazismo è pero Rosenberg \t che scrive un libro che parlava di sottouomini, che dovevano essere schiacciati dalla dottrina nazista \t nel "Mito del ventesimo secolo" \\
"Mein Kampf" viene scritto quindi nei primi anni 20 \t quindi era noto

\v

Immediatamente sopprime tantissimi giornali, e da carta bianca alle SH, ovvero le squadre d'assalto, che si scagliano contro la sinstra
\end{document}
