\documentclass[12pt]{article}

\usepackage[a4paper, total={6in, 8in}]{geometry}
\usepackage{textcomp}

\begin{document}
\setlength{\parindent}{0pt}

\def \t {\textrightarrow}
\def \v {\vspace{1em}}
\def \bi {\begin{itemize}}
\def \ei {\end{itemize}}
\def \s[#1] {\section*{#1}}
\def \ss[#1] {\subsection*{#1}}
\def \sss[#1] {\subsubsection*{#1}}

\s[Guerra civile in spagna]
È il momento che cambia le sorti dell'europa \t sara la prova generale della WW2 \\
Hitler prova dei nuovi aerei, che usera nel corso della guerra poi \\
La guerra civile di spagna nasce nello scontro tra le destre e le sinistre \t la spagna era arretrata, alfonso 13 aveva favorito un colpo di stato fascista per mantenere il governo \\
Nel 31 ci sono le amministrative, dove i repubblicani vincono \t il re viene cacciato e viene proclamata la repubblica \\
Dal 31 al 36 c'è un alternarsi al governo tra forze di sinistra e di destra \t questo in un contesto difficile, con continue insurrezioni anarchiche e socialiste \t e vengono represse nel sangue \\
Le destre salgono poi nel 33 al potere \t che era preoccupante visto situazione in italia e germania \t quindi le forze di sinistra decidono di allearsi e creare fronte compatto, per impedire un ascesa fascita \\
Nasce cosi il fronte popolare \t in cui sono anche liberali e democratici \t perfino i comunisti entrano, su invito di Stalin \t che è preoccupato per la diffusione del fascismo \\
Tutte le forze non di destra fanno parte \\
Nel 36 il fronte popolare vince le elezioni \t e da vita a un governo difficile: è una coalizione pero molto frammentaria \\
È un governo moderato di stampo liberal-democratico \t e hanno l'appoggio esterno dei socialisti \\
La difficolta si trova subito \t le aspettative del popolo sono molte alte e si aspettavano delle riforme sociali, ma non è un governo socialista \t è anche liberalre \\
La componente moderata ostacola le riforme sociali \t quindi rinascono proteste e rivolte \t e vengono colpite le forze piu moderate, quindi il clero, i proprietari terrieri, i conservatori \\
La spagna presenta una tensione sociale molto profonda, e sembra essere sull orlo della rivolzuione \\
La destra armata interviene, e attua un colpo di stato \t che viene portato dalla falange, che è un movimento non nuovo nella visione spagnola \t era gia stato fondato da Antonio I de Rivera (figlio del primo dittatore) \t ed è il partito fascista \\
È un movimento autoritario, ostile a ogni forma di democrazia e si ispira alle forme del fascismo italiano \t e non è un movimento di massa in spagna, è un organizzazione di cui Franco si serve (ma che non sara fondamentale neanche nella sua ascesa) \\
Il sostegno di franco arrivera infatti dall'esercito, e lui infatti è un generale \t e l'esercito è supportato dalla chiesa \t esercito e chiesa sono le forze della dittatura di franco

\v

Franco pero appartiene lo stesso alla falange, e alla morte di Primo de Rivera guida la falange, e portera al suo interno anche altri gruppi di destra \t diventera quindi come un nuovo partito, con un aspetto religioso e militare marcato \\
Franco è un generale alla guida della falange che si trova in marocco al momento \t guida alcuni reparti dell'esercito li \\
Il colpo di stato prevedeva l'ammutinamento delle truppe in marocco, dell'esercito quind in spagna \t e poi va in spagna \\
L'aviazoine tedesca e italiana aiuta questo spostamento dal marocco \t Hitler aiutera con i mezzi, mussolini con gli uomini \\
C'è un altra lettura di questo evento: quando vince il fronte popolare, la destra non vuole cedere e il colpo di stato è il tentativo di rovesciare il risultato elettorale \\
Franco non vuole riportare la spagna all'ordine \t fa colpo di staot perche destra non vuole cedere potere al fronte popo. che ha vinto le elezioni 

\v

Nel 36 scoppia quindi guerra civile \t da un lato c'è la falange, dall'altro l'esercito del governo liberare-democratico \\
Org. filofasciste si uniscono a franco \t la falange accresce le sue fila \\
La guerra diventa subito una questione internazionale \t il governo repubblicano è consapevole della sua debolezza (esercito sta con franco, che si unisce alle truppe del marocco) \\
Quindi il governo chiede aiuto dalla francia, che nel 36 ha governo socialista \t ma la francia non interviene, e anche l'inghilterra:
\bi
    \item francia perche vuole evitare polemiche interne, e vuole evitare tensioni con inghilterra
    \item ing. temeva una vittoria della sinistra radicale \t e non era cosi favorevole a combattere franco
\ei
Quindi francia e ing. promuovono un patto internazionale di non intervento (le potenze europee si tengono a non intervenire), ma italia e germania intervengono lo stesso anche se avevano partecipato al patto \\
Si schierano con l'intenzione internazionale, ma poi fa il contrario \t la guerra di spagna ha avuto successo grazie alle potenze dell'asse \\
L'unica nazione che interviene è la URSS \t non interviene direttamente, ma interviene come Comintern (attraverso la 3 internazionale) e aiuta il fronte repubblicano, mandando aiuti \\
Vengono organizzate le brigate internazionali \t ovvero reparti di militari antifascisti che vengono da tutti i paesi del mondo \t molti giovani partecipano e partono per la spagna \\
Nonostante la collaborazione di queste brigate, i repubblicani sono inferiori militarmente \t e sono anche frammentari: ci sono anarchici, socialisti, democratici, liberali, repubblicani, coministi filo-sovietici = grosse divisioni \\
A volte divisioni scoppiano anche in conflitti interni all'esercito repubblicano

\v

Vince quindi franco, ed entra a Barcellona e poi Madrin nel 39 \t quando cade madrid, la spagna è nelle mani di franco \t inizia la dittatura franchista, che durera fino al 75 \\
La guerra di spagna è quindi un momento imporante \t ed è uno dei momenti di vicinanza tra germania e italia \t si va verso un progressivo asservimento di mussolini sotto hitler \\
Italia, dopo le leggi razziali, si sta piegando a hitler \t non c'era componente antisemita cosi forte in fascismo

\v

Nel frattempo pero Hitler porta avanti il piano espansionistico intorno ai suoi confini \t nel 34 aveva provato ad annettere Austria con Anschluss, ma non c'era riuscito \t nel 38 viene invitato da un governo filonazista a annetter austria \\
In realta non è proprio invasione, lui risponde all'invito dell'austria \t il terreno è favorevole, e in austria c'è propaganda nazista forte \t infatti viene acclamato \\
Poi pero hitler vuole anche i sudeti \t confine con cecoslovacchia \t e questa regione era stata annessa a ceco. con Versaille, ma vivono solo tedeschi \\
Era una regione anche molto militarizzata dalla ceco. \t pero Hitler lo rivendica, sulla base della naturale appartenenza al mondo tedesco di coloro che erano di stirpe germanica \\
Il governo cecosl. non è disposto a trattare \t ma Hitler minaccia invasione, c'è rischio reale di una guerra \t la cecosl. aveva anche firmato un accordo di mutua assistenza con francia, che sarebbe dovuta intervenire \\
La francia non vuole assolutamente intervenire \t e vuole quindi mediare \\
L'aiuto francese non arriva, e mussolini viene chiesto di mediare \t nella confeenza di monaco del 38, ci sono Chanderlain, Daldier, Hitler e Muss. \\
Mentre i rappresentatni della cecosl. vengono fatti aspettare in anticamere \t conf. di Moncaco è il momento + vergnoso dei cedimenti delle potenze democratiche agli assolutismi \\
Qui i sudeti venngono ceduti \t e mussolini era gia favorevole a questa annesione, mentre fr. e ing. non vogliono nuovo coflitto mondiale \\
Il resto della cecoslv. deve pero rimanere indipendente \t e la cecosl. non puo che accettare \t mentre l'europa va verso la guerra \\
Chanderlain e Daldier vengono considerati salvaguardatori della pace, ma c'è consapevolezza di essere a un passo dalla guerra

\v

Pochi mesi dopo Hitler invade Boemia e Moravia \t mentre slovachia rimane formalmente indipendente, ma diventa stato vassallo della germania \\
Queste annessioni vengono accompagnate da aggressione violenta alle popolaizoni non germaniche, che pero vivevano li \t Hitler li considera degli occupanti illegittimi \\
Poi nel 21 marzo del 39, hitler chiede a polonia la citta di danzica \t ma cosi le potenze democratiche intervongo, e in parlare inghilterra \t dice che se hitler invade, dichiarano guerra \\
Ma lui non ci crede, oppure non gli interessa \t ma aveva bleffato molte volte \\
Nel 1 sett. 39 lui invade, e intervengono potenze e scoppia guerra

\v

Pero prima: Mussolini nel frattempo nel aprile 39 occupa l'albania, che viene anessa all'impero \t e cosi muss. rivendica anche Tunisi, Corsica, Nizza, Savoia \\
Il 22 maggio viene firmato il patto d'Acciaio \t che è diverso dall asse roma berlino: questa è un alleanza militare, mentre l'altro era un patto di vicinanza \\
IN caso di guerra offensiva o difensiva, le due potenze si sarebbero aiutate a vicenda \t ma era chiaro che hitler andava verso una guerra, mentre italia non era minacciata \\
Questo vincola quindi l'italia ad entrare in una qualsiasi guerra che hitler avrebbe voluto \\
Ma non è l unico patto: c'è anche il patto con l'URSS \t Hitler vuole iniziare la guerra, ma deve avere le spalle coperte \t se Stalin avesse intervenuto in polonia, sarebbe stato rischioso \\
Hitler e Stalin si odiavano, ma si accordano e il 23 agosto del 39 firma il patto Molotov-Ribbendroff, nominato il "patto scellerato" \t è un patto di non aggressione per 10 anni \\
I due stati si impegnano a non aggredirsi per 10 anni \t e aveva un protocollo segreto, che verra reso pubblico negli USA nel 48 pero verra confermato dall URSS solo nel 90 \\
Il protocollo segreto prevede la spartizione della polonia \t ci doveva essere un compromesso tra i due \\
Prevedeva anche che la zona occidentale della liturania erano il massimo della germania, mentre estonia e lettonia e finlandia erano le zone di espansione sovietica, e Stalin si prende anche la Vessaravia

\v

Hitler invade la polonia e quindi il 3 settembre ing. e francia dichiarano guerra

\s[Il crollo di Wall Street]
Gli USA escono da WW1 in condizione migliore dell europa \t in suolo statunitense non viene combattuta la guerra, e hanno capitali in eccedenza \\
Tra il 22 e il 28 c'è crescita economica senza precedenti, con produzione industriale altissima che cresce quantitivamente e qualitativamente \t quindi cresce la produzione di massa in tutti i settori \\
Per questo gli USA plasmano anche dei consumatori di massa \t questi beni devono essere acquistati \\
Nasce quindi una innovativa produzione pubblicitaria, e i magazzini per distribuiri i beni \t e nascono i pagamenti in rate: tutti possono comprare \\
Si diffono molti prodotti nuovi, e soprattutto la radio \t Roosevelt la usa \t e hanno anche energia elettrica \\
Questo accade in una condizione di isolamento \t gli usa sono la prima potenza mondiale, e si da quindi vita a una posizione isolaizonista (che c'era gia stata) torna a essere presente \\
Nel 20 i repubblicani vincono con Harding, il cui programma accogli volonta isolazioniste della cittadinanza \t si vede che non entrano nella societa delle nazioni, e che non firmano il trattato di versaille \\
Il paese quindi non si preoccupa delle politica estera \t e questo porta a razzismo e xenofobia \\
Sacco e Vanzetti erano due anarchici che vengono condannati a morte anche se non erano colpevoli di una rapina, e nel 27 vengono giustiziati \\
E si limita anche ogni forma di migrazione, e il KuKlux Klan cresce in questo periodo \\
Nasce anche la legge sul probizionismo, perche i grandi bevitori erano i tedeschi e gli irlandesi \t quindi si voleva colpire queste minoranze \\
A partire dal 21 viene eseguita, ed è controproducente \t alimenta anzi il contrabbando, corruzione e mercato nero \t sono gli anni di Al Capone

\v

Questo è il clima in cui si trova questa crescita degli investimenti in finanza \t il cittadino medio è portato a fare investimenti anche rischiosi grazie a clima ottimista \\
Le previsioni sui profitti sono ottimistiche, e i prezzi delle azioni crescono e tutti sono portati a comprare le azioni
\end{document}
