\documentclass[12pt]{article}

\usepackage[a4paper, total={6in, 8in}]{geometry}
\usepackage{textcomp}

\begin{document}
\setlength{\parindent}{0pt}

\def \t {\textrightarrow}
\def \v {\vspace{1em}}
\def \bi {\begin{itemize}}
\def \ei {\end{itemize}}
\def \s[#1] {\section*{#1}}
\def \ss[#1] {\subsection*{#1}}
\def \sss[#1] {\subsubsection*{#1}}

\s[Fascismo agrario]
Squadre fasciste nelle campagne vanno a far rispettare l'ordine, dove lo stato si assenta \\
Si pensa che Mussolini sia manovrabile \t non si capisce la gravità della situazione \\
Mussolini da una veste legale al suo operato \t e capisce di poter ambire a una rappresentaza politica \t decide quindi di:
\bi
    \item cerca di evitare il dilagare delle squadre \t vuole limitare i RAS (ovvero i capi delle squadre fasciste) \t ma non ci riuscira
    \item vuole trasformare il suo movimento in un partito \t ambisce a rappresentanza politica 
\ei
Nasce quindi il partito fascista \t mossa che permette a mussolini di entrare in parlamento \\
Le elezioni vengono indette da giolitti \t e fa un errore, ovvero quello di accettare la composizione di liste comuni di centro, liberali e fascisti \\
Fascisti si candidano con i liberali e centristi \t ma non si tiene conto che fanno violenza, che continua durante la campagna \\
Giolitti vuole ridimensionare i popolari e comunisti, che pero sono in parlamento \\
Alla fine pero 35? seggi vanno ai fascisti \\
Mussolini vuole essere un leader credibile, e quindi fa azioni moderate \t cerca di limitare squadre e mantenre dialogo con altri partiti (di centro), per mantenere consenso

\v

Poi fascismo diventa anche cittadino \t citta del nord vedono l'affermazione del partito fascista alle elezioni amministrative, e quindi le citta vengono amministrate dai fascisti \\
Controllano gia il nord italia e le campagne, ma il governo è ancora un altro \t quindi il govenro deve essere fascista \\
Marcia su roma è l'affermazione del potere fascista ??????? CHIEDERE IMPORTANTE \\
Governo bonomi dura 6 mesi e viene sostituito dal governo Facta \t ultimo governo democratico prima del fascismo \\
Viene appoggiato dai liberali e dai popolari, che però hanno scarsa intesa \t Facta inoltre non era determinato \\
Mussolini invece diventa monarchico \t il san sepolcro prevedeva la repubblica \t abbandona critica al capitalismo e diventa liberalista \\
Abbandona anche l'anticlericalismo \t abbandona completamente il programma di sant sepolcro \\
Critica inoltre il partito popolare, come se fosse eversivo contro la chiesa \t in realta è il fascismo che era anticlericale \\
Tutto questo lo rende + credibile \t quindi il 24 ottobre del 1922 si va a prendere il governo \t a napoli raduna le camice nere perche prepara la marcia su roma \\
Facta lo sa, e quindi chiede al re di proclamare il stato d'assedio e mobilitare l'esercito \t ma alla fine non lo firma \\
Quindi il 28 ottobre camice nere entrano a roma (lui era a milano pronto a scappare) \t marcia condotta da 4 leader \\
Il 30 ottobre gli viene conferito l'incarico di formare un nuovo governo \\
Dal 22 al 25 mussolini guida governo di coalizione, in cui sono presenti i popolari (anche contro i luigi sturzo) \t fase legalitaria del fascismo \\
Chiamata cosi perche mussolini si muove ancora dentro le strutture dello stato liberale \t non c'è trasformazione verso stato totalitario \\
Si consolida nelle strututre dello stato liberale \\
Ha l'appoggio del re, degli apparati di stato (esercito, magistratura che lo appoggia, industriali, proprietari terrieri) e anche dei liberali che erano alla maggioranza \\
Liberali pensavano che il fascismo poteva portare a + ordine e contrasto al socialismo \\
Mussolini si presenta il 16 novembre in parlamento e fa discorso, che viene votato da 300 deputati \t votano contrari socialisti e comunisti \\
Lui ora deve mantenere le sue promesse, che sono di destra e conservatrici \t esatto contrario del programma sant sepolcro \\
C e pressione di finire violenza squadrista \t quindi Mussolini legalizza lo squadrismo creado la milizia volontaria per la sicurezza nazionale \\
Invece di smantellare le squadre, le trasforma in una struttura paramilitare dello stato \\
Nel 23 mussolini perde appoggio dei popolari \t che tornano alle idee di Sturzo, che era antifascista \\
Mussolini è credibile anche dal punto di vista internazionale \t si propone come il garante dell'ordine e della pace, e come colui che puo impedire un altra ondata rivoluzionaria (come in russia) \\
Stati democratici come la francia e inghilterra sono a favore \t era prioritario sconfiggere il comunismo \\
Anche se le violenze squadriste continuavano (come Amendola viene bastonato perche liberale) \\
Riforme:

\ss[Riforma della scuola di Gentile]
Nell'aprile del 23 \t giovanni gentile, filosofo neo-idealista che adersice al fascismo \\
Metteva in primo piano la formazione umanistica e classica \\
Liceo classico \t permetteva accesso a tutte universita \t mentre liceo scientifico solo a facolta scientifiche \\
Poi c'erano istituti professionali/tecnici/magistrali per lavorare \\
Gentile aderisce al fascismo perche era hegeliano \t mentre Croce firma gli intellettuali antifascisti \t completamente contro, ma visione filosofica era la stessa

\ss[Legge Acerbo]
Legge con cui viene promossa la riforma elettorale \t approvata dal parlamento, con solo liberali e fascisti (popolari se ne erano andati) \\
Legge dice che il partito che alle elezioni avesse ottenuto la maggioranza relativa, avrebbe ottenuto i 2/3 dei seggi al parlamento (quindi la maggioranza assoluta) \\
Maggioranza relativa verra stabilita (dopo) al 25 per cento \\
Legge maggioritaria e inequa, che permette al fascismo di governare \\
La legge acerbo permette di andare alle elezioni del 24 \t dove viene presentato il listone fascista \t lista unica fascista, con anche cattolici e liberali \\
Salandra, Orlando e altri erano in questa lista, mentre Giolitti si oppone \\
Lista unica \t dai voto alla lista e quindi vanno al capolista (se non esprime preferenze) \\
Giolitti si oppone, ma le forze di opposizione sono troppo divise \t non riescono a fare fronte al fascismo \\
Inoltre elezioni sono durante clima di violenza e intimidazione \t il listone ottiene il 65 per cento dei voti

\v

A questo punto Matteotti si oppone a questa legge e alle elezioni \t era un socialista che nel 24 fa un discorso in parlamento \\
Nel discorso denuncia le violenze, scandali elettorali, inequita della legge acerbo (in particolare, che non poteva essere una legge garantista) \\
Matteotti viene rapito e ucciso, ritrovato vicino a roma \t gli esecutori furono arrestati, mentre i mandanti mai scoperti \\
Momento matteotti = momento di minore consenso \t ma opposizione è ancora troppo divisa e non riesce \\
Il popolo italiano pero si risveglia \t partito comunista vuole org. sciopero generale ma viene rifiutato \\
I parlamentari non fascisti decidono di non parteciapare piu, e si riuniscono in altra sede \t sarebbero tornati solo dopo l'abolizione della milizia volontaria e ripristino della legalita \\
Chiamata la cessione dell'Aventino \\
I parlamentari speravano che intervenisse il re \t italia spera che il re ritiri l'incarico a mussolini \t i savoia vennero poi accusati \\
Lui chiude il parlamento, poi lo riapre (ma loro non tornano) \t quindi mussolini incolpa loro, non li aveva cacciati \\
Quindi il 25 tiene un discorso alla camera, dove afferma che se il fascismo è una associazione delinquere si prende la resp. \\
Questo discorso chiude la fase legalitaria e apre la fase totalitaria
\end{document}
