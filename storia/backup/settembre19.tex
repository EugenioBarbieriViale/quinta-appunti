\documentclass[12pt]{article}

\usepackage[a4paper, total={6in, 8in}]{geometry}
\usepackage{textcomp}

\begin{document}
\setlength{\parindent}{0pt}

\def \t {\textrightarrow}
\def \v {\vspace{1em}}
\def \bi {\begin{itemize}}
\def \ei {\end{itemize}}
\def \s[#1] {\section*{#1}}
\def \ss[#1] {\subsection*{#1}}
\def \sss[#1] {\subsubsection*{#1}}

\s[Giolitti]
Eta giolittiana parte nel 1903 quando diventa presidente del consiglio \t in realta il vero primo era tra i due governi Crispi \\
Eta giolittiana finisce nel 1914 \t ma quando ci saranno problemi ritornera, perche rimane un riferimento pericolo \\
Egli imprime ai suoi governi e alla politica italiana i suo caratteri specifici \\
È un liberale moderato riformista (sinistra storica), ma aggiunge dei caratteri propri \t il primo è il trasformismo \\
I suoi caratteri:

\ss[Trasformismo]
Durante l'eta giolittiana diventa il carattere principale \\
Trasformismo = ricerca delle alleanze parlamentari pur di mantenere la maggioranza (perfino con l opposizione) \\
Non è una prassi limpidissima \t in parlamento sono stato letto con un programma, poi in parlamento mi accordo con altre ideologie \t quindi devo derogare dal programma con cui sono stato votato \\
Esempio piu eclatante \t nel 1903 chiede a Filippo Turati (partito socialista riformista) di entrare a far parte del suo governo \\
Il socialismo nel 1903 era pero veramente all opposizione \t non aveva nulla da condividere ocn liberalismo \\
Ma lo fa perche:
\bi
    \item vuole mostrare di essere aperto a tutte le forze politiche
    \item vuole indebolire il partito socialista \t si aspettava un rifiuto, ma lo stesso questo presuppone che c'era un dialogo \t mette in diffiolta turati
\ei
Ovviamente Turati rifiuta \\
Un altro esempio sara il patto Gentiloni \t giolitti firmo un accordo con i cattolici (insieme ai socialisti costituiva l opposizione)

\ss[Dimissioni]
Si dimette ogni volta che il parlamento non è con lui \t l eta giolittiana prevede 3 governi (1903-1905, 1906-1910, 1911-1914) \\
Si dimette mettendo al suo posto uomini di fiducia, e quando il problema finisce ritorna \t lo puo fare perche ha sempre la maggioranza \\
Lui avanza una proposta sulle ferrovie ? non voluta da nessuno \t proposta non passa, ma si dimette prima \t il presidente del consiglio è un suo uomo \t la proposta passa, perche in realta era giolitti che non piaceva, non la proposta \\
Lo fa l'ultima volta nel 1914 \t si dimette perche aveva promosso troppo a troppa gente \t al suo posto mette Salandra, che non è un burattino nelle mani di giolitti \\
Salandra affronta la "settimana di sangue" \t proposte forti, e poi scoppia la prima guerra \t scenario politico cambia, quindi non riesce a tornare (nel 1920 viene richiamata) \\
Quello del 1903 è il secondo governo in realta 

\ss[Politica del doppio volto]
Politica non molto limpida \t adotta una strategia al nord e un altra al sud \\
Al nord è politico molto liberale e aperto \t attua una politica del non intervento, e appoggia la nascita del triangolo industriale (torino-milano-genova) \\
Al sud invece ha un atteggiamento repressivo e lascia il controllo totale ai prefetti, che diventano come dei governatori \\
Le politiche di crescita del nord non vengono attuate \\
Viene definito "il ministro della malavita" da un parlamentare chi??? 

\ss[Politca del non intervento]
Non intervento in materia sociale \t dove nascevano scioperi e manifestazioni e occupazioni delle fabbriche \t non interviene \\
Perche ci devono essere degli sfoghi sociali \t queste manifestazioni, una volta fatte si spengono \\
Nel 1904 viene procalamato il primo sciopero della storia d italia, e lui non agisce \\
Non agisce con la froza, ma mantiene l ordine pubblico \\
Poi scioglie le camere e si va a nuove elezioni \t lui stravince, mentre i socialisti perdono elettoralmente, che avevano promosso lo sciopero, perche fa la figura del buono \\
Giolitti si trova a che fare con due forze d'opposizione: cattolici e socialisti 
\bi
    \item ala riformista: + moderata, guidata da Turati \t sono favorevoli a un dialogo con il governo \t sono ancora marxisti, ma la rivoluzione non si poteva attuare in quel momento \t non c'era la disposizione
    \item ala massimalista: + radicali, nessun dialogo con governo, vogliono attuare la rivoluzione \t uno dei suoi leader sara Mussolini
\ei
Lo sciopero generale del 1904 viene proclamato dalla dirigenza del partito socialista, che nel 1904 era massimalista \\
Il problema di giolitti era che ala riformista e massimalista si alternavano a guidare \\
Con turati c'erano un dialogo, ma quando massimalisti il dialogo cessa completamente \\
Se avesse intervenuto, sarbbe scoppiata una rivoluzione? \t a posteriori no, perche la classe operaia non era ancora abbastanza forte in italia \\
Socialismo pero cresce e nel 1906 viene fondata la CGL \t sindacato non rivoluzionario, caraterizzato da riformismo \\
A questo sindacato si affianca anche la Federterra \t federazione degli agricoltori (gia nel 1901 nasce, ma ora cresce \t agricoltura rimane fondamentale per l'italia, anche con sviluppo industriale) \\
+ del 55 percento della popolazione lavorava la terra \\
Nel 1910 gli industriali danno vita a Confindustria = associazione rappresentativa del mondo industriale \\
Giolitti cerca sempre di mediare \t non vuole mai attaccare le organizzazioni, non è mai repressivo \\
Dopo 1904 gli industriali pero sono meno propensi a mediare con le forze operaie \t erano infastiditi dallo sciopero \\
Durante giolitti gli scioperi sono tantissmi

\v

Governi giolitti varano anche leggi a favore dei lavoratori, che regolamentano il lavoro femminile e dei bambini. Inoltre viene garantito un giorno di riposo a settimana \\
Nel 1903 impone assicurazione obbligatoria sugli infortuni sul lavoro \\
Tra 1903/4 \t crea leggi che favoriscono ceti rurali \\
Vuole tutelare di piu il mondo dei lavoratori \t capisce che bisogna guidare la popolazione per nuove vie, che non sono quelle della repressione \\
1911 legge sull'istruzione elementare pubblica \\
Poi viene istituito l INA \t istituto nazionale assicurazioni \t stato ha il monopolio su tutte le assicurazione della vita, i cui utili venivano dedicato a sostenere i lavatori vecchi che non potevano + lavorare 

\v

Tra 1907/8 arriva crisic economica in europa \t anche in italia \\
Affligge le banche e le industrie \t accresce la diffindenza verso i sindacati \t le industrie sono in crisi, quindi i sindacati non servono \\
Massimalisti indicono nuovo sciopero \t va avanti diversi mesi \t l'intento era di scatenare la rivoluzione, ancora una volta non interviene, dopo un po si scioglie

\ss[Con la Chiesa]
L'altra forza di opposizione era la chiesa \t il non expedit vale ancora (di Pio 9) \\
Le forze cattoliche non hanno rappresentanza politica \t è grande problema elettorale per giolitti \\
Non ha forza antisocialista con cui allearsi, che pero non esiste in parlamento \\
Il problema è la chiesa, che non consente un partito cattolico \\
Giolitti cerca di convincere il papa della minaccia del socialismo al potere \t è interesse della chiesa che cio non succeda \\
Nel 1904 ottiene un'attenuazione del "Non expedit" \t minimo di apertura \\
Papa è Pio 10 \t giolitti si impegna a non assumere carattere anticlericale \t permette ai cattolici di partecipare alle elezioni a titolo personale \\
Non esiste una lista cattolica \t ma i cattolici possono entrare in parlamento attraverso altri partiti, non attraverso un partito cattolico \\
Questo permette l'ingresso di alcuni cattolici, che vengono eletti, ma a titolo individuale rappresentano gli interessi della chiesa \\
Giolitti mantiene governo grazie a questa nuova forza \t i massimalisti avevano fatto lo sciopero \\
Seconda attenuazione del "Non exp" nel 1909 \t per stesso motivo: isolare ala rivolzionaria \\
Nel 1919 verra abrogato definitivamente, quando viene fondato il "Partito popolare" da ??? \\
Nel suo 4 governo firma il patto gentiloni con i cattolici (ma non con la chiesa) \t viene firmato con ottorino gentiloni, rappresentante dell'"Unione elettorale cattolica" \t non è un partito e non è espressione della chiesa \\
Al congresso di reggio emilia, l'ala massimalista guida il partito \t che decide di bloccare la produzione industriale e prova la rivoluzione \\
Agisce su due fronti:

\sss[concede suffragio universale maschile nel 1912]
Vara una riforma elettorale \t toglie ai socialisti il cavallo di battaglia, che chiedeva il suffragio \t giolitti puo essere favorito dal suffragio e un modo di far vedere che il governo è dalla parte del popolo \\
Per analfabeti e disoccupati sopra i 30 anni, per istruiti o leva militare sopra 21 anni \\
Vanno a votare ora i contandini, sopratutto del sud che erano stati esclusi = nuovo scenario elettorale

\sss[Patto Gentiloni]
Non tutti sono contenti \t i piu conservatori soprattutto \\
Nel 13 firma patto per avere + conseno dai conservatori \\
I cattolici avrebbero votato per i liberali, e il suo governo si sarebbe impegnato a non assumere posizioni anticlericale \\
Nel 1913 vince alle elezioni, grazie a questo patto \\
Porta in parlamento tanti deputati, che pero sono molto frammentati \t la maggioranza non è in grado di governare, non riesce a varare riforme e no transformismo \t no mediazione tra cattolici e socialisti \\
Cosi Giolitti si dimette e sale Salandra 

\ss[Politica coloniale]
Conquista della Libia sotto il 4 governo di giolitti \\
Si inserisce nel contesto dell'imperialismo \t decenni tra 800 e 900 sono definiti imperialisti \\
Imperialismo = unica ragion d'essere è la volonta di potenza, che è diversa dal colonialismo (che ha anche ragioni di natura economica e demografica) \\
Anche italia si butta in questa impresa, con la guerra di Libia \t si vuole anche riscattare le sconfitte africane precedenti \\
C'è anche un momento di instabilita internazionale \t quindi tenta di sfruttarla \\
1911: la libia era governata dalla turchia, che pero ha problemi nel controllare problemi nei balcani (nasce Lega Balcanica supportata dalla russia, che rivendica autonomia balcanica dalla turchia) \\
Bisogna fare un passo indietro \t Bismark dopo porta una politica di consolidamento, che vede anche la convocazione del Congresso di Berlino (1878) (e poi la conferenza) \\
Qui le potenze europee si incontrano per definire delle questioni \t in particoalre come spartisi i territori balcanici \\
Mentre la conferenza del 95 piu sulla spartizione dei territori africani
\end{document}
