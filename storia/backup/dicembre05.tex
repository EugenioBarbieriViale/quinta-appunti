\documentclass[12pt]{article}

\usepackage[a4paper, total={6in, 8in}]{geometry}
\usepackage{textcomp}

\begin{document}
\setlength{\parindent}{0pt}

\def \t {\textrightarrow}
\def \v {\vspace{1em}}
\def \bi {\begin{itemize}}
\def \ei {\end{itemize}}
\def \s[#1] {\section*{#1}}
\def \ss[#1] {\subsection*{#1}}
\def \sss[#1] {\subsubsection*{#1}}

\s[Fascismo]
1925 si inaugura la fase totalitaria del fascismo \t fase in cui mussolini trasforma lo stato italiano in uno stato totalitario:
\bi
    \item a partito unico
    \item le strutture del partito coincidono con quelle di governo (come in Stalin) \t partio diventa organo dello stato
    \item l'unica ideologia consentita attraversa tutta la vita \t nell educazione, radio, giornali \t è pervasiva e non può non essere condivisa
\ei
Non tutte le dittature sono totalitarismi \t solo 3: stalinismo, nazismo, fascismo \t sono questi 3 storicamente i totalitarismi \t sono i 3 modelli \\
In realta quello italiano è un totalitarismo imperfetto \t perchè c'è la monarchia, formalemente stato italiano è monarchico \\
A partire dal 25 fa approvare le "Leggi fascistissime" \t segnano la trasformazione dello stato italiano in una dittatura \\
Governo fascista è:
\bi
    \item antidemocratico
    \item autoritario
    \item conservatore \t appoggia capitalismo e condanna comunismo
\ei
Le leggi fascistissime vanno dal 25 al 28 \t sono tutte le leggi comprese in questo periodo \t e vengono ispirate da Alfredo Rocco \t il codice Rocco è il codice fascista \\
Nel 25:
\bi
    \item unico parito riconosciuto \t ovvero il partito nazionale fascista \t tutti gli altri fuori legge
    \item il presidente del consiglio viene sostituito al capo del governo (presid. del cons. fa fronte a una assemblea ed è tra pari) \t il capo del governo viene riconosciuto dal re e non dal parlamento e ha molti + poteri dei ministri \t inoltre ha potere legislativo e parlamento non ha significato
    \item eliminate elezioni comunali (no sindaco, ma il podesta, nominata dal governo direttamente = presenza forte sul territorio)
    \item limitata liberta di stampa e di associazione \t no sciopero
\ei
Nel 26 sciolti i partiti di opposizione e chiusi i giornali antifascisti
Poi nasce l'OVRA (vedere acronimo) \t polizia segreta, doveva individuare ed arrestare gli oppositori \t poi veninvano giudicati dal Tribunale speciale per la difesa dello stato (istituito nel 26) \\
Questo tribunale condannera a morte molte persone e accumula 28.000 anni di carcere 

\v

Allo stesso tempo normalizza il suo partito \t ha sempre il problema delle squadre, che non sono utili ne opportune \\
La direzione del parito viene tolta a Farinacci (che era capo dello squadrismo ed era il + violento) e tutte cariche all'interno del parito vengono assegnate da mussolini \\
Parito fascista viene quindi riorganizzato \t nelle diverse regioni ci sono i prefetti (scelti da mussolini) \t mentre il vertice è il gran consiglio del fascismo, ovvero l'unico organo del parito \\
Qui si discuteva la linea politica \t e capo era mussolini \\
Nel tempo il gran consiglio assumera anche compiti di rilevanza costituzionale e di governo \t per esempio il gran consiglio sceglieva il capo di governo (il parlamento è un organo inutile) \\
Il culmine è nel 28 \t stato diventa veramente totalitario con nuova legge elettorale, che prevedeva che il gran consiglio del fascismo presentasse una lista unica di candidati \\
Se la lista avesse ottenuto la metà + 1 dei voti, sarebbe stata approvata in toto \t i cittadini possono solo approvare la lista proposta dal partito, non possono scegliere \\
Diventa come un plebiscito, non un elezione \\
Prime elezioni con questa legge nel 29 ci sono dei contrari (con 1.5 per cento), mentre poi nel 0.15 per cento (non andavano a votare)

\v

A tutto questo si affianca una propaganda per costruire il consenso \t si costruisce facendo vedere che lo stato fa cose per il popolo, e poi andando a diffondere l'ideologia \\
Diventa obbligatorio avere la tessera del partito \t per lavori pubblici bisogna averla \\
Poi creata l'opera nazionale del dopo lavoro \t organizzava il tempo libero dei lavoratori e li intratteneva gratuitamente (con feste, gite) \\
Nasce anche il CONI, di stato, doveva stimolare l'attivita fisica \\
Poi vengono creati dei gruppi per eta, in cui venivano educati \t per esempio i Balilla, che venivano educati al fascismo \\
Poi cerano i GUF \t gruppi universitari fascisti \\
Fascismo ha componente utopistica marcata \t fascismo vuole creare uomo nuovo, ovvero il nuovo dux romano \t si esalta la potenza della roma imperiale \t roma che ha anticipato tutta la gloria del fascismo, e quelle sono le radici dell italia \\
Per questo mussolini crea saluto romano, il fascio littorio, si fa chiamare dux \\
La societa doveva essere maschile e maschilista, e si riprende il cameratismo nato nelle trincee \t la prima guerra mondiale viene esaltata dal fascismo \t aveva generato cameratismo e nobilta e virilita \\
I socialisti e i liberali erano degli omuncoli, erano intellettuali, chiacceroni e non sportivi \t antitesi dell'uomo fascista \\
Inoltre fascismo si presenta sempre come rivolto al futuro \t fascismo porta alla modernita, a un italia nuova \\
Viene fondata l'EIAR \t ente radiofonico gestito direttamente dal regime e diventa un mezzo di comunicazione potente \t attraverso radio fa conoscere cio che vuole diffondere \\
Le radio erano nei locali pubblici, ma poi anche nelle case \t e poi c'era anche il cinema, che viene sfruttato \\
Nasce l'istituto luce \t guidato direttamente da mussolini \t ogni gestore di una sala cinematografica era costretto a proiettare i cinegiornali dell'istituto luce \\
Nel 37 nasce il MINCULPOP \t ministero della cultura popolare \t aveva l'obbiettivo di controllare e orientare tutti gli aspetti della vita culturale italiana \t ma non ce la fa perche c'è la chiesa, che è una grande difficolta \\
In molte zone d'italia c'erano le parrocchie, che erano dei centri di aggregazioni non controllati dal fascimo \t per questo mussolini cerca un accordo con la chiesa \\
Sarebbe difficile governare senza la chiesa o contro la chiesa \t popolo italiano è cattolico

\v

Nel 29 firma i patti Lateranensi (firmati nei palazzi del Laterano) \t firmati da cardinale Gasparri e Mussolini \t cosi si chiude la questione romana del 1870 \\
Tre punti:

\ss[Trattato]
È politico \t tra due stati: chiesa riconosce lo stato italiano e roma come capitale, e ottiene invece la sovranita sul vaticano \\
Comprende la basilica di san pietro e i palazzi intorno

\ss[Concordato]
Documento che regola i rapporti stato-chiesa, e stabilisce che:
\bi
    \item cattolic. è la religione di stato \t stato italiano è confessionale \t quindi viene insegnata obbligatoriamente nelle scuole \t catt. è il fondamento e coronamento dell'istruzione pubblica
    \item riconosciuti effetti civili del matrimonio
    \item chiesa puo amministrare i suoi beni, scegliere i vescovi, ed avere delle organizzazioni cattoliche intorno a se (ma dovevano essere apartitiche)
    \item vescovi pero dovevano giurare fedelta allo stato italiano
\ei
Il papa è Pio 11 \t e fa errore: si dichiara soddisfatto degli accordi e parla bene del fascismo \t Papa dice che Mussolini è l'uomo della provvdenza \\
Sturzo si oppone e dice che è una vergogna \t e la chiesa non avrebbe dovuto firmare \\
Solo due anni dopo infatti il regime fascista cercherà di demolire le organizzazioni cattoliche, con le squadre fasciste violentemente \t ma non ce la farà

\ss[Convenzione finanziaria]
Accordo che impegna l'italia di versare un indennita al vaticano, che aveva perso lo stato pontificio

\v

Costituizione li ha mantenuti nella clausola 7 \t i padri costituenti avevano scritto che si potevano cambiare pero \\
Craxi nel 84 firma nuovo accordo, che rende stato laico \t cosi religione nella scuola non + obbligatoria
\end{document}
