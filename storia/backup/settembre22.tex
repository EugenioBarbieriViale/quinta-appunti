\documentclass[12pt]{article}

\usepackage[a4paper, total={6in, 8in}]{geometry}
\usepackage{textcomp}

\begin{document}
\setlength{\parindent}{0pt}

\def \t {\textrightarrow}
\def \v {\vspace{1em}}
\def \bi {\begin{itemize}}
\def \ei {\end{itemize}}
\def \s[#1] {\section*{#1}}
\def \ss[#1] {\subsection*{#1}}
\def \sss[#1] {\subsubsection*{#1}}

\s[Giolitti - politica esteri]
Le potenze europee puntano alla conquista di diversi territori \\
Giolittti mantiene l'italia nella triplice alleanza, ma non si tiene amici anche i paesi della triplice intesa \t politica trasformista anche in affari esteri \\
Mantiene delle amicizie con tutti \\
Giolitti vuole applicare la politica coloniale nella libia, sotto controllo turco \\
Turchia si trova in diffocolta sul fronte balcanico \\
Politica coloniale in libia perche:
\bi
    \item prestigio internazionale
    \item soddisfare le esigenze delle industrie e banche 
    \item accontentare l'opinione pubblica
\ei
Prima la direzione era l'etiopia \t lui cambia per vantaggio turco \\
Nel ?? la francia aveva occupato il marocco \t firmato accordo con l'italia che riconosceva, ma concedeva all italia il diritto di prelazione \\
Poi la francia prende il marocco \t poi italia dichiara guerra alla turchia per conquistare la libia \\
All inizio opposizione è facile \t ma poi la guerriglia locale crea resistenza \\
L'italia reagisce in modo molto duro, ma la resistenza libica continuava \\
Poi l italia va a conquistare il Dodecaneso (12 isole) e i turchi temono che l italia voglia occupare i dardanelli + mianccia sempre balcanica \\
A quel punto i turchi cedono e nel 1912 cedono, firmando la pace di Losanna \t che prevedeva che il dodecaneso venissere restituito (accadra nel 40) \\
Conquista della libia pero era costata molto, ma era priva di risorse da sfruttare \\
Inoltre le popolazioni locali dell entroterra resisteranno sempre \t l'italia puo accedere solo alla zona costiera 

\v

Le banche invece traggono vantaggio \t la popolazione invece non guadagna nulla \\
La pop. non è contenta perche non ha avuto benefici reali, non sono contenti neanche i nazionalisti, che volevano uno scontro maggiore contro i turchi \\
Cosi Giolitti supera il momento di difficolta, ma poi nel 14 si deve dimettere 

\s[Questioni internazionali europee]
Gli anni dell'imperialismo sono fine 800 fino alla prima guerra mondiale \\
Nel 1890 Bismark si ritira e governa Guglielmo II, molto nazionalista, che da vita a una politica estera aggressiva \\
Nel 1907 nasce la triplice intesa con francia, gran bretagna e russia \t è un successo per la francia, perche bismark l'avevava isolata \\
La germania invece si trova in difficolta, perche è accerchiata dall alleanza \\
La situazione europea entra in crisi perche Guglielmo II vuole far vedere che la germania è uno stato forte \t ed entra inquestioni internazionali che non lo riguardano \\
Per esempio partecipa nelle "Crisi marocchine" \t crisi diplomatiche e militare a seguito del suo intervento \\
Marocco era l'unico territorio nordafricano indipendente e nel 1904 c'è accordo:
\bi
    \item alla gran bretagna egitto
    \item alla francia il marocco (che prendera nell 1911)
\ei
Guglielmo vuole pero che non passi alla francia \t e quindi nel 1905 la germania si presenta come la garante dell'indipendenza marocchina \\
Viene convocata la conferenza di Algesiras del 1906: alla francia viene concesso il protettorato sul marocco e non l'annessione diretta \t inoltre doveva rispettare di controllare solo la zona costiera \\
Nel 1911 pero la francia occupa il paese \t guglielmo II accusa la francia di aver trasgredito gli accordi \t quindi manda una corazzata nella rada di Agadir \\
Arriva pero anche la flotta inglese \t davanti alla flotta britannica guglielmo si tira indietro, e si stringe l'accordo che:
\bi
    \item il marocco viene riconosciuto alla francia
    \item in cambio la germania si puo prendere una parte del congo, che era importante
\ei
Tutto questo genera delle tensioni tra le potenze euorpee

\v

L'altro focolaio di tensioni era la penisola balcanica:
\bi
    \item per l'austra era la zona di espansione naturale
    \item la russia voleva uno sbocco sul mare attraverso i dardanelli \t non voleva avanzata austria \t giustifica interesse dei balcani perche si fa difensore dei popoli balcanici e ortodossi
    \item anche l'italia si interessa \t consolidamento sull'adriatico
    \item la serbia che aveva ottenuto l indipendenza nel 78 con il congresso di berlino \t voleva insorgere contro i turchi e realizzare "la grande serbia" \t voleva l'egemonia sui balcani
\ei
La russia supporta la serbia in funzione antiaustriaca \\
In turchia nasce il movimento dei "Giovani turchi", che voleva annientare l'assolutismo monarchico \\
Ad istanbul fanno rivolta, che pone fine all'assolutismo del sultano \t ma crea conflitti nei balcani \\
I giovani turchi volevano trasformare la turchia in una monarchia costituzionale \t ma cosi facendo si genera una disgregazione \\
Questa rivolta fa nascere movimenti indipendentisti negli altri paesi

\v

Bismark convoca il congresso di Berlino per redimere la question coloniale e europea, mentre nel 95 nella conferenza si parla della spartizione dell africa \\
Contesto non facile perche nazionalistica \t bismark cerca politica di equilibrio \\
Nel 88? viene affidata all austria il protettorato sulla bosnia erz. \\
Il protettorato non implica l'annessione, ma poi l austria la annette unilateralmente \t nascono quindi le proteste della serbia \\
Insieme alla serbia protesta anche la russia \\
La germania invece sostiene l austria nel caso la russia dovesse sostenere esplicitamente la serbia \\
Assetto anomalo \t la russia, austria e germania avevano affinita perche erano imperi \\
Russia riconosce l'annessione austriaca, perche viene minacciata pesantemente dalla germania \t questo porta la russia a portarsi verso inghilterra e francia \\
Nel frattempo i turchi nel 12 stanno combattendo la guerra di libia \t quindi la russia sollecita la nascita della lega balcanica \\
Questa lega ha il dichiarato intento di combattere contro i turchi \t nascono due guerre balcaniche \\
La PRIMA: Grecia serbia e bulgaria vanno contro i turchi per il controllo dei macedoni, che era turca \t si conclude con il successo, e la macedonia viene spartita \\
La SECONDA: la bulgaria pero vuole fare rispartizione \t quindi scoppia guerra tra serbia e bulgaria \t la serbia viene appoggiata dall grecia \\
Bulgaria viene sconfitta e nel 13 pace di bucarest \t consente alla turchia di prende un pezzo di tracia e la macedonia viene spartita tra Grecia e Serbia (bulgaria viene sconfitta) \\
In realta la serbia voleva pero sbocco sul mare e si vuole espandre in albania \t austria interviene e le impedisce questo allargamento, creando il protettorato albanese \\
Serbia è quindi rafforzata territorialmente \\
Questo è il motivo dell'assassioni di Sarajevo \\
Francesco Ferdinando sostenva il progetto trialistico \t nel 1907?? l'austria diventa austrungarica \t la nazionalita austriaca richiedeva un riconoscimento??? \\
Compresso???? chiedere \\
Gli slavi del sud a questo punto chiedono anche loro una rappresentazione \t e questo progetto trialistico veniva sostenuto da ferdinando, che era dalla parte dei balcani \\
Viene ucciso perche si trovava a sarajevo \t il suo omicidio era solo il casus belli

\v

La triplice intesa nasce come? \t gugliemo II applica sempre politiche aggressive \\
A fronte di cio francia e inghilterra superano il loro conflitti e nel 1904 firmano "Intes cordiale" \t inglesi egitto, francesi marocco \\
Nel 1907 diventa triplice intesa \t la russia si unisce in funzione anti austriaca \\
Nasce come un accordo difensivo, che in realta non è \t cosi la germania viene accerchiata, 3 intesa rappresenta una minaccia per la germania \\
Nel 1907 c'è triplice alleanza e triplice intesa
\end{document}
