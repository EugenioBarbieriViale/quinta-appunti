\documentclass[12pt]{article}

\usepackage[a4paper, total={6in, 8in}]{geometry}
\usepackage{textcomp}

\begin{document}
\setlength{\parindent}{0pt}

\def \t {\textrightarrow}
\def \v {\vspace{1em}}
\def \bi {\begin{itemize}}
\def \ei {\end{itemize}}
\def \s[#1] {\section*{#1}}
\def \ss[#1] {\subsection*{#1}}
\def \sss[#1] {\subsubsection*{#1}}

\s[Mussolini e Hitler]
La politica estera di mussolini avvicina l'italia alla germania nazista \\
Nel 33 Hitler prende il potere nella repubblica di Weimar come cancelliere, mentre mussolini era gia al potere da 11 anni \\
All'inizio Mussolini diffida della politica di Hitler \t e viene + volte chiamato in causa dall potenze europee per mediare con la germania \\
Quando Hitler vuole annettere l'austra, l'italia occupa il confine (sentinelle del brennero) e Hitler non invade \\
L'avvicinamento si compie nel momento della guerra d'etiopia \t societa delle nazioni impone sanzioni all italia (che in realta saranno inutili) \\
L'unica nazione che appoggia Mussolini è la germania (nel 36) \\
Poi la germania e l italia partecipano nella guerra di spagna \t Francisco franco aveva fatto colpo di stato e dara vita al fascismo spagnolo \\
Italia fornisce mezzi, come nuovi aerei, mentre germania manda piu uomini \t i critici dicono che la guerra di spagna è stato il campo di prova per la WW2 \\
Nel 36 viene firmato anche l'asse Roma-Berlino \t primo documento che sancisce l'avvicinamento, ma non è ancora alleanza militare \t solo un patto d'amicizia \\
Mussolini pensa di poter usare questo documento per esercitare pressione sull europa e espandersi colonialmente \t muss. pensa di poter sfruttare Hitler \\
Nel 37 l'asse diventerà Roma-Berlino-Tokyo \t hitler aveva stretto accordo con giappone, che si stava espandendo sul pacifico con una politica aggressiva fascista

\v

Nel 38 viene firmato il manifesto della razza (da un gruppo di intellettuali fascisti), che è la premessa delle leggi razziali \\
Questo documento dice che gli ebrei non appartengono alla razza italiana \t qualche mese dopo vengono pubblicate le leggi razziali, imitando le leggi di Norimberga del 35 \\
Hitler, che sale al potere nel 33, inizia la persecuzione degli ebrei \t mentre questa politica nel fascismo nasce nel 38 (mentre mussolini era salito nel 22) = evidente influenza \\
Queste leggi vengono firmate dal re \t impediscono matrimonio misto, scuola, servizio militare per gli ebrei (e di svolgere funzioni pubbliche) \\
Questa fase è stata definita come la fase della persecuzione dei diritti \t fino al 43, ebrei non vengono uccisi (ma gli vengono negati i diritti) \\
Poi ci sara la persecuzione della vita \\
Nella scuola viene introdotta la dottrina razziale

\v

Questo suscita perplessita nell opinione pubblica \t le leggi razziali vengono considerate come il piegarsi a un regime straniero, e inoltre vengono condannate dalla chiesa \\
In occasione della conferenza di Monaco, l'italia viene chimata a mediare tra germania e europa \\
Qua, dopo che Hitler chiede di annettere la regione dei Sudeti, mussolini viene invitato per mediare (in quanto amico) e quindi viene convocata la conferenza di monaco (cove partecipano francia, germania, italia e ingh.) \\
Gli ambasciatori cechi non vengono fatti entrare, e si vedono sottratti senza partecipare una regione \\
Questo è stato fatto perche si voleva evitare una guerra \t potenze europee pensavano che Hitler si sarebbe accontentano, ma non è successo \\
Quindi è un continuo cedimento \\
Cause della WW2:
\bi
    \item aggressivita nazista
    \item politica di accoscidentenza dell'europa nei confronti di hitler \t che cede alle richieste di hitler per evitare la guerra
\ei

\ss[Patto d'acciaio]
L'ultimo passo che lega la germania all italia è il patto d'acciaio del 1939, che è una vera alleanza militare: \\
Le due nazioni si impegnano ad aiutarsi reciprocamente nel caso di guerra, si offensiva che difensiva \\
Questo è un suicidio per l'italia \t era evidente che hitler avrebbe scatenato a una guerra, mentre l'italia non era minacciata \\
Italia non era pronta e non aveva necessita della guerra \t pero si lega a una nazione che sicuramente scatena la guerra \\
Questo patto porta l'italia quindi verso la WW2

\v

Nel 39 l'italia occupa l'albania, che viene annessa al grande impero italiano d etiopia etc. \\
Mussolini rivendica anche la savoia, nizza, etc. \\

\s[Hitler]
La fine della WW1 porta in italia un senso di rivalsa (mito della vittoria mutilata) \\
In germania, il trattato di versaille, crea enorme malconento \\
Si crea una grande campagna socialista e aleggia l'idea di una rivoluzione bolscevica \t vengono creati i primi consigli su modello del soviet \\
Nel 1918, la monarchia viene travolta e a berlino dichiarata repubblica (Gluglielmo II non voleva dimettersi ma scappa in Oladna) \\
11 novembre 1918 \t firmato l'armistizio \\
Il governo provvisorio deve portare la germania alla normalizzazione e alla legalita \t c'è un governo repubblicano (che firma Versaille) \\
Questo governo, guidato da Ebert, indice delle elezioni per un assemblea costituente \t di fronte a queste elezioni la sinistra tedesca di mostra divisa \\
La social democrazia considera i soviet come un qualcosa di transitorio \t e pensano che vengano smantellati una volta che si va verso la democrazia \\
La social dem. vuole evitare ogni forma di scontro e radicalizzazione contro le opposizioni \t ma non tutti sono d'accordo nelle sinistre \\
Infatti c'è chi sostiene i consigli operai \t i piu radicali sono il partito social democratico indipendente (nel 17) e la Lega di Spartaco (guidata da Rosa Louxemburg, e i membri sono detti spartachisti) \\
GLi spart. nel 18 danno vita la partito comunista, ma il governo che si è formato è a maggioranza social democratica \\
Quindi , quando spartachisti si staccano dal part. socialista, nascono proteste contro il governo liberale ?? che non impedisce rivoluzione \\
Spartachisti fanno quindi tentativo rivoluzionario, che pero viene stroncato dalle forze armate \t Rosa e i capi vengono uccisi, e alcuni membri giustiziati \\
Per uscire da questa situazione si impone l'opzione moderata \t e quindi si va verso la costituente

\v

Attorno alla soci. dem. si allacciano i conservatori \t pensano di poterla sfruttare per vincere le elezioni \\
Nel 19 si tengono elezioni a suffragio universale \t stravicne la socialdemocrazia ma senza maggioranza assoluta \\
QUindi nasce governo di coalizione: social dem con parito cattolico e i liberali democratici \t coalizione repubblicana, democratica, parlamentare \\
Ebert viene chiamato presidente della repubblica \t e questa assemblea lavora a Weimar (piu sicura di berlino) e redime la costituzione di Weimar
\bi
    \item la germania diventa repubblica federale in 17 regioni
    \item potere leg. al parlamento \t eletto con sistema proporizonale a suff. univ.
    \item consiglio federale che a veto sul poter legislativo
    \item potere esecutivo al cancelliere (che è un primo ministro) ed è responsabile di fronte al parlamento \t quindi repubb. parlamentare
\ei

Pero cè presidente, che da carattere molto presidenziale a questa repubblica:
\bi
    \item ??
    \item comandare l esercito
    \item puo chiamare i referedum popolari per le leggi del parlamento
    \item beneficia dell articolo 48 della cost., che da al presidente poteri straordinari per ristabilire l'ordine quando dovesse valutar esituazione di pericolo, e puo anche intervenire con l'esercito
    \item art. 48 dice anche che il presidente puo anche sospendere in tutto o in parte i diritti fondamentali
\ei
Hitler sfruttera molto l articolo 48 \t e giustifichera le sue azioni con la costituzione di weimar 

\v

Il governo provvisorio era stato accusato di aver accettato la sottomissione della germania, che aveva firmato armistizio ingiusto, etc. \\
A versaille nel 19 vengono stilate le condizioni di pace \t che accrescono il disprezzo per il governo provvisorio \t la cost. di Weimar viene fatta nel 19 \\
Grande risentimento \t la destra accusa i socialisti e democratici al govenro di aver tradito la germania \t la germania doveva pagare 1/4 del suo pil ogni anno \\
Nel 1921 viene assassinato il ministro che aveva firmato l'armistizio

\v

Gli anni tra il 19 e il 23 sono come il biennio rosso \t ci sono forti tensioni politiche e socali, dopo la lega di spartaco e la nascita del movimenti nazionalisti di destra, che parlano di pace ingiusta e hanno carattere antisemita \\
Hitler afferma che gli ebrei avevano lavorato per la guerra di Versaille, insieme ai marxisti \t volevano danneggiare la germania \\
Nel 1920 si assiste a un primo tentativo di colpo di staot \t Wolfgang Kapp tenta, ma viene fermato da uno sciopero generale org. dalle forze democratiche \\
Il partito nazional socialista dei lavoratori è il gruppo di destra + forte \\
Hitler entra nel 20 \t e lo ingradisce, e prende il controllo \t poi organizza il Putsch di Monaco, e viene incarcerato (dove scrive Mein Kampf, che è il suo programma politico) \\
Il putsch di monaco viene sventato da Stresermann, che era il leader del partito popolare (una formazione di stampo liberale) 

\v

Str. sara il fautore del consoldiamento della repubblica di W., che vive per anni in stabilita \t fino al 28, quando muore \\
Stresermann porta la germania nel 25 al congresso di Locarno, primo congresso a cui germ. partecipa dopo la guerra \\
Sters. si allea con un partito del centro e con i socialisti \t vuole aprire il dialogo con euorpa e risanare l'economia \\
Fa riforma monetaria \t introduce una nuova moneta, garantita dalle industrie e agricoltura tedesche \\
Poi cerca di risolvere il conflitto con la francia, e repirmera l'opposizione sia di destra sia di sinistra \t ma non ha fatto tutto da solo \\
Infatti USA interviene \t capisce che il mercato europeo è inesistente ed è da ricostruire \t usa vuole questo mercato e europa è indebitata con usa

\v

Da vita piano chiamato triangolazione finanziaria \t accettato nel 24 che deve ricostruire la germania \\
Prevede che alla germania siano forniti dei capitali, in modo che la germania faccia ripartire la sua produzione, e in modo che possa pagare i debiti di guerra franco-inglesi, cosi che francia e ingh. possono resituire soldi a usa \\
E soprattutto, in modo che il mercato si rimetta in moto e l'europa possa ripartire \\
Triangolazione finanz. \t usa-germania-francia e ingh. \\
USA puo fare questo perche ha un sacco di soldi \t ha capitale in eccedenza, e quindi gli conviene investire in euorpa \\
Streserman accetta aiuto delgi usa, e a partire dal 25 si vedono gia i risultati, quando produzione tedesca supera quella precedente alla guerra \\
La ripresa della germania sara ininterrotta \t e si riprende perche è finanziata da capitali stranieri: quando arriva il crollo nel 29, pero, crolla anche la germania (che non era cosi forte per essere indipendente)

\v

La questione francese, risolta da Str. \t nel 23 francia coglie come occasione un mancato pagamento e occupa la Rur, e la occupa come garanzia di pagamento \\
Gli operai tedeschi fanno resistenza passiva \t boicottano e non collaborano \\
I francesi si aspettavano vantaggi da questa occupazione (zona ricca) \t e quindi la germania è sull'orlo del crollo totale \\
Per garantire un sostegno a lavoratori, la banca tedesca stampa soldi = inflazione senza precedenti \t nel 1923, un dollaro veniva cambiata con 400 miliardi di franchi \\
Il 23 è l anno in cui stresermann forma il governo ? \t e resistera a questa situazione drammatica \\
In tutto questo l'industrie tedesche grosse fanno i miliardi \t mentre il ceto medio arriva alla poverta, chi possedeva fabbriche o aveva valuta straniera, hanno vantaggio enorme da inflazione \\
Si crea divario assurdo tra chi diventa ricchissimo e chi è nella totale povertà

\v

Nel 25, la germania partecipa al congresso di Locarno \t Stresermann stringe accordi con Brian (primo ministro francese) e questo porta a una normalizzazione dei rapporti \\
Il cong. di Loc. da vita allo "spirito di locarno", ovvero un alleggiamento nell aria di distensione e di pace
\end{document}
