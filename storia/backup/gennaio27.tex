\documentclass[12pt]{article}

\usepackage[a4paper, total={6in, 8in}]{geometry}
\usepackage{textcomp}

\begin{document}
\setlength{\parindent}{0pt}

\def \t {\textrightarrow}
\def \v {\vspace{1em}}
\def \bi {\begin{itemize}}
\def \ei {\end{itemize}}
\def \s[#1] {\section*{#1}}
\def \ss[#1] {\subsection*{#1}}
\def \sss[#1] {\subsubsection*{#1}}

\s[Hitler]
Definisce in Mein Kampf la sua dottrina \t quando diventa cancelliere della rep. di Weimar, da inizio al suo programma \t non c'è fase di consolidamento come in Mussolini \\
Sale al governo, attua un epurazione dell amministrazione pubblica, nel 33 incendia il parlamento e lui accusa i comunisti, che diventano l'obbiettivo delle persecuzioni \\
Cosi introduce anche delle misure speciali \t in cui le misure costituzionai vengono sospese e alcuni articoli in cui tutelavano i diritti \t Hindenburg sciogli il parlamento ed elezioni \\
Nazisti + nazionalisti garantiscono la maggioranza assoluta del parlamento \t queste solo le elezioni del 3 marzo \t cosi inizia la nazificazione dello stato, in cui esso si sottomette agli ideali nazisti \\
Hitler impiega 3 mesi per passare da repubblica di weimar a uno stato totalitario \\
Prima seduta del parlamento fa votare la legge sui pieni poteri \t con i quali il parlamento esautora se stesso, ovvero il parlametno delega il potere legislativo al governo \t Hitler fa le leggi e ha potere esecutivo \\
Comunisti non ci sono + nel parl., ma ci sono socialdemocratici che votano contro \t ma tutto il resto si piega alla volonta di Hitler, e inaugura la dittatura del fuhrer \\
Nel 33 viene stabilito con una legge il partito unico tedesco \t che coincide con lo stato \t tutti i sindacati soppressi, anche organizzazioni come associazioni culturali, sociali, scolastiche, sportive, vengono nazificati = sottoposte a delle regole rigide emanate da Hitler \\
In questo modo penetra l'ideologia nazista in ogni aspetto del sociale \\
Poi nasce il fronte del lavoratori ??? su modello fascista \\
Poi crea la Gestapo, ovvero la polizia segreta di stato \t inizialmente guidata da Goring, poi da Himmler (che è anche capo dell'SS, organo paramilitare) \\
Magistratura è sotto il controllo del governo \t legalita = volonta di Hitler

\v

Esiste pero un opposizione nel partito nazista \t non tutti approvano alcune iniziative, come lo spostamento di Hitler con il mondo dell'industria e il legame con il mondo della grande finanza \\
Esiste un aria rivoluzionaria anticapitalista del nazismo \t questa parte è guidata da Romm \t che guida anche le SA, le squadre d'assalto (altra formazione paramilitare, che nasce come la sua guardia del corpo) \\
Hitler decide che questa ala di dissenso non deve esistere, quindi epura anche all'interno \t la notte del giugno del 34 la gestapo e le ss uccidono 1000 persone \t incluso Romm \t nella notte dei "coltelli?????" \\
L'apice si ha con la morte di Hindenburg, per cui Hitler assume le funzioni di presidente \t non ci sono + elezioni presidenziali, e quindi si passa a un altro tipo di regime che non è la repubblica \\
Dal 34 si proclama capo assoluto del terzo Reich (secondo di Bismark) e Fuhrer 

\v

Non incontra grande resistenze pubbliche \t perche era blanda, e la macchina repressiva dello stato era troppo forte \\
La gestapo ripulisce la societa intera dai nemici del popolo, e vanno nei Lager controllati dall SS \t prima degli ebrei vanno i comunisti, gli esponenti degli altri partiti, i parlamentari (uccide il 23 per cento) e tutti i non ariani, quindi gli invalidi, omosessuali, zingari etc. \\
Lui ha in mente una societa perfetta, senza oppositori, slavi, ebrei, senza criminali, senza malati di mente \t tutti coloro che sono difformi dal modello dell'ariano perfetto \\
Opposizione c'è tardivanmente (dopo il 36) nella chiesa \t infatti all'inizio Hitler ha buini rapporti con la chiesa cattolica e protestante \\
Nel 33 infatti aveva firmato un concordato con il papa, che garantiva alla chiesa liberta nell'organizzazione della chiesa in germ. e libertà di culto \\
Il loro partito pero, che era del centro, era stato sciolto \t ma non c'era nonostante cio senso di opposizione \\
"Con cocente dolore" \t in questa enciclica del 37 vengono condannati il razzismo, la persecuzione degli ebrei, la violazione del concordato del 33, e l'idolatria dello stato \\
Dal 37 inizia quindi la persecuzione dei cattolici \t e nel campo di Dakau ci sono il blocco 26 e 28, interamente riservati ai sacerdoti cattolici \\
La chisa protestante invece si piega \t e nel 38 presta giuramento, con pochissimi casi di dissidenti

\v

Lo sterminio degli ebrei diventa l'elemento di coesione di questo stato \t insieme alla razza, e quindi l'odio razziale \\
All'inizio vengono discrimanti, poi perseguiti e infine sterminati \\
Dal 33 al 35 \t prima fase, in cui viene creata una violentissima propaganda perche i tedeschi provassero odio verso la comunita ebraica \t i negozi vengono contrassegnati con le stelle di david e vandalizzati \\
Vengono anche lincenziati gli ebrei dalla pubblica amministrazione (e in generale i non-ariani, e anche chi aveva un nonno non-ariano) \\
Secondo momento dal 35, quando vengono emanate le leggi di Norimberga \t si fondano su criteri biologici/razziali che tolgono la cittadinanza tedesca agli ebrei \t non hanno alcun diritto politico nei civile, e quindi sono fuori dalla comunita tedesca \\
Sono untermenschen, razza inferiore \t e quindi esclusi dalle scuole, universita, mezzi comunicazione sociale, non possono fare una serie di lavori come medico, insegnante et.c \t non hanno diritto di cariche pubbliche, perche non sono cittadini \\
Diventa reato frequentare (e sposare) un ebreo \t quindi molti emigrano versono in palestina e usa \t emigrano quindi molti intellettuali, come Einstein, Mann, Fromm, Freud \t e i loro libri vengono bruciati in piazza a berlino \\
Terzo momento a partire dal 38, un ebreo polacco uccide un diplomatico tedesco a Parigi \t e questo scatena ogni violenza in germania \\
Tra 9/10 novembre del ?? \t notte dei cristalli, in cui chiunque puo distruggere neogozi, sinagoghe, incendiare le case, uccisi ebrei \t qualunque violenza era permessa \\
Da qui sara un crescendo di violenze

\v

La soluzione finale viene concepita quando la guerra è gia scoppiata, e in due momenti \\
\bi
    \item primo momento è l'invasione dell URSS \t vengono organizzate delle unita dell SS che seguono l'avanzate dell'esercito e uccidono tutti gli ebrei (che erano anche slavi)
    \item secondo momento è la conferenza di Van See \t nel 42 i + vicini a Hitler si riuniscono, calcolano numero di ebrei (11 milioni, di cui 6 moriranno) e pianificano il loro sterminio
\ei
Nel 42 viene quindi organizzata sistematicamente la deportazione \t che è l'attuazione della soluzione finale, mentre la sua progettazione era la conferenza, che era un incontro segretissimo \\
Viene pero redatto un verbale \t e Eichman in quella sede presenta un piano dettagliato \\
Re in italia firma le leggi razziali nel 38 \t quando in germania è successo tutto questo \\
Accompagnato il tutto da una campagna propagandistica molto forte \t in cui vengono stereotipizzati \\
Il primo campo di concetramento è Dakau, ed era stato creato gia nel 33 \t che viene considerato uno strumento ordinario per risolvere il problema dei non ariani nella razza \\
Nel 33 viene anche emanata la prima legge demografica \t in cui c'è sterilizzazione eugenetica \t si possono riprodurre solo coloro che hanno dei caratteri genetici in accordo con la razza ariana \\
Verrano imposti circa 30 mila aborti a donne che non hanno questi caratteri \\
Anche i malati mentalmente vengono sterilizzati \t e a partire dal 40 viene istuita la operazione eutanasia \t in cui cittadini tedeschi, come infermi, malati, vecchi, malformati, con handicap \t non possono riprodurre e vengono uccisi con il CO \\
Interviene la chiesa e la protesta delle famiglie \t quindi viene fermata, ma gia 80 mila persone erano state uccise

\v

La repressione è accompagnata dalla propaganda, affidata a Gobels \t che diventa ministro per l educazione e la propaganda \\
Tutto viene utilizzato per far passare l'idea dell'uomo ariano \t e nel frattempo vige anche la censura \\
Questo viene passato in una rigorosa gioventu \\
Il regime nazista ha caro anche la difesa della famiglia \t perche è dove l'ariano cresce \t e la natalita viene sostenuta dal regime, i giovani poi vengono educati dal regime \t nella gioventu hitleriana \\
Dai 6 ai 18 anni i ragazzi venivano inquadrati in gruppi militari \t e ognuno aveva un libretto in cui erano annotati i miglioramenti nella dottrina nazista e nello sport \\
A 10 anni si faceva esame di atletica, campeggio e storia nazificata \t per cui si entrava nella gioventu hitleriana \\
A diciotto anni, il giovane passava automaticamente al lavoro obbligatorio o all esercito \t e valeva anche per le ragazze, che pero a 18 anni andavano nelle aziene agricole dove per un anno facevano lavoro obbligatorio \\
Nelle scuole naziste si studiava il mein kampf, che diventa l'orientamento pedagogico, e si studiano le leggi razziali (che esaltano i tedeschi come razza superiore, e gli ebrei come il male) \\
Gli insegnanti venivano formati in scuole speciali, sia sulle teorie razziali sia sulla dottrina nazista \t e dovevano prestare giuramento al fuhrer \\
Anche all'universita i prof. dovevano essere selezionati da persone esperte in base al loro orientamento nazista \\
Questo è a grande danno dell'istruzione \t livello dell'educazione si abbassa, e in non vanno universita ma lavorano \\
La storia insegnata proponeva la schiavitu della razze inferiore, razzismo, patriottismo \t tutti gli argomenti venivano insegnati in un altra visione (nazista)
\end{document}
