\documentclass[12pt]{article}

\usepackage[a4paper, total={6in, 8in}]{geometry}
\usepackage{textcomp}

\begin{document}
\setlength{\parindent}{0pt}

\def \t {\textrightarrow}
\def \v {\vspace{1em}}
\def \bi {\begin{itemize}}
\def \ei {\end{itemize}}
\def \s[#1] {\section*{#1}}
\def \ss[#1] {\subsection*{#1}}
\def \sss[#1] {\subsubsection*{#1}}

\s[Recuperare 28 ottobre]
Calo demografico in tutta europa \t dilaga la febbre spagnola \t c'è il problema della conversione industriale e inflazione alta \\
Grande malumore verso le democrazie liberali \t le promesse al popolo non vengono rispettate \\
Ci sono molti scontri nelle industrie e nelle campagne \t che hanno un disprezzo delle istituzioni liberali \t polarizzazione delle posizioni verso estreme destra e sinistra \\
Nasce, in italia e germania, grande paura nei confronti della sinistra \t si era vista la rivoluzione in russia \t qundi si tende piu alla destra, posizioni + conservatrici \\
Nasce nel ? il cominter \t la terza internazionale \t Lenin vuole generare una rivoluzione comunista europea, che doveva essere coordinata dal cominter \\
Nel 20 c'era il secondo congresso dell'internazionale comunista a mosca, in cui fanno 21 punti \t requisiti per entrare al cominter e prevedeva che il partito comunista di un paese veniva sottomesso a quello russo \\
Nasce spaccatura tra massimalisti e riformisti \t i massimalisti vanno verso il comunismo russo, mentre i riformisti rimangono tali

\ss[Biennio rosso]
In italia cresce anche il movimento operaio \t classe operaia diventa una classe vera e propria \t aumenta adesione ai sindacati \\
Quindi il popolo partecipa ora alla politica \t prima no \t i lavoratori europei ed italiani vogliono portare il cambiamento \\
Si richiedono salari giusti, condizoni di lavoro migliori, etc.. \t questo cambiamento è il biennio rosso \\
Tra il 19 e il 20 nascono dei consigli di fabbrica a modello dei soviet \t inoltre partito socialista diventa sempre piu schierato 

\v

Nel 19 nascono il partito popolare di Sturzo, il partito dei "fasci italiani di combattimento" \\
Nel 19 c'è sistema proporzionale alle elezioni, voluto dai massimalisti \t seggi in proporzione ai voti ottenuti dalla lista \\
Vince il partito socialista, che diventa il primo partito italiano (32 perc.) \\
Poi il paritto popolare (appena nato), poi ? e poi i gruppi liberali \t che vedono un drastico ridimensionamento dei seggi \\
Questi risultati sono preoccupanti \t è difficile dare stabilita al governo \t inoltre il part. socialista rifiuta dialogo con i gruppi borghesi ? \\
I liberali vedono quindi una via di salvezza nell'alleanza con i popolari \t opposizione ma antisocialisti \\
Nel 20 i lavoratori occupano le fabbriche \t la FIOM (sindacato dei metalmeccanici) voleva aumenti di stipendio, ma indusriali avevano rifiutato le richieste \\
Ind. sono intransigenti \t proclamato lo sciopero bianco \t entrano in fabbrica, ma una volta all'interno non lavornano \\
Cosi gli indus. chiudono gli stabilimenti \t e alla fine gli operai occupano le fabbriche, guidati dai sindacati rossi \\
I sindacati liberali ?? non partecipano \\
Occupazione è enorme (300 fabbirche nel nord) e partecipano 400.000 lavoratori \t operai prendono controllo di stabilimenti \\
In alcuni casi lavorano e tentano di continuare la produzione \t tutto nell'ottica di un processo rivoluzionario che roverscera lo stato \\
Ma non sara cosi perche il movimento operaio era molto frammentatno \t non c'era unita di intenti tra fabbriche e non c'era progetto, una strategia rivoluzionaria (come le tesi d'aprile) \\
Gli ordinovisti sono pero molto attivi \t indicano la via della rivoluzione \t Gramsci dira che i consigli di fabbrica sono gli strumenti per la riv. e dovevano garantire unita \t ma non accade \\
Il governo liberale non sa quindi cosa fare \t situazione è disastrosa \t Nitti da le dimissioni \\
Viene chiamato Giolitti (che ha 80 anni), risolve la questione di fiume (trattato di Rapallo) \t lui ha atteggiamento di non intervento, anche se subisce pressioni dagli industriali \\
Fa quindi strada della mediazione e riconciliazione tra CGL e industriali \t e qua è vincente \t fa inoltre ottenere agli operai l'aumento di salari \\
Inoltre promette agli operai maggiore controllo delle aziende che non accadra \t in cambio gli operai liberano le fabbriche \\
Gli industriali pero sono scontenti \t non la considerano una mediazione giusta \t quindi mondo dell'industria non è contenta di questo governo liberale e giolitti \\
Gli operai invece speravano di ottenere molto di piu \t la grande industria e borghesia vede invece in Giolitti un cedimento alla sinistra \\
Quindi prende piede l'idea di una possibile soluzione reazionaria per affrontare quello che succedeva \t Mussolini capisce e quindi si sposta su posizioni + conservatrici \\
Il partito socialista è sempre + diviso al suo interno (tra rif. con Turati e mass.) \\
Nel 21, congresso di Livorno del part. soc., ala massimalista si stacca, si riuniscono insieme agli ordinovisti e nasce cosi il partito comunista italiano \t con Gramsci a capo \\
I riformisti infatti non volevano sottostare ai 21 punti di Lenin \t e lo stesso Lenin aveva fatto pressione perche venissero applicati (forse) \\
Gramsci non era soddisfatto di questa scissione \t una sinistra frammentaria avrebbe indebolito la rappresentanza degli operai
\end{document}
