\documentclass[12pt]{article}

\usepackage[a4paper, total={6in, 8in}]{geometry}
\usepackage{textcomp}

\begin{document}
\setlength{\parindent}{0pt}

\def \t {\textrightarrow}
\def \v {\vspace{1em}}
\def \bi {\begin{itemize}}
\def \ei {\end{itemize}}
\def \s[#1] {\section*{#1}}
\def \ss[#1] {\subsection*{#1}}
\def \sss[#1] {\subsubsection*{#1}}

\s[Biennio rosso]
Al congresso di Livorno nasce quindi il partito comunista \t Togliatti e un altro che guidano \\
Mussolini invece si era gia staccato nei fasci italiani \\
Negli anni del bienno rosso \t italia è sull orlo della guerra civile \t massimalisti da un lato, contro le forze di destra (rappresentato dai fasci italiani di combattimento e dai conservatori e irredentisti) \\
Queste forze di destra vedono nella prassi di mussolini, che è violenta, un modo per mantenere l'oridne \\
Mussolini si mostra con le squadre fasciste = gruppi fascisiti che terrorizzavano prima solo le campagne, e poi anche le citta, con la scusa di far rispettare l'ordine \\
E lo facevano con la violenza, picchiando e poi anche rapendo \t rasstrellavano e le persone rapite non si vedevano + \\
Azione era fuorilegge, ma stato liberale non interviene perche:
\bi
    \item fenomeno fascista viene sottovalutato, inoltre si pensava di poter manovrare mussolini 
    \item poi si voleva sfruttare mussolini \t dove lo stato non puo arrivare, arrivano le squadre fascite a mantenere effettivamente l'ordine
\ei
Agiscono nella pianura padana principalmente \t i contadini, a differenza degli operai, erano riusciti nella pianura padana a ottenere successo \\
Infatti erano nate delle coperative cattoliche e delle leghe socialiste \t e cosi riuscivano effettivamente a controllare il mercato del lavoro \\
E queste associazioni erano come dei sindacati \t contrattavano le ore lavorative e i salari \t struttura associazionistica che però funzionava \\
I contadini erano riusciti, grazie a quelle forze di opposizione al fascismo \t e quindi le squadre fasciste distruggono

\ss[Eccidio di bolonga]
Bologna era il centro dle movimento sindacale \\
Nel 20 i socialisti avevano vinto le amministrative a bologna \\
E quando i socialisti si devono insediare nel palazzo della citta, il sindaco si affaccia per salutare la folla \\
Scoppiano dei colpi di pistola \t i socialisti, che facevano servizio d'ordine, sparano sulla folla anche loro \\
E i fascisti che erano tra la folla sparano ai socialisti (che sembra che fossero stati i primi a sparare nella folla) \\
I fatti di bologna segnano l'inziio del fascismo agrario = azione dei fascisti nelle campagne che doveva intimidire e colpire il partito socialista \\
Il fascismo agrario vede consenso nei propretari terrieri, che erano contro queste associazioni contadini \\
Mussolini è ancora ininfluente sul piano politico \t queste azioni sono sul territorio, non ricadono sulla politica \\
Questo trova consenso nei giovani, negli industriali e nella piccola borghesia \t piccola borghesia si vuole disinguere dalla massa operaia, e trova questo modo per farlo \t di sostenere il fascismo \\
Dopo bologna le azioni punitive crescono molto \t al punto che alcuni mitilanti socialisti saranno costretti addirittura a lasciare l'italia \\
Succsesso dello squadrismo è dato anche dalla neutralita delle forze dell'ordine \t stato non interviene, la classe dirigente non si esprime, mentre le forze dell'ordine si mostrano indifferenti (e quindi condivono) \\
Questo perche c'era la volonta di usare mussolini per ripristinare l ordine, e poi di riassumere il controllo
\end{document}
