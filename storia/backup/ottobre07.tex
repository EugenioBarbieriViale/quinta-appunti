\documentclass[12pt]{article}

\usepackage[a4paper, total={6in, 8in}]{geometry}
\usepackage{textcomp}

\begin{document}
\setlength{\parindent}{0pt}

\def \t {\textrightarrow}
\def \v {\vspace{1em}}
\def \bi {\begin{itemize}}
\def \ei {\end{itemize}}
\def \s[#1] {\section*{#1}}
\def \ss[#1] {\subsection*{#1}}
\def \sss[#1] {\subsubsection*{#1}}

\s[Conseguenze della Prima Guerra Mondiale]
Riconosce la germania come unica responsabile della pima guerra mondiale \\
Anche patto Saint German viene considerato punitivo \t l'italia ottiene i territori entro i suoi confini naturali \t nasce il mito della vittoria mutilata \\
Per l'italia era una vittoria parziale (mutilata) \t non aveva acquistato tutte le terre previste dal patto di Londra \t questo alimenta un senso di rivalsa verso l'europa 

\v

Governo nuovo tedesco viene accusato diessersi piegato ai patti dei vincitori \t patto di versaille viene considerato come un patto che svilisce il popolo tedesco \\
Ebrei verrano accusati di aver partecipato a questo patto \t verra strumentalizzato \t diventera un motivo di propaganda forte \\
Da questo governo nasce la Repubblica di Weimar \t che viene accusata di lavorare per i suoi interessi e di essersi piegata alle potenze europee \t sara un apparato governativo instabile \\
Il primo presidente della repubblica di Weimar è un monarchico (Inderburg??) \t i voti della sinistra si frammentano, non c'è coesione politica 

\v

Si creano risentimento sia in italia sia in germania \t si vuole riscattare il prestigio di nazione tedesca \\
Su questo terreno si baserà la propaganda dei futuri regimi \\
Dal punto di vista territoriale:
\bi
    \item nascono nuovi stati: ungheria, jugoslavia, cecoslovacchia, estonia, lettonia, lituania
    \item danzica viene dichiarata libera
    \item austria estremamente ridotta \t ha perso 7/8 del suo impero
    \item turchia \t ha perso tutti i territori europei
    \item la palestina è l'iraq vengono affidati all inghilterra come mandato
    \item la siria va alla francia
    \item italia riceve trentino, venezia giulia, trieste
\ei
Europa è molto diversa territorialmente ma anche politicamente \t sono crollati 4 imperi: turco, tedesco, austriaco e russo \\
Inoltre Europa non è piu al centro del mondo \t i veri vincitori sono gli USA \t diventano la prima potenza politica ed economica \\
Inoltre USA diventano i principali creditori degli stati \t e quindi debitori \\
Nel 1919 l'europa deve agli usa 7 miliardi di dollari \t prima della prima guerra mondiale era il contrario, usa doveva 5 miliardi all'europa 

\v

Vittorio emanuele orlando abbandona le trattative di pace \t le forza richiedendo anche Fiume \\
L'italia si allontana, ma nel frattempo le colonie tedesche vengono spartite \t quando torna non ha + l'occasione \\
Gli animi dei nazionalisti si accedono per il non rispetto del patto di Londra \\
I nazionalisti, guidati da d'Annunzio, occupano la città di Fiume \t si voleva generare una repubblica parallela a Fiume (che riceve anche costituzione) \\
Si voleva dimostrare che l'italia poteva prendersi quello che si voleva \t ma questo piano vanno contro le trattative di pace \\
Sembrava che l'italia avesse ordito un modo per danneggiare \\
Nitti era al governo \t non riesce a risolvere questo problema (nel 19) \t d'annunzio guida 2500 legionari \\
Fiume viene presa senza incontrare resistenza \t Nitti manda truppe guidate da Badoglio, che sconsiglia intervento armato \\
Ma con d annunzio si stavano unendo diversi soldati \t anche i soldati di badoglio avrebbero potuto disertare \\
Scrive costituzione del Carnaro \t ma legionari erano fuori controllo, il tutto si realizza in un clima di violenza \\
Al governo viene chiamato Giolitti \t nel 20 firma il trattato di Rapallo, che mette fine a questo evento
\bi
    \item la jugoslavia ottiene dalmazia senza zara
    \item istria va all italia
    \item italia deve liberare fiume
    \item fiume diventera citta libera sotto il controllo della societa delle nazioni
\ei
Cosi l'esercito entra a fiume \t ci sono scontri \t ma viene liberata \\
Mussolini dichiara questa azione fratricida \t ma poi riconoscera che trattato di rapallo era l unica via possibile

\v

A Mussolini non piace d annunzio, verso cui ha un atteggiamento distaccato \t ma capisce che lo puo sfruttare (era famoso e nazionalista) \\
Mussoloni si ispira anche a d annunzio \t d annunzio era un capo a cui ci erano ispirate molte persone 

\s[Rivoluzione russa]
Impero zarista è molto arretrato \t praticamente medievale \t risorsa fondamentale era l'agricoltura \\
Potere autocratico dello zar, non permette opposizione \t sistema praticamente feudale \t tutti i contadini sono servi della gleba \\
Situazione di grande malcontento \t l unico che tenta una riforma è Alessandro II (sale al potere nel 55) \\
Nel 1861 abolisce la servitu della gleba \t senza pero che ci fossero le condizioni \t riforma è inutile \\
In realta riforma era + articolata e prevedeva che i contadini affitassero la terra \t ma non avevano soldi \t umore si inasprisce \\
La russia esporta cereali e materie prime \t e importa tutto il resto \t non è competitiva sul mercato \t dipendeva economicamente dall'occidente \\
Sul finire dell 800 gli Zar volevano pero consolidarsi anche all'estero \t a partire dagli anni 60 dell 800 si fanno sforzi per sviluppare un industria nazionale (con dei capitali stranieri) \\
Francia, Germania e UK investiranno \t in russia arrivano esperti stranieri, che dovevano far funzionare la russia ma insegnare anche \\
Tra 85 e 98 \t si verifica un boom \t industria cresce del 400 per cento \\ 
Ma crescita disomogenea \t industrie nascono solo a Mosca, Pietroburgo e Baku \\
Ma decollo industriale non è legato a una crescita interna, a un mutamento sociale interno \t è solo legato ai capitali stranieri e arriva dall'alto, dal governo \\
La societa russa rimane pero arretrata \t scollamento tra politiche economiche e societa russa 

\v

Si genera cosi un opposizione allo zarismo \t non partitica perche non esistono partiti per ora \\
È un opposizione culturare \t gli intellettuali si oppongo e si dividono in:
\bi
    \item occidentalisti
    \item slavofili
\ei
Entrambe le posizioni voglionon indicare delle vie allo zarismo per favorire la crescita \t è opposizione perche vogliono dire allo zar cosa fare \\
Occidentalisti voglio economia capitalistica e democrazia, introdurle in russia \t bisogna ripercorrere la strada di sviluppo dell'occidente \\
Gli slavofili invece vogliono puntare sulle risorse interne, su cio che è peculiare della russia \t ovvero l'agricoltura, dove vogliono puntare \\
Per gli slavofili è come un vantaggio \t si possono cosi evitare gli errori gia commessi dall occidenti \t arrivano quasi ad idealizzare la classe contadina \\
Gli slavolifi quindi volgiono sensibilizzare i contadini e mostrare il cambiamento che volevano portare \t i contadini non danno retta \\
Prospettiva slavofila (molto propagandista) si evolvera nel populismo \t che inneggerra alle risorse della russia e alla classe agricola \\
Ma populismo non ha successo \t per farsi ascoltare alcuni fanno terrorismo \t nel 98?? zar viene ucciso da un populista \\
Social rivoluzionari sono gli eredi dei popolusti

\v

Altra fonte di opposizione allo zarismo dall'esterno è marxismo \t si diffonde a partire dal 1848, ma non arriva in russia \t non c'è industria \\
Quando si sviluppa industria \t alcuni intellettuali esuli vogliono riproporre il marxismo \t padre del marxismo russo sara Plekanov \\
Plekanov (in esilio) pensa che il marxismo di Marx non è praticabile in russia \t dove c e societa agricola \\
Prima in russia ha bisogno della rivolzuione capitalista \t bisogna produrre una borghesia dal feudalesimo \t borghesia deve produrre una societa capitalista \\
Cosi nasce la classe operaia \t a quel punto si potra attuare rivoluzione \\
Il capitalismo occidentale deve essere importato \t poi andra distrutto, ma manca forza che si oppone \\
I socialisti russi fonderanno nel 98 il partito social democratico russo \t sulla base di queste idee \\
A partire dal 1903, si divide in due correnti:
\bi
    \item bolscevichi: guidati da Lenin, si rispechiano nel testo di lenin (corrsipondono ai massimalisti)
    \item menschevichi: sono guidati da Martov \t sono + moderati, il modello di martov è la social-democrazia tedesca \t partito + elettorale e democratico
\ei
Nel 1912 frazione diventa una frattura \\
Lenin espone la sua tesi nel Che Fare? \t riconosciuto come un documento programmatico, in cui spiega come intende la politica \\
Tesi di aprile è un documento interventista \t programmava la rivoluzione \\
(1) per lui la rivoluzione deve avere la precedenza su ogni rivendicazione sindacale \t i sindacati non funzionano, il sistema è sbagliato \\
I sindacati trattano con uno stato che è in essere \t non va bene, si deve fare la rivoluzione (prospettiva massimalista) \\
(2) rivoluzione non puo essere spontanea perche fallisce se no \\
(3) riv. deve essere guidata da dei professionisti della politica \t che non sono dei dirigenti, ma la punta del proletariato, che guida la classe operaia \\
(4) il partito deve essere monolitico \t non aspira a un partito democratico \t partito deve avere una idea che deve essere condivisa da tutti, ed è l'idea che porta avanti la rivoluzione \\
Non ci sono visioni diverse all interno del partito, che si possono confrontare

\v

Il che fare viene pubblicato da Lenin nel 1903 \t descrive la sua idea di rivoluzione \\
Tutto questo accade in esilio \t la russia non prevede delle opposizioni interne \\
Prima opposizione si verifica nel 1905  \\

\ss[Rivoluzione del 5]
Russia esce da conflitto con giappone, combattuta in crimea \t guerra peggiora situazione della popolazione \\
Nel gennaio del 5 \t 40 mila persone sfliano guidate da Capon??? (un prete) \t arrivano al palazzo d'inverno, e volevano la protezione dello zar \\
Ma l'esercito apre il fuoco sulla folla \t passa alla storia come la domenica di sangue \\
Questo scatena delle rivolte in tutto il paese \t si ribella la borghesia, che non aveva possibilita di arricchimento con lo zar \\
Nasce partito democratico costituzionale \t chiamato dei cadetti \t hanno una visione liberale e chiedono un sviluppo ecomico \\
Nicola II si spaventa \t dalle fila della borghesia si leva opposzione \t quindi promette molte libera politiche e concede la Duma \\
È una assemblea consultiva, dove poteva essere rappresentatno anche il popolo \\
Ma la protesta continua e si allarga all esercito \t l armata Potionki ammutina e soldati si rifiuteranno di aprire il fuoco sui rivoltosi \\
Culmine in ottobre \t si crea il primo soviet a san pietroburgo \t il capo eletto è il menscevico Trozky \\
Soviet = consiglio \t è un consiglio della fabbrica \\
I lavoratori si riconsocono nel soviet anche politicamente (e non nella duma) \t soviet è la vera istituzione politica per gli operai \\
Ma non nasce con scopo politico \\
Le dume successive non avranno mai successo \t verrano sciolte ogni volta che rappresentavano minima opposizione

\v

Dopo scoppia la guerra \t ma l economia era troppo debole per sostenere una guerra del genere \\
Durangte la guerra tutte le risorse economiche vengono orientate alla guerra nelle altre nazioni \t ma russia non puo farcela \\
I contadini vengono arroulati \t i campi rimangono incolti, quindi si generano carestie \t inoltre non addestrati \\
La guerra è il detonatore della rivoluzione \t in questo contesto la russia arriva al 17 \\
Lenin viene fatto rientrare in russia nell aprile, su un treno tedesco \t la germania aveva interesse che cio succedesse per indebolire russia
\end{document}
