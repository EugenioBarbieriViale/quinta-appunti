\documentclass[12pt]{article}

\usepackage[a4paper, total={6in, 8in}]{geometry}
\usepackage{textcomp}

\begin{document}
\setlength{\parindent}{0pt}

\def \t {\textrightarrow}
\def \v {\vspace{1em}}
\def \bi {\begin{itemize}}
\def \ei {\end{itemize}}
\def \s[#1] {\section*{#1}}
\def \ss[#1] {\subsection*{#1}}
\def \sss[#1] {\subsubsection*{#1}}

\s[Fascismo]
\ss[Politica economica]
Può essere divisa in due momenti \t prima fase fino al 25 \t è ancora liberista qua e si muove ancora nelle strutture dello stato liberale \\

\ss[Prima fase]
Libera impresa, incentivato dallo stato l'iniziativa privata, ridotta la spesa pubblica \t quindi italia cresce e approfitta di una congiuntura economica internazionale favorevole \\
La lira è ancora fruttuante come valore sul mercato \\
A partire dal 1926 mussolini cambia radicalmente la sua politica e chima Giusppe Volpi come ministro delle finanze \t vuole raggiungere la stabilita della lira e quindi attua politica interventista \\
Fa misure protezionistiche, dazi doganali (soprattutto su merci dalla francia), potenzia l'intervento dello stato, potenzia l'area dell'agro pontino e costruisce nuove citta (come Sabaudia e Latina) \\
Migliora anche le tecnice di irrigazione e incrementa il terreno coltivabile \\
Lui vuole raggiungere l'autarchia ed essere totalemente autosufficiente \t soprattutto nel settore agricolo e nei cereali (per questo la battaglia del grano con la francia) \\
Anche Hitler vuole raggiungere questo \t per mussolini pero lo scopo è rendersi forte e lavorare per stabilizzare la lira \t mentre per hitler lo scopo è la guerra \\
Hitler vuole autarchia alimentare per la guerra \t germania deve essere autosufficiente \\
Mussolini vuole evitare di dipendere dalle importazioni stranieri \t la stabilita della lira viene raggiunta nel 1927 circa

\v

In generale pero politica economica ha costi elevati \\
Per esempio battaglia grano (ovvero i dazi) va a trascurare altri settori come l'allevamento e tutte le colture che vengono esportate \\
Gli altri settori risentono di questa politica \\
Inoltre Italia è povera di materie prime \t quindi politica autarchica porta a una mancanza di materie prime \\
Anche la rivalutazione della lira ? \\
Ci sono quindi ricadute negative sul lato sociale ed economico 

\ss[Seconda fase]
Corporativismo \t doveva abolire tutte le rivendicazioni sociali \t e scioperi vanno aboliti \\
Nel 25 Confindustria e i sindacati fascisti firmano accordo (legge poi nel 26) \t hanno validita giuridica solo i sindacati fascisti \\
L'azione dei sindacati socialisti e com. (che era fuorilegge) viene bandita completamente ora (prima rimaneva) \\
Corporativismo doveva essere un modo per far parlare operai e imprenditori \t e doveva riprendere le corporazioni, dove c'erano sia gli operai sia gli imprenditori \\
Qui ci sarebbero dialogo pacifico \t e viene presentato come metodo innovativo (ma era gia medievale) \\
Le corporazioni falliscono pero \t perche vengono controllate dagli imprenditori e dialogo non accade \\
Sistema corporativo viene legittimato nel 27 con la Carta del Lavoro \t tutti i settori si devono organizzare in corporazioni (ovvero organizzazioni con lavoratori e imprednitori dello stesso settore) \\
Nasceva anche un miniestero che regolava queste corporazioni \t poi intervento statale diventa ancora + intenso col crollo di Wall Street nel 29 \\
Lo stato italiano deve intervenire sulla politica economica + pesantemente \t nel 31 viene fondato l' INI (istituto mobiliare italiano, beni mobili = soldi)  = ente di credito pubblico che sosteneva le industrie in difficolta \\
Stato si vuole sostituire alle banche \\
Nel 33 IRI (istituto ricostruzione industirale) \t è un ente statale che entra nel privato comprando azioni (azionista di maggioranza pero) e prende controllo di alcune aziende grandi \\
Prima fa credito e basta, poi entra attivamente come socio di maggioranza \t e questo sistema salva molte imprese, grazie ai finanziamenti pubblici \\
L'IRI pero dal 37 diventa permanente \t uno dei presidenti sara poi anche Romano Prodi \\
Questo è da leggere anche nella propaganda \t se si salvano imprese, questo genera consenso perche si salvano posti di lavoro e genera consenso nelle classi dirigenti \t stato fascista diventa punto di riferimento dell'economia italiana \\
Allo stesso tempo si favorisce economia

\ss[Guerra in Etiopia]
Tutto questo con nazionalismo, che nasce con il mito della vittoria mutilata (patto di Londra non rispettato) \t l'italia doveva avere un riscatto \t Mussolini cavalca l'onda \\
Nazionalismo lo spingera a intraprendere la guerra d'Etiopia \\
Nel 34, quando decide di conquistare l'etiopia, i fatti cambiano (riguardo ai rapporti con le altre potenze) \\
Prima c'era dialogo con le democrazie occidentali \t che lo ritenevano come un baluardo contro il comunismo \\
Viene molto considerato e i rapporti sono buoni \t fino al 34, quando inizia politica coloniale \\
Voleva dare all'italia un impero \t e c'era illusione che Etiopia poteva diventare meta per imigrazione italiana \\
Senza dichiarazione di guerra invade l'etiopia nel 35 \t e l'etiopia era uno stato sovrano ed era una monarchia \\
Italia si muove dall'eritrea e nel 36 viene presa Adis Sabeba \t re etiope deve scappare, ma etiopi si oppongono \\
C'è grande resistenza e un impegno bellico molto superiore a quello preventivato \t inoltre mussolini era convinto che la comunita internazionale sarebbe stata indifferente \\
Poi pero Societa delle Nazioni condanna l'italia come un paese aggressore, e quindi vengono imposte sanzioni economiche (anche etiopia era membro della societa delle nazioni) \\
Queste sanzioni in realta sono inutili \t vietano solo la vendita all'italia dei beni militari \t ma non toccano materie prime \t in realta quindi societa nazioni non era interessata (tantè che non verrano neanche rispettate dagli stati europei) \\
Mussolini pero puo ancora far leva su questo fatto e critica le potenze europee \t propaganda \t episodio dell Oro alla patria \t invita i cittadini a donare l'oro per sostenere questa causa \\
La stampa nel frattempo denigrava gli etiopi \t e non c'era voce di opposizione \t momento di + consenso al fascismo \\
Sempre nel 36 muss. inaugura l'impero dell'africa orientale italiana \t e offre al re la corona \\
Poi tutte sanzioni della societa delle nazioni vengono ritiare e gran bretagna e francia riconoscono l'impero d'etiopia (in realta inglesi non volevano combattere con gli etiopi contro italia, quindi accetta) \\
È evidente che societa delle nazioni è inutile 

\v

Ma etiopia è comunque povera di risorse naturali \t quindi il ritorno economico è minimo \t ma l'importante era aver creato l'impero
\end{document}
