\documentclass[12pt]{article}

\usepackage[a4paper, total={6in, 8in}]{geometry}
\usepackage{textcomp}

\begin{document}
\setlength{\parindent}{0pt}

\def \t {\textrightarrow}
\def \v {\vspace{1em}}
\def \bi {\begin{itemize}}
\def \ei {\end{itemize}}
\def \s[#1] {\section*{#1}}
\def \ss[#1] {\subsection*{#1}}
\def \sss[#1] {\subsubsection*{#1}}

\s[Rivoluzione russa]
Lenin deve affrontare la guerra civile e le armate dell'intesa che invadono la russia \t volevano riinsediare il governo zarista \\
Ma le due guerre si intrecciano \t potenze straniere sostengono l'armata bianca (i fedeli allo zar) \\
Lenin combatte contro entrambi \t schiera l'armata rossa \t esercito bolscevico del 1918 di Trozky \\
3 milioni di morti \t tra cui lo zar e la sua famiglia \t erano stati fatti prigionieri e nel 18 vennero giustiziati \t erano il simbolo del potere passato \\
C'era il timore che l armata bianca li potesse liberare \\
Guerra civile è molto violenta e aggressiva \t la storiografia si è interrogata come Lenin abbia potuto superare tutto qeusto \\
Vince grazie ai contadini che stanno dalla sua parte \t scelgono il male minore \t la rivoluzione si puo fare solo col popolo contadino \\
Nel 20 rivoluzione si conclude \t ultimo attacco all armata rossa arriva dalla popolnia \t russia è dilaniata, quindi polonia cerca di riappropriarsi dei territori persi durante trattato di Versaille \\
Russi respingono e con controeffensiva arrivano a Varsavia \\
Nel 21 si conclude la guerra con l'acquisizione della polonia di parte della Bielorussia e della Ucraina

\v

Nel 18 entra in vigore la 1 costituzione sovietica 
\bi
    \item repubblica federale
    \item a questa repubblica si possono aggregare altre repubbliche socialiste (sia in russia, sia straniere)
\ei
La costituzione afferma i diritti del popolo sfruttato e soppresso \\
Nel 22 nasce la URSS \t unione delle repubbliche socialiste sovietiche \\
Nel frattempo le guerre civili avevano fatto si che i bolscevici accentuassero i caratteri autoritari \t menscevichi e social-riv. erano fuori legge \\
Inoltre viene reintrodotta la pena di morte (prima rimossa con la riv.) \\
Inoltre nasce la keka \t polizia molto violenta \t polizia di regime \\
Ma lo spirito iniziale del bolscevismo non viene rispettato \t il potere è nei commissari del popolo ed è totalitario \\
Inoltre la condizione economica è grave \t sia per costi di attuare questo sia per la guerra \\
Poi la situazione diventa disastrosa \t i contadini che avevano ricevuto parte della terra, lavoravano per il loro sostentamento \\
Producevano lo stretto necessario per sostenersi \t nelle citta non arrivava nulla \\
Il governo è nel caos \t non è in grado di riscuotere le tasse \t inflazione sale perche stampa solo soldi \\
Inoltre gli operai che si sono appropriati delle fabbriche le sfruttano per loro

\v

Nel 18 Lenin approva la politica economica del "comunismo di guerra" = politica autoritaria \\
\bi
    \item terre vegnono nazionalizzate
    \item medie industrie dello stato
    \item soppresso libero mercato \t contrllo statale e distribuzione dei beni
\ei
Per risovlere rifornimenti alimentari \t mandati commissari bolscevichi che gli strappano i loro beni per ridistribuirli \\
I contadini si oppongono, e uccidno il bestiame e bruciano i campi \t pur di non darle allo stato \t le ritenevano un loro diritto \\
Inoltre ci sono sommosse \\
L'alleanza con contadini e bolscevichi si incrina \t fino al 21 \\
Nel 21 Kronstad \t si ribellano i marinai \t prima avevano appoggiato i bolscevichi \t si diffonde malcontento \\
La rivolta di kronstad viene repressa spietatamente \\
Nel 21 Lenin inaugura la NEP \t nuova politica economica \t segna fine del comunismo di guerra (18-21) \\
Viene inaugurata al decimo congresso del partito comunista (a mosca) 

\ss[Nep]
La nep consiste:
\bi
    \item ai contadini viene concesso di coltivare per la loro necessita, una parte allo stato e poi possono vendere cio che è in eccedenza
    \item viene legalizzato il mercato nero dei beni di prima necessita \t cosi si permette una maggiore circolazione dei beni, e si vuole aumentare il potere economico dei commercianti \t commercio libero di beni di prima necessita
    \item viene creato un sistema industriale di produzione misto \t intervento del privato all'interno delle fabbriche \t stato mantiene il controllo delle industrie, mentre il privato entra in quelle medie (come socio)
\ei
Cosi i vecchi proprietari delle industrie si riappropriano in parte delle loro industrie \t perche finanziano, diventano soci \\
Nep contestata perche reintroduce la proprieta privata \\
Viene contestata anche dai bolscevichi \t nep non coinvolge gli operai, che vengono trascurati \t vengono favoriti solo i contadini cosi \\
Nep piace poco \t ma è vincente \t le condizioni dei contadini aumentano \t tornano i beni di consumo nelle citta \\
Nel 26 la produzione industriale e agricola tornano ai livelli pre-bellici \t si chiude la parentesi delle guerre \\
Nep era stata voluta da Lenin \t le sue azioni si concentrano nella politica interna \\
Negli altri stati europei c erano dei tenativi insurrezionali che per falliscono \t l unica rivoluzione socialista sarabbe stata solo in russia \t negli altri paesi no \\
Aveva quindi l urgenza di stabilizzare il potere politico ed economico della russia \t all inizio voleva esportare la rivoluzione, ma si rende conto che non si puo

\v

Nel 10 congresso viene vietato anche il frazionismo = correnti all'interno del partito \\
Leinin nel che fare voleva un paritito monolitico \t unica idea \t all interno del partito comunista non ci dovevano essere contrasti \\
Viene affermato il centralismo democratico \t vuol dire che una volta che il partito aveva assunto una posizione, nessuno poteva contestarla \\
Se qualcuno non la condivideva, poteva esprimersi all'interno del paritio, ma non doveva uscire dal parito \t la posizione di facciata doveva rimanere quella \\
Questo vuol dire autoritarismo \t si accentua il carattere autoritario del paritio \t nessuno puo assumere posizione diverse ufficialmente \\
La dittatura del proletariato sta diventano la dittatura di alcuni direttori bolscevichi \t tra cui stalin 

\v

Nel 22 Lenin ha un ictus \t muore nel 24 \\
Gia a partire dal 22 si apre il problema sulla successione \t si scontrano Trozky e Stalin, che simboleggiavano le due anime del partito \\
Lenin le aveva sempre tenute insieme \t lui pero inizia a cedere \t unita viene meno \\
Stalin è un dirigente del paritio \t è conosciuto e ha posizione politica favorevole \t è inserito nel contesto della riv. \\
Trozky era l'ideologo della rivolzuione \t ma non aveva posizioni plitiche \t è l'intellettuale \\
Gode di un prestigio personale, ma non ha l'appoggio del parito \\
Stalin e Trozky sono in disaccordo su:

\sss[La gestione del partito]
Gestione autoritaria per Trozky non va bene \t troppo autoritaria \t serve partito + democratico e pubblico \\
Trozky si fa promotore di un manifesto con dei firmatari che si oppono all'autoritarismo del partito \t quando ancora Lenin era in vita \\
Stalin invece è daccordo con lenin \t vuole partito monolitico \t è dirigente del partito e vuole potere

\sss[Nep]
Per Trozky favorisce solo i contadini e tradisce i valori del comunismo \\
T. vuole accellerare il processo di industrializzazione e collettivizzazione forzata \t cosi che stato ha controllo totale sull economia \\
Questo è quello fara Stalin quando abbandona la nep nel 27 \\
Ora pero Stalin dice che nep va bene e funziona 

\sss[Esportazione della riv.]
Trozky dice che deve essere esportata \t se la riv. rimane un esperimento solo sovietico riv. verra distrutta \t le potenze intorno schiaccerano la russi a\\
Lui sosteiene la rivolzuione permanente \t bisogno esportare la rivoluzione in tutto l'occidente \\
Stalin invece sostiene "il socialismo in un solo paese" \t anche lui pensava che socialismo si doveva diffondere, ma nel breve tempo pensava che non c era la possibilta \\
Voleva consolidare il socialismo in russia \t russia doveva diventare forte per reggere contro il mondo capitalista \t potenze non schiacceranno russia se diventa forte

\v

Stalin vince \t è segretario del parito \t e perche riesce a far riconoscere la russia come stato \\
Nel 24 ottiene riapertura della russia nelle comunicazioni diplomatiche con altri paesi \t altri stati riconoscono URSS \\
Quando ottiene questo riconoscimento da l idea che il momento riv. si è concluso \t per urss nuova fase storica, che è quella della stabilita \\
Urss doveva diventare potenza militare ed economica \\
Lenin nel 26 muore \t Stalin al potere \t nel 27 trozky viene espulso dalla russia e verra ucciso nel 40 in messico da un sicario di Stalin\\
Al fianco di Stalin rimane solo Bucharin \\
Quando Stalin sale al potere lascia in vigore la nep \t fino al 27 \t peggiore situaizone economica della russia \t poi predne politca di Trozky della prima
\end{document}
