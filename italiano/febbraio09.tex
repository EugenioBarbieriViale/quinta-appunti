\documentclass[12pt]{article}

\usepackage[a4paper, total={6in, 8in}]{geometry}
\usepackage{textcomp}

\begin{document}
\setlength{\parindent}{0pt}

\def \t {\textrightarrow}
\def \v {\vspace{1em}}
\def \bi {\begin{itemize}}
\def \ei {\end{itemize}}
\def \s[#1] {\section*{#1}}
\def \ss[#1] {\subsection*{#1}}
\def \sss[#1] {\subsubsection*{#1}}

\s[Pascoli]
\ss[Arano]
Specifismo pascoliano, bozzetto, segantini per la natura, le caratteristiche testuali dell'autore (sinestesie anagologie etc.) sono presenti \\
Presenza umana e le presenza che sostituiscono l'uomo \t la solitudine umana \\
È il ritratto della rivoluzione che coinvolge il paesaggio del fine 800 \t Myricae ha come hanno cardine il 91, ovvero la prima edizione \\
I canti di castelvecchio invece sono 1912 \\
Lo specifismo e l'attenzione alla zoologia, l'attenzione al mondo animale in tutte le sue caratteristiche \t questo si aggiunge al macrotema dell'operosita dell uomo \\
Nello spirito di Rousseau, evoca una natura che protegge l'uomo dalla contaminazione dei vizi \t gli uomini qua svolgono un lavoro manuale

\v

Inizia con una atmosfera ovattata dell ambiente autunnale \\
Sinestesia, analogia e onomatopea in un solo verso (quello finale) \t condensazione di figure retoriche \\
Sottile tintinno è una sinestesia \t mentre tintinno evoca l'onomatopea, mentre come d'oro è un analogia (non similitudine perche non c'è il terzo in comparationis) \\
Roggio = rosseggiante, colore dell'autunno, e pampano = foglie della vite \t specifismo \\
Simmetria che dovrebbe portare ordine \\
Questo presenta una routinarieta \t l'operosita dell uomo si configura poi come cenrale \\
Fratta = cespuglio \\
Verso 3/4 enjambement \t mentre nebbia mattinale c'è iperbato \\
Il vapore caratterizza molti ambienti paessaggistici autunnali, e qua il vapore esala e sale \t una nuvola che sale \\
La presenza impersonale \t non c'è soggetto \\
Lente lente \t anafora e chiasmo \\ 
L'uno e l'altro \t impersonalita, non c'è soggetto \\
Il silenzio umano è interrotto dalle grida \t accompagnano all'atteggiamento di questi animali, che si muovono lentamente \\
Marra = bastone \t è paziente perche i tempi della natura e degli animali sono lenti \\
Il passero saputo ricorda il passero solitario \\
Il passero spia da una visuale ampia senza interferenze \t i rami irti, immagine che si ritrova in montale \\
L'azzurro del cielo e questo paesaggio che non ha confini \t è concepito dal passero in modo limitante, è saputo ma non puo avere una dimensione totalizzante del paesaggio \t allusivita molto forte \\
L'uomo non puo conoscere tutto, gli è negata la possibilita di consocere la totalita \\
Passero \t indica una genericita, per poi nella specificita con pettirosso \\
I rami del moro sono irti perche pungenti \t e il moro è il gelso \\
S'ode \t non si ha soggetto e forma impersonale \t la forma impersonale caratterizza il testo in una forma sospesa, in cui le percezioni visive sono nella prima parte, uditive nella seconda \t no percezioni tattili \\
L'immagine finale suggella la percezione evocativa \\
Bozzeto pascoliano \t completato dalla nebbia \\
Nella lirica di pascoli la natura è come una protezione nell'attesa, l'attesa della morte \t che viene anticipata da sensazioni sonore e uditive

\ss[Assiuolo]
Anche qua presenza animale \\
L'assiuolo è un uccello \t e l'ornitologia è approfondita da Pascoli \\
Critici come Emengaldo, Cataldi \t hanno evidenziato questo, ovvero la competenza nel mondo animale dell'autore \\
Perche lui si dedica a osservare le differenze tra un animale e l altro? \\
L'immagine che traspare è decadentista, e giustificata dalla presenza di visioni sovrapposte tra loro \t che in parte evocano
\end{document}
