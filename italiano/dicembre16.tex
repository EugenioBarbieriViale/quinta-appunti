\documentclass[12pt]{article}

\usepackage[a4paper, total={6in, 8in}]{geometry}
\usepackage{textcomp}

\begin{document}
\setlength{\parindent}{0pt}

\def \t {\textrightarrow}
\def \v {\vspace{1em}}
\def \bi {\begin{itemize}}
\def \ei {\end{itemize}}
\def \s[#1] {\section*{#1}}
\def \ss[#1] {\subsection*{#1}}
\def \sss[#1] {\subsubsection*{#1}}

\s[Zibaldone]
Opera parcellizzata e filosofica \t la sua scrittura è palpitante e lo vede impegnato saltuariamente \\
Lo zibaldone è una pietanza mista \\
Opera filosoficamente concepita \t ma leopardi non fu mai filofo tale da abbracciare un certo pensiero, ma costruisce la sua filosofia \\
La disperazione dei suoi giorni e i tradimenti che vengono dai cari \\
Porta alle illusioni, i ricordi, la ricerca di un piacere mai perfetto, che si scontra con il pericolo di essere profondamente infelice \\
Bukowski parla della disgrazia di essere intelligenti e di porsi delle domande \t di porre sempre sotto osservazione i nostri comportamenti \\
Bukowski non è lontano dalla dimensione di Leopardi \\
Lo zibaldone accompagna la produzione dell'autore \\
Noi leggiamo nell'opera tutto cio che troviamo nelle sue liriche \t se la domanda è le fasi del pensiero leopardiano, si parla di pessimismo e nei termini della sua formazione \\
I pensieri dello zibaldone sono incastonati in un percorso evolutivo \t le ultime lettere di Jacopo Ortis sono un romanzo, questo non lo è \\
È una raccolta che permette di accedere alla vigta dell'autore \\
Nelle lettere del compito c'erano delle lettere a cuore aperto e molto personale \\
L'assassino di Umbero Saba \t che è psicanalitco prima della psicanalisi, come Leopardi \t in Saba l'assassino è il padre \t conflittualita con il padre e disallineamento, mancanza di comunicazione \\
Fa perdere il senso di un legame forte, che è quello genitoriale
\end{document}
