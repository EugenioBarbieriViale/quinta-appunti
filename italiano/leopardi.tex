\documentclass[12pt]{article}

\usepackage[a4paper, total={6in, 8in}]{geometry}
\usepackage{textcomp}

\begin{document}
\setlength{\parindent}{0pt}

\def \t {\textrightarrow}
\def \v {\vspace{1em}}
\def \bi {\begin{itemize}}
\def \ei {\end{itemize}}
\def \s[#1] {\section*{#1}}
\def \ss[#1] {\subsection*{#1}}
\def \sss[#1] {\subsubsection*{#1}}

\s[Giacomo Leopardi]
Distanza temporale grande \t ma l'anima incontra quella di ognuno \t ambito topografico circoscritto, è riuscito a parlare il linguaggio di un romanticismo europeo, senza spostartsi dalla penisola \\
È più rivolto all'interiorita, è introflesso \t introflessione è importante \t questo chiama a ase un altra considerazione \\
Quella che si apre con la domanda: fu poeta-filosofo o filosofo-poeta? \\
Profonda analisi del cuore e dei sentimenti \\
Autore si pone domande esistenziali \t che riguardano lo stare dell uomo nella realta 

\ss[Vita]
Leopardi nasce in un contesto difficile \t a recanati, piccolo borgo \t ha pero il pregio di affacciarsi sul mare ed essere vicino alle montagne \\
Presenta pero limitazioni politiche \t e presenta tradizioni conservatrici, retrogrado dal punto di vista della mentalita \\
Il conte Monaldo \t è il padre \t ha dei possedimenti terrieri e vive normalmente, moglie ?? \t difende i principi conservatori \\
Famiglia è poco propensa a una prospettiva di innovazione dei figli \\
Però il padre era protettivo dei figli \t in particolare di giacomo \t che era esile e aveva problemi di salute fin da piccolo \\
Madre ha atteggiamento di protezione e ha relazione fredda con lui

\v

Presto sfrutta la biblioteca paterna \t legge molto \\
Legato alla figura dei fratelli, ma soprattutto di Paolina \t è colei che lo protegge e supporta \t viene anche aiutato dal fratello Carlo \\
Però l'esclusione non viene dall'esterno \t viene dalla situazione stessa \t non sono gli altri che lo escludono, ma il suo stato di salute \\
Il prodotto della sua letteratuta \t non viene influenzato dalla sua salute \\
Poi si scontra con De Sinner \t lui lo accusa di essere in una fase della sua produzione troppo vittimista \t leopardi risponde che lui tratta il dolore dell'umanita, non il suo \\
Non è la circostanza, ma la sostanza

\v

La prima fase \t dagli 1 ai 9 anni \t fase dell'erudizione (primo periodo leopardiano) \t conosce il greco e il latino, l' arabo antico \t vive questa fase con una voracità di lettura \\
Legge i volumi della biblioteca paterna \t traduce a vista classici latini e greci \\
Rimane pero sempre rinchiuso in casa \t non comunica molto con i coetanei \t per lui leggere era varcare i limiti spaziali \\
La sua vita non scorre nell'assenza di contatti \t ha contatti amorevoli nella famiglia \t che pero pone dei divieti, e viene sostenuto dai fratelli \\
La famiglia è comunque anaffettiva \\
Paolina è importante \t poi va a Napoli, da Ranieri, che ha una figlia?? che lo accudira fino alla sua morte e si chiama anche lei Paolina 

\v

Poi fase della scoperta del bello \t conversione estetica \t la composizione, poesia, creativita \t dall'erudito al bello \\
Il bello non basta pero \t l'uomo innegabilmente si scontra contro qualcosa che non puo dominare \t fase successiva descrive la sua figura in toto 

\v

Fase della conversione al vero \t conversione filosofica ??? \\
\end{document}
