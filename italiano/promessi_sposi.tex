\documentclass[12pt]{article}

\usepackage[a4paper, total={6in, 8in}]{geometry}
\usepackage{textcomp}

\begin{document}
\setlength{\parindent}{0pt}

\def \t {\textrightarrow}
\def \v {\vspace{1em}}
\def \bi {\begin{itemize}}
\def \ei {\end{itemize}}
\def \s[#1] {\section*{#1}}
\def \ss[#1] {\subsection*{#1}}
\def \sss[#1] {\subsubsection*{#1}}

\s[Promessi sposi]
Valenza del romanzo 
\bi
    \item come strumento di diffusione di una societa borghese
    \item come superament totale della visione giansenista e calvinista
\ei
1861 \t "Della lingua italiana e dei mezzi per diffonferla" \t la lingua italiana si deve diffondere \\
Ciò che dante aveva gia predisposto nel suo "De vulgaris eloquentia" \t ma manzoni deve scieglere il fiorenti, e sceglie qullo della borghesia colta \\
È un volgare purificato dalle sporie ecessivamente lombarde e francesi a cui l autore era incline (lui permane in francia) \\
C'è una presenza naturale del lombardo, rielaborata con il francese \t chi sono i due personaggi che danno a manzoni il metodo storico? Muratori e Fourielle \t dalla composizione delle tragedie stesse \\
La riflessione storica sulla dominazione longobarda, spunti sulla peste del 1648 \\
Ogni capitolo dei promessi sposi porta la data di ambientazione (mese + 1648) \t scansione temporale è importante

\v

PS: potente presenza storica e ampia stesura \t ha avuto successo nel pubblico dell'epoca \t perche argomenti storici avrebbero fatto sentire il popolo italiano unito \\
Per rendere il romanzo fruibile in tutta la penisola \t risolve la questione linguistica \\
Romanzo ha successo straordinario \t e viene chiamato per disporre le linee guida dell istruzione della nuova nazione italiana 

\ss[Gli snodi]
Monaca di Monza e il Conte del Sagrato (l'Innominato) \\
Innominato è l'esempio del calvinismo / giansen. \t quando si converte e piange, ha in mente le parole di Lucia (dio perdona ..?) \\
Un gesto di comprensione e di accettazione verso l'altro ha la valenza di redimere \\
Cardinale federigo borromeo \t buono, comprensivo, ma incarna anche il lato umano di una figura religiosa \t perdona ma tiene ancora conto del passato dell innominato \\
Figura narratologica = narratore interno, narratore zero, velocita di narrazione, analessi, prolessi \\
Don Abbondio rappresenta l emblema dell crisitano finto ?? \t nel testo è recitente \t reticenza è la figura narratologica che caratterizza la figure di don abbondio \\
Don abbondio \t "non era certo un cuor di leone" e vaso di ceramica tra i vasi di ferro (fig. retorica, metafora)

\ss[Le donne]
Figure femminili \t Perpetua puo essere associata ad Angese \t parole chiavi che identificano Agnese è "cuore e destrezze", e poi "coraggio" \\
Cuore e destrezza = rappresenta la sua saggezza popolare \t Perpetua non ha la stessa sagezza \t si lascia ingannare con un discorso sui suoi findanzamenti remoti, fatto da Agnese \\
Agnese accudisce anche \t analisi transazionale, Enric Bern, noi comunichiamo anche secondo un bagalio relazionale \t ribaltamento di ruolo \\
Agnese conforta Perpetua, ma non in modo onesto \t l'ha ingannata \\
Neanche luica è sempre onesta \t episodio di Fra Galdino e le noci \t la raccolta delle noci era scarsa, e quando lui arriva gli da un compito \t gli da una grande quantita di noci (redarguita da Agnese) \\
Ma non era un gesto di carita \\
Anche se è il simbolo della purita e del candore \t lei da qualcosa per avere qualcosa in cambio

\v

Agnese e perpetua: stessa estrazione sociale ma comportamenti diversi \\
Lucia \t colei che fa da motre immbolie, è il centro di un disegno provvidenziale \t tutto si snoda attraverso di lei \\
È pedina della providenza \t fa sua l'espressione "Dio che a terra suscita affanna e consola" \\
Segue la mano della provvidenza e realizza la computezza della sua figura = candore, pudore, rossore \t si esprime con questa sensibilità \\ 
Lucia è più sentimentale che sensibile \\
Pudore e rossore \t è pudica \t in tutto il romanzo non c'è uno scambio affettivo di intimità tra i due \t la cultura del tempo così voleva \\
Ma Lucia si distingue dalla sua madre, che è pratica, mentre lei è introspettiva 

\v

Monaca di monza \t figura reale storica \t attraverso la figura di Gertrude replica il reale \\
Moncaca ha anche valore educativo (rigurado ai giovani, comparati a dei fiori sferzati dal vento) \\
Importante è l'incontro tra Gertrude e Lucia \t Gertrude come donna, il ciuffo che fuoriesce (emblema del suo essere sprezzante delle regole), ma anche gli occhi che guizzavano \\
Occhi importanti in manzoni (come anche per san cristoforo), cintola che mostrava una femminilità \\
Monaca di monza è l'emblema di una femminilità costretta \t veste da suora copre la sua volonta di essere donna e adempie la costrizione di essere monaca \\
È una donna diversa da lucia, che però esprime la propria identità \\
Le due si incontrano e si vedono le diverse personalità \t lucia è rispettosa e riguradosa \\
La monaca fuoriesce dal suo ruolo ottemperando il male \t quando si ordisce il rapimento di lucia, architetta il tutto e la manda fuori dal convento con una scusa \t ha un senso di rimorso e la chiama indietro, ma poi la fa proseguire \\
Volonta di vendicarsi della sua costrizione \\
La conversione dell'innominato parte poi dal rapimento di Lucia \\
Monaca ha una repulsione delle regole 

\v

Lucia pero incontra anche la vecchia \t nel castello dell innominato \t aveva avuto l'ordine di farle coraggio \\
L'altra è donna Prassede \t che è la moglie del sarto \t le prepara un brodo ed è amorevole \\
La casa del sarto aggancia le figure femminili \t dove vi è anche un'immensità di libri \t ma il sarto non aveva letto nulla \\
Rappresenta solo lo sfoggio della cultura \\
Donna prassede suggerisce aiuti non richiesti e sottolinea la difficolta altrui \t Prassede corregge le lettere del marito \\
Tra le donne, quelle istruite erano la monaca di monza e donna prassede \t lucia non molto \\
L'analfabetizzazione era una piaga, che non riguarda solo le donne \t ma tutti i ceti umili \\
Cultura = interiorizzazione dell'istruzione \\
Azzeccagarbugli esercita una violenza su renzo, ma anche la monaca di monza su lucia \\
Se cultura rimane uno sfoggio, non è mai diventata cultura \t es. don abbondio \t non rispetta i valori che sono propri di un prete \\
Il clero consapevole \t cardinale federigo borromeo

\s[Recuperare da qui in poi - 07/10]


\end{document}
