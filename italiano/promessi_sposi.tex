\documentclass[12pt]{article}

\usepackage[a4paper, total={6in, 8in}]{geometry}
\usepackage{textcomp}

\begin{document}
\setlength{\parindent}{0pt}

\def \t {\textrightarrow}
\def \v {\vspace{1em}}
\def \bi {\begin{itemize}}
\def \ei {\end{itemize}}
\def \s[#1] {\section*{#1}}
\def \ss[#1] {\subsection*{#1}}
\def \sss[#1] {\subsubsection*{#1}}

\s[Promessi sposi]
Valenza del romanzo 
\bi
    \item come strumento di diffusione di una societa borghese
    \item come superament totale della visione giansenista e calvinista
\ei
1861 \t "Della lingua italiana e dei mezzi per diffonferla" \t la lingua italiana si deve diffondere \\
Ciò che dante aveva gia predisposto nel suo "De vulgaris eloquentia" \t ma manzoni deve scieglere il fiorenti, e sceglie qullo della borghesia colta \\
È un volgare purificato dalle sporie ecessivamente lombarde e francesi a cui l autore era incline (lui permane in francia) \\
C'è una presenza naturale del lombardo, rielaborata con il francese \t chi sono i due personaggi che danno a manzoni il metodo storico? Muratori e Fourielle \t dalla composizione delle tragedie stesse \\
La riflessione storica sulla dominazione longobarda, spunti sulla peste del 1648 \\
Ogni capitolo dei promessi sposi porta la data di ambientazione (mese + 1648) \t scansione temporale è importante

\v

PS: potente presenza storica e ampia stesura \t ha avuto successo nel pubblico dell'epoca \t perche argomenti storici avrebbero fatto sentire il popolo italiano unito \\
Per rendere il romanzo fruibile in tutta la penisola \t risolve la questione linguistica \\
Romanzo ha successo straordinario \t e viene chiamato per disporre le linee guida dell istruzione della nuova nazione italiana 

\ss[Gli snodi]
Monaca di Monza e il Conte del Sagrato (l'Innominato) \\
Innominato è l'esempio del calvinismo / giansen. \t quando si converte e piange, ha in mente le parole di Lucia (dio perdona ..?) \\
Un gesto di comprensione e di accettazione verso l'altro ha la valenza di redimere \\
Cardinale federigo borromeo \t buono, comprensivo, ma incarna anche il lato umano di una figura religiosa \t perdona ma tiene ancora conto del passato dell innominato \\
Figura narratologica = narratore interno, narratore zero, velocita di narrazione, analessi, prolessi \\
Don Abbondio rappresenta l emblema dell crisitano finto ?? \t nel testo è recitente \t reticenza è la figura narratologica che caratterizza la figure di don abbondio \\
Don abbondio \t "non era certo un cuor di leone" e vaso di ceramica tra i vasi di ferro (fig. retorica, metafora)

\v

Figure femminili \t Perpetua puo essere associata ad Angese \t parole chiavi che identificano Agnese è "cuore e destrezze", e poi "coraggio" \\
Cuore e destrezza = rappresenta la sua saggezza popolare \t Perpetua non ha la stessa sagezza \t si lascia ingannare con un discorso sui suoi findanzamenti remoti, fatto da Agnese \\
Agnese accudisce anche \t analisi transazionale, Enric Bern, noi comunichiamo anche secondo un bagalio relazionale \t ribaltamento di ruolo \\
Agnese conforta Perpetua, ma non in modo onesto \t l'ha ingannata \\
Neanche luica è sempre onesta \t episodio di Fra Galdino e le noci \t la raccolta delle noci era scarsa, e quando lui arriva gli da un compito \t gli da una grande quantita di noci (redarguita da Agnese) \\
Ma non era un gesto di carita \\
Anche se è il simbolo della purita e del candore \t lei da qualcosa per avere qualcosa in cambio
\end{document}
