\documentclass[12pt]{article}

\usepackage[a4paper, total={6in, 8in}]{geometry}
\usepackage{textcomp}

\begin{document}
\setlength{\parindent}{0pt}

\def \t {\textrightarrow}
\def \v {\vspace{1em}}
\def \bi {\begin{itemize}}
\def \ei {\end{itemize}}
\def \s[#1] {\section*{#1}}
\def \ss[#1] {\subsection*{#1}}
\def \sss[#1] {\subsubsection*{#1}}

\s[Promessi sposi]
Valenza del romanzo 
\bi
    \item come strumento di diffusione di una societa borghese
    \item come superament totale della visione giansenista e calvinista
\ei
1861 \t "Della lingua italiana e dei mezzi per diffonferla" \t la lingua italiana si deve diffondere \\
Ciò che dante aveva gia predisposto nel suo "De vulgaris eloquentia" \t ma manzoni deve scieglere il fiorenti, e sceglie qullo della borghesia colta \\
È un volgare purificato dalle sporie ecessivamente lombarde e francesi a cui l autore era incline (lui permane in francia) \\
C'è una presenza naturale del lombardo, rielaborata con il francese \t chi sono i due personaggi che danno a manzoni il metodo storico? Muratori e Fourielle \t dalla composizione delle tragedie stesse \\
La riflessione storica sulla dominazione longobarda, spunti sulla peste del 1648 \\
Ogni capitolo dei promessi sposi porta la data di ambientazione (mese + 1648) \t scansione temporale è importante

\v

PS: potente presenza storica e ampia stesura \t ha avuto successo nel pubblico dell'epoca \t perche argomenti storici avrebbero fatto sentire il popolo italiano unito \\
Per rendere il romanzo fruibile in tutta la penisola \t risolve la questione linguistica \\
Romanzo ha successo straordinario \t e viene chiamato per disporre le linee guida dell istruzione della nuova nazione italiana 

\ss[Gli snodi]
Monaca di Monza e il Conte del Sagrato (l'Innominato) \\
Innominato è l'esempio del calvinismo / giansen. \t quando si converte e piange, ha in mente le parole di Lucia (dio perdona ..?) \\
Un gesto di comprensione e di accettazione verso l'altro ha la valenza di redimere \\
Cardinale federigo borromeo \t buono, comprensivo, ma incarna anche il lato umano di una figura religiosa \t perdona ma tiene ancora conto del passato dell innominato \\
Figura narratologica = narratore interno, narratore zero, velocita di narrazione, analessi, prolessi \\
Don Abbondio rappresenta l emblema dell crisitano finto ?? \t nel testo è recitente \t reticenza è la figura narratologica che caratterizza la figure di don abbondio \\
Don abbondio \t "non era certo un cuor di leone" e vaso di ceramica tra i vasi di ferro (fig. retorica, metafora)

\ss[Le donne]
Figure femminili \t Perpetua puo essere associata ad Angese \t parole chiavi che identificano Agnese è "cuore e destrezze", e poi "coraggio" \\
Cuore e destrezza = rappresenta la sua saggezza popolare \t Perpetua non ha la stessa sagezza \t si lascia ingannare con un discorso sui suoi findanzamenti remoti, fatto da Agnese \\
Agnese accudisce anche \t analisi transazionale, Enric Bern, noi comunichiamo anche secondo un bagalio relazionale \t ribaltamento di ruolo \\
Agnese conforta Perpetua, ma non in modo onesto \t l'ha ingannata \\
Neanche luica è sempre onesta \t episodio di Fra Galdino e le noci \t la raccolta delle noci era scarsa, e quando lui arriva gli da un compito \t gli da una grande quantita di noci (redarguita da Agnese) \\
Ma non era un gesto di carita \\
Anche se è il simbolo della purita e del candore \t lei da qualcosa per avere qualcosa in cambio

\v

Agnese e perpetua: stessa estrazione sociale ma comportamenti diversi \\
Lucia \t colei che fa da motre immbolie, è il centro di un disegno provvidenziale \t tutto si snoda attraverso di lei \\
È pedina della providenza \t fa sua l'espressione "Dio che a terra suscita affanna e consola" \\
Segue la mano della provvidenza e realizza la computezza della sua figura = candore, pudore, rossore \t si esprime con questa sensibilità \\ 
Lucia è più sentimentale che sensibile \\
Pudore e rossore \t è pudica \t in tutto il romanzo non c'è uno scambio affettivo di intimità tra i due \t la cultura del tempo così voleva \\
Ma Lucia si distingue dalla sua madre, che è pratica, mentre lei è introspettiva 

\v

Monaca di monza \t figura reale storica \t attraverso la figura di Gertrude replica il reale \\
Moncaca ha anche valore educativo (rigurado ai giovani, comparati a dei fiori sferzati dal vento) \\
Importante è l'incontro tra Gertrude e Lucia \t Gertrude come donna, il ciuffo che fuoriesce (emblema del suo essere sprezzante delle regole), ma anche gli occhi che guizzavano \\
Occhi importanti in manzoni (come anche per san cristoforo), cintola che mostrava una femminilità \\
Monaca di monza è l'emblema di una femminilità costretta \t veste da suora copre la sua volonta di essere donna e adempie la costrizione di essere monaca \\
È una donna diversa da lucia, che però esprime la propria identità \\
Le due si incontrano e si vedono le diverse personalità \t lucia è rispettosa e riguradosa \\
La monaca fuoriesce dal suo ruolo ottemperando il male \t quando si ordisce il rapimento di lucia, architetta il tutto e la manda fuori dal convento con una scusa \t ha un senso di rimorso e la chiama indietro, ma poi la fa proseguire \\
Volonta di vendicarsi della sua costrizione \\
La conversione dell'innominato parte poi dal rapimento di Lucia \\
Monaca ha una repulsione delle regole 

\v

Lucia pero incontra anche la vecchia \t nel castello dell innominato \t aveva avuto l'ordine di farle coraggio \\
L'altra è donna Prassede \t che è la moglie del sarto \t le prepara un brodo ed è amorevole \\
La casa del sarto aggancia le figure femminili \t dove vi è anche un'immensità di libri \t ma il sarto non aveva letto nulla \\
Rappresenta solo lo sfoggio della cultura \\
Donna prassede suggerisce aiuti non richiesti e sottolinea la difficolta altrui \t Prassede corregge le lettere del marito \\
Tra le donne, quelle istruite erano la monaca di monza e donna prassede \t lucia non molto \\
L'analfabetizzazione era una piaga, che non riguarda solo le donne \t ma tutti i ceti umili \\
Cultura = interiorizzazione dell'istruzione \\
Azzeccagarbugli esercita una violenza su renzo, ma anche la monaca di monza su lucia \\
Se cultura rimane uno sfoggio, non è mai diventata cultura \t es. don abbondio \t non rispetta i valori che sono propri di un prete \\
Il clero consapevole \t cardinale federigo borromeo

\s[Recuperare parte finale - 07/10]

Renzo e la sua involuzione \t chiude il cerchio con lucia \\
Inoctra al lazzaretto \t figura di padre cristoforo \t fa da collante per fare ritrovare tutti i personaggi, ma poi muore \t come se avesse realizzato il suo scopo \\
Renzo e lucia si ritrovano dopo una lunga separazione \t importanza dell'alfabetizzazione \t non hanno neanche contatti di scrittura \t separazione fisica ma non di anima \\
Tutto accade al lazzaretto \t ancora piu sintomatico: ritrovamente avviene in un luogo di dolore \\
Nel lazzaretto venivano ospitati i colpiti dalla peste \t rappresenta il flagello di dio \t come padre cristoforo che si ammala, anche lucia viene ricoverata????? \\
I critici parlano del romanzo di renzo \t egli arriva a milano dopo lasciare monza \t renzo prende la strada per milano \t milano viene descritta come una meraviglia e renzo rimane estasiato dal comportamento dei cittadini \\
Trovarsi in un luogo nuovo è per lui come trovarsi in una bolla \t renzo è pero la causa del suo male \t sceglie in nome della liberta e si distacca dalla provvidenza \\
Poi approda al convento di Bonaventura \t chiede al guardiano e porta con se una lettera \\
Ma al posto di aspettarlo nella chiesa esplora la citta \t curiosita della gioventu \\
Sceglie pero la via della perdizione anche in base al luogo che ritrova davanti a se \t ha due possibilita
\bi
    \item il bene = rimanere in chiesa ed attendere l arrivo del padre a cui consegnare a lettera
    \item il male = uscire e seguire la propria curiositas \t Apuleio
\ei
Inizia cosi una perdita del se, dei valori \t lui si ritrova davanti a una realta piu allettante \t ma di salvifico per renzo non c'è ancora niente \\
Per renzo milano è come il paese dell meraviglie \t ma presto si rende conto dello spreco di pane (c'era stata anche carestia, mancanza di grano) \\
Visione realistica di milano \t donne che fanno scorta di pane e hanno le pance piene \\
Non realizza pero che si sottendono delle tensioni politiche \\
Questo conduce alla perdita dell identita di renzo \t si confonde nella folla, sperimenta la violenza, si alimenta di quei valori che nella sua formazione sono gli antivalori \\
Si snatura \t diventa altro da se, mentre in tutto questo Lucia è assente \\
In questo contesto i fatti degenerano \t finche si ritova nell osteria della luna piena \t rappresenta la realta falsata, illusione, l inganno e l ignoto \t la luna piena era l emblema di un momneto di apice e si facevano riti \\
Mentalita contadina si riconosce \t milano era popolata comunque da contandini, nella periferia \\
Tutto questo renzo lo vive sulla propria pelle \t la fuga dall'osteria lo pone davanti a diversi ostacoli \\
Riaffiorano anche i ricordi dolorosi della lontanza da lucia, che equivale al perdersi \t il momento consolatorio diventa una rete che lo cattura e lo imbriglia \\
All osteria viene sollecitato a bere il "vino sincero" \t all osteria è il bravo giovane che rinnega se stesso, è l apice che diventa un rovinoso scendere verso l annullarsi \\
Quando deve fornire le sue generalita \t si ricorda di lucia \t la consolazione si incarna nel vino \\
Ma l epilogo successivo lo vede in fuga dalle guardie (perseguitato come riottoso) \t intraprende una nuova strada verso crescenzago???? \\
Poi arriva all adda \\
Osteria era luogo di ritrovo \t in questo caso è luogo di perdizione \t ma quando si trova li si sente libero di parlare e questo gli libera l'animo \\
Scappa dalle guardie \t velocita con cui renzo si aggrappa per risalire \\
La fuga avviene la mattina dopo \t poi va in altre osterie \t in una si rifocilla \t il suo cambiamento è altro che in atto e la sua interazione è moderata \\
Il suo parlare è controllato \t in qualche modo renzo non si autuolimita perche gli altri glielo impongono \t lo fa per proteggersi \\
La sua figura lascia presagire l'avvenuta maturazione \t parabola ora ascendente \\
La panca è affettivita, stabilita, ultimo assaggio di una visione familiare (che non ha piu, sua famiglia era lucia ed agnese) \\
A gorgonzola \t le panche lo vedranno più scaltro e attento, consapevole

\v

Incontra un mercante che gli ricorda i fatti di milano \t parla di un ragazzo scappato dalla guardie, ma lui non reagisce \\
Inizia quindi a proteggersi \t e imposta la prosecuzione del suo viaggio \t renzo sara a contatto con la natura del lombardo-veneto \\
In tutto questo aveva l'obbiettivo di raggiungere il cugino a bergamo \t aveva occupazione nel campo tessile \\
Renzo però è estenuato dal lungo viaggio \t anche clima non era favorevole (clime di novembre) \t approda in un punto che gli sembra protetto \\
Prima svolta \t ritrova la forza di pregare, come quando era bambino \t le preghiere lo portano a un sonno ristoratore \\
Poi si risveglia col suono dell adda e delle campane \t si sente a casa \t l adda viene anche menzionato all inizio del romanzo \\
Per lui l' adda è casa \t Bortolo è possibilita di una nuova vita \\
Parabola ascendente di renzo \t fase evolutiva della storia \\
Dopo che ha perdonato finalmente don rodrigo \t il cielo si fa scuro e il clima diventa afoso \t inizia a piovere, pioggia purificatrice e liberatrice \t peste si dissolve \\
Qui si attua veramente un atto di misericordia \t dio perdona renzo \t inizia un ciclo nuovo e i fatti portano a una conclusione \\
Romanzo senza idillio \t immagine rappresentativa di una ricomposizione perfetta dell'ordine \\
Idillium \t idilli di mosco \t un quadretto campestre, interagire umano con la natura \t con la quale uomo ha equilibrio perfetto \\
Pioggia risolutrice \\
Ma la fine del romanzo celebra l'idillio borghese dei personaggi \t no colpi di scena \t vicenda ed epilogo borghesi \t trionfa la normalita \\
Risulta eccezionale pero di fronte ai personaggi \t questi parametri corrispondo alla loro felicita (anche se felicita si rifersce a una morale laica, mentre di manz. è cattolica \t felicita terrena = centuplo) \\
Don rodgrigo realizza il centuplo \t perche muore perdonato da renzo

\v

Questa opera è la sola che rende gloria a manzoni? \t no \\
Il sistema manzoniano comprende anche le odi, 5 maggio, napoleone \t fama che varca il configini della peisola \\
Manzoni degli inni sacri \t feste religiose \t ma soprattoutto è il manzoni delle tragedie 

\s[Morte di Ermengarda]
Viene ripreso il genere della tragedia \\
Le tragedie di manzoni arrivano dopo quelle di Alfieri \t senechiane, sanguinose \\
Accanto alla storia sono presenti le fonti bibliche (il "Saul", ...) \t qui niente di tutto cio \\
Genere storico \t nel passato il centro della cultura era il nord, con Carlo Magno che era considerato il salvatore \\
Dall'altro lato i Longobardi erano i devastatori \\
Manzoni sceglie di guardare la faccia piu intima della storia \t Adelchi, figlio di Intimo, sovrano dei longobardi, e figlia Ermengarda \\ \t qui niente di tutto cio \\
Genere storico \t nel passato il centro della cultura era il nord, con Carlo Magno che era considerato il salvatore \\
Dall'altro lato i Longobardi erano i devastatori \\
Manzoni sceglie di guardare la faccia piu intima della storia \t Adelchi, figlio di Intimo, sovrano dei longobardi, e figlia Ermengarda \\
Ermengarda è sposata a carlo magno \t visto qui come sovrano ribaltato \t è il perfido che vuole mettere le mani sui possedimenti dei longobardi \\
La ragione dell inganno è Ermengarda \t che è santa del suo patire \t è vittima, colei che paga per salvare il suo popolo \\
Nel coro (il momento in cui l'autore poteva esprimersi direttamente, in manzoni si chiama il cantuccio dell'autore)

\v

Nel testo si focalizza sulla morte di ermengarda \t lo scopo di questo coro descrive il suo abbandonarsi all'aldila \\
In questo testo si ha la consapevolezza che manzoni concretizza la frase "algi umili non possono che subire????"

\v

Si trova in un convento a brescia \t si sta approcciando alla morte \\
Testo inizia con accusativo alla greca (sparsa le trecce, ma anche lenta le palme) \\
Pia \t religiosa \\
Tremolo lo sgurado \t leopardi usa questo termine nel ? alla luna \t sta combattendo per gustare gli ultimi attimi di vita ed approcciarsi all aldila \\
Cerca il cielo \t il centuplo non è possibile nella vita terrena \\
Le suore pregano per lei 

\v

Poi inizia una fase di rievocazione in cui la donna vorrebbe dimenticarsi \t sperimenta un delirio prima della morte \\
Una carrellata di immagine le scorrono davanti \\
Lui tradisce lei sotto i suoi occhi \t lei ricorda le batture di caccia e gli sguardi che rivolgeva alle altre \\
Ha volonta di dimenticare i torti \t prova ancora amore per lui \\
Lei chiude una prima parentesi della donna angelo \t è bionda, occhi azzurri, ha ancora nobilta d animo \t vive un amore interiorizzato non corrisposto \\
Immagine dell amore al femminile \t prima volta che si vede nella letteratura \t fino ad adesso non si è letto di una donna che ama non corrisposta \\
Manzoni la descrive nella sua purezza
\end{document}
