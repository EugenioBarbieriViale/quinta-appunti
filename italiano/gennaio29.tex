\documentclass[12pt]{article}

\usepackage[a4paper, total={6in, 8in}]{geometry}
\usepackage{textcomp}

\begin{document}
\setlength{\parindent}{0pt}

\def \t {\textrightarrow}
\def \v {\vspace{1em}}
\def \bi {\begin{itemize}}
\def \ei {\end{itemize}}
\def \s[#1] {\section*{#1}}
\def \ss[#1] {\subsection*{#1}}
\def \sss[#1] {\subsubsection*{#1}}

\s[Verga - Ripasso]
I beni materiali sono l'occasione, la possibilità di evolvere \t sono l'identificazione della persona, e io esisto perche ho \t non perche sono \\
I ganci con le novelle sono: "Nedda" che da il via all versimo, "La roba" che presenta con Mazzaro una personalita improntata sull'avere, e l'avere diventa occasione di accumulare \\
Le strategie narrarative dell'opera di mastro don sono il verismo e la cultura dell'inquietudine?? \\
Da un lato c'è la sicilia, dall'altro il nord \t da un lato una lettura 8centesca, mentre con Buzzati un esistenzialismo letterario \\
Esistenzialismo letterario permette anche di agganciarsi agli indifferenti di Moravia \t la figura di gesualdo presenta quindi diversi ganci al 900 \\
È un documento umano, ma non sul modello francese \t è un opera che mette gia in discussione il verismo, e in particolare il determinismo che dovrebbe dare sostegno ai personaggi \\
Il progresso viene criticato da verga, ed è il punto stonante con la scienza \t positivismo \\
Il romanzo fa evolvere la consapevolezza di un verga, che demistifica la fiducia nella scienza \t è quindi un romanzo spartiacque \\
Anche il modernismo di Fogazzaro \t parla dei confini della scienza e della fede \t dove si ha anche un punto di connessione con il manzonismo e il sensualismo d'annunziano \\
Nei malavoglia, ci sono elementi storici? \t no, non presenta prolusioni storiche con presunzioni dei veridicita (che invece c'è in mazoni, come la colonna infame) \\
La presunzione dell'autore è nella discussione dei valori del positivismo \t che supera e demistifica il verismo

\v

Il passaggio da verga all'inettitudine e la crisi dell uomo moderno \t sono gli ultimi due: onorevole scipione e ?? \\
SOno coinvolti anche i membri della borghesia e della casta politica \t a Roma \\
Onorevole scipione è vittima di corruzione \t e Roma viene descritta come era nell'antichita \t roma fotogrofa non i valori del successo dell'italia unita, ma una decadenza della classe politica \\
Roma viene descritta anche nell'uomo di lusso \t ed è la parabola che coinvolge tutti i gradi della societa \\
C'è un degradarsi valoriale, e tutto l'impianto post-unitario crolla attraverso la vuotaggine di questo personaggio, che ambisce a una vita di lusso ma è un artista \\
Questo è un antenato della figura di Stefano Balli \t "Senelita di Italo svevo" \\
La realizzazione di un ciclo, nel quale tutti i gradi della societa sono coinvolti nel deteriorarsi valoriale, è avventua con successo \t anche se le opere sono incomplete \\
Le prime due infatti avevano gia successo, e aveva altri progetti + si profilavano venti nuovi, che veicolavano l'eco di canzoni monotematiche: l'uomo non trova la felicita con la scienza, e non puo essere studiato a fondo da essa \\
L'uomo è una selva di simboli ed è l'espressione dell "onconnui" \t l'uomo è sconosciuto, e sono si puo studiare totalmente \t l'animo umano non puo essere rivelato \\
La lettura dell'animo umano, che ora non ha una sola dimensione cognitiva \t ma Freud apre a 3 diverse dimensioni, es ego superego \\
Le dimensioni relazionali vengono analizzate anche da Gardner, che parla di intelligenze multiple e intelligenza emotiva \\
INoltre si usa l'ipnosi per conoscere il se profondo \t la psiche umana non puo essere svelata, perche dipende da una volonta del soggetto \t il soggetto svela quello che ritiene di svelare \\
La psicanalisi arriva dopo la letteratura \t zeno cosini rivela quello che si sente di rivleare \\
La delusione post-compitana si ha ancora in  atto, e l'infelicita umana è da superare \t ma il come superarlo è ancora sconosciuto \\
Il percorso parte dalla svalutazoine del determinismo, si concentra su zeno cosini che anticipa la psicanalsi, si afferma in freud e sfocia nella scoperta dell'intelligenza multipla \t dimostra che l uomo non ha capito che prima di risolvere occorre accettare \\
"Piccolo mondo antico" di Fogazzaro \t diventa diverso a seconda delle potenzialita dei due personaggi: madre e padre perdono la bambina, che annega nel lago \\
Fogazzaro si configura come un erede di manzoni, ma estremamente innovativo, recuperanto il topos del lago \\
Il padre affronta la morte con il conforto della fede, mentre la madre non puo dimenticare il passaggio viscerale dal parto alla morte nel lago \\
Con l'aggravante del senso di colpa che si radica in lei \t scaturito dalla perdita, con questo la incapacita di accettare quel dolore \\
Solo in faccia al dolore e attraversandolo si puo superarlo \\
La morte della bambina è innaturale e viene gestita diversamente dai genitori \t nella visione della fede, la morte va in una dimensione piu grande, nella dimensione laica è una cosa innaturale \\
Carlo Gadda?? \t la concezione del dolore 

\v

Se prima c'era un post manzoni, ora c'è un post verga \t ovvero il neorealismo, e Luchino Visconti \t il verismo ha molto da comunicare a chi vive i drammi devastanti delle due guerre mondiali \\
Si ha una distruzione delle anime, dei valori e della fiducia dell'uomo in un mondo nuovo \t è il momento della ricostruzione, che fa il nostro 900 post bellico \\
Nel cinema si ha il neorealismo \\
"Ladri di biclette" di Visconti mette a nudo una societa che si arrampica per trovare una nuova vita, dove non sono i potenti a essere protagonisti \t questo nasce da Verga \\
Poi si traduce in letteratura neorealista \t che prende il via da Cesare Pavese \\
Cesare Pavese si suicida \t il mestiere di vivere, ma anche Vittorini, Calvino, Silone \t tante sono le sfaccettareure del neorealismo, con l'amore del vero \\
"Canto il vero" \t contestazione scapigliata

\v

Di queste figure rimane un attenzione verso il meridione \t è stat istituita una casa del mezzogiorno

\s[Decadentismo]
Nesso tra decadenza e decadentismo \t ci troviamo davanti a traccie di una romanita consunta, giardini isolati, colonne distrutte \t con fontane asciutte, statue mozze, una natura che porta con se le traccie della desolazione \\
Nel giardino del dolore di Leopardardi c'era gia \t ma il dolore esistenziale che si era tramutato in una risorsa, qua si fa come prospettiva di futuro e speranza \\
Si parla di un decadentismo europeo, e di un estetismo nostrano \t ma cosi si parla di D'annunzio, che pero c'è anche in europa \\
C'è una ventata di rigorismo estetizzante \t c'è l'immagine di un eroe, e di modelli che rifugiano la insoddisfazione per la vita nella bellezza e nell'espressione totalizzante di essa \\
Non si puo avere nel decadentismo un termine ad quo e ad quem \t la scuola siciliana cessa di essere dopo la morte di federico II, mentre qua ancora oggi il prodotto di quel decadentismo estetizzante depaurpeato di quei valori si configura nei social \\
Nella vacuita che paion persona \t un movimento che portano e promuovono che arrivano d'oltr'alpe, come i poeti fiamminghi \\
È visto come l'appendice della poesia del romanticismo europeo? in parte, leopardi si ritrova \\
Ci sono 3 poeti eurpei spefici di fondamentale importanza: Rodenbach, Jammes, Meterlink \t sono rilevanti perche promuovono dei bozzetti di paesaggi e ambientazioni, che si ritrovano in pascoli \\
Ma anche la fetta estetizzante, ha dei riferimenti \t in per esempio Dorian Grey e Karl Hyusman in "A reboirs" = "Controcorrente" \\
L'idolo per D'annunzio diventa Dorian Grey (andreas perelli) e la bibbia del decadentismo è di Leone De Castris, che scricve "Il decadentismo" \\
Il protagonista di "A reboirs" è Desseints \t è un esteta che si circonda della bellezza, e la sua casa è un tempio della bellezza \t con tappeti, mobili, etc. \t la casa è lo specchio del personaggio \\
D'annunzio con perelli in "Il piacere", Desseints in "A reboirs" con Hyusman e Dorian Grey con Wilde \t triangolo
\end{document}
