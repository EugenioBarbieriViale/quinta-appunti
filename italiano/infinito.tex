\documentclass[12pt]{article}

\usepackage[a4paper, total={6in, 8in}]{geometry}
\usepackage{textcomp}

\begin{document}
\setlength{\parindent}{0pt}

\def \t {\textrightarrow}
\def \v {\vspace{1em}}
\def \bi {\begin{itemize}}
\def \ei {\end{itemize}}
\def \s[#1] {\section*{#1}}
\def \ss[#1] {\subsection*{#1}}
\def \sss[#1] {\subsubsection*{#1}}

\s[L'infinito]
1818/19 \t fa parte della raccolta "I canti" (uscito definitivamente nel 31, curato da Ranieri) \t a Recanati \\
Romanticismo \\
Paradigma del limite \t testo dell'astrazione, che non è tangibile ma esiste \\
Leopardi è il poeta dell'infinito \t l'infinità del male non lo fa soccombere, non lo schiaccia \t ma lo fa trovare nel limbo, nell'ovattamento, protezione che solo il ventre materno sa riprodurre \\
Critica psicanalistica \t mare, liquido amniotico \\
L'io solo davanti all'infinito \\
Un colle, rappresenta solitudine \t monte Tavor, nel quale l'autore andava per riflettere \\
Siepe \t rappresenta il limite \t l'osservazione di ciò che è altro e oltre il limite \\
Strategia testuale \t corrisponde alle sue emozioni interiori \t la natura qui non è ancora matrigna, ma avvolge, è possibilita di altro \\
Un altro che non centra con il superuomo di Nietzsche \\
Ermo colle \t provocazione \t ermo è difficile da raggiungere, astrazione è in cio che non possiamo toccare con mano \\
Astrazione è oltre, e paradossalmente dentro di noi \t senso di totalita, che azzera la propria insignificanza \\
Nella sua unicità è carico di significato, ma nel sistema è un punto

\v

Sempre caro mi fu \t iperbato \\
Caro e ermo \t chiasmo \\
Evidenza concreta che apre un varco verso l assoluto \t questa siepe, che inoltre rappresenta un limite \\
Ma dove la vista viene meno, il cuore legge con profondita \t cio che descrive il mistero umano è cio che è invisibile agli occhi \\
Ultimo orizzonte \t quella linea che si percepisce come nitida tra cielo e mare \t linea in se non esiste, ma si vede \\
Avversativa al verso 4 \t permette di trovare congruenza tra l infinito e la descrizione di esso \\
Sedendo e mirando \t gerundi \t quasi a climax, interminati \t la percezione dell'assoluto \\
Interminati \ spazi \t enjambement \t profondissima quiete \t triade (sovrumani spazi, silenzi, profondissima quiete) di immagini che evocano sensazioni uditive \t infinito supera la parola \\
Quiete da "quiesco", o "requiesco" \t la morte \\
L'assertivita massima giunge al verso 7 \t "io nel pensiero mi fingo (immagino, latino)" \t io sprofondo nel pensiero e divento tuttuno con un mondo che posso raggiungere solo con l immaginazione  \\
"Ove per dove lo cor non si spaura" \t l'essere su un filo immaginario, tra la vita e la morte \\
La sensazione di un momento in cui siamo colti da una mancanza di respiro \t battito di ciglia in cui si percepisce di essere in un punto di non ritorno \\
Momento di smarrimento davanti a una natura sublime \t es. Didone che maledice Enea dagli scogli \\
Sublime secondo Edmond Burk \t oppure pittoresco \\
Lo stormire tra queste piante paragonato al silenzio dell'infinito \\
Smarrimento di rendersi conto di essere di fronte all infinito \\
L'io limitato dell uomo \\
Nell immensita l eterno è descritta anche attraverso la successione temporale \\
Presenta e viva \t sinestesia \t dove è l'infinito? \t nell'immaginazione \t essa permette di superare il limite e fluire nell infinito \\
La sostanza è l'interiorrizzazione del'assoluto \t il cielo non è in alto, ma è dentro \t l'assoluto non si puo toccare ma è dentro di noi \\
Se l'assoluto è dentro, il limite dove è? \t è fuori di noi, quindi non ci appartiene \t non dipende da me \t io devo sentire la compiutezza del mio essere
\end{document}
