\documentclass[12pt]{article}

\usepackage[a4paper, total={6in, 8in}]{geometry}
\usepackage{textcomp}

\begin{document}
\setlength{\parindent}{0pt}

\def \t {\textrightarrow}
\def \v {\vspace{1em}}
\def \bi {\begin{itemize}}
\def \ei {\end{itemize}}
\def \s[#1] {\section*{#1}}
\def \ss[#1] {\subsection*{#1}}
\def \sss[#1] {\subsubsection*{#1}}

\s[Verga - Malavoglia]
Se si parla di progetto/progettualità, si intendeno 5 opere \t non realizzazione perche sono state composte solo nel numero di due: i malavoglia e mastro don gesualdo \\
I malavoglia parte da Padron Toni \t attorno al quale girano altri personaggi, con i valori siculi e sicilianità e le sfumature linguistiche si ritrovano in Sciascia \\
Sciascia parla di un abbassamento tonale \t quindi di discorso linguistico: Verga parla come i suoi personaggi, che esprimono la loro condizione geografica \\
Per parlare come questi c'è bisogno di alcune strategie \t l'opera deve sembrare essersi fatta da se = criterio dell impersonalita, che richiede il discorso diretto libero \\
Questo è correlato con il linguaggio teatrale \t infatti Verga è anche drammaturgo \t "La cavalleria rusticana" viene messa in musica con le musiche di Giuseppe Verdi infatti \\
Questo si riscontra anche nella novella "Nedda", ma si perfeziona nei Malavoglia \t questo è il suo stile

\v

Si ritrova tutto qeusto nel registro del passato-presente \t la trama dei Malavoglia è esisgua e complessa contemporaneamente \\
Esigua perche è tutta una famiglia \t un microcosmo, un limite oltre al quale si incontra la morte \\
Ma complessa per le vite pregresse dei personaggi \t "la casa del nespolo è il luogo dell'anima" di tutta la famiglia \t esprime l'identita di apparteneza della famiglia a quella casa \\
E perdere quella casa significava perdersi \t e c'è un gran numero di personaggi \\
La famiglia si identifica nella casa del nespolo \t perderla significa perdere la famiglia \\
La casa pero è a rischio \t e la trama nelle sue complesse semplicita presenta una necessita: quella di approvvigionarsi e di sopravvivere \\
Sono pescatori \t e hanno solo un carico di lupini che va a male, in cui avevano investito e si erano indebitati, ovvero la barca che usavano per approvvigionarsi \t a causa di una tempesta \\
Qui è quindi presente un fulcro manzoniano \\
Intorno alla casa e alla barca provvdienza \t gira il sistema dei personaggi e degli eventi (ovvero la ricerca di cibo) \t sono pescatori \\
Devono quindi ricostruire la barca e si sono anche indebitati e non hanno il cibo che hanno perso \t questo apre il varco tra passato e presente \t casa del nespolo e provvidenza sono passate e future \\
La provvidenza è disfatta, quindi la casa deve essere concessa per il debito \\
All'interno di questo sistema è presente il passato e i personaggi del passato quando avevano speranza, ma poi la provvidenza si distrugge e quindi sorge la delusione e il crollo della speranza \\
La famiglia vive sempre questa situazione \t sta nella casa del nespolo ma non è piu sua, e ripara la barca ma non ha soldi \t il problema della poverta rimane \t ogni tentativo è vano \\
Sono vinti per questo \t non riescono a uscirne \t è diverso dagli inetti e dagli umili di manzoni \t che in qualche riescono \\
Sono vinti in ogni gradino della scala sociale \t sono vinti tra gli umili pescatori i malavoglia, vinti in miglioramento in mastro don ges., etc \\
La storia \t inchista Sydeney-Sonnino \t ha abbandonato il sud \\
Sono vinti da dio (provvidenza laica, è una barca) \t vinti da loro stessi, perch quello stato di impossiblita non li fanno muovere \\
Primo inetto è Petrarca \t non puo trasformare pensieri in azioni
\end{document}
