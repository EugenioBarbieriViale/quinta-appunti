\documentclass[12pt]{article}

\usepackage[a4paper, total={6in, 8in}]{geometry}
\usepackage{textcomp}

\begin{document}
\setlength{\parindent}{0pt}

\def \t {\textrightarrow}
\def \v {\vspace{1em}}
\def \bi {\begin{itemize}}
\def \ei {\end{itemize}}
\def \s[#1] {\section*{#1}}
\def \ss[#1] {\subsection*{#1}}
\def \sss[#1] {\subsubsection*{#1}}

\s[X agosto - Pascoli]
Prima strofa immette in medias res nel motivo che precede la composizione, ossia la ricorrenza della morte del padre (avvenuta quando l'autore aveva 12 anni), presenta un universo accogliente nel momento di san lorenzo \\
È la giornata delle stella cadenti \\
Le stelle cadenti nel immaginario si ricollegano alla stella cometa \t la scia luminosa è rapidissima, e il mistero sta nel coglierla nel suo manifestarsi \\
Il blu del cielo in una sera estiva è un'immagine \t l'aria è calda alla sera, e il fuoco che rappresenta il passaggio della stella cadente \\
Enj. al verso \t cielo concavo è il cielo che si percepisce come volta celeste che avvolge il globo \\
Lo sfavillare si riconnette alla luce delle lacrime \t un viso irrorato di pianto, le lacrime luccicano con la riflessione della luce \\
Il primo verso presenta una certezza: "io lo so" \\
Lui aveva presente gli assassini del padre \t e decise di non vendicarsi \t ma questo ebbe ricadute sulla sua salute: il fegato, anche in seneca nel De Ira, è l'espressione della rabbia soffocata \t infatti pascoli muore di un problema al fegato \\
Biografia che diventa poesia, e vissuto che non puo essere scisso dall'espressione artistica

\v

Nella seconda strofa c'è la descrizione attraverso il linguaggio del nido e lo specifismo \\
Rondine \t precisione ornitologica \\
Ritornava \t è un imperfetto che rappresenta una continuita nel passato \\
L'uccisero \t fulimneita dell'evento e l'impossibilta di tornare indietro \\
La rondine è l'alter ego del padre \t e i rondinini sono la famiglia, che veninvano sostentati dal padre \t l'armonia si rompe quando il padre viene ucciso 

\v

Rilegge il vissuto con linguaggio ornitologico \t evoca la croce come in croce \t pascoli evoca una crocifissione, e la croce ha anche il significato di non poter reagire \t "essere messi in croce" \\
Tende \t enj., mentre quel e quel è un chiasmo e anafora \\
Il nido attende il ritorno della rondine, come la famiglia attende il padre

\v

Anche un uomo tornava \t di nuovo uso di imperfetto e poi passato remoto \t è presente un alternarsi tra questi due tempi \\
Nella strofa precedente campeggia il presente \\
Disse: perdono \t discorso indiretto libero \t e il grido negli occhi è il grido non espresso, soffocato, di chi esala l'ultimo respiro \t grido di Munch

\v

Usa un linguaggio che ricorda l'abitudine del bambino, come era l'autore \t aspettano in vano nella casa \\
Attonito \t ricorda il 5 maggio di Manzoni \\
Le bambole \t anafora \t sono le figlie, ed il padre è come se chiedesse una protezione per le figlie \\
Visto come una sorta di vittimismo \t il padre protegge le figlie, e non lui

\v

Chiusura che ripercorre l'abitudine di diversi autori \t ovvero quella del climax \\
La chiusura presenta il determinismo \\
Tu cielo \t apostrofe del cielo \t mondi / sereni è enj. \\
Cielo e Male sono in un chiasmo \t entrambi con la maiuscola, e sono a inizio e fine in un incrocio \\
L'atomo ai tempi era l'unita di misura della realta \t e riflessione in merito alla terra, vista come atomo rispetto al cielo \\
È da ricordare anche il Somnium Scipionis \t in cui l'africano mostra all'emiliano la terra dal cielo, che appare minuscola \\
Da riconnettersi anche con la relativita \\
Atomo opaco \t è una sinestesia \t e che sia opaco, significa che la terra non presenta luce \t e quindi la terra è male \t quindi l'uomo vive innegabilmente in una condizione di male \\
Il male viene visto come un parametro assoluto \t "Al di la del bene e del male" di Nietzsche \\
Il male con la maiuscola \t Petraca e la pregevolezza che dava alla morte e all'amore \\
Tendenza al determinismo meccanicistico \t che era gia presente in Leopardi, che in leopardi era piu solo meccanicismo e basta \\
Nel decadentismo, si mira alla base dell'impianto comptiano \t che prevedeva una totale fiducia nella scienza \\
Rientra nel tema del decadentismo della morte, che è presente assiduamente nella vita

\s[Lampo-tuono-temporale]
Tuono \t aspetti uditivi, mentre lampo aspetti visivi, e temporale tutti 

\ss[Tuono]
Il moto di una culla \t cantilena, donodolia \t si ricollega all'assiuolo e lavandare \\
Il fanciullino è anche manifestazione visiva in ogni sua lirica \t ed è il presupposto per la poesia stessa \\
Il padre costituisce l'irrisolto, la madre l'edipo \\
Il ritomo narrativo parte da E \t la notte nera rafforza, quasi pleonastico \t ed è il perdersi di ogni senso \t ed è anche una sinestesia \\
Inoltre analogia \t notte come il nulla \t celebrazione della morte del tutto \\
La fulimneita \t impredivibilita della natura, e al secondo verso c'è una allitterazione della R \\
Il dirupo è scosceso, e il suo franare (enj v. 2/3) \t sensazioni visive ed uditive vengono mischiate usando l'imperfetto \\
Climax al verso 4 \t ed evidente richiamo di un altro verso del 5 maggio di Manzoni, dove si evoca la fine di Napoleone \\
E tacque sembra finire il chiasmo, che pero riprende \\
Mare e cielo sembrano in simbiosi \t dirupo e rimareggiare evoca il mare, mentre il tuono evoca il cielo \\
La fine del tutto: il punto di chiusura è vanì \t la fine con la morte del padre, il nido è rotto \\
A meta verso il tono cambio, che ripropone l'inizio del tutto \t il nulla porta a una vita, il punto zero e il circolo \t clinamen, e anche in Democrito \t atomi dopo morte si ricompongono per formare una nuova vita
\end{document}
