\documentclass[12pt]{article}

\usepackage[a4paper, total={6in, 8in}]{geometry}
\usepackage{textcomp}

\begin{document}
\setlength{\parindent}{0pt}

\def \t {\textrightarrow}
\def \v {\vspace{1em}}
\def \bi {\begin{itemize}}
\def \ei {\end{itemize}}
\def \s[#1] {\section*{#1}}
\def \ss[#1] {\subsection*{#1}}
\def \sss[#1] {\subsubsection*{#1}}

\s[Boudelaire - Spleen]
Spleen \t termine che caratterizza cio che abbiamo fatto \t assimilabile alla noia, al tedio, l'atmosfera tetra \t in cui ci sono ganci (le corrispondenze) tra le cose lontante e i nessi razionali \\
Equivalente dell'assenza di vita, coperta da un nebbia fitta che occulta \t Giovanni Svani, che è Pascoli \\
Si ritova anche nella letteratura di Elliot e di Joyce \t anche in Montale, quell'atmosfera umida, e una cappa di umido e calore insopportabile \\
Qui è presente il linguaggio traslato dell'autore, che presenta una ribellione nei confronti dell'ipocrisia della societa \\
Testo esprime la frammentarieta \t si procede per illuminazione, e i flash che si colgono nella sovrapposizione di immagini \\
Dal punto di vista archiettonico \t è presente un'anafora del quando (prima, seconda e terza strofa) \t e questa armonia stona con i contenuti \\
Cielo basso \t sinestesia \\
Nella seconda strofa, la cella evoca le urne sepolcrali \t ed è anche la cella delle prigioni (probabilmente questa è l'interpretazione giusta, perchè poi parla di delle sbarre che è la pioggia che cade) \\
L'immagine della terra, vista non come un entita di madre, ma nella sua materialita e nel suo elemento recettivo \t la terra che è umida per la pioggia \\
Speranza = riminescenza petrarchesca \t la speranza è una personificazione (per la maiuscola) come in Petrarca \\
La speranza viene paragonata a un pipistrello \t considerato animale orribile e simbolo di morte, e portano malattia \\
Evocano una sensazione di mistero e di notte \\
Ali timide sono in contrasto col fatto che sbattono contro i muri \t e i soffitti marci danno l'idea di una consunzione \\
Urtando \t sembra un volere abbattere quei muri marci \\
La pioggia a raffiche, imita le sbarre di una grande prigione \t è talmente fitta che evoca le sbarre di una prigione \t questo evoca un bisogno di liberta, anche nel pipistrello \t che se sbatte contro i muri, non è libero \\
Il popolo silenzioso \t di solito sono le formiche \t qua i ragni, poi enjambement \t tessono nei cervelli, come se non avessimo la liberta di pensiero \\
Cervelli e ragni \t evocano le briglie valoriali \\
La nostra capacita congitiva è indirizzata, quindi non libera \\
Le campane \t sono uno degli elementi focali nella poesia di Pascoli, che è decadentista \t ma è molto legato alla scapigliatura \\
Le campane possono assumere tanti significati \t in manzoni evocano un periodo (la notte degli imbrogli) o un messaggio per la comunita (l'arrivo del cardinale borromeo) \\
Mentre in pascoli rappresentano un senso di unita nel paese, e sono un evocazione onomatopeica \t cosi evocano l'elemento della culla, che si muove \\
Qua le campane pero scoppiano e producono urla furiose \t un crescendo verso figure sempre piu racapriccianti \\
Tutto questo paesaggio è interiorizzato nell'animo del poeta \\
L'angoscia è dispotica (sinestesia) \t perchè non si puo frenare (mentre la speranza si) \t soprattutto l'angoscia nei confronti di cio che non si conosce \\
Il cranio chino \t metonimia, per arrivare all'essenza, il centro dei pensieri \t pianta il suo vessillo nero: una sorta di bandiera, che presagisce solo morte \\
Testo carico di evocazioni \\
O albatro, o attello ebbro
\end{document}
