\documentclass[12pt]{article}

\usepackage[a4paper, total={6in, 8in}]{geometry}
\usepackage{textcomp}

\begin{document}
\setlength{\parindent}{0pt}

\def \t {\textrightarrow}
\def \v {\vspace{1em}}
\def \bi {\begin{itemize}}
\def \ei {\end{itemize}}
\def \s[#1] {\section*{#1}}
\def \ss[#1] {\subsection*{#1}}
\def \sss[#1] {\subsubsection*{#1}}

\s[Assiuolo - Pascoli]
L'assiuolo è un piccolo rapace simile al gufo \t la vita e il mistero, il presagio di morte \t un'aldila che cozza con il suo meccanicismo materialistico, fatto di una concretezza che l'autore mostra, privilegando una visione socialista \\
Onomatopea è ripetuta dopo ogni strofa \t chiu \\
Le risonanze sono presenti nel testo, e c'è un incipit leopardiano \\
Inoltre enj. 1/2 verso \\
Disquisizione interpretativa \t alba di perla è una sinestesia, ed è anche presente però una analogia \t colore madreperla \\
Prospettiva di osservazione è limitata dal mandorlo e dal melo \t un orto e un giardino propongono una delimitazione spaziale precisa \\
Parevano ergersi \t come se fossero personificati \\
Soffi di lampi, nero di nubi = immagini consuete in pascoli \t la perla dell'alba contrsta con le nubi che minacciano tempesta, e morte \t e il suo che arriva dal silenzio: il verso dell'assiuolo \\
Immagini sovrapposte, frammentismo pascoliano delle liriche

\v

Singulto è il singhiozzo legato al pianto \\
La nebbia di lato \t metafora e analogia sottesa, e anche sinestesia \\
Nebbia \t filone "foggy" che caratterizzera le opere di Joyce, che ritrae un mondo ovattato \\
Questa immagine di una mente che è ottusa \t ottusa è una persona \t la nebbia nasconde ciò che è morto ?????? \\
Verso 11 \t cullare del mare: evocazione pascoliana \t il mare come madre nel liquido amniotico \t collegabile con le lunghe catilene del testo prima, delle cantilene che fanno da culla, e Hans in sotto la ruota \\
Fru fru \t anticipazione montaliana \t e le fratte sono i cespugli (scpecifismo) \\
Onomatopea, evocazione sonora molto forte e assonanza \\
Io sentivo \t soggettivismo di Pascoli, tutto è filtrato secondo il suo punto di vista \\
Verso 13 arriva il climax finale \t evocativita della morte si legge nella lettura dei fenomeni naturali: natura è una selva di simboli \\
Sonava è una sinestesia \t un singhiozzo non suona

\v

Qua c'è epilogo che conduce alla chiusura \t lucide vette è sinestesia \\
Prima soffi di lampi, ora sospiro di vento \t il vento che sospira è di un essere morente, a fronte di un dolore \\
Sistri \t sono monete che evocano l'antichita, ed erano dei soldi di poco valore \t il cui suono evocava un tintinnio evocativo di morte
\end{document}
