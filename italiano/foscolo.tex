\documentclass[12pt]{article}

\usepackage[a4paper, total={6in, 8in}]{geometry}
\usepackage{textcomp}

\begin{document}
\setlength{\parindent}{0pt}

\def \t {\textrightarrow}
\def \v {\vspace{1em}}
\def \bi {\begin{itemize}}
\def \ei {\end{itemize}}
\def \s[#1] {\section*{#1}}
\def \ss[#1] {\subsection*{#1}}
\def \sss[#1] {\subsubsection*{#1}}

\s[Foscolo]
Nato il 6 febbraio 1778 \t muore nel 1827 \\
Zante \t grecia classica \t 
Attaccamento alla terra madre (grecia) \\
Sua madre è anche greca \t attaccamento alla madre \t per lui corrisponde a una privazione \t si deve allontanare da Zante \\
Questo crea un grande senso di bisogno \t di privazione \\
Equazione tra la grecia con i suoi paesaggi e la madre nel senso generazionale \\
Amore e passione per la politica \t padre è veneziano \t attaccamento anche alla penisola \\
Eta napoleonica \t 1815 congresso di vienna, che immette nell'inzio del romanticismo italiano \\
Foscolo vive il periodo alfierano \t alfieri ha tempra e viaggia molto \t intellettuale cosmopolita \\
Entrambi viaggiano molto e soffrono la stanzialità \t foscolo viaggia anche a Londra, simbolo dell innovazione \\
Alfieri come foscolo vive il tormento della mancanza di liberta \t scrive trattato sulla tirannide 1777 \\
Foscolo è vittima del regime napoleonico \t foscolo vive il tradimento di napoleone nel:
\bi
    \item la cessione del Lombardo Veneto agli austraci
    \item trattato di Campoformio ed editto di Saint-Claude
\ei
"Le ultime lettere di Jacopo Ortis" sono un'innovazione non solo per il genere ma anche per la caratteristica interna al genere \\
Hanno il retroscena della delusione napoleonica, ma poi sono la specularita dell'autore \t il protagonista trasmette se stesso \\
Jacopo e Foscolo sono congruenti \\
Il genere: romanzo epistolare \t è un romanzo storico? non veramente \t per l'autore è contemporaneita \\
C'è analisi degli eventi ma non c'è il distacco storico \\
È romanzo \t questo genere in italia è nuovo \t in questo caso è ibrido, perche epistolare \t è il primo romanzo \\
Boccaccio \t "Elegia a madonna fiammetta" \t primo romanzo in volgare \t poi genere diventa silente \t foscolo lo riprende \\
Seneca \t "Epistolae ad Lucillum", oppure Cicerone: "Ad atticum fratem", e poi le lettere di Abelardo ad Eloisa (nel medioevo, fusione tra produzione letteraria e filosofica) \\
Petrarca è il maestro dello scavo interiore nelle lettere \t es. "Le familiares", oppure quando scrive a persone del passato \\
"Laura lettere" 1798/99 \t 1801 "Ultime lettere di Jacopo Ortis", che viene completato nell'edizione finale con la "Lettera al Parini" \t 1802 definitiva \\
Contesto politico, esilio, forte sentire, temrpa alfierana, cosmopolitismo, amore e passione per la vita e nel contempo suicidio eroico, focus sul suicidio \t opera si chiude col suicidio per motivo politico e amoroso \\
Ipotizza e immagina la sua morte una tomba pianta dalla tomba amata \t fiori e lacrime sono la consolazione per l'autore (che è jacopo) \\
La coscenza di Zeno anteceduto da "La vita" \t Alfonso di suicida perche si sente incapace di vivere, pensando ad Annetta che piange sulla sua tomba (richiamo foscoliano) \\
Forte peso politico: l'opera non è consequienziale negli eventi \t le lettere che vengono datate spesso sono lettere di risposta (l'interlocutore Lorenzo) \\
Lettura del se attraverso la verbalizzazione
\end{document}
