\documentclass[12pt]{article}

\usepackage[a4paper, total={6in, 8in}]{geometry}
\usepackage{textcomp}

\begin{document}
\setlength{\parindent}{0pt}

\def \t {\textrightarrow}
\def \v {\vspace{1em}}
\def \bi {\begin{itemize}}
\def \ei {\end{itemize}}
\def \s[#1] {\section*{#1}}
\def \ss[#1] {\subsection*{#1}}
\def \sss[#1] {\subsubsection*{#1}}

\s[Foscolo]
Nato il 6 febbraio 1778 \t muore nel 1827 \\
Zante \t grecia classica \t 
Attaccamento alla terra madre (grecia) \\
Sua madre è anche greca \t attaccamento alla madre \t per lui corrisponde a una privazione \t si deve allontanare da Zante \\
Questo crea un grande senso di bisogno \t di privazione \\
Equazione tra la grecia con i suoi paesaggi e la madre nel senso generazionale \\
Amore e passione per la politica \t padre è veneziano \t attaccamento anche alla penisola \\
Eta napoleonica \t 1815 congresso di vienna, che immette nell'inzio del romanticismo italiano \\
Foscolo vive il periodo alfierano \t alfieri ha tempra e viaggia molto \t intellettuale cosmopolita \\
Entrambi viaggiano molto e soffrono la stanzialità \t foscolo viaggia anche a Londra, simbolo dell innovazione \\
Alfieri come foscolo vive il tormento della mancanza di liberta \t scrive trattato sulla tirannide 1777 \\
Foscolo è vittima del regime napoleonico \t foscolo vive il tradimento di napoleone nel:
\bi
    \item la cessione del Lombardo Veneto agli austraci
    \item trattato di Campoformio ed editto di Saint-Claude
\ei
"Le ultime lettere di Jacopo Ortis" sono un'innovazione non solo per il genere ma anche per la caratteristica interna al genere \\
Hanno il retroscena della delusione napoleonica, ma poi sono la specularita dell'autore \t il protagonista trasmette se stesso \\
Jacopo e Foscolo sono congruenti \\
Il genere: romanzo epistolare \t è un romanzo storico? non veramente \t per l'autore è contemporaneita \\
C'è analisi degli eventi ma non c'è il distacco storico \\
È romanzo \t questo genere in italia è nuovo \t in questo caso è ibrido, perche epistolare \t è il primo romanzo \\
Boccaccio \t "Elegia a madonna fiammetta" \t primo romanzo in volgare \t poi genere diventa silente \t foscolo lo riprende \\
Seneca \t "Epistolae ad Lucillum", oppure Cicerone: "Ad atticum fratem", e poi le lettere di Abelardo ad Eloisa (nel medioevo, fusione tra produzione letteraria e filosofica) \\
Petrarca è il maestro dello scavo interiore nelle lettere \t es. "Le familiares", oppure quando scrive a persone del passato \\
"Laura lettere" 1798/99 \t 1801 "Ultime lettere di Jacopo Ortis", che viene completato nell'edizione finale con la "Lettera al Parini" \t 1802 definitiva \\
Contesto politico, esilio, forte sentire, temrpa alfierana, cosmopolitismo, amore e passione per la vita e nel contempo suicidio eroico, focus sul suicidio \t opera si chiude col suicidio per motivo politico e amoroso \\
Ipotizza e immagina la sua morte una tomba pianta dalla tomba amata \t fiori e lacrime sono la consolazione per l'autore (che è jacopo) \\
La coscenza di Zeno anteceduto da "La vita" \t Alfonso di suicida perche si sente incapace di vivere, pensando ad Annetta che piange sulla sua tomba (richiamo foscoliano) \\
Forte peso politico: l'opera non è consequienziale negli eventi \t le lettere che vengono datate spesso sono lettere di risposta (l'interlocutore Lorenzo) \\
Lettura del se attraverso la verbalizzazione 

\v

Nesso tra foscolo ed alfieri, ma anche con la letteratura estera del tempo \\
Romanticismo si differenzia a seconda del luogo in europa \t in foscolo non si parla di anticipazione del romanticismo \\
Lettere portano al confronto con "Didimo Chierico" \t opera di foscolo, in cui c'è un immagine pungente e dissacatrice di Foscolo \\
Nell opera combatte le illusioni foscoliane, che sono state deluse \\
Le lettere iniziano con "il sacrificio della patria è consumato" \\
Il Didimo guarda con occhio beffardo e dissacratore il giovane foscolo, è come l'uomo maturo che guarda al suo passato \\
Didimo trova ragione \t perchè è doppio: il foscolo maturo e il foscolo giovane \\
Foscolo ha illusioni della patria e dell'amore \t foscolo ha un senso di appartenenza doppia: italia e grecia \\
Lo spirito di foscolo è doppio anche nel senso di terra madre (grecia) e di essere proiettato nella patria paterna (italia) \\
Fratello Giovanni si suicida per debiti di gioco \t foscolo non era lontano, aveva debiti di soldi (infatti scappa a Londra, dove morirà) \\
Londra dove ci sono le innovazione \t è l opposto di Zante, che rappresenta il ricordo \\
Sturm und drang è il tratto distintivo di Foscolo \t lui è assalto, impeto, istinto \\
Ma ha un controllo neoclassico sulla sua forza interiore, tutta preromantica \t lotta tra ragione e istinto \\
Duplicita \t viene presentata da una forma neoclassica che controlla contenuti romantici \\
Mito foscoliano = emblema di un linguaggio nuovo che veicola contenuti nuovi (nuova sensibilita) \\
Es. Dante usa Virgilio per veicolare contenuti nuovi attraverso una figura credibile \\
(leopardi) Passione viene definta soprattuto nel "Alla sera" \t lo spirito guerriero dorme, è una calma appartente \\
Duplicita \t matrice petrarchesa/agostiniana \\
Alla sera, in morte al fratello giovanni, carme 101, caro fratello addio, a zacinto, il ritratto (solcata ho la fronte, ?)

\v

Primo snodo: umili e potenti - santi e birboni - istruzioni e cultura - genere enciclopedico - il ritratto (Perpetua, Agnese, Monaca di Monza, Donna Prassede, Moglie del Sarto) - rossore/candore/pudore - i potenti e il clero - parabola involutiva di Renzo - osterie come luoghi di socializzazione, dove perdersi e ritrovarsi - peste

\ss[A Zacinto]
1803 \t "Stagione dei sonetti" \t escono pubblicati con delle aggiunte di Pisa, comprensi i due autoritratti \t nell'edizione definitiva si chiama "Poesie" \\
Sonetti tratteggiano l'evoluzione interiore dell'autore \t dissidio interiore \\
Conflittualità passionale dal padre e razionalità dalla madre greca, che rappresenta per Foscolo la classicità \t il bello classico è l'equilibrio \\
Foscolo subisce le critiche di Guillaume \t lo accusa di un linguaggio troppo elettivo per il tempo \\
Linguaggio e mito sono veicoli di contenuti nuovi 

\v

Sonetto: due quartine (narazione), due terzine (riflessione introspettiva) \t questa lettura da Giorgio Barberi Squarotti (colui che definisce i promessi sposi come romanzo contro la storia) \\
Negazione iniziale \t disfattismo con cui il sonetto inizia, dolore per non poter stare nella terra di origine, caratterizza un parallelismo con il scacrificio della patria consumato (le lettere, in cui c'è tutto foscolo) \\
Sacre sponde \t metonimia molto evasiva \t ricorda l'affacciarsi della sua terra sul mare \\
Giacque \t passato remoto indica qualcosa di chiuso e definitivo \\
Zacinto mia \t senso si appartenenza \\
"nell onde / del greco mar" \t enjambement \\
Valore di venere duplice \t dea della bellezza, ma anche dea della nascita e della generazione \t si lega quindi a zacinto, terra madre \\
Paesaggio quasi incantato della grecia, lugo ameno \\
Acque fatali \t anticipa Ulisse, che peregrina per mare \\
Ulisse viene reso come grande, bello attraverso la fama e le sue peripezie \t Ulisse fa ritorno alla sua patria, mentre Foscolo no \\
Tocco, canto, ? \\
Illusioni foscoliane rappresentano dolore, perchè non si concretizzano \\
Petrosa \t riminescenza dantesca delle rime petrose \\
Il doppio \t ulisse e foscolo \\
A noi = "pluralia maestatis" \t illacrimata sepoltura \t aggancio al tema dei "Sepolcri" e la parte finale delle lettere jacopo ortis (che escono con il titolo "Laura lettere", e poi anche "Storia di due amanti infelici") 

\v

Viaggio sentimale \t nel Didimo chierico finde di aver trovato la traduzione del viaggio dello Stern??? \\
Parte da quello per diramare due opere: il didimo e il "Sesto tomo dell'io" = abbozzo autobiografico di foscolo, che non è pubblicato integrale \\
Rappresenta l'introspezione, e del viaggio dentro di se \t in manzon il viaggio di Renzo e di purificazione dentro di se, + il viaggio dentro di se del 900 (es. "La crisi dell'io" di Pirandello) \\
Altri collegamenti
\bi
    \item seneca e cicerone per la lettera introspettiva
    \item lettere di abelardo ad eloisa
    \item scavo nell'io di matrice agostiniana (Confessioni)
    \item petrarca con le sue lettere e il "Mondo de secretum"
\ei

\ss[Alla sera]
1803, nello stesso blocco di sonetti che compaiono nel libro "Poesie", senza una vera e propria struttura \\
Canzionere di Saba = psicanalitico prima della psicanalisi

\v

Ancora la focosita e duplicita interiore dell'autore, e della presenza razionale che non puo del tutto placare \\
Immagine della sera epicurea, come emblema della morte \\
Ciclicità della sera e del giorno \t meccanicismo materialistico \\
Immago = immagine, prefigurazione \t impianto classico \\
Pace dopo una giornata tempestosa, come la pace dopo la tempesta \\
"Sembri scendi evocata e le secrete / vie del mio cor soavemente tieni" \t la sera percorre le vie del cuore dell'autore, che aspetta la sera \\
Vie \t viaggio interiore e dinamismo \\
Spirito guerriero si assopisce, non muore \\
Reo tempo \t il tempo presente in cui il dolore è l'incapacita di non trovarsi nel suo tempo \t riminescenza petrarchesca 

\v

\ss[Solcata ho fronte]
Autoritratto di Goldoni nei "Memoir" \t anche in questo sonetto \\
Crin \t termine elettivo, classico \\
Aridito aspetto \t rimanda alla focosita interiore \\
Immagine di impetuoista contrasta con "capo chino" \t l'altra parte di se, +introspettivo \t dualismo e contrasto, opposizione interiore \\
Sobrietà nel vestire ed equilibrio armonico \\
Verso 6 \t climax ascendendte \\
Verso 7 \t passa dagli aspetti esteriori a quelli interiori \t dagli accidenti alla forma \\
Verso 8 \r spirito vittimista del romanticismo  \\
Da lode alla ragione \t ricerca il controllo, ma corro ove al cor piace \t di nuovo dualismo
\end{document}
