\documentclass[12pt]{article}

\usepackage[a4paper, total={6in, 8in}]{geometry}
\usepackage{textcomp}

\begin{document}
\setlength{\parindent}{0pt}

\def \t {\textrightarrow}
\def \v {\vspace{1em}}
\def \bi {\begin{itemize}}
\def \ei {\end{itemize}}
\def \s[#1] {\section*{#1}}
\def \ss[#1] {\subsection*{#1}}
\def \sss[#1] {\subsubsection*{#1}}

\s[Pascoli - Novembre]
L 11 novembre è una data importante \t è un giorno particolare che viene descritto qua in modo singolare \\
L'autore trasfigura un paesaggio dedicato a San Martino \t noto per essere sceso da cavallo ed aver donato il proprio mantello a un povero infreddolito = espressione di carita \\
Lui sopravvisse perche dio attenuo quindi il freddo di quella giornata, in modo che lui sopravvivesse senza un mantello \t Pascoli non era credente, pero a san martino ci si aspetta il sole per questa ragione \\
Qua troviamo un paesaggio primaverile \t ma è un illusione: è ancora inverno \\
Sarebbe la primavera d'autunno \\
Ultima strofa \\
Estate fredda è un ossimoro

\v

Prima strofa \t sono presenti le figure retoriche tipiche \\ 
L'aria è splendente come la gemma \t che è la pietra dura lavorata \t che è una metafora (non c'è tertium comparationis), e appartiene anche alla sinestesia \\
Il sole cosi chiaro \t si riconduce a gemmea e aria con sole \t chiasmo \\
E tu ricerchi \t tu generico, colloquialita che si fa strada nella lirica \\
Dall'odore dolce dei fiori di albicocco a quello amaro e del biancospino \\
Compone anche "La digitale purpurea" \t i farmici digitalici, legati ai problemi cardiaci \\
Odorino amaro è un vezzeggiativo \t e odorino amaro è un po una sinestesia \t evoca odore pungente \\
Finora tempo presente domina \\
Gemmea = forma elettiva, si intende "gemmeva" \t l'imperfetto, tempo della continuita e di una azione protratta nel tempo 

\v

Seconda strofa \t ma avversativo cambia direzione \\
Verso 6 è fondamentale per immetersi nel contesto del limite \t vedremo poi scaglie di cielo, prima avevamo visto straci di nubi \\
Trame \t sono i rami, che sono secchi e quindi scuri \\
Il cielo è vuoto \t in X agosto era concavo \\
Nei sepolcri \t c'è il viandante e gli scricchiola il terreno sotto i piedi \t e lui non si sofferma sulle voci dei morti \t cavo al pie sonante sembra il terreno \t sembra, non è = poetica dell'indefinito, pur con la volonta di chiamare le cose con il loro nome \t Contini, Emengaldo \\
Il pie sonante \t il terreno è secco \\
Chiasmo al vero 7 \t dopo enj. si chiude il testo \t notare l'armonia tra spazi bianchi e la parola scritta

\v

Terza strofa \t sensazioni uditive \\
Anastrofe/iperbato \\
Da giardini ad orti è importante \t Metterlinch, Rodenbach etc. rietrano in questa immagine di giardini ed orti \t tipica immagine del decadentismo \\
Odi \t di nuovo seconda persona \\
Cader fragile \t al verso, anastrofe e cader fragile ricorda lavandare \t soffia il vento e nevica la frasca \\
È l'estate, spazio biano sotto \t quindi il focus è nel verso finale \t è un illusione è san martino, novembre ci presenta illusione d'estate \\
Il mese di novembre è il mese dei morti \\
Questa è la lirica della discordia con D'annunzio \t l'uno accusa l'altro di plagio \\
Questo portera d'annunzio a definirlo un pantofolaio \t e l'altro reagisce pubblicando un bozzetto satirico di d annunzio con il cane
\end{document}
