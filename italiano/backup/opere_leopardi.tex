\documentclass[12pt]{article}

\usepackage[a4paper, total={6in, 8in}]{geometry}
\usepackage{textcomp}

\begin{document}
\setlength{\parindent}{0pt}

\def \t {\textrightarrow}
\def \v {\vspace{1em}}
\def \bi {\begin{itemize}}
\def \ei {\end{itemize}}
\def \s[#1] {\section*{#1}}
\def \ss[#1] {\subsection*{#1}}
\def \sss[#1] {\subsubsection*{#1}}

\s[L'infinito]
1818/19 \t fa parte della raccolta "I canti" (uscito definitivamente nel 31, curato da Ranieri) \t a Recanati \\
Romanticismo \\
Paradigma del limite \t testo dell'astrazione, che non è tangibile ma esiste \\
Leopardi è il poeta dell'infinito \t l'infinità del male non lo fa soccombere, non lo schiaccia \t ma lo fa trovare nel limbo, nell'ovattamento, protezione che solo il ventre materno sa riprodurre \\
Critica psicanalistica \t mare, liquido amniotico \\
L'io solo davanti all'infinito \\
Un colle, rappresenta solitudine \t monte Tavor, nel quale l'autore andava per riflettere \\
Siepe \t rappresenta il limite \t l'osservazione di ciò che è altro e oltre il limite \\
Strategia testuale \t corrisponde alle sue emozioni interiori \t la natura qui non è ancora matrigna, ma avvolge, è possibilita di altro \\
Un altro che non centra con il superuomo di Nietzsche \\
Ermo colle \t provocazione \t ermo è difficile da raggiungere, astrazione è in cio che non possiamo toccare con mano \\
Astrazione è oltre, e paradossalmente dentro di noi \t senso di totalita, che azzera la propria insignificanza \\
Nella sua unicità è carico di significato, ma nel sistema è un punto

\v

Sempre caro mi fu \t iperbato \\
Caro e ermo \t chiasmo \\
Evidenza concreta che apre un varco verso l assoluto \t questa siepe, che inoltre rappresenta un limite \\
Ma dove la vista viene meno, il cuore legge con profondita \t cio che descrive il mistero umano è cio che è invisibile agli occhi \\
Ultimo orizzonte \t quella linea che si percepisce come nitida tra cielo e mare \t linea in se non esiste, ma si vede \\
Avversativa al verso 4 \t permette di trovare congruenza tra l infinito e la descrizione di esso \\
Sedendo e mirando \t gerundi \t quasi a climax, interminati \t la percezione dell'assoluto \\
Interminati \ spazi \t enjambement \t profondissima quiete \t triade (sovrumani spazi, silenzi, profondissima quiete) di immagini che evocano sensazioni uditive \t infinito supera la parola \\
Quiete da "quiesco", o "requiesco" \t la morte \\
L'assertivita massima giunge al verso 7 \t "io nel pensiero mi fingo (immagino, latino)" \t io sprofondo nel pensiero e divento tuttuno con un mondo che posso raggiungere solo con l immaginazione  \\
"Ove per dove lo cor non si spaura" \t l'essere su un filo immaginario, tra la vita e la morte \\
La sensazione di un momento in cui siamo colti da una mancanza di respiro \t battito di ciglia in cui si percepisce di essere in un punto di non ritorno \\
Momento di smarrimento davanti a una natura sublime \t es. Didone che maledice Enea dagli scogli \\
Sublime secondo Edmond Burk \t oppure pittoresco \\
Lo stormire tra queste piante paragonato al silenzio dell'infinito \\
Smarrimento di rendersi conto di essere di fronte all infinito \\
L'io limitato dell uomo \\
Nell immensita l eterno è descritta anche attraverso la successione temporale \\
Presenta e viva \t sinestesia \t dove è l'infinito? \t nell'immaginazione \t essa permette di superare il limite e fluire nell infinito \\
La sostanza è l'interiorrizzazione del'assoluto \t il cielo non è in alto, ma è dentro \t l'assoluto non si puo toccare ma è dentro di noi \\
Se l'assoluto è dentro, il limite dove è? \t è fuori di noi, quindi non ci appartiene \t non dipende da me \t io devo sentire la compiutezza del mio essere

\v

Canto dell'infinito \t i canti sono la raccolta in versi \\
Permette di individuare Leopardi come il poeta dell'infinito \\
I temi principalemente hanno un paradigma esistenziale gia fissato \t teoria del piacere \\
Lo stordimento del sublime \t permette di abbracciare l'infinito \\
Nel testo si ha una struttura evocativa dell'infinito \t e anche dei tempi verbali: con il gerundio e i paritici, si descrive qualcosa che va oltre la temporalita \\
Immensita dell'orizzonte che leopardi descrive non rappresenta un traguardo raggiungibile con le risorse materiali umane, ma solo con l'immaginazione \\
Irraro presente \t ai versi 12 e 13: le morte stagioni \\
Dimensioni temporali, indicazione di un tempo non tempo, limite è valicabile \\
Il muro è proibitivo, respingente, è un limite evidente \t mentre qua il limite non si vede \\
Muro in mentale \t pero con realismo sferzante, in cima al muro ci sono cozzi aguzzi di bottiglia \\
"Horuts conclusus" \t orto perimetrato \t espressione di d'annunzio \\
Dove è il limite \t anche Nella poesia di novembre di pascoli \t il cielo si osserva da sprazzi, la vista del cielo è limitata dai rami \t cielo si vede da un rete \\
Limiti sono anche quelli della societa \t pregiudizi e preconcetti \t come la societa che Leopardi vive a Recanati \t non trova neanche la scappatoia a Roma \\
Anche da Roma viene deluso dalla falsità \\
L'amicizia con Ranieri è imporante, come quella con Giordani \\
Tra i percorsi dell'infinito \t è presente il mare \t "Il vecchio e il mare" di Hemingway \\
Psicanalisi Freudiana \t gli inganni della coscienza di zeno \t illudersi per sopravvivere \\
Leopardi è antisegninano di un scavo interiore \t e di un meccanicismo che portera al positivismo di Compt (ultimi dell 800) \\
Visione della realta deterministica, di cause ed effetti \t che porta alla risoluzione dei mali della umanita \\
Nella filosofia Comptiana sono presenti diverse sfaccettature \t uomo si vede come prodotto di una serie di fattori \\
Freud \t "Psicoanalisi della vita quotidiana????" \t chiedere cosa ha detto qua \\
Il castello di franz kafka è una descrizionie di limite \t come l'assaggio dell'acqua in sotto la ruota ripropone l'aggancio con la mdare \\
Leopardi ebbe invece una madre totalemtene anafettiva \t l'uomo cerca quello che gli manca \\
Interiorizzazione dello spazio, infinito spaziale, uditivo e ontologico \t che pero esiste solo nella immaginazione \\
Viandante sul mare di nebbia

\s[La sera del di di festa]
Chiedere info. generali \\
Il testo è spezzato in due \\
Prima parte referinziale: osservazioine della natura, paesaggio pittoresco e allineato al romanticismo straniero \t squarcio paesaggistico dalla finestra \t montagna fa da limite che protegge \t invocazione alla donna \\
Apostrofe alla donna amata \\
Sinestesia \t i sentieri tacciono \\
Continuo alternarsi di io e tu \t piani disallieati \\
Non ti morde \t suono della R \\
Piaga in mezzo al petto \t ricorda un iconografia cristiana \t anche se lui non era cristiano \\
Lei non sa di averlo ferito \t lui soffre \t verso 11 emblematico, tu/io nello stesso verso \\
Cielo appare benigno, "l'antica natura ognipossente" \t altro verso importante \\
"Mi fece" \t linguaggio quotidiano \\
Mi e a te \t discorso libero \t anticipazione di Joyce con stream of consciousness \t discorso diretto libero \\
"Nego mi disse anche la speme" \t chiasmo \\
Verso dopo \t vittimismo leopardiano \t guardare dentro di se, rimando a seneca \\
"Mi getto, e grido, e fremo" \t perdita di controllo \\
Eta verde \t sinestesia \t "fiore degli anni caduto", espressione carducciana \\
Dal verso 21 continua riflessione filosofica \\
"Come tutto al mando passa" \t riflessione sul tempo, seneca \\
"Orma" \t termine manzoniano \t l'orma che Napoleone ha voluto lasciare, 5 maggio \\
Ciclicita del tempo \t il non essere per sempre, il tempo che fugge, il tempo che gia è oltre \\
Elegia per le fasi storiche passate \t gli antichi che sapevano illudersi \\
Allitterazione della R \t roma, armi, fragorio \\
Dopo queste domande torna la riflessione filosofica \\
Premea le piume \t materasso

\v

Una parte di rappresentazione, poi rimbalzo che anticipa la riflessione, poi rimando al tema che ha scatenato la composizioine \\
Poi di nuovo il blocco conclusivo filosofico \\
È presente cicerone nelle domande retoriche \\
Endecasillabi sciolti \t mette in campo anche il monologo (donna non risponde) \t anche la luna compare \\
Luna è femminilita, mistero (ariosto) \t luna leopardiana è la interlocutrice privilegiata \\
La luna è puramente ascolto, abbraccio \t questo distacco consente all autore la verbalizzazione \t valore catartico che porta alla maieutica \\
Il dottor S nella coscienza di Zeno \t lascia trasparire un minimo di giudizio \\
Debussy \t Claire di Lune \t futurismo che distrugge la luna \t "Falce di luna calante" d'annunzio \\
"Ciaula scopre la Luna" \t pirandello \t "La luna e il falo" di Pavese \\
"La volta degrignata al plenilunio" di Ungaretti

\s[Il sabato del villaggio]
Questo testo è della stagione del 28 \t Leopardi è già maturo come poeta e come persona \\
È un testo speculare a "La sera dei nidi di festa" \\
Parte dall' hic et nunc \\
Idillio campestre \t si intravedono le parti all'interno dell'opera \t si trova il fatto filosofico \\
Anche in questo componimento (si configura negli idilli) \t in conclusione si ha la "cogitatio" \t ovvero la riflessione sulla vita e sul tempo 

\v

Tramonto \t dimensione campestre \t figura anonima addolcita da un avvezzativo, tipico medievale \t "donzella" evoca anche atmosfera tassesca \\
Operosita \t reca un fascio d'erba \t in mano ha anche un mazzolino di rose \t Pascoli: contesta a leopardi la mancanza di uno specifismo, l'approssimazione \t rose e viole non fioriscono nello stesso periodo \\
Ma lui è il poeta dell'infinto \t non gli interessa, vede oltre \\
Alter ego della donzella sara la vecchia \\
Termini elettivi e quotidiani vengono mischiati \\
Attesa, aspettative, speranza \t prima fase: la memoria abbrevia il corso \\
Spazialita \t donzelletta viene dai campi, che sono in basso \\
La vecchia siede in alto \t lei ha gia salito la scala della vita \t è gia al vertice della vita \\
Ha gia varcato la soglia del "discovrir del vero" \\
Operosita campestre \t lei lavora e fila \\
Ricorda di come si ornava la testa per la festa \\
Insieme aspetti ludici, convivialita, gioiosita e condivisione \t il vivere in una dimensione simmetrica \\
Poi l'attenzione si sposta all'esterno \t atmosfera idilliaca di interazione con la natura \t individuo sperimenta una visione edenica del mondo, ecorapporto con la natura (che non è matrigna qua) \\
Luna ricopre un ruolo importante \\
Suono delle campane \t la festa che si avvicinando \t tono colloquiale \t "il cor si riconforta" \t cuore aveva gia seguito il ciclo dello sconforto, e poi del riconforto \\
Montale \t "I limoni" \\
Piazzuola \t luogo limitato, ludico, circoscritto \t non conosce il dolore, è un limite di protezione \\
Dimensione di gioia viene descritta anche con il danzare e i salti di gioco dei bambini \\
Operosita \t il zappatore che fisichia \\
"Tutto tace" \t personificazione e sinestesia \\
"odi la sega / del legnaiuol" \t enjambmement \t lucerna, energia elettrica \\
Operosita precede il giorno festivo \\
Esterno - esterno - interno ?? \\
Manca ora la parte riflessiva, cogitativa \\
"Diman" \\
Il giorno in cui si sperimenta la felicita \t che è correlata alla convivialita???

\v

La riflessione inizia già nel blocco precedente \\
La parola "travaglio" \t dal francese vuol dire faticare, lavorare \t permette di creare un ponte di connessione con montale, dove viene presentato il travaglio della vita \\
Montale parte da una formazione scientifica \t mentre Leopardi da una onnivora \\
Ultima strofa \t focus sul garzoncello \t strofa filosofica e meditativa \\
Garzoncello \t attenuativo, vezzeggiativo affettuoso \t che è un ragazzo che lavora \\
Viene messa in risalto la sua giovane età, ma anche essere il suo lavorativo \\
\textbf{Perchè è scherzoso}? \t si approccia alla vita ed è nel contesto del sabato del villaggio \\
Monito di godersi questi momenti \t età fiorita \t Catullo, età come un fiore \t celebrazione del messaggio epicureo, ciclicità si legge con il giorno \\
Catullo indiviuda nel nuovo giorno una rinascita \t lucenti soli sta per i giorni \t poi dice che ci aspetta la morte: il giorno culmina con la sera, e il sonno è come una morte = chiusura del cerchio nell'attesa di un nuovo giorni, circolarità \\
Animo sereno quando ci si alza \t con l'animo pieno di aspettative, come nella vita \\
Minore è la lunghezza della vita, maggiori sono le aspettative \\
"Godi, fanciullo mio" \t messaggio epicureo \\
Specularità tra l'io leopardiano e questa figura del giovane \\
"Stagione lieta" è da intendersi come una sinestesia \t fase della vita \\
Inizio delle delusioni \t godi l'infanzia il + possibile, che significa non approcciarsi alla festa

\s[Alla luna]
Monte dove l'autore era solito recarsi, immedesimazione con la natura, che lo accudisce \t la percepisce come madre, ma già non è completamente positiva \\
Natura viene definita "graziosa" \t luna comunica eticità e bellezza, ed è come un faro nella notte \\
Luna è il faro oggettivo, ma anche il suo faro personale \\
Luna rappresenta anche la fertilità \\
Sia leopardi sia il paziente vedono l'interlocutore come neturo \t aprono cosi l animo senza alcun vincolo \t luna è psicanalista per leopardi \\
Psiconalista infatti non puo dare una risposta \\
L'animo si transponde \t 29 giugno, colle, osservazione della luna \\
Luna pendeva \t come se fosse appesa nel cielo \\
Riminiscenza dantesca \t selva, che protegge \\
Prima presentazione idilliaca \t poi presenta il dolore \\
Nebuloso e tremulo \t per le lacrime davanti agli occhi \\
"Che mi sorgea sul ciglio" \t sineddoche, metonimia \\
L'imperfetto rappresenta la repitività di un'azione \t mentre passato remoto è conclusa \\
Visione pessimistica (non vittimistica) \t "o mia diletta", dal latino, = scegliere \\
"E pur mi giova / la ricordanza" \t verso spaccato \t riflessione filosofica \t entra l'ente consolatorio del ricordo \\
Il ricordo è in grado di tramutarsi in emozione \\
"noverar l'etate / del mio dolore" \t espressione vittimistica \t Bergson parla di ricordi che vengono scelti \t tempo interiore segue ritmi emozionali \\
Esclamazione del verso successivo \t la memoria nel tempo giovanile è corta, mentre le aspettative sono molte \\
Leopardi ha già fatto esperienza della vita \t ha già conosciuto l'arido vero \t ovvero la festa = incipit dell'arido vero \\
Ci apre alla verità ma porta dolore 

\v

Testo sembra diviso in due, dialogo diretto con la luna \t che non risponde \t flusso di pensiero nel rispetto della sintassi \\
Stream of consciusness di Joyce non segue la sintassi invece \t è forma + libera \\
La pulsione verso l'infinto si esprime attraverso i modi infiniti dei verbi (rimembrare, ...) + enjambement anticipa lo stile franto del ciclio di Antasia ??

\s[Recuperare 28 ottobre]
\s[La quiete dopo la tempesta]
Testo è stato composto in settembre, nel 1829 \t è collocabile a Recanati, come emerge nel testo \\
Mostra affinita con il canto notturno di un pastore errante per l'Asia \t che viene composto dopo \\
Idillio = componimento campestre che trasmette l'equilibrio tra uomo e natura \\
Impiega 3 giorni per comporre questa poesia \t processo lento \t ma non è un componimento problematizzante \t rimanda alla dimensione epicurea \\
Chiaro nella valle il fiume appare \t il fiume è limpido, non è contaminato ne offuscato da detriti \\
Diomensione idilliaca dell'inizio \t questo è il canto del ritorno alla vita dopo un dramma, la ripresa della natura \\
Gallina \t rientra in un paesaggio campestre \\
Linguaggio non elettivo, non specifista come direbbe Pascoli \\
Prima evocazioni uditive, poi visiva (montagna) di pace \t montagna rappresenta a volte anche un limite \\
Leopardi non da indicazioni \t ripresa della vita atemporale \t non da coordinate temporali, solo dpo la tempesta \\
Evoca un meccanimso insito nella natura \t dopo il buio viene la luce \\
No termini elettivi, semplicita espressiva \\
Operosita del borgo leopardiano \t è presente anche qua \t operosità genuina che si delcina nelle figure dell'artigiano, etc.. \\
"Con l'opra in man cantando" \t con i suoi attrezzi da lavoro \t si affaccia per verificare come il cielo si sia pacificato \\
Tempesta della vita \t dopo subentra la pace \\
Eros a tanatos \t espansione e risoluzione \t anche seneca, che spiega i meccanismi naturali \\
Anche la ginestra \t i fenomeni naturali distruggono, senza tenere conto dell'uomo \t Pascoli, l'ampio tuono scompare \\
Manodopera è deficitaria \t ma le professioni artigianali sono in uno stato di sofferenza \\
Femminetta \t vezzeggiativo nel dispreggativo \t femmina non è proprio la parola corretta, come donzella o ragazza \\
Apre terrazzi ... \t chiasmo + enjambement + iperbato \t apre, apre \t anafora \\
Odi \t chiama in causa un interlocutore \t un tu generico \\
Il carro stride del passegger \t enjambement + onomatopea + Carducci, inno al progresso, stridio di freni nella "Ferrovia in una mattina di autunno" \\
Al verso 25 cambia rotta \t fino al verso finale, ampia parentesi filosofica, fatta di domande (senza risposta) \\
Sono domande esistenziali profonde \t domande di tutti, non solo di chi è affliatto (come leopardi) \\
Atarassia \t epicureo \\
Anafora di quando \t domande intensamente esistenziali e leopardiane \\
Il piacere è figlio di affanno \t stato di non piacere momentaneo \\
Anche chi rifiutava la vita, viene spaventato dalla morte \t invito alla resilienza \\
Ungaretti dice che non è mai stato attaccato alla vita ?? \\
Fredde, tacite, smorte \t climax ascendete, ma anche Folgori ... \\
Verso 41 \t dolore è cio che piu ineliminabile nella vita umana \\
Sudar le genti \t sinestesia \\
Cortese \t intende il contrario in realta \\
Se montale maglia che si è allentata porta al miracolo \t ovvero il barlume di piacere \t nasce dall'affanno, è prodotto del dolore \\
Ultimi versi: al verso "mostro miracolo" \t monstrum, mostrare, Livio e miraculum, sempre di Livio in dimensione cristiana \\
Mirare \t esprime la meraviglia \\
Il ripetere ritmico delle domande porta come un tormento \\
Visione di un epicureismo che non trova le risposte, non risolve i drammi della vita (come si propone l'epicureismo, col quadrifarmaco)

\s[Il passero solitario]
È un componimento descrittivo, come quelli analizzti prima \t analizza la quotidianità \\
Composto tra il 29 e il 35 \t delinea prepotentemente la sua visione epicurea e la vita \\
Il passero solitario è sulla vetta di una terra \t che ricorda il saggio epicureo, il quale si eleva da una torre di avorio \\
La torre si eleva sopra gli uomini \t torre d'avorio è l'ottica del distacco totale \\
La torre d'avorio è super partes e vede da lontano il tormento umano, che vede gli uomini dall'alto \\
Il passero solitario vede una selittivita nei confronti degli altri \t ma una mancanza di dialogo che possa creare un collegamento \\
Assistiamo a una simbisiosi tra leopardi e il passero \t si tratta di un transfer \\
Questa somiglianza non si coglie subito all'inizio \t parallelsimo si coglie dopo, prima solo descrizione del passero \\
Asse viene spostato dalla descrizione poi all argomento filosofico \t dal verso 45 

\v

Fino al verso 31/32: \\
La vetta della torre antica \t della chiesa di Sant Agostino, a Recanati = zona frequentata \\
Vetta fa pensare alla montagna \t la superiorita del saggio e l'elevatezza di queso posto \t permette di vedere senza essere visti \\
Non specifica il tipo di uccello \t specifismo pascoliano, che aveva contestato le viole e le rose che non fioriscono insieme \\
Prime quattro strofe descrivono la solitudine del passero \\
Al verso 6 \t brilla nell'aria... \t chiasmo \t natura alleggerisce le pene del cuore \\
Verso 8 \t viene descritto il comportamento del passero \t odi è stato visto in modo equivico: tu lettore o tu passero solitario o tu generico \\
Anche verso 8 è un chiasmo \\
A gara contenti \t contento è composto dall'associazione di cum e tendo \t che conciliazione, che deriva dal latino \\
Importanza di guardare il cielo, ovvero l'immenso \t e liberta del cielo (libero ciel) perchè permette agli uccelli di esprimere se stessi \\
Opposizione tra loro e tu \t verso 12 \\
Si vede gia distacco e atteggiamento epicureo \t non volonta di mescolarsi a un gruppo \t il passero è diverso, si distingue \\
Pensoso \t evoca linguaggio petrarschesco, in cui il poeta si misurava con il contesto \\
Poi serie di negazioni \t non compagni non voli \t è un'anafora \\
Compagni indica la comitas \t uccelli volteggiano nel cielo \\
Verso 11 \t focus ciceroniano \t tu il tutto miri \\
13 \t anafora di non, frase nominale \\
Ritorna la metafora del fiore \t si configura come l'eta della speranza \\
Prima descrive il passero, poi dal 17 fase della immedesimazione \\
Oime \t esclamazione che pone l'accento sulla somiglianza tra l'autore e il passero \\
German di giovinezza \t germano significa fratello \t si trova anche, inteso come sorella, nell'Enede \t Anna viene chiamata germana \\
Sospiro acerbo \t amore è fonte di sposiri, dei giorni in cui ci si misura con la vita \\
Autore pone attenzione sull'eta giovanile \t che viene vista come rimpianto \t "provetti", giorni rimpianti \\
Anafora di quasi \\
Si arriva al tramonto \t è consuetidine festeggiarlo nel borgo \t iperbato al verso 28 \\
Versi 29/30 \t hanno sfumatura uditiva e incessante anafora + tono colloquiale \\
Operosita garantisce la vita del borgo \t e si mostra anche al tramonto \\
Ferree canne sono gli attrezzi da lavoro \\
Verso 34 \t iperbato, chiasmo \\
Mira ed è mirata \t reciprocità \t e il cor si allegra \\
Descrizione idilliaca e conviviale che caratterizza il borgo \t ciceroniano \t il focus dagli altri si sposta su se stesso, io solitario \t ora tratta il poeta \\
Dal 36 al 44 \t evidenziata la tipologia di comunicazione di leopardi \t isolato \\
In modo epicureo il tramonto viene paragonato alla vita \t si mostra in tutta la sua debolezza e fragilità \\
Beata gioventu \\
Io / tu \t verso 45 \t strategia ciceroniana \t che usa ego nelle arringhe \\
Versi fino al 60 \t mostrano autonalisi, autore si chiede cosa ne sara della sua vita \t adesso passero da solo \\
"Sera / del vivere" \t enjambement \\
Dal 45 \t serie di domande \t che sorte avrai? \\
"Certo del tuo costume non di darrai" \t gli animali per natura non hanno la consapevolezza del dolore \t gli uomini soffrono \\
A me \t tutto in rimbalzo, tu / me \\
Impetro \t latinismo \\
Che cosa ne sara del mio desiderio se gli occhi saranno chiusi e non potranno vedere il mondo? \\
Che ne sara di questi anni \t autore non da risposta \\
I versi finali contengono pentimento e il termine sconsolato \t sconsolato esprime l'apice, impossibilita di avere il conforto di qualcuno \\
Autoritratto dell'autore si apre con la solitudine e finisce con essa \t ma passero non prova dolore, mentre autore si \\
Questo testo rappresenta l'occasione per riflettere sul dolore umano e sullo stato di grazia degli animali \t puo essere confrontato con "Il giardino del dolore" \\

\s[Canto notturno]
Ribatte il concetto degli animali che non soffrono \t mentre l'uomo si \\
Il gregge è costituito da pecore \t l'autore ha letto la testimonianza del viaggio da Orenburg a Bancara \\
Questo testo viene scritto nel 30 \t ha come retroscenza la lettura di questo testo del viaggio \\
L'autore entra a conoscenza con questo testo attraverso la lettura del "journal de savantant" \t di Meyendodoff
Da questa lettura trae l'ispirazione \\
È un testo filosofico, che focalizza le domande esistenziale, fortemente argomentativo \t in una parte mostra cosa è il tedio \\
Gli animali si riposano invece, giocano, mentre l'uomo quando non ha nulla da fare viene preso da mille pensieri \\
Uniti il tema del transfer, del viaggio, i temi che coinvolgono la differenza tra uomo e animale \\
Pastore è uomo semplice \t le domande esistenziali toccano tutti \t che pero non possono essere poste dagli animali \\
Asia rappresenta l'altro, il nuovo e il diverso \\
"Viaggio di ora Bacara" ??? \\
Tardo leopardi, nel aprile 29 \t ma finisce nell'ottobre del 30 \\
Verso da 1 a 10 \t domande dirette all luna, valore psicologico dei versi: è come un transfert a cui rivolge le domande di senso \\
La luna è ferma e non risponde \t raccoglie solo le domande \\
Luna contempla \t non è una prosopopea, ma una personificazione \\
Deserto, luogo solo e desolato \t ha valore particolare per leopardi \t desere = abbandonare \t \\
Elena landoni \t "Questa luna nel deserto" ?? \t nichilismo leopardiano \t smonta l'idea del deserto come nichilismo \t filosofia di Nietzsche \\
Ciclicita della natura \t non distrugge \\
Continua con domande non ciceroniane, ma leopardiane \\
Vago = desideroso \t anafora di ancor, v 5 e 7 \\
Come il pastore repititivamente segue il suo percorso che fa ogni giorni \t ciclicita nella vita \\
Prima trance, da 1 a 9 \t la vita di leopardi è uguale a quella del pastore \t l'autore osserva, cosa si fa quando si ha un transfert: ci si presenta in terza persona \\
La luna, come lo spicanalista, è lo speccio della persona \t non giudica, perchè non parla \\
Pastore / luna \t poi il pastore diventa leopardi, perchè si fa le stesse domande del pastore \t triade in relazione \t la ricerca di risposte e la folta selva di domande, come in un cerchio che non si chiude \\
Poi strofe descrittive \t greggia / greggi \t evoca un poliptoto o un'anafora correttiva \\
Altro non spera \t non ha desideri \t questo è esistere, non vivere \\
Domanda di senso: il pastore veramente non si pone domande? poi se le porra piu avanti nell opera \\
Vero 15 \t iperbato \\
"La vostra vita a voi? ... corso immotrale?" \t filosofo, domande + sostanziose \\
Vago, in latino, vagare indefinito \t dove tende l'esistenza, qual'è il traguardo \t invece percorso immortale della luna \\
Vecchierello \t riminiscenza petrarchesca \t mezzo svestito \t messa in evidenza la sua fragilità \\
Paesaggio arido, faticoso \t e l'escursione termica \t il pastore non si risparmia \\
Il varco \t riminescenza dantesca (forse) \\
Cade, risorge, s'affretta \t climax ascendente \\
Nell abisso il pasore perde completamente traccia di se \t è il tornare al nulla \\
"Tale è la vita mortale" \t lettura laica \t vita mortale ma non viene specificato dell'uomo \t chiusura di cerchio che lascia poco spazio alle gratificazioni \\
Catone \t il vecchio \\
Sofferenza proptratta nella sofferenza di tutta una vita \\
A partire dal 38 \t versi + descrittivi e filosofici \\
Le domande si spiegano da se \t la nascita avviene faticosamente per il parto \t poi subito il tormento della vita \t il pianto \\
Perche mettere al mondo se la vita è solo sofferenza

\s[Ciclo di Aspasia]
Opera la stagione del 29 \t che viene dopo il 24, anno delle operette morali \\
Componimenti sono percorsi da domande \t riflessione filosofica intensa \\
Nelle operette morali \t c'è prosa, quindi riflessione è diversa \t mentre nelle opere liriche ci sono descrizione dell'idilliaco e riflessione filosofica \\
Kierkegaard \t non cercare il consenso fuori ma dento \t rischiare significa perdere equilibiro \t ma non rischiare è peggio \\
Paolina \t Ranieri \t Napoli, il triste epilogo \\
Ha crisi respiratoria e muore \t il vesuvio, la sofferenza che si traduce nell annullamento \t questo è il clima dell arido vero \\
Delusione da Fanni Tozzetti \t momento cruciale dell'arido vero \t dinamiche insite nella stagione delle operette morali \t ma questa opera è in poesia, mentre quelle in prosa \\
L'amore sboccia da entrambi le parti \t "mai ho amato nessun uomo come ho amato giacomo leopardi" \t ma non ha portato nella concretizzazione del matrimonio \\
L'intensita del sentimento è tanto forte quanto l'intensita della delusione \\
Triangolo \t relazione ranieri, fanni e giacomo \t amicizia che lega i tre sulla base di interessi comuni culturali e politici \\
Una amicizia anche solida, ma non priva di ombre \t dettate dal fatto che l'attenzione di Fanni è asimmetrica \t è sposata ma ha relazione con Ranieri \\
Delusione nasce perche fanni trascorre giornate giocose con i due \t che portano a delle aspettative tradite \t Eneide, Didone, "chi puo ingannare l'animo di una donna che ama?" \\
Intermediario \t è ranieri \t leopardi matura un sentimento autentico per fanni, che lo porta a sperare nel matrimonio \t questo significava che fanni doveva divorziare \\
Ma lei non è pronta a questo \t ha figlie, e il matrimonio era una copertura \\
Fanni porta a leopardi in lettera il suo garbato rifiuto \t ma lei non rifiuta la relazione con giacomo, ma la relazione amorosa \\
Aspasia \t da vita al ciclo di aspasia \t preso da "Se stesso", testimonia il crollo di tutte le speranze e totale rinuncia \\
Il dialogo di antica tradizione con il suo cuore \t con il quale parla \\
Leopardi muore gia in quel momento \t solo ranieri e sorella fanno da conforto, da ortus conclusus \\
Aspasia è il nome che prende dalla moglie di Pericle \t simbolo della femminilita greca, le donne sposata ai capi erano intellettivamente preparate \\
Anche sempronia è moderna \t il ritratto di Sallustio \t era sensuale \t sapeva leggere e scrivere pero \t queste figure erano chiamate hecire

\s[A se stesso]
Il testo rappresenta la chiusura del cerchio \t 1833 \\
Leopardi aveva gia preso il peso della forte delusione della storia con Fanni \t lei morira nell 89 \\
Fanni mantiene il matrimonio con il suo marito, per copertura \t non lo interrompe anche se l'amore non c'era \\
Dimensione di dualismo \t poi parte e giacomo rimane da solo \t anche ranieri rimane deluso, perche era presente una sodalitas tralisitca (tra i tre) della convivialita \\
Compare nei canti nell'edizione del 35 \\
Anche la sintassi segue il cuore lacerato e deluso di Leopardi \\
L'infinita vanita del tutto \\
Climax ascendente che percorre tutta l'opera \t dialogo con il proprio cuore, ricorda il dolce stil novo \\
Sintassi è frammentata, numerosi enjambement \t non creano continuita nel testo \\
Presenza dell'irrisolto nel risolto \t nel risolto non c'è piu niente da fare \t gioco tra il futuro e il perfetto \t situazione va vista col numero nel suo modulo \\
Assai palpitasti \\
Per sempre \t richiama all'investire su cio che non sappiamo \\
Si rivolge direttamente al cuore \t cari inganni (cari perche permettono di sognare, di andare oltre il contingente) \t è un ossimoro \\
Speme correlato ad inganni \t il desiderio è spento poi \t la situazione è risolta, ma una delusione è presente \t quindi l'irrisolto è presente \\
Non val cosa nessuna \t epicureo \\
Dinamica dei ricordi \t ripercorre il suo passato \t delusione in toto \\
Amaro e noia la vita altro mai nulla; e fango è il mondo \t nichilismo, la negazione del tutto, nulla vale \\
Amaro e noia \t sinestesia \\
Leopardi ama Tasso \t fango è il mondo \t Tasso scrive la creazione del mondo in 7 giorni, come secondo la bibbia \\
Il fato \t seneca, necessitas \t no egoriferico, il genere nostro, e non dice io \t smontato il vittimismo di cui Sinner lo accusa \\
Vittmista perche capisce prima cosa, poi perche non si riesce ad illudere come gli antichi, poi il vittimismo generale \\
Pascoli \t 10 agosto, "questo atomo opaco del male che è la terra" \t non è un epoca, non è un movimento, ma è l'uomo che genera questo stato di disfattismo \t in ogni tempo

\s[Tramonta la luna]
Rapporto uomo natura, meccanicismo, forza ciclica del nascere-morire \t di matrice lucreziana \\
Apoteosi della natura interiorizzata, o dei sentimenti interiorizzati nella natura \\
Rimane l'indefinito \t secondo lo Zibaldone si configura come poeticissimo \\
La luna ogni giorno nasce e cade \\
Canto catulliano \t tibi candidi soles \\
Prima macrosequenza di versi \t esprime la visitatrice che è la luna, che percorre \t lo sguardo della luna si proiettta sui paesaggi \\
La luna è un interlocutore muto e per questo gradito \\
Excursus paesaggistico \t che porta alla triste dolcezza degli ultimi versi (16-19) \t si presenta lo spaccato paesaggistico che mette a fuoco l'operosita del borgo (nella figura del carrettier) \\
Seconda parte \t passaggio dal paesaggio naturale alla riflessione che subito specchia l'essere umano \t la fine della giovinezza corrisponde al fine della vita \\
Dilettosi inganni \t sono quelli amorosi, e delle illusioni giovanili \t sono parte di un fallace piacere, che si scontra con il discovrir del vero \\
Lontane speranze (sinestesia), poi enjambement \\
Mortal natura \t il mondo della natura e il mondo dell'uomo \\
Non c'è prolissita nei versi, che invece sono franti da enjambenet (come in se stesso) \\
Confuso viatore ed invano \t triade che mostra il brancolare umano nel buio \\
Questa meta e questa ragione sono estranei all'uomo \t se si ha una meta la si vuole raggiungere \t ma posso quando mi do una ragione per farlo \t atltrimenti si esiste e non si vive \\
Materialismo, ciclicita, dolore \\
La sorte umana è misera \t se lo stato giovanile, che è frutto carico di mille pene (iperbole), durasse per tutta la vita \\
Se nel mezzo della vita si viene troncati con la morte \t dolore grande \t seneca: bisogna imparare a morire per vivere \\
All'uomo è dato di provare sempre meno e senza alcuna compensazione di bene \\
Lui sperimenta questa privazione, ma muore precocemente (39 anni) \\
Nella parte finale si rivolge alla natura \t la natura è ciclica, il sole risorgera ancora \t vita umana no \\
Il testo si chiude in modo drammatico \\
Pregnanza di meccanicismo materialistico, che induce a riflettere sulla vita

\s[Versi importanti]

\ss[Infinito]
Sempre, importanti i tempi (alternare passato remoto e forme indefiniti), indifinito, sovrumani, profondissima, scansione che arriva al verso 7 (nel pensier mi fingo), verso 8 inframezzato, lettura personale (anticipazione naturalista, e come il vento che stormisce tra le piante \t poeta/natura), antipascoli: non dice quali piante \\
Importante al verso 7 e 8 \t Io ripetuto \\
Percezioni \\
Inoltre è presente il gerundio e infinito dei verbi \t non c'è un approdo \\
Arco spazio temporale \t le morte stagioni (ossimoro) e al v. 11 l'eterno
Mi sovvien eterno \t è il primo carattere atemporale \t stagioni vivono, se no non sarebbero stagioni \t ossimoro come vita morta \\
Presente viva \t pleonastico \t con morte stagioni fa un chiasmo \\
Il naufragar m'è dolce \t sinestesia \\
In guerra e pace di Tolsoj ci sono dei riferimenti all'infinito

\ss[Passero solitario]
Spazialità, visione epicurea, relazione tra io del poeta e socialita, parallelismo tra uomo e natura, il saggio \\
Parte finale (come in tutti i testi) aspetto filosofico \\
Verso 45 - alla fine \t parte molto filosofica \\
Inconnue \t l'inconoscibile del decandentismo \\
L'infinito è dentro di noi, non nella realta \t ma nell'animo \\
Verso ? \t confronto tra tu e me \t variatio fino al verso 50, con qualche variatio \\
Le domende ultime non sono + domande ciceroniane, ma leopardiane \t in particolare da Romano Luperini \\
I due versi finali sono fondamentali \t pentirommi, sconsolato \t termini + esistenziali (pentimento deve essere un punto di partenza)
\end{document}
