\documentclass[12pt]{article}

\usepackage[a4paper, total={6in, 8in}]{geometry}
\usepackage{textcomp}

\begin{document}
\setlength{\parindent}{0pt}

\def \t {\textrightarrow}
\def \v {\vspace{1em}}
\def \bi {\begin{itemize}}
\def \ei {\end{itemize}}
\def \s[#1] {\section*{#1}}
\def \ss[#1] {\subsection*{#1}}
\def \sss[#1] {\subsubsection*{#1}}

\s[Lampo]
Apocalittica è la visione iniziale del testo \t ricorda la passione di cristo, e in questo si spiega la nominazione di terra e cielo \\
Immagine fulminate, che ricorda il giorno di passione di cristo \t in cui il cielo si scurisce, cristo muore e l'umaita si salva \\
La terra ansante, livida in sussulto \t climax ascendente e personificazione della terra \t viene vista come una persona morente \\
Il cielo ingobro di nubi \t che creano immagine cupa di imminente tragedia \\
Nel climax ascendete c'è il contrasto cromatico \t qui si ha l'immagine del bianco e del nero, nascita e morte \t e una casa che appare bianca bianca (ripetizione come nelle filastrocche dei bambini) \\
Tacito tumulto \t di nuovo si ritrova nel 5 maggio, ed è un ossimoro \\
Un immagine umana e una paesaggio in tempesta, un sublime tragico \\
Il climax ascendente ricorda cadde, risorse, giacque \t tipico del testo il 5 maggio di manzoni \\
La casa è l unica presenza umana \t ed evoca una natura tormentata, ribelle \\
Nella fulmineita apparisce e scompare \t la fulmineita, e lo sbattere delle ciglia come riflesso \t movimento non razionalmente deciso \\
Movimento rapido, fulminante, è presente \t la similutidne come un occhio \\
Immagine della notte nera, e l immagine dell'occhio esterefatto \t l'imagine è inquietante, e la terra è ansante \t un presagio di morte \\
E il contrasto con l immagine di vita, che è spettatrice di morte \t uno sguardo che segue sempre \\
Si puo leggere l'occhio del padre che osserva dal cielo, ma bisogna specificare che l autore non ebbe un fede certa e non si converti \t l autore fatica a vedere una lettura cristologica di suo \\
Si puo notare l equilibrio tra spazi bianchi e parola scritta \t il testo si presenta come compatto

\v

Si ha un primo verso e un interruzione \t sia in "Tuono" sia in "Lampo" \t primo verso, uno spazio bianco e poi gli altri versi \\
Equilibrio tra spazio bianco e parola scritta \t testualita franta \t l'osservazione del testo comunica \\
Il gioco di tempi verbali va analizzato \t il perfetto, i participi perfetti, e dove si parla di casa si parla il perfetto al primo verso (e anche quando si parla di cielo e terra) = meccanicismo \\
L'immagine di una apertura, che nella sua fulineita si apre e si chiude istantaneamente

\ss[Temporale]
Nell'ordine di composizione si ha lampo, tuono e poi temporale \\
Questa è l'epigolo della triade \\
Testualita franta, che ripropone un rinnovato equilibirio tra spazi bianchi e parola scritta \\
I versi sono lapidari \\
Incomprensibilita di un ala di gabbiano che chiude il testo \\
Testo enigmatico che ricollega alla incontrollabilita della natura \t che viene resa con sensazioni uditive, garantite dall onomatopea (bubbolio) \\
Il bubbolio \t è il rumore del tonfo che accompagna il temporale \\
Rosseggia l orizzonte \t immagini che evocano sensazioni visive \\
Il rosseggiare evoca anche l'immagine di uno spazio marino, dove il sole sempre affocare (infoucoarsi) e affogare (col tramonto nel mare) \\
Il cromatismo di questi versi \t che sono topici e nucleari, ma non sintatticamente coerenti \t sono versi nominali, ci sono pochi versi \\
Sono una serie di immagine sovrapposte, una che subentra all altra \t come se fosse un quadro \t e il crimatismo è diverso, c'è nero di pece e rosso \\
Il nero di pece, è il nero che ha una sfumatura bluastra \t evoca un episodio biblico e un testo di Giustino, per cui la pece serviva per traghettare il cibo da una sponda all altra del fiume \\
Nero di pece è una forma pascoliana, che nel range dei tropi ?? \\
Tra il nero un casolare \t c'è chiasmo tra verso 4 e verso 6, dove ci sono i due neri \\
Stracci di nubi \t nubi non hanno contorni definiti, vede delle forme che non identifica \t stracci di nubi è un tropo (ovvero un traslato, e quindi una metafora/analogia, oltre ad essere una sinestesia) \\
Nubi non permettono di avere una visione totalizzante del cielo \t lo stesso cielo che manda la pioggia di stelle nel X Agosto \\
La pioggia piu greve o men greve (d annunzio) \t è fatta sempre di goccie \\
Abbiamo una percezione che non permette di abbracciare la totalita \t gli stracci di nubi sono una prima immagine di limite nel decadentismo \\
L'ala di gabbiano evoca la liberta, è l emblema della liberta \t il gabbiano vive librandosi nell aria \\
Viene posta alla fine del testo \t il sentore è la volonta di liberarsi dal peso nella natura opprimente \\
La frammentarieta del testo premia versi nucleari e nominale, dove predominano le figure retoriche pascoliane \t dal primo verso, siamo rapiti in un silenzio assordante \\
Le varie immagini evocate sono di silenzio
\end{document}
