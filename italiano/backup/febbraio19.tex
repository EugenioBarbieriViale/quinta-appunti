\documentclass[12pt]{article}

\usepackage[a4paper, total={6in, 8in}]{geometry}
\usepackage{textcomp}

\begin{document}
\setlength{\parindent}{0pt}

\def \t {\textrightarrow}
\def \v {\vspace{1em}}
\def \bi {\begin{itemize}}
\def \ei {\end{itemize}}
\def \s[#1] {\section*{#1}}
\def \ss[#1] {\subsection*{#1}}
\def \sss[#1] {\subsubsection*{#1}}

\s[Pascoli- Nebbia]
\ss[Canti di Castel vecchio]
Nebbia e canti di .. sono complementari \t sono legati come fenomeno metereologico e luogo \t Castel Vecchio di Barga, qua si individua una natura campestre e rurale \t qua è frequente la nebbia \\
Sono connessi, e rispondono al legame fortissimo tra la biografia, il luogo e la consapevolezza di un destino che accompagna l uomo a una meta \t meta che non ha risvolto religioso \\
L'elemento di collegamento è anche con Giuseppe Nava \t che ha curato le ediziondi di Myricae \t e in Nava si legge la convergenza con ??? il matrimonio della sorella \\
Il matrimonio della sorella viene visto come una rottura del nido dell autore \t e lui non va neanche al matrimonio \\
Il matrimonio simboleggia anche l'irrecuperabilita \t e questo ha tratti morbosi \t anche nel gelsomino notturno si trova una sessualita morbosa \t il cognato viene visto come colui che ha rubato la sorella

\v

In tutto questo la nebbia sta per la confusione \\
Nebbia trova espressione in "foggy" \t che limita e ottunde la chiarezza mentale \\
I canti di castel vecchio \t hanno meno successo quando si ricerca nella lirica un andamento meno frammentario \t frammentarieta è presente fortemente in Mirycae \\
Qua invece si ha come una narrazione intera \t si ritova quindi Leopardi (anche lui compone i canti) \\
Tema sempre naturalistico, e aspetti biografici che emergono \t osservazione della segreta immensita dell'universo, tema della morte che si riflette \\
La nebbia ricorda in parte Myricae, ma si differenzia con la presenza del male che domina il mondo \t decadentismo

\ss[La poesia]
Si rivolge direttamente alla nebbia \\
Che io veda solo la siepe dell'orto \t paradigma del limite condensati nella strofa \t e orto = hortus conclusus \\
La valeriana è un rimedio naturale che viene impiegato nella farmacologia di ampio spettro \t ha valore sedativo e valium è rilassante \\
La valeriana è all'interno delle crepe \t nelle crepe dell'anima, le crepe del muro il quale crea una corazza contro il dolore \\
In montale invece ci sono le biche di minuscole formiche \t qua un operosita conquistata tra i muri della vita, qua invece una presenza massiccia delle crepe dell'animo e la volonta di tenere il dolore placato \\
Due meli \t si immagina un giardino, un frutteto \t la visuale rimane in questo orzzonte limitato \\
Che io veda solo quel bianco di strada, che porta al campo santo \t \textbf{solo} = orizzonte proibitivo, circoscritto, limitato, è per Pascoli protezione \t la strada è bianca = purezza profonda e sentita \\
Contrasto cromatico invece con il nero mio pane nella strofa precedente

\v

Andalento cantilenante di nascondi nascondi nascondi \t anafora che crea un ritmo, come se fosse una canzone \t crea un ritmo che genera una simmetria e armonia testuale \\
Presenza di un tono esclamativo riferendosi alla nebbia \t proprio come Leopardi si riferiva alla luna \\
Le cose lontante = un lessico quotidiano \t accosta termini elevati a termini meno elettivi \\
Rampolli = fai figli \\
Aspetto unificante unito all'aspetto apocalittico di lampi notturni e frane \t i tonfi di Lavandare ritornano \t e si ritrova la violenza dei fenomeni naturali in queste immagini \\
La nebbia è impalpabile (non si puo toccare) e scialba (senza colore) \\
Nascondi = anafora correttiva \\
Le cose lontane sono il dolore, e il nido integro che ormai non esiste +

\v

Le cose da nascondere sono le cose ebbre (piene) di pianto \\
Chio veda soltanto = iperbato \\
Poi verso fatto da una parola soltanto \t equilibrio spazio bianco e parola

\v

Che danno soavi lor miele \t addolciscono i dolori \t usl pane nero loro addolcirebbero \t il pane nero era il pane del periodo di guerra \\
Nella allusivita pascoliana, è un pane carico di dolore \\
Che io ami \t qua c'è il senso del perdono \t non ha mai perdonato gli uccisori del padre ma neanche vendicato \\
Stanco don don di campane \t hanno un valore molto importante \t onomatopea \t sono le campane che preannunciano la morte, infatti è il stanco don don \\
Anche qua è presente il fanciullino \t i bambini si esprimono per onomatopee \\
Che io ami e che io vada \t la volonta di andare oltre il contigente \t che io vada è andare oltre la contingenza

\s[Gelsomino notturno]
Viene visto come il \\
Oscurantismo = anche barriere comunicative \t Tertulliano = non bisogna vergnarsi di cio che dio ha dato all uomo \\
Siamo al 1901, e si va proiettati verso un nuovo secolo \t tanto atteso quanto temuto \t il secolo della svolta e la conquista di una societa avanguardistica \\
La poesia della notte \\
Testo allusivo e narrativo nel contempo \\
Il gelsomino è un fiore che di notte si schiude per rinchiudersi il mattino \t si dice che abbia odore seducente \t ed è una pianta rampicante

\v

Qua viene descritto nell'ora serale \t i suoi petali si aprono ora \\
Le ciglie evocano la piogga del pineto \\
Specifismo pascoliano: viburni sono le farfalle notturne e crepuscolari \t è importante per il movimento del crepuscolarismo \\
La festa è la festa del matrimonio \t sono grida di divertimento \\
Là da sola \t è lontana \t una casa bisbiglia è onomatopeico \t e di nuovo riferimento ornitologico \t il nido interrotto \t gli occhi sotto le ciglia, i nidi sugli alberi \\
I calici aperti del gelsomino \t rimandano all'utero \\
Il nido è visto come la vita e nella dimensione descritta qua, è la nascita della vita = momento della fecondazione \\
I nidi non sono visti nella loro accezione normale della famiglia \t qua il nido è la famiglia che deve nascere \\
Le senzazioni visive sono sempre + incalzanti \t e gelsomino + odore di fragole = passione che scoppia, le fragole rosse sono l'allusione all'apice della passione \\
Il lume nella sala splende \t e l'erba che nasce sulle fosse è discusso \t come nell'utero si crea un germoglio nuovo \t utero paragonato all'erba sulle fosse \t paragone stonante, e fa riflettere sul mancato apprezzamento dei metodi riproduttivi femminili \\
Forse autore è stato anche bisessuale
\end{document}
