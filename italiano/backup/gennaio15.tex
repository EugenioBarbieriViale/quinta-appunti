\documentclass[12pt]{article}

\usepackage[a4paper, total={6in, 8in}]{geometry}
\usepackage{textcomp}

\begin{document}
\setlength{\parindent}{0pt}

\def \t {\textrightarrow}
\def \v {\vspace{1em}}
\def \bi {\begin{itemize}}
\def \ei {\end{itemize}}
\def \s[#1] {\section*{#1}}
\def \ss[#1] {\subsection*{#1}}
\def \sss[#1] {\subsubsection*{#1}}

\s[Scapigliatura]
La prosa \t è antimanzoniana e anticlericale \t Tarchetti ed Ecletto Arrighi 

\ss[Ecletto Arrighi]
Ha scritto diversi romanzi, tra cui "Gli sposi non promessi" \t esatto ribaltamento dei promessi sposi \\
Lo ribalta per:
\bi
    \item la negazione di tutti i valori risorgimentali
    \item per quegli standard che caratterizzano la figura femminile in Manzoni
    \item i valori del buono e del bello vengono ribaltati
\ei
È cambiato il sentire e la visione che la politica deve essere una condivisione \\
In questo romanzo cambiano le figure femminili \t percorrono il percorso dalle madonne alle donne \\
Lo stereotipo di Lucia in manzoni è gia cambiato \t che infatti è contadina, è scura di capelli (non piu donna d'angelo) \t è presente uno stereotipo che coinvolge la madonna \\
Manzoni pero non ritrae la donna come gli scapigliati \t manzoni vuole ritrarre il vero cristiano \\
Lucia pero lavora \t non è quindi una figura passiva \t e fa sentire la sua voce con l'innominato \\

\ss[Tarchetti]
Scrive invece "Fosca" \t fosca è l'ombra \\
L'autore fa finta di trovare un manoscritto \t manzoni è presente, ma viene ribaltato \\
Muore di tifo \t opera viene lasciata incompluta, e completata da Slavatore Farina (che ha il ruolo di occupari all'universo editoriale dell opera) \\ 
Fosca ha un male psicologico \t che la debilita anche fisicamente \t si parla di una malattia che preclude il contatto con l'altro \\
Si innamora di Giorgio, con il quale vive una relazione santuaria (ma assidua) \t anche se non ufficiale, perche lei era sposata \\
Ma lei è corrotta, non è pura ?? \\
L'immagine di clara rappresente questo: un legame disimpegnato, che nella societa borghese non esiste \t e la relazione continua \\
Ma giorgio si infatua di Fosca ?? \t che ha malattia psicologica, ed era costretta a rimanere a letto \\
Fosca  il personaggio principale, che è luce nonostante il nome \t mentre clara è l'amante di Giorgio, che invece è ombra \t è la moglie borghese perfetta, ma tradisce \\
Mentre fosca rappresenta la malattia, clara la salute \\
La figura di fosca tiene in mano le redini dell evoluzione della vicenda \t giorgio, quando lei muore, si ammala alla fine \t sia dell'animo sia del corpo \\
La simbiosi emotiva era stata cosi forte, che ha prodotto questo \\
Vivono una continua fuga e ricerca \t che crea una interdipendenza dettata dalla complementarieta \t i due si appoggiano l'uno all'altro \\
Ci sono scene in cui si verificano amplessi con lei morente (rappresenta il vero) \\
La malattia ora è sdoganata nella letteratura \t i deliri amorosi sono presenti nella letteratura in ampia possibilita di espressione \t alda merini ha avuto anche lei deliri, che nascono nel non riconoscimento dell'altro \\
La malattia psicologica e la degenerazione fisica \t incipit della psicanalisi

\v

"Le libere donne di Majano" nel 1900, Marlo Tobino \t nelle colline toscane, un medico ?? \\
Calvino \t ritrae la perfezione nell'imperfezione \\
Qua la malattia è ritratta senza filtri (in fosca) e giorgio e fosca sono legati da un amore viscerale \t alla fine muoiono entrambi (giorgio forse, romanzo incompiuto \t non c'è l'idillio) \\
Dall'opera non si richiede una caratterizzazione dei personaggi \t come in Pierpaolo Pasolini, si trova il turpe, il brutto, il malvagio, un umanita sporca ma viscerale, pura, autentica \t come in "Ragazzi di vita" \\
Specchio di una societa che non c'è piu, che guarda l'utile e il denaro \t e dimentica l'interiorita umana \\
Opera fondamentale, insieme a "The picture of dorian gray" e "A rebouers" di Karl Huysmans (tradotto: Controcorrente) \t sono le bibbie del mondo decadente \\
Nel decadentismo, si trova sempre una realta caratterizzata sempre da un atomo di male (Pascoli) \\
Fosca nel 99 ?? \t delusione risorgimentale presente 

\v

Stile è fortemente irrazionale \t e piena di digressioni, che mostrano il buio e la luce \t molte digressioni introspettive, in cui anche giorgio parla il linguaggio del dolore profondo \\
Fosca si attacca all'amore con giorgio, come per aggrapparsi alla vita \t simbiosi totale, che spiega come il distacco tra lucia e renzo non è possibile (che è figlio di valori e di un era che non esiste piu) \\
Qua invece c'è epsressione di un morbosita \\
Come nell'Innocente di d'Annunzio, o il romanzo e il fuoco \t Fosca è un punto di partenza ineliminabile \\
Introspezione e analisi interiore dei personaggi ??? \\
Questa è espressione di un anticonformismo, che diventa conformismo \t all'epoca la malattia era sondata, ma non identificata \\
La sua malattia è legata con l'isteria \t ma lui vede lei, non la sua malattia \\
Il primo incontro avviene alla caserma dei carabinieri, dove lui lavora \t lui accoglie una sua denuncia \\
La malattia non è per lui inibente, ma è una realta che si azzera con il loro amore \\
Dopo una notte d'amore, diventa piu vicini e percorrono cammini paralleli \t sono simmetrici: anche lui moriira per malattia \\
È pero una distruzione reciproca \t lui distrugge lei, lei distrugge lui \\
La tematica della follia e della malattia \t è spesso collegata con la guerra 

\v

L'unita d italia viene contestata con una malattia che coinvolge il mondo militare \\
Punti chiave:
\bi
    \item malattia e sensibilita
    \item ossesione per la morte
    \item follia coniugata all'isteria 
\ei

\s[Dopo la Scapigliatura - Positivismo]
Si fanno strada delle illusioni scientifiche, che passano sotto il positivismo con Compte \t la scienza è La Risposta alle odmande di spiegazione sul reale \\
Il positivismo vuole essere il rimendium doloris \t la scienza annulla il dolore \t si crede in un meccanicismo rigido, di cause ed effetti \t il mondo è tutto studiabile e comprensibile \\
La filosofia è il positivismo \t bisogna prima parlare della letteratura europea \\
Questa volonta si era espressa nel realismo francese: Zolà, i fratelli De Concuer, una moda che vuole descrivere la realta come essa è, senza filtri \\
Ci sono testi che si modulano su cio che succede in una sala operatoria, c'è volonta di far aderire parola e vero \t questo era possibile in francia, perche non c'erano le maglie restrittive di stampo cristiano \\
In italia c'è infatti il papa, ed italia è fortemente cristiana \\
La francia è all'avanguardia nei movimenti di protesta e di ribellione \\
In francia la prosa è centrale: il romanzo ritrae la realta oggettiva e la letteratura diventa strumento di conoscenza e denuncia sociale \\
Verga non è propriamente verista \t lui contesta e le sue opere rappresentano il fallimento dei valori del verismo \t introduce il discorso indiretto libero, il coro del villaggio \\
Verga si oppone al positivismo \t nella prefazione ai malavoglia, l'attaccamento alla terra (come l'ostrica attacata allo scoglio) e se ci si stacca, ci si perde \\
In francia siamo ancora nella fase in cui si denuncia ancora la realta \\
Il punto di falla del positivismo è che la psiche umana non puo essere misurata scientificamente \t e questo è il punto di partenza del decadentismo \\
La nascita delle scienze umane e della sociologia, e lo studio di taine verte sulla race, milieu e momon \\
Accanto a questo: Giuseppe Lombroso, la nascita della fisiognomica = lettura del volto che suggerisce la caratterialita, e la nascita della criminologia
\end{document}
