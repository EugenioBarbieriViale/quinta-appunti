\documentclass[12pt]{article}

\usepackage[a4paper, total={6in, 8in}]{geometry}
\usepackage{textcomp}

\begin{document}
\setlength{\parindent}{0pt}

\def \t {\textrightarrow}
\def \v {\vspace{1em}}
\def \bi {\begin{itemize}}
\def \ei {\end{itemize}}
\def \s[#1] {\section*{#1}}
\def \ss[#1] {\subsection*{#1}}
\def \sss[#1] {\subsubsection*{#1}}

\s[Verga]
La novella "Nedda" è la consacrazione al verismo \t che vive questo passaggio dalla scapigliatura al verismo \\
Prima di aprire lo sguardo sul vero, l'autore passa attraverso una parentesi duratura della fase scapigliata \t che si configura come terreno fertile per i romanzi scapigliati \\
Soprattutto "Eros", "Tigre reale" ed "Eva" (anche un pochettino "La capitanera") \\
Apre le porte a un vero che puo essere anche turpe, malato \\
A milano comprende che bisogna guardare la realta con un occhio aperto \t questo è il punto di partenza per il suo verismo \\
Verismo inteso come punto di passaggio dalla sua consapevolezza per amore del vero \\
Verga parte come verista ed approda alla visione positiva del reale \t infatti: \\
verismo vuol dire apertura totale alla realta, linguaggio che fotografa le figure, che sono radicate in un ambiente di una sicilia vittima di preconcetti \\
C'è un perimetro ben recintato in cui i suoi personaggi si muovono \\
Adesione alla realta ci apre alla contestazione stessa del verismo e alla sua messa in discussione \t con la metafora dell'ostrica attaccata allo scoglio \\
L'ostrica è l'emblema di una visione dell'essere umano che deve stare nel suo ambiente \t e se esce, si perde e si snatura \t incontra un mondo nel quale non puo identificarsi \\
L'introduzione ai "Malavoglia" si configura come una messa in campo del macrotema del progresso \t che è da vedersi in una discussione del progresso stesso \\
Verga si mostra reticente contro il progesso, che snatura l'uomo e allontana dalla natura e avvicina alla civilta (che toglie all'uomo lo stato di natura che fa vivere all uomo stesso una solitudine, e mancanza di senso e di appartenenza) \\
È quello che si legge nella selezione della specie, nel progresso non puo trovare altro se non la morte  \\
Verga parte da un esigenza positivista, ma se è vincolante si nega il principio primo della scienza \t la conoscenza nel progresso rimane aperta \\
Questo Verga lo teme ???? \t aderire a una visione verista della realta implica il fotografare la realta per quello che è \\
Questo neccessita di una serie di strategie testuali, come \textbf{far in modo che l'opera d'arte sembra essersi fatta da sè} \\
Ne consegue che la mano dell'artista che scrive scompare \t quindi impersonalita, e presenza nuda e cruda di personaggi che si muovono e che sembrano comunicare indipendenti \t con il loro linguaggio e dialetto siculo \\
Si parla quindi di un abbassamento tonale dell'autore \t quello che disse Sciascia, che studia il linguaggio \t che parla di una sicilianita verghiana, e una abbassamento tonale in funzione di questa sicilianita \\
Anche nei romanzi scapigliati c'è abbassamento tonale \t perche vengono trattate figure di donne conpromesse \\
C'è di solito un altro personaggio, ovvero il popolo e il gruppo del villaggio \t che giudica e si configura come coro manzoniano \\
La figura della donna sta cambiando, e Lia soffre la vita di questi borghi \t spera di affermarsi laddove c'è una visione + aperta dell'essere umano \t vede nella Parigi avanguardista \\
Muore nella purezza e nell'integralita valoriale \t es. Luciano Anceschi perde l'aureola \\
Inotlre quando Lia viene esclusa dalla famiglia e dal paese, che è il coro giudicante \t quel gruppo consolidato sui pregiudizi e sui preconcetti \t su quella visione preclusiva e delusa del reale \\
Il messaggio è che è impossibile abbattere le differenze, la poverta, conquistare la liberta totale, vivere una dimensione nella qualle essere riconosciuti per se \\
I nuovi linguaggi della comunicazione \t popolazioni isolate culturalmente e geograficamente \t infatti la sicilia è un isola

\v

Verga è toltamente prosa \t compone il paritcolare di quelle aree, che conosce altro dal suo mondo siculo, e vive una temperia culturale per la ricerca di una liberta autentica, che ricercavano gli sacpigliati (di pensiero, di anima)\\
Se l'anima è libera, sa anche scegliere dove orientarsi \\
Ciclio dei vinti \t progettualita che non si mette in atto \\
Malavoglia ritraggono una visione paesaggistica \\
Presso Treves (casa editoriale anche di Leopardi), verga pubblica "Padron Toni" \t che poi gli da la notorieta \\
La sponsorizzazione di un opera avveniva attraverso le riviste e il passaparola \\
Nel 79 scrive "Il capuana" \t si propone di avere un opera che presneta una ricostruzione intellettuale e una rappresentaziozne vera della reatla (parole di Verga che passano attraverso capuana) usando il metodo scientifico \\
Verga vuole studiare un villaggio marinaresco siciliano \\
Sidney Sonnino \t uno dei ??? \\
Inchiesta Sonnino \t il sud aveva la piaga del brigantaggio e analfabetismo \t ma grazie a Vilgari e Franchetti, Verga approda ad alcune informazioni utili, che gli consentano di pubblicare nel 81 "I Malavoglia" \\
Quest'opera si scontra con un clamoroso insuccesso, e un fiasco completo \\
Ritrarre queste condizioni \t pubblico ancora troppo manzonizzato respingono questo \\
I malavoglia mostrano la rivoluzione del progetto, mentre il contesto dei lettori non è ancora pronto \\
Sicilianita = quell insieme di valori che si ritrova anche in pirandello \t l'atteggiamento di difensiva verso l'altro, l'attaccamento alla terra, il senso di appartenenza e la visione dell'amore come legata al triangolo amoroso \\
Sicilianita si applica ad Aiala ??? in Pirandello, che viene giudicata dal paese, e da Lia \\
Tutti questi aspetti confluiscono nel sistema dei personaggi
\end{document}
