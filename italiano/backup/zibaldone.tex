\documentclass[12pt]{article}

\usepackage[a4paper, total={6in, 8in}]{geometry}
\usepackage{textcomp}

\begin{document}
\setlength{\parindent}{0pt}

\def \t {\textrightarrow}
\def \v {\vspace{1em}}
\def \bi {\begin{itemize}}
\def \ei {\end{itemize}}
\def \s[#1] {\section*{#1}}
\def \ss[#1] {\subsection*{#1}}
\def \sss[#1] {\subsubsection*{#1}}

\s[Zibaldone di pensieri]
Zibaldone \t era un piatto con molti ingredienti mischiati \t qua usato metaforicamente per indicare
Vita di un anima in senso parcellizzato \t nel qui e ora \\
Zibaldone di pensieri \t è un autobiografia con tutte le crisi della modernita \\
Legato a "La coscienza di zeno" \t Zeno Cosini \\
Modernita anche di seneca \t verbalizzazione è fondamentale anche il lui, e anche sant'agostino \\
Parte da petrarca con consapevolezza del suo dualismo interiore \\
Leopardi \t quest opera è il magazzino che si è radicato nel tempo, e non ha avuto sistemazione organica \t non sono articolati perche spontanei \\
La struttura è un magna \t e c'è la logica nella illogicità \t infatti i pensieri non sono sequenziali, ma un flusso \\
Elemento di grande modernita \t per Freud ma anche per Bergson, che riflette sulla sequenzialita del tempo

\v

Non ci sono date interne se non quelle poste dall'autore \\
C'è interruzione nel dicembre del 17 \t da quella data non compone, mentre dal 22 \t abbandona definitivamente \\
Abbandona non lo spirito costruttivo che pervade la sua mente \\
Rimangono dello zibaldone alcuni difetti \\
Ci sono dei momenti tratteggiati che sono definiti dalle emozioni, e si colgono attraverso la parabola che ha momenti di dolore + o - forti \\
Ranieri modifica \t ma anche giordani \\
Lettera al padre \t "Complesso di Telemaco" di Recalcati \t analisi del rapporto padre/figlio \t riflette anche Far Danson?? \\
Leopardi \t "Il giardino del suffrance" \t e teoria del piacere \\
Punto in cui viene descritta una mancanza di controllo \t inusuale per la tempra dell'autore \\
Nello stile compositivo ed etico dell'epoca di cedere al fuoco della disperazione \\
Passaggio dal determinismo Comptiano al decadentismo \t esprime l'aspetto + irrazionale dell'uomo \\
Le scienze naturali sono costrette da un continuo movimento \t anche essere umano non puo rimanere costatne \\
Filone conduttore che salva l'opera (come anche nel "Mestiere di vivere") \t ovvero il porre al centro elementi costitutivi dell'essere umano, come uomo nelle sue impossibilita, etc
\end{document}
