\documentclass[12pt]{article}

\usepackage[a4paper, total={6in, 8in}]{geometry}
\usepackage{textcomp}

\begin{document}
\setlength{\parindent}{0pt}

\def \t {\textrightarrow}
\def \v {\vspace{1em}}
\def \bi {\begin{itemize}}
\def \ei {\end{itemize}}
\def \s[#1] {\section*{#1}}
\def \ss[#1] {\subsection*{#1}}
\def \sss[#1] {\subsubsection*{#1}}

\s[Pascoli - La mia sera]
Testo fa riflettere sui capisaldi della poesia pascoliana \t forte contenuto di figure retoriche, e componente delle sfere sensoriali \t inoltre tono cantilenante, che riporta alla poetica del fanciullino \\
Poesia frammentaria \\
Il testo si correla con il paradigma della luna e della notte di Leopardi \\
Fusione tra mondo naturale/animale e mondo umano \t vissuto dell'autore e dimensione ad ampio specchio che caratterizza la natura pascoliana \\
In pace la sera, con venatura di dolore e di sofferenza del distacco, di un nido infranto \t la cena ha comportato una porzione ridotta per tutti \t viene ridotto \\
Viene conglobato nell'immagine di nido infranto, interroto, nella sua esperienza autobiografica \\
Termine sera, qua viene correalto da aggettivo mia \t mostra affinita con riflessione profonda che subentra alla fine del giorno \t la morte del padre avviene alla fine di quel tragico giorno, che ha sconvolto la famiglia \\
Nella lirica non è fuori luogo l'incidenza che ha l'esclamazione \t presente in diversi versi \t l'esclamazione è propria di chi si stupisce di tutto, e lo stupirsi è tipico del fanciullino

\v

Il giorno e la sera si confrontano di continuo, in un dualismo \t tacite stelle è sinestesia \\
Gregre ranelle \t onomatopea e ripresa della lettera R, che produce suono duro \\
E anche le tremule foglie dei pioppi \t di solito i pioppi sono gli alberi che conducono ai cimiteri \t non c'è evocazione diretta della morte, ma gli oggetti a cui si riferiscono \\
Tremule ricorda alla sera di leopardi \t tremula immagine della luna \t e ricorda una delicata sfumatura che non scuote le cime, ma le foglie \\
Trascorre e leggera \t la pensatezza evocata è nella R in parole che indicano quasi evanescenza \t la R esprime la respingenza, la pesantezza \\
Gioia leggera \t sinestesia \\
Espressioni nominali nei versi iniziali \t non ci sono verbi, e si procede per immagini

\v

Le stelle appaiono, non si aprono \t utilizzo verbale simile a una sinestesia \\
Cielo vivo è anch'esso una sinestesia \\
Allegre ranelle \t anche qua la R è insistente \t ranelle lo ripete anche dal verso 4 \t queste parelle urticanti con la R precedono Montale \\
\textbf{singhiozza monotono un rivo.} \t sonorita molto forte, e anticipa il male di vivere di Montale \t il rivo singhiozzato che gorgoglia ?? \t ed evoca i poeti fiamminghi, che descrivono paesaggi secchi, pozze acquitrinoso, pioggerellina domenicale \\
Immagine che mostra pascoli proiettato nella dimensione europea, sprovincializzante \\
Riferimento montaliano \t il male di vivere \\
Non rimane che il dolce singulto \t sinestesia e ossimoro \t il singhiozzo non è dolce \t anche umida sera è una sinestesia \\
Immagine continua che rimanda all'infinita tempesta \t e subito dopo finita \t paradosso e anafora correttiva \t la tempesta ha forza colossale che finisce in un rivolo \\
Rivo canoro si allinea alle Stirpi Canore di D'Annunzio \t e rivo canoro è una sinestesia \\
Fulimini fragile è una sinestesia \t non un'antitesi, perche si deve avere "non questo ma quello" \\
\textbf{cirri di porpora e d'oro} \t colore delle nubi al tramonto \t colore che è anche quello delle foglie d'autunno \\
Stanco dolore \t sin. \t poi esclamazione \\
Nube verso 22 \t anastrofe \t anche il dolore + grande letto nella sera diventa roseo \t grazie al distacco della sera \\
E la prima persona \t chiave biografica \\
"Ultima sera" \t punto molto discusso, che evoca la morte del padre, ucciso di sera

\v

Voli di rondini, e la fame del povero giorno \t garula insieme a cirri = Pascoli specifista e classicheggiante \t garulo = loquace \\
Sociologicamente Pascoli da + attenzione ai ceti + bassi \t compartecipazioni a movimenti filo-socialisti, che poi lo portano anche all'incarcerazione ed estromissione dall'universita \\
Prospettiva per la famiglia \t nido interrotto, che non ha le stesse potenzialita di vita \\
Aspetti biografici ritornano \t prima persona e sempre evocazione del nido, la morte è sempre pesente, la poverta che la famiglia ha dovuto sopportare \\
Ne io \t di nuovo evocazione prima persona \t il giorno si conclude come pace totale nella sera \t e non si configura in uno spirito bellicoso \t "dorme lo spirito guerriero ch'entro mi rugge" \\
Testo propone immagine talvota spensierata, ma minata dall'ombra della morte \t che si traduce nella conseguenza della fame della famiglia \\
Il mondo animale è ancora presente \t rondini offrono possibilita di riflessione maggiore \t le rondini tornano in primavera \\
Il passero \t l'assiuolo \t il gabbiano \t ora la rondine

\s[Speranze e memorie - paranzelle in alto mare]
Aspetti biografici, vicinanza con natura, immagine dell'immensita funestata sempre dalla morte, prima persona e tono cantilenante \\
Perche il titolo è speranze e memorie? \t si riconduce a Leopardi, anche per il mare e le memorie \\
Onomatopee: palpitare \\
Elemento cantilenante: bianche bianche, nere nere \\
Equilibrio tra spazi bianchi e parola scritta \t testo molto frammentato, ma rinnovato equilibrio tra spazi bianchi e parola scritta \\
Di nuovo potere dell'esclamazione che riporta al fanciullino \\
Punteggiatura è degna di rilievo \t alternarsi tra punteggia debole e enjambement nella prima strofa, quidi puneggiatura frammenta \\
Inoltre dopo ! c'è spazio bianco \\
Ali di sogni \t come se avessero le ali \\
La metrica si sta disperdendo ? \t carducci, in "Odi barbare", vuole portare nella lingua comune la metrica classica \\
Ombra \t gia evoca qualcosa di evanescente \t non afferrabile \t i sogni infatti non sono una realta \\
Il sogno è democratizzato ma ha perso la sua qualita premonitrice \\
I sogni sono del futuro, mentre le ombre sono qualcosa di proiettato nel passato \\
Serrata razionalita, un mondo altro \t meccanicismo \\
Specifismo delle paranzze (verso 1 e 9) \\
Le speranze di azzerare un dolore e le memorie insistenti di quel dolore, l'andare avanti ma l'essere sempre tirati indietro \\
Mentre alza gli occhi al cielo, crede che siano nere nere \t nel cielo c'è la morte, il padre è nel cielo \\
Le paranzelle sono barche da pesca \t si perdono nell orizzonte, come la petroliera nella casa dei doganieri di Montale
\end{document}
