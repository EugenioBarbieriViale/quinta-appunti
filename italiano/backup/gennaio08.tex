\documentclass[12pt]{article}

\usepackage[a4paper, total={6in, 8in}]{geometry}
\usepackage{textcomp}

\begin{document}
\setlength{\parindent}{0pt}

\def \t {\textrightarrow}
\def \v {\vspace{1em}}
\def \bi {\begin{itemize}}
\def \ei {\end{itemize}}
\def \s[#1] {\section*{#1}}
\def \ss[#1] {\subsection*{#1}}
\def \sss[#1] {\subsubsection*{#1}}

\s[La scapigliatura - letteratura dell'Italia unita]
Anche detta letteratura della scapigliatura \t nasce dopo l'unita d'italia, che rappresenta una delusione \\
La scapigliatura si manifesta a milano \t manzoni era il prototipo del letterato pro unita d'italia e indipendenza dallo straniero \\
Nei promessi sposi, che sono il punto di partenza per il blocco di letteratura post-risorgimentale (e dei manzonisti regionali), con la scapigliatura c'è lo spartiacque \\
Sara fondamentale per il decadentismo, e anche per il positivismo \t scapigliatura è anche il varco per la letteratura d'oltr'Alpe \\
Era presente una contestazione giovanile, e la contestazione della scapigliatura è valoriale \t anche dei poeti maledetti, con Boudelaire \t le flors du mal (1857) \t ossimiro che accosta i fiori con il male \\
Snaturasi dell'idea del fiore, che è freschezza, nuova generazione, vita \t anche Leopardi lo usa molto come metafora per la vita \t e anche in Carducci \\
Qua invece i fiori si accostano al male \\
Per boudelaire, ogni aspetto del reale presenta un atomo di male, e anche nell essere umano \t che è angelo e diavolo \\
L'apparato valoriale si colloca in una selva di simboli, provocatori e repellenti nei confronti del bello, buono e vero manzoniani \\
Boudelaire e la letteratura francese sono paralleli a una visione della vita come una selva di simboli \t questa letteratura non puo essere interpretata alla lettera, ma è evocativa \\
La letteratura francese è il paradigma di un mondo, nel quale si avverte una crisi di valori \\
La scapigliatura milanese diventa il paradigma dei valori che diventano antivalori \\
Erano tutti giovani ed avevano una vita dissoluta, per la loro fiducia nella provocazione \\
Bohemien \t è la scapigliatura francese \\
Anche Verga ha una fase scapigliata \t la scapigliatura richiede il vero, che si traduce nel turpe, malato, deforme, tutto cio che prima era rifiuto \\
Quindi amplia i confini del reale ed evade il perbenismo manzoniano \t e sprovincializza la cultura italiana, aprendosi alla lett. oltr'alpe

\v

Scapigliatura non è una scuola, ma un movimento \t mentre i francesi hanno intenti comuni, ma ognuno ha la propria unicita (Pau Verlaine, Boudelaire, ...) \t condividono lo stile di vita \t stavano sull'Arigos a Parigi ? \\
L'Annconnui, l'inconoscibile, dimensione del mistero, dove il poeta (come dice Luciano Angeschi) ha perso l'aureola \t non sono + santi, sono umani \t il poeta è colui che legge la societa malata \\
Boudelaire trasmette nella natura la volonta di un contatto di anima, e fa delle corrispondenze dei singoli e delle allegorie \\
Attraversa le apparenze e si serve del potere dell'immaginazione \t ricorda Leopardi \\
Il poeta inizia quindi "l'avventura del poeta" \\
Divorizio totale da manzoni e dai valori

\ss[Corrispondenze]
I simboli rendono le corrispondenze di ? \\
Percepire le corrispondenze coincide con vivere con i 5 sensi \t profondamente seduttivo, le essenze caratterizzano la seduzione \\
L'ambra come principio olfattivo \t l'ambra esprime sensualita \\
Il muschio, e benzoino \t sfumature calde di potere seduttivo \t e questo permette di capire che le corrispondenze sono a pelle \t è l'anima che sente \\
Sono affini alle affinità elettive? si \\
La percezione attraverso i sensi \t è sensuale, la natura è l'emblema di queste corrispondenze che diventano simboli \\
Le parole sono confuse \t spesso la parola è muta e non riesce ad esprimere in toto \\
Tenebrosa, oscura, inafferrabile \t il sovrasenso di questi sensi non è immediato, ma frutto di una ricostruzione tecnica e mirata \\
Tenbroista pregna di senso \t espressa al meglio alla notte e dal suo chiarore \t duplicita che si oppone \\
Insiste sull'olfatto \t il profumo ha una dimensione intima, che esprime pero un messagio sensualistico (perche percepibile) \\
Associa alle sensazioni olfattive quelle uditive e visive \t e gli altri sono corrotti \\
L'ambra, l'incenso, il censoino (usati forse anche da Cleopatra), sono corrotti \t perche esprimono il completo coinvolgimento dei sensi \\
Questo non sarebbe mai potuto essere concepito da Manzoni \\
Simboli si esprimono in una dinamica di corrispondenze \t allegoria diverso da simbolo, simbolo incarna qualcosa
\end{document}
