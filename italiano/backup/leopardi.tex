\documentclass[12pt]{article}

\usepackage[a4paper, total={6in, 8in}]{geometry}
\usepackage{textcomp}

\begin{document}
\setlength{\parindent}{0pt}

\def \t {\textrightarrow}
\def \v {\vspace{1em}}
\def \bi {\begin{itemize}}
\def \ei {\end{itemize}}
\def \s[#1] {\section*{#1}}
\def \ss[#1] {\subsection*{#1}}
\def \sss[#1] {\subsubsection*{#1}}

\s[Giacomo Leopardi]
Distanza temporale grande \t ma l'anima incontra quella di ognuno \t ambito topografico circoscritto, è riuscito a parlare il linguaggio di un romanticismo europeo, senza spostartsi dalla penisola \\
È più rivolto all'interiorita, è introflesso \t introflessione è importante \t questo chiama a ase un altra considerazione \\
Quella che si apre con la domanda: fu poeta-filosofo o filosofo-poeta? \\
Profonda analisi del cuore e dei sentimenti \\
Autore si pone domande esistenziali \t che riguardano lo stare dell uomo nella realta 

\ss[Vita]
Leopardi nasce in un contesto difficile \t a recanati, piccolo borgo \t ha pero il pregio di affacciarsi sul mare ed essere vicino alle montagne \\
Presenta pero limitazioni politiche \t e presenta tradizioni conservatrici, retrogrado dal punto di vista della mentalita \\
Il conte Monaldo \t è il padre \t ha dei possedimenti terrieri e vive normalmente, moglie ?? \t difende i principi conservatori \\
Famiglia è poco propensa a una prospettiva di innovazione dei figli \\
Però il padre era protettivo dei figli \t in particolare di giacomo \t che era esile e aveva problemi di salute fin da piccolo \\
Madre ha atteggiamento di protezione e ha relazione fredda con lui

\v

Presto sfrutta la biblioteca paterna \t legge molto \\
Legato alla figura dei fratelli, ma soprattutto di Paolina \t è colei che lo protegge e supporta \t viene anche aiutato dal fratello Carlo \\
Però l'esclusione non viene dall'esterno \t viene dalla situazione stessa \t non sono gli altri che lo escludono, ma il suo stato di salute \\
Il prodotto della sua letteratuta \t non viene influenzato dalla sua salute \\
Poi si scontra con De Sinner \t lui lo accusa di essere in una fase della sua produzione troppo vittimista \t leopardi risponde che lui tratta il dolore dell'umanita, non il suo \\
Non è la circostanza, ma la sostanza

\v

La prima fase \t dagli 1 ai 9 anni \t fase dell'erudizione (primo periodo leopardiano) \t conosce il greco e il latino, l' arabo antico \t vive questa fase con una voracità di lettura \\
Legge i volumi della biblioteca paterna \t traduce a vista classici latini e greci \\
Rimane pero sempre rinchiuso in casa \t non comunica molto con i coetanei \t per lui leggere era varcare i limiti spaziali \\
La sua vita non scorre nell'assenza di contatti \t ha contatti amorevoli nella famiglia \t che pero pone dei divieti, e viene sostenuto dai fratelli \\
La famiglia è comunque anaffettiva \\
Paolina è importante \t poi va a Napoli, da Ranieri, che ha una figlia?? che lo accudira fino alla sua morte e si chiama anche lei Paolina 

\v

Poi fase della scoperta del bello \t conversione estetica \t la composizione, poesia, creativita \t dall'erudito al bello \\
Il bello non basta pero \t l'uomo innegabilmente si scontra contro qualcosa che non puo dominare \t fase successiva descrive la sua figura in toto 

\v

Fase della conversione al vero \t conversione filosofica ??? \\
Fase dell erudizione arriva fino al 1815 \t poi fase della creativita \t dall erudizione al bello \\
Loepoardi studia i classici, i contemporanei \\
Pensiero leopardiano è dinamico \t fa riflessione, che si concretizza nel "Discorso sulla felicita degli antichi" \\
Felicita degli antichi è una chimera da raggiungere \t essa consente di costuire sull'immaginazione e speranza \t gli antichi scrivevano favole di fantasia e mettevano al centro il piacere \\
Teoria del piacere \t viene identificata nella ricerca quotidiana ai bisogni \t cerca una risposta, una soddisfazione dei bisogni \\
In questa ricerca se non c'è appagamento, arriva il dolore \t spesso si trova quello che non si cerca \t serendipity \\
Leopardi indaga sul bisgno dell uomo di vedere soddisfatte le necessita primarie \t elabora teoria con al centro la delusione dell uomo moderno e la felicita dell antico, che viveva di favole \\
Stato di grazia non dura in eterno \t continuo alternarsi di piacere e di dolore \t aspetto fondamentale, piacere e il dolore (come costitutivo dell essere umano) e piacere come intervallo tra due dolori \\
A recanati respira calma, ma anche sofferenza \t neanche il paesaggio è di conforto: anche la natura risente di questo ondeggiare tra piacere e dolore \\
Schopenhauer è collegato a questa visione \t il piacere si insedia nella vita umana ma viene surclassato subito dal dolore \t il piacere è come un momento di attesa per il dolore \\
Nuovo piacere richiede una nuova costruzione \t che fa sperimentare la difficolta del destino dell uomo \t sempre correlazionato che ci compare complice del nostro dolore ?? \\
Leopardi sperimenta un dolore del suo tempo \t l uomo non vive secondo le favole degli antichi, non vive di illusione \\
L'uomo del suo tempo è pessimista \t non riesce a ricreare la fantasia dell'illusione \\
L'uomo del tempo non riesce ad essere felice \t è il suo tempo, la sua storia ?? \\
Prima fase è pessimismo personale, ma poi rappresenta il pessimismo storico \t capisce che uomo non si puo agganciare a illusioni \\
Terza fase \t dal bello leopardi passa al vero \t la terza fase viene chiamata conversione filosofica, dove si trova la delusione + amara \t convizione che tutto sia male, e che l'uomo è figliastro della natura \\
Gli uomini vivono una condizione di continua minorita nei confronti della natura \t natura cortese \\
Gia dalla nascita l uomo si approccia al dolore \t con lo stacco dal corpo della madre \t primo momento di abbandono \t capisce di essere solo \\
Momento in cui leopardi si avvicina alla speculazione filosofica \\
In realta anche nelle fasi precedenti la speculaizoine filosofica con le domande esistenziali era gia presente \\
Ma in questa fase filosofica si dedica di + alla prosa \t "Operette morali" \\
Lo Zibaldone accompagna tutta la sua vita \t accompagna la riflessione filosofica alla composizione

\v

Fase dell erudizione, fase estetica, fase filosofica \t con 3 tipi di pessimismo \t nella terza fase dal pensiero di un pessimismo storico (l'uomo vive infelice perche non sa piu illudersi) capisce che il dolore non è solo del suo tempo \\
"Il giardino del dolore" \t nello zibaldone \t dice di entrare in un giardino e viene rapito dalla bellezza dei fiori e dai colori \\
Poi si addentra e piu si avvicina piu vede che alcuni sono gia putrefatti, altri hanno i petali sgualciti \t in nessun luogo c'è la felicita totale \t niente è esente dal dolore, che fa parte dell uomo \\
Nella terza fase approda al pessimismo cosmico \t la natura è matrigna, tutto è male \t con altre due riflessioni \\
Delusione totale della vita \t con Fannì fallisce la redazione?? \t la donna che amava aveva trovato in lui in interiorita profonda, anche se era umile
A Fannì attraverso Ranieri fa arrivare il \textbf{garbato rifiuto} \t declina l invito di una stabilita con delicatezza e garbo \\
Si capisce dall opera che ranieri pubblica: "Sette anni di sodalizio poetico con Giacomo Leopardi" \t nel 70 \\
Sodalizio \t poeti si identificano nell'anima e nel vissuto \t amicizia molto intensa \\
Teoria che dice che Ranieri sfrutta Leopardi per farsi conoscersi \t ma in realta no \t perche ?? \\
Ranieri è rispettoso dell amico \t ma non sempre, per esempio Ranieri è stato amante di Fanni Tozzetti \\
Testo viene diffuso tra chi voleva dissacrare Leopardi e esaltare Ranieri \\
Ma questo testo è molto provocatorio, e letto da chi era interessato in che relazioni Leopardi stava con Ranieri \t viene messa in luce le crisi respiratorie di Leopardi \\
Tra i due c'era però una comunicazione molto intensa \t e lo stato di salute di Leopardi viene esorcizzato attraverso la sua produzione letteraria

\v

Vero \t corrisponde alla terza fase \\
Non solo le operette morali \t che sono pregne di contenuto e di provocazioni \\
Ciclo di Aspasia \\
Quest'opera esce inoltre dopo la morte dell'autore  \\
Abbraccia tutta la sua vita \t e racconta la storia di un anima, non logicamente sequienzale \t lo zibaldone non segue un iter logico \\
È scritto dinamico e non statico \\
Correlata a un pessimismo storico \t chiama in causa \t il suo secolo riflette la falla + grande: l'impossibilta di credere alle illusioni e di concentrarsi sulla realta \\
Vero dramma è quando non si ha un respiro dal dolore \t il vero nega la possibilta dell uomo di vedere oltre \\
Si rende conto che tutto cio in cui aveva sperato è fallace \t tutto cio che costituisce la dinamica essenziale e deterministica, di illusioni e ricordi \\
Soggetta a un nulla ontologico \t se il piacere sta nell illusione o nel ricordo \t io mi rendo conto che guardare la realta mi reca dolore \t dove mi posso ancora illudere, conosco la possibilita di un minimo di piacere \\
La dove posso ricordare conosco uno scampolo di piacere \t se risiede dove non sta piu, il piacere non ha sostanza ontologica \t è una astrazione \\
Addirittura approda Leopardi in questa constatazionie \t nel momento in cui non vivo il dolore, non è piacere, ma un vaccum (vuoto) \t una negazione che si puo esprimere con "taedium vitae" di cui parla seneca \t esprime la mancanza di possibilita di avere un senso, il nulla \\
Leo anticipa i poeti maledetti \t parlano di un tedio che uccide l'anima, caratterizzato da un colore senza definizione propria \t non ha identita precisa, è il grigio plumbeo \\
Leopardi prota a superare cio che i maledetti hanno semplicemente denunciato successivamente \\
Leopardi muore di una crisi respiratoria mentre contempla il Vesuvio \\
La terra guardata dal cielo sembra un punto \t si traduce nel 900 nel relativismo \t la realta è la stessa ma la guardiamo da un punto di vista diverso \\
Dialettica delle illusioni \t di nulla valenza ontologica \t descritta come nell eta giovanile la memoria "ha breve il corso", mentre piu avanti contrario \\
Infanzia ed adoloscenza si costruiscono sulle illusioni \t mentre l eta adulta il rimbembrar del ricordo
\end{document}
