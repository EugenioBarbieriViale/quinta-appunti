\documentclass[12pt]{article}

\usepackage[a4paper, total={6in, 8in}]{geometry}
\usepackage{textcomp}

\begin{document}
\setlength{\parindent}{0pt}

\def \t {\textrightarrow}
\def \v {\vspace{1em}}
\def \bi {\begin{itemize}}
\def \ei {\end{itemize}}
\def \s[#1] {\section*{#1}}
\def \ss[#1] {\subsection*{#1}}
\def \sss[#1] {\subsubsection*{#1}}

\s[Verga]
I malavoglia \t romanzo arriva dopo i promessi sposi \t "Fede e bellezza" di tommaseo

\ss[Mastro don Gesualdo]
Per la prima volta esce un romanzo che ha dietro un lavoro radicato in diverse fasi evolutive nell'arte dell'autore, e nel suo essere passato in diverse realta \\
È un paradosso: conosce Milano, firenze, roma, ma poi scrive di un ambiente perimetrato e limitato \t la sua scelta si manifesta nella difesa dei confini, al di fuori del quale i personaggi si perdono \\
C'è un sistema di personaggi che permette all'autore di parlare, come se l'opera fosse fatta da se \t come se fosse impersonale \\
L'autore apre gli orizzonti di un bello che era anche bello e buono \\
Usa discorso diretto libero \t e una sicilianita che Sciacia connota?? \\
E forte connivenza tra le novelle e i romanzi \t l'autore è come se portasse avanti un progetto, ma poi capisce l'inefficacia di questo progetto \\
Anch'egli è vinto da se stesso \t i suoi personaggi hanno la malattia della poverta, del non riconoscersi in un ambiente sociale che mangia i + deboli \\
Non riconoscersi significa non appoggiare questo meccanimso \\
Il secondo vinto è mastro don gesualdo \t che caratterizza una svolta del ciclo dei vinti \\
Il titolo presenta gia la duplicita \t esce nel 1889 \\
Se si vive senza un preciso riferimento identificativo, non si è nessuno \t qua si capisce, ma c'è anche un altra transazione \\
Lui constata di non poter essere don, e si attacca a cio che lo fa sentire don: i soldi \\
Lui è il primo inetto \t vinto dalla sorte, dalla storia, dal sistema, da se stesso \t perche non potra mai essere mastro \\
Viene considerato sempre mastro \t si è scollato dalle sue radici, e vive la condizione di vinto, inetto \t non perche non agisca per dominare la realta, ma perche è incapace di vivere la sua realta \\
Scambia i rapporti falsati per quelli autentici \t viene tradito dal sistema e dalla famiglia \t cerca di salire l'ascensore sociale sposando una nobile decaduta \t bianca ?? \t ma aveva solo il titolo e non aveva soldi \\
Si ritrova cosi deluso, frustrato, non puo salire la scala sociale e non puo vivere il riscatto economico sperato \\
Nell "Roba" \t Mazzaro dice "roba mia vieni con me" \t come se il possedere superare l'essere, cosa che in mastro don ges. il possedere sostituisce l'essere \\
Mazzaro e Don gesualdo sono costretti a continuare ad essere cio che erano \t ma si sforzano ad essere altro cio che non sono \\
La novella "Roba" non fa parte del ciclo dei vinti, ma di fatto mostra un vinto (mazzaro) \t è attaccato al denaro come l'ostrica allo scoglio \\
Don gesualdo è invece egoista nell'avere e nel possedere, attraverso il quale vuole cambiare lo stato e la sua posizione sociale \t ma rimarra sempre mastro \\
Ci sono pero passaggi anche in cui mastro fa donazioni ai contadini del paese, per raggiungere una considerazione da parte del paese 

\v

Don gesualo è solitamente considerato solo e avido di denaro, ma non solo \t ha volonta di donare ai paesani, per cui egli non dona per riceve dei favori, ma vuole essere riconsociuto, indentificato e non schernito \t mentre rosso malpelo viene deristo e isolato \\
Il suo voler dare è quello che si confronta con le persone che non si risparmiano nulla nel lavoro, per avere una considerazione piena del loro operato \\
Quadrilatero amoroso, si tradiscono ed è la celebrazione di un antimanzonismo, moderato e circoscritto allo stile siculo, ma utile a far capire che lo spartiacque 800/900 è gia qui \\
Non si riesce a realizzare, come a partire da Federico Tozzi si chiamano inetti, poiche sono incapaci a vivere \\
Nel titolo è gia presente una dicotomia \t e di un uomo che ha un esperienza campestre, che aspira a diventare don ma non riesce \\
L'opera si muove tra il 20/21 e il 48/49 tra catania e parlermo \t ci sono carboneria e moti di rivoluzione, e l'epidemia di colera del 37 \\
Parallela alla peste di manzoni \\
Sono 21 capitoli \t e ci sono diversi bozzetti = pittura campestre rurale, usata nelle liriche pascoliane che sono vicine a Segantini \\
C'è filo conduttore che mostra conme il criterio dell'impersonalita e la volonta di descrivere cio che ci arriva come in uno schermo \t la fotografia era la novita del scolo anche, e la volonta di fissare qualcosa nei suoi contorni coincide nell'impersonalità \\
Altra parola chiave è interesse \t i personaggi si muovono sia che siano nobili, popolani \t e lo fanno gli animali \\
Descrizione del cercare per avere, e quindi il criterio dell'utile \t e le dinamiche borghesia, ceti, classi \t Marx \\
Si vede ribellione della classe sociale umile, che si configura anche nell'introduzione di nuovi srrumenti di lavor \t e il male della poverta è una malattia di queste popolazioni \\
C'è un distacco tra il momento compositivo e quello narrato \t quindi puo essere un romanzo storico \t ma non puo essere considerato completmamente storico
\end{document}
