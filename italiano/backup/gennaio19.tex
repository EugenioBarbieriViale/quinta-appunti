\documentclass[12pt]{article}

\usepackage[a4paper, total={6in, 8in}]{geometry}
\usepackage{textcomp}

\begin{document}
\setlength{\parindent}{0pt}

\def \t {\textrightarrow}
\def \v {\vspace{1em}}
\def \bi {\begin{itemize}}
\def \ei {\end{itemize}}
\def \s[#1] {\section*{#1}}
\def \ss[#1] {\subsection*{#1}}
\def \sss[#1] {\subsubsection*{#1}}

\s[Verga]
Testi saranno: come novelle: "La lupa", "Rosso mal pelo" \\
Verga è appartenuto alla Scapigliatura \\
Tratta l'analfabetismo, il lavoro minorile \\
La nascita della sociologia, l'amore per l'osservazione, gli umili che verranno chiamati vinti \t orizzonte limitato \\
Vengono fotografati ceti sociali prevaricanti, contrapposti all'umila plebe, che viene ritratta da Verga \t che sceglie di fotografare gli umili del sud \\
Non abbiamo in realta non equivalente rappresentazione nel nord \t non c'è la stessa pregnanza della Brianza di Manzoni \\
Verga risente al dopo dell unita d'italia \t chiamata la fase degli anni 80 \\
Federio de Roberto (autore dei vicere) commenta Verga \t è uno dei pochi che lo frequentano e vive da vicino \\
Parla di un Verga riservato, schivo, che sente la responsabilita forte di sanare lo stato \t fortemente limitato a catania \\
Fu anche drammaturgo e compositore teatrale \t e fu attivo nella milano della scapigliatura \\
Resta condizionato dalla scapigliatura, senza la quale non avrebbe potuto tratteggiare la parabola dei vinti \\
La sua penna non faceva sconti \t che invece i post manzoniani facevano \t nei confronti della classe prevaricante

\v

Giovanni verga nasce a catania nel 40 \t da una famiglia modesta, antiborbonica e liberale \t antibaronato e antinobilta, contro i privilegi ereditati dalla nobilta \\
Lui vive le vicende del tempo \t come i moti del 48, a cui assiste \\
Inizia la carriera universitaria e studia giursprudenza \t ma aveva tanti interessi \t studia il diritto, da cui impara l'amore del rispetto dei diritti di tutti \\
Inoltre aveva una passione per Victor Hugo, Alexandres Dumas 
Prima va a roma, poi a firenze interrompe gli studi \t sente l'eco di ribellione e dell'unficazione dell'italia \t partecipa cosi con distanza al progetto dell unita \\
La scuola italiana si deve uniformare alla progettualita di formare \\
La realta italiana era molto frammentata e diversa \t fondamentale è il suo arrivo a milano, dopo un soggiorno a Firenze (dove conosce Luigi Capuana) \\
Nel 66 pubblica "Una peccatrice" e "Storia di una capinera" \\
Nel 72 arriva a milano \t che si presenta come un ambiente pieno di intellettuali, e qui conosce Salvatore Farina \\
Viene a contatto con questo mondo, e rimane fino al 93 \t Milano ha quello che non ha ne il sud ne firenze \\
Questo è l'unico luogo dove la promiscuità è condivisoine, dialogo, curiosita per un mondo nuovo \t nella cerchia degli Scapigliati \\
La sua cultura era formata su una fomrazione molto campestre \t si forma a Catania \t parlera di Milano in "per le vie"

\v


Compone:
\bi
    \item "Eva" nel 73
    \item "tigre reale" e "Eros" nel 75
\ei
Sono romanzi scapigliati, che trattano donne condannate a darsi alla prostituzione, controllate da un potetne, o che si lasciano prendere dalla passione \t passione deleteria: "amor condusse noi a una morte" \\
Collabora con "Scienza e arti" \t rivista

\ss[Nedda]
\textbf{La svolta è "Nedda", che è una novella \t è vista come la conversione al verismo} \\
È una novella e la protigonista è una ragazza, che raccoglie le olive \t rappresenta la sicilia rurale e abbandonata a se stessa \\
Rappresenta il tratto distintivo per cui l'autore, nel 80-81, fara uscire il romanzo che lo rende famoso: "I malavoglia" \\
Nedda è anche orfana, e si cimenta con orari durissimi a lavora \t sfruttamento femminile e minorile, lavoro terribile, orari lunghissimi \\
In questo di innamora di Janu, e rimane incinta \\
Durante la gravidanza, sulla scala per raccoglierre le olive, cade e la gravidanza non è compromessa \t pero la salute del bambino si \\
Poi pero Janu muore e lei si trova sola, con questo lavoro che ormai è diventato impossibile \t e il figlio nasce compromesso \\
La sua adolescenza è quindi oppressa, ed è isolata

\v

La progettualita di Verga inizia adesso \t perche qua si matura l'idea del personaggio marinaresco (Padron Toni), che diventera il reale romanzo dei "Malavoglia" \\
Ultimo ventennio dell 800 ha molti romanzi diversi \\
I malavoglia si candida per essere il primo romanzo che fa parte di un progetto \t il "Ciclo dei vinti" \t la parola chiave è progettualità \\
I malavoglia fa parte di un progetto, di connessione con Compte e la causa/effetto, determinismo \\
Nel ciclio i romanzi sono:
\bi
    \item i malavoglia \t vinti sono chi appartengono ai ceti umili (operai, agricoltori, raccoglitori, pescatori)
    \item mastro don gesualdo \t vinto è colui che è mastro, ma vuole diventare don \t vuole scalare la scala sociale
    \item parte da mastro don gesualdo, perche tra i personaggi del romanzo c'è un personaggio che da il titolo al terzo, che rimarra in completo e non lo finisce \t Verga non ha piu la motivazione per finirlo \t ovvero "La ducessa di Leyra", che dara il via per quello successivo \t vinta è la nobilta, perche la ducessa vive con un patrimonio assente e mantiene solo il titolo nobiliare
    \item "L'onorevole Scipioni" \t anche lui è un progetto, che inizia quello successivo \t vinti sono nella politica, se vuole essere onorevole deve scendere a patti con la corruzione
    \item "L'uomo di lusso" \t vita passionale e di artista, che vuole realizzare la sua eccentricita \t il suo essere quasi un Andreas Perelli, protagonista \t l'artista voleva vivere nel lusso e nella passione
\ei
\end{document}
