\documentclass[12pt]{article}

\usepackage[a4paper, total={6in, 8in}]{geometry}
\usepackage{textcomp}

\begin{document}
\setlength{\parindent}{0pt}

\def \t {\textrightarrow}
\def \v {\vspace{1em}}
\def \bi {\begin{itemize}}
\def \ei {\end{itemize}}
\def \s[#1] {\section*{#1}}
\def \ss[#1] {\subsection*{#1}}
\def \sss[#1] {\subsubsection*{#1}}

\s[Paradiso]
Come l'inferno, è la cantica + nota \\
Idea di un mondo perfetto \\
"Paradise lost" \t Milton \t idea di un mondo edelico, nel quale le divergenze si ricompongono \\
Si raggiunge nell'osservazione dei principi cristiani \\
Nel mondo crisitiano parlare del paradiso significa parlare di anime salve \t non si sono pentite all'ultimo, ma hanno fatto una costruzione nel corso della vita \\
Ci sono le figure speculari della chiesa \t come i padri della chiesa \t e Beatrice, capace di amare al di sopra di ogni limite e per la quale dante ha scelto la via della salvezza \\
È la cantica che ciude il cerchio \t rasseregna l'animo \t la luce accompagnata dalla preghiera e dalla coralita del purgatorio, che era riblatato risp. all'infenro (montagna vs voragine) \\
In paradiso c'è universo etereo che non è tangibile \t dante prende in esame attraverso varie fonti \t e "Il somnium scipionis" \\
Dante consulta le fonti sacre \t la bibbia, i salmi, il vangelo \t il vangelo perchè parla di un umanita salvata \\
Ma qual è il testo che parla di + del paradiso? \t il somnium scipionis \t l'ultimo libro del de repubblica, pubblicato autonomanente \\
Movimenti celesti, un universo immenso rispetto all'occhio umano \\
Per la prima volta nella lett. latina \t un oratore filosofo parla di un argomento scientifico/filosofico \t parla dei moti celesti, e di uno stretto rapporto tra movimento e suono \t principio petagorico \\
In cicerone è presente come una summa \t che Dante prende in esame \\
Dante analizza (come petrarca) i tesi antichi \t le opere del medioevo si servono dell'apparato classico per agganciarvi contenuti nuovi \\
Per esempio Virgilio \t Dante lo sceglie per l'egogla (nona, delle Bucoliche) che parlava della nascita di un bambino \t che viene strumentalizzata in ambito cristiano 

\v

La numerologia delle cantiche \t delle terzine \t cosa rimane? \t rimane tutto l'apparto che fa si che Dante sia autor e actor \\
Cantica porta a una difficolta maggiore nella lettura \t viene definito da Erich Auerbach, da Dante Isella, Leo Spitzer e studi recenti \t quest'opera è la + carica di senso e ricca dal punto di vista dottrinale \\
Dante pero non risulta del tutto cambiato \t parla di un processo di "indiarsi" (neologismo dantesco) \t = diventare dio \\
Canto di Giustiano è centrale \t si trova una rivisitazione del "Codex iustinaieus??" \t dante dice "del troppo e del vano" ??? \\
Regni romano/barbarici \t contatto con lo straniero è tema importante \\
Lo straniero nella letteratura latina è il nuovo \t tutto il percorso dell "Asinus aureus"

\v

Movimento concentrico, proporzionalità \t movimento delle sfere e desiderio di godere dio \\
A seconda della distanza dalla rosa dei beati \t movimento è più lento o più veloce \\
Forte relazione con la musica, coralità ed evanescenza delle anime \t diventano sostanza nel paradiso \\
Linguaggio presenta nuovi neologismi danteschi (come "indiarsi") \\
Sono presenti personaggi del contesto e filosofia medievale \t Francesco Domenico \\
La vicinanza ai deboli francescani è immediatezza sul campo \\
Anche giustiniano è presente \t fa riflettere sui temi della giustizia \\
Figura che si ritrova (solo apparentemente era stata persa, è sempre presente in dante) si ritrova \\
L'ineffabile è cio che non puo essere descritto a voce \t cio che con le parole non si puo dire \\
Luce viene direttamente da dio \\
Sfere + lontante ricevono luce, ma non hanno stessa forza di quelle vicine \t principio gravitazionale \\
Trasumanar = andare oltre l'uomo \\
Tutta l'opera è chimata "summa" \t ci sono riferimenti filosofici, dottrinali e teologici, ma anche scientifici (dell'epoca) \\
Cantica del paradiso permette di individuare connessioni con la poesia di montale

\s[Canto I]
Prime due terzine \t i versi sono emblematici \\
Dante vede cose che non può ridire \\
Avvicinandosi all'oggetto del suo desiderio \t l'intelletto si immerge tanto che la memoria non puo seguire \\
Memoria esercita un valore sul nostro animo \t memoria puo essere confortante (I sepolcri) \\
Memoria è un ostacolo che non lo accompagna nel suo viaggio \\ 
Terza terzina mostra gli intenti danteschi \t le prime due sono una captatio benevolentiae \\
Abbraccia tutte le branchie del sapere del tempo \\
Ora ha bisogno dell'intervento di Apollo \t dio della poesia e delle arti in generale \\
Fa tutto un discorso esortativo \\
La cultura pagana è veste per incaranre dei veicoli nuovi (Virgilio) \\
Forma asincopata e anastrofe \t "vedra mi" \t esigenza sintattica \\
Materia torna \t e usa metonimie, es. legno per albero \\
Cesare \t imperatore o poeta \t in particolare qua ha la minuscola \t il potere terreno è subordianto a quello divino \\
L'alloro dovrebbe generare letizia a dio, nel momento in cui qualcuno ne manifesta il bisogno \\
Sono presenti diverse figure \\
A partitre dal 36 c'è descrizione paesaggistica di quel mondo \t con visione cosmologica aristotelica/tolemaica

\s[Gennaio 14]
Verso 37 \\
Verso 40 \t anafora e chiasmo \\
Qua inizia il racconto cosmologico \t c'è l'esempio della perfezione: la croce esprime la trinita, mentre il cerchio il ritorno al punto d'origine \\
Qui la mondana cera viene plasmata secondo la luce del sole \\
Compare quindi beatrice sul lato sinistro con il sole in tutto lo splendore, che lei riesce a guardare senza problemi e senza limitazioni (che gli umani hanno) \\
Un aquila, che ha valore politico \t poi si ritrovera nel canto politico del paradiso (il sesto) \\
Qui si vede il valore congiunto di teologia, politica e conoscienza \\
L'aquila sembra essere in grado di fissare il sole \\
Il secondo raggio suole uscir del prmo \t in fisica, l'ottica \\
Come cosi (verso 49-53) \t parallelismi danteschi, che portano armonia \\
Tramite beatrice, sperimenta l'indiarsi \t attraverso i suoi occhi vede il sole, l'immeso e quindi dio (verso 54) \\
Infuso \t da infundere, equivale a trasmettere \\
È presente simmetria totale tra dante e beatrice, e quindi dio 
\end{document}
