\documentclass[12pt]{article}

\usepackage[a4paper, total={6in, 8in}]{geometry}
\usepackage{textcomp}

\begin{document}
\setlength{\parindent}{0pt}

\def \t {\textrightarrow}
\def \v {\vspace{1em}}
\def \bi {\begin{itemize}}
\def \ei {\end{itemize}}
\def \s[#1] {\section*{#1}}
\def \ss[#1] {\subsection*{#1}}
\def \sss[#1] {\subsubsection*{#1}}

\s[Recuperare appunti]
\s[Manzoni]
1810 \t conversione \t dovuta a principi che l'autore aveva gia in se \t affini all'illuminismo \t non è propriamente vero che l'illuminismo sia ateo \\
Conversione avviena a seguito della processione, cui perde la moglie e la ritrova su una panchina che contemplava l altare \\
Conversione si puo vedere nelle opere \t ma i valori illuministi ?? rimangono ??

\ss[Inni sacri]
Gia dal 1812 escono opere che segnano la conversione \t inni sacri sono la prima testimonianza della conversione \t sono delle composizioni religiose che seguono il ritmo nel canto \\
Ciascuno è dedicato alle principali ricorrenze religiose \t ovvero l attivita di maria, la pentecoste, l'assunzione di maria, ... \\
Inni sono in versi \t e trasponde molto dell apparato biblico e dei testi sacri in generale \t sono anche molto settoriali \t per un pubblico circoscritto \\
Sarebbero 12 ma non li finisce tutti \\
Lo spirito santo durante la pentecoste scende con ligue di fuoco sugli apostoli \t cosi gli apostoli possono comunicare con il mondo intero \\
Spirito parla tutte le lingue \t se no torre di Babele \t "De vulgari eloquenza" di Dante \t con il volgare vuole uniformare \\
Manzoni sente il bisogno di testimoniare la sua conversione \t ma ciò che era calvinismo non viene cancellato in un attimo

\v

Prima era calvinista \t sentiva il peso della predestinazione \t questo si associa a i "Discorsi sulla morale cattolica" \\
Sono uno scritto teorico \t che focalizza la teoria del gian-senismo \t da Gian Senio \t l opera è dedicata a Sismonde di Sismondì \\
L'interlocutore vede lentamente smontare l'apparto del gian senismo \t ma cosa ha in comune? \t idea che dio scende per salvare \t e sceglie chi non salvare \\
Preghiere servono a chi deve essere salvato 

\v

Visione della predestinazione è quella che permane nel romanzo \t ma la vera risoluzione di questo calvinsimo rimangono fino al romanzo \\
Solo col romanzo Manzoni risolve \\
Nel romanzo:
\bi
    \item nella prima c'è
    \item nella seconda viene epurata
    \item nella terza \t viene risolto???
\ei

\v

"Discorsi sulla morale cattolica" \t argomentazione sul gian-senismo \\
Nel romanzo ci sono esempi di calvinismo e gian-senismo \t quando c'è un senso di condanna per l'agire dell'uomo \t es. le peripezie dei personaggi, la peste come punizione divina, quando renzo non puo sposare lucia \\
Le sorti dei personaggi non possono essere risolte \t idea della provvida sventura, legata all' "Adelchi" \\
Ermengalda \t per quanto riguarda la narrativa (dal 16 al 21) compone due tragedie: "L'adelchi" e "Il conte di Carmagnola" \\
Qui contesta le unita aristoteliche \t la lettera a Messiue sull'unita di tempo, luogo e azione \t manzoni mette in discussione i principi aristotelici \t la tragedia doveva avere luogo in un unita di tempo e di luogo \\
Nell adelchi ribalta la storia (1821) \t mostra come carlo magno, re dei franchi, è stato un traditore \\
Aveva sposato Ermengarda (che era Longobarda) per impossessarsi dei suoi territori \t ma poi l'abbandona \t si lascia morire in un convento, a Brescia \\
Lei è vittima qui \t ma nella storia ufficiale, i Longobardi erano i forti e prepotenti \\
Qui carlo magno è un sovrano brutale, ma nella storia è ricordato come un grande re \\
= ribaltamento storico \t "agli umili non resta che far torto o patirlo" \\
Ermengarda è l'emblema della provvida sventura \t non raggiunge la felicita su questa terra, ma si proietta verso l'aldila \t visione calvinista \\
Diversamente da lucia \t lei realizza il centuplo \t sperimenta la felicità terrena \\
Dopo la conversione: inni sacri (1812), Adelchi (1821), Dissertazione sulla morale cattolica a Sismonde e argomento=confutazione del gian-senismo e come fonti usate i testi sacri

\v

Riflette sul genere del romanzo \t lettera a Barchese d'Azeglio?? \\
I suoi trascorsi lo portano a far fruttare gli studi storici che conduce su varie epoce \t è colpito dalla somiglianza tra il suo periodo, dove l italia è vittima degli austriaci, e il 600, italia vittima spagna \\
Cosi decide di realizzare una composizione lunga \t con grande pubblico, che aspetta volta formativa \\
Questo romanzo viene ampiamente progettato e schematizzato \t e supportato da opere che dimostrano andamento parallelo tra studi storici e l'opera stessa dei promessi sposi \\
Manzoni analizza le fonti \t tratteggio un opera che fu in illustrazione nitida della peste, e compose cosi "Della colonna infame" \\
Passione storica anche nell "Adelchi" \t prima compone "Discorso sopra alcuni punti della dominazione longobardica in italia" (di supporto all adelchi) \\
1648 \t peste a milano \t di supporto alla peste "Historia della colonna infame" \\
La peste di Atene, ma anche quella di Firenze del 1348 \t dove muore Laura, la musa di Petrarca \t e le testimonianza del Decameron 

\v

I "Promessi sposi" ha una serie di step di composizione:
\bi
    \item prima edizione: la scrive nel 17 ed esce nel 21, sotto il nome "Fermo e Lucia" \t ventana, molto storica, dilatazione dei capitoli sulla peste e monaca di monza (per cui è come un romanzo nel romanzo)
    \item seconda edizione: ventisettana (nel 1827), sfronda le parti storiche (soprattutto quelle della peste) \t cambia titolo a "Gli sposi promessi"
    \item terza edizione (titolo definitivo, 1840): da "i sposi promessi" a "i promessi sposi", la quarantana \t vera e propria congettura linguistica e politica \t si reca infatti a firenze e "sciacqua i panni in Arno" = recepita la necessita di una rivoluzione linguistica, elimando i lombardismi e dialettismi e francesismi, che avrebbero ostacolato la diffusione del romanzo
\ei
Per le prime edizioni la lingua non sarebbe stata comprensibile a tutti \t sceglie quindi il fiorentino parlato da una classe medio-colta \t dà voce a una classe definibile borghese \\
Rivoluzione linguistica ha anche finalità politica e di diffusione \\
Non solo traguardo linguistico, ma auspica anche traguardo politico: l'unificazione d'italia \\
Torre di babele \t lingua comune rende unite le persone \t "questione della lingua", avviata da Dante con il suo "De vulgari eloquentia" e 1525 Pietro Bembo, "Prose della volgar lingua" \\
Anche machiavelli riflette sulla lingua della corte \t fino al saggio di Melchiorre Cesarotti, "Saggio sulla filosofia della lingua" in clima preromantico \\
Nasce con manzoni lo spirito della editoria dell'opera \t non si promuovevano le opere attraverso i canali mediatici \t manzoni lo fa \\
Al gabinetto Vieuesseux di Firenze \t si incontrano Manzoni e Leopardi \t Leopardi elogia il romanzo, che ebbe una diffusione colossale \\
Fu il punto di incontro tra letteratura e risorgimento \t manzoni è voce del risorgimento (ovvero romanticismo italiano) \\
Valore sociologico di questa opera \t la nascita dell'editoria e la diffusione \t con manzoni nasce l'idea dell'opera di massa 

\v

Dopo manzoni in molti provano a riprodurre un opera simile \t poi crisi di fine 800 \\
Ippolito Nivo scrive "Le confessioni di un italiano" 

\v

L'opera non segue la fabula ma l'intreccio ?? \\
Due voci:
\bi
    \item romanzo constro la storia \t Giorgio Barberis Squarotti
    \item romanzo senza idillio \t Ezio Raimondi
\ei
Senza idillio \t romanzo si chiude con il matrimonio, come una soddisfazione borghese, che viene realizzata \\
Il centuplo è possibile anche sulla Terra \\
Realizzano il piano divino \t ma raggiungono il centuplo \t la costruzione è avvenuto come traguardo \t renzo cresce, mentre lucia cambia poco
\end{document}
