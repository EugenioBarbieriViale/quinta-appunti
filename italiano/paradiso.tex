\documentclass[12pt]{article}

\usepackage[a4paper, total={6in, 8in}]{geometry}
\usepackage{textcomp}

\begin{document}
\setlength{\parindent}{0pt}

\def \t {\textrightarrow}
\def \v {\vspace{1em}}
\def \bi {\begin{itemize}}
\def \ei {\end{itemize}}
\def \s[#1] {\section*{#1}}
\def \ss[#1] {\subsection*{#1}}
\def \sss[#1] {\subsubsection*{#1}}

\s[Paradiso]
Come l'inferno, è la cantica + nota \\
Idea di un mondo perfetto \\
"Paradise lost" \t Milton \t idea di un mondo edelico, nel quale le divergenze si ricompongono \\
Si raggiunge nell'osservazione dei principi cristiani \\
Nel mondo crisitiano parlare del paradiso significa parlare di anime salve \t non si sono pentite all'ultimo, ma hanno fatto una costruzione nel corso della vita \\
Ci sono le figure speculari della chiesa \t come i padri della chiesa \t e Beatrice, capace di amare al di sopra di ogni limite e per la quale dante ha scelto la via della salvezza \\
È la cantica che ciude il cerchio \t rasseregna l'animo \t la luce accompagnata dalla preghiera e dalla coralita del purgatorio, che era riblatato risp. all'infenro (montagna vs voragine) \\
In paradiso c'è universo etereo che non è tangibile \t dante prende in esame attraverso varie fonti \t e "Il somnium scipionis" \\
Dante consulta le fonti sacre \t la bibbia, i salmi, il vangelo \t il vangelo perchè parla di un umanita salvata \\
Ma qual è il testo che parla di + del paradiso? \t il somnium scipionis \t l'ultimo libro del de repubblica, pubblicato autonomanente \\
Movimenti celesti, un universo immenso rispetto all'occhio umano \\
Per la prima volta nella lett. latina \t un oratore filosofo parla di un argomento scientifico/filosofico \t parla dei moti celesti, e di uno stretto rapporto tra movimento e suono \t principio petagorico \\
In cicerone è presente come una summa \t che Dante prende in esame \\
Dante analizza (come petrarca) i tesi antichi \t le opere del medioevo si servono dell'apparato classico per agganciarvi contenuti nuovi \\
Per esempio Virgilio \t Dante lo sceglie per l'egogla (nona, delle Bucoliche) che parlava della nascita di un bambino \t che viene strumentalizzata in ambito cristiano 

\v

La numerologia delle cantiche \t delle terzine \t cosa rimane? \t rimane tutto l'apparto che fa si che Dante sia autor e actor \\
Cantica porta a una difficolta maggiore nella lettura \t viene definito da Erich Auerbach, da Dante Isella, Leo Spitzer e studi recenti \t quest'opera è la + carica di senso e ricca dal punto di vista dottrinale \\
Dante pero non risulta del tutto cambiato \t parla di un processo di "indiarsi" (neologismo dantesco) \t = diventare dio \\
Canto di Giustiano è centrale \t si trova una rivisitazione del "Codex iustinaieus??" \t dante dice "del troppo e del vano" ??? \\
Regni romano/barbarici \t contatto con lo straniero è tema importante \\
Lo straniero nella letteratura latina è il nuovo \t tutto il percorso dell "Asinus aureus"
\end{document}
