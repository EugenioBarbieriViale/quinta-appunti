\documentclass[12pt]{article}

\usepackage[a4paper, total={6in, 8in}]{geometry}
\usepackage{textcomp}

\begin{document}
\setlength{\parindent}{0pt}

\def \t {\textrightarrow}
\def \v {\vspace{1em}}
\def \bi {\begin{itemize}}
\def \ei {\end{itemize}}
\def \s[#1] {\section*{#1}}
\def \ss[#1] {\subsection*{#1}}
\def \sss[#1] {\subsubsection*{#1}}

\s[Recuperare 2/3 Febbraio]
\s[Pascoli]
Il nido insanguinato \t suo padre muore quando aveva 10 anni \t e il suo obbiettivo sara di ricomporre il nido \t ha dieci fratelli \\
Il nido viene circoscritto alle sorella Ida e Maria (detta Mariu) \t il nido domestico è un limite, connotato da un senso di gelosia delle sorelle e ha portato a rimuovere i contatti tra sorelle ed esterno \\
Presunta omossesualita dell autore, e incestualita presente nella famiglia ridotta a loro 3 \\
Ostruzionismo della vita delle sorelle \t che ha portato all'impedimento di una costruzione di relazioni proprie da parte delle sorelle \\
Maria è colei che alimenta di piu questo disagio di pascoli \t soprattutto quando si sposa e va a vivere via \\
La piscoogia di pascoli si deve leggere con il parallelismo con il mondo naturale \\
La difesa dei propri cuccioli puo essere relazionata alla difesa dell'autore delle sorelle \t che in realta pero sono esseri umani, e neanche figlie dell'autore = animo malato dell'autore \\
Gli anni goliardici dell unviersita e la famiglia \t non lo precludono dal dolore, e la vicinanza con gli amici di bevuta con i compagni di uni. e i suoi studenti una volta professore \\
Morira di cirosi epatica a Bologna nel 12 \t per la compressione del fegato dovuta a un abuso dell'alcol, e dalla degenrazione delle cellule epatiche dovute a un infenzione dell'epatite c \\
Luciano Ancessi \t l'artista ha perso l'aureola \\
La sua vita continua con un costante dolore, che non trova risoluzione, e una cultura si radica nel suo animo e si fa portavoce di uno spirito oltr'alpa \\
Il suo decadentismo è simbolico, allusivo e fortemente intimistico \t per questo euorepeo, non estetizzante \\
Il decadentismo si trova in due: "Mirice" (91) che presneta diverse vicende filologiche, e Gino Ferrari???? che dice in latino "non omen iuvant umilesque Mirice" = a non tutti piacciono le piccole merici (forte defezione da dannuncio con la pioggia nel pineto), tamerici sono cespugli \\
Tameririci ricodano la umilta lucreziana ?????? \\
L'altra opera è "Canti di Castelvecchio" \t "canti" evoca Leopardi, siamo nel 97 \\
La differenza sta nell'espansione, il luogo, la riminiscenza leopardiana \\
La gestazione filologica non mostra tanto la differenza tra queste due raccolti diversi, ma i "canti di cast." partono da luoghi diversi e partono da un universo come quello di mirice \\
Jammes, Metterling e Rondenbach \t poeti fiamminghi
\end{document}
