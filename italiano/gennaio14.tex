\documentclass[12pt]{article}

\usepackage[a4paper, total={6in, 8in}]{geometry}
\usepackage{textcomp}

\begin{document}
\setlength{\parindent}{0pt}

\def \t {\textrightarrow}
\def \v {\vspace{1em}}
\def \bi {\begin{itemize}}
\def \ei {\end{itemize}}
\def \s[#1] {\section*{#1}}
\def \ss[#1] {\subsection*{#1}}
\def \sss[#1] {\subsubsection*{#1}}

\s[Scapigliatura]
Bohemien vuol dire disordinato, scapigliato \t una cultura ai margini che ha un manifesto \\
6 febbraio 1862 \t manifesto di Cletto Arrighi, che pubblica questo manifesto di intenti \t in cui spiega come deve essere l'arte scapigliata, e come si deve comportare l'artista scapigliato e soprattutto cosa contesta \\
Questo manifesto da un vademecum di tutto cio che caraterizza la nuova era: una repulsione per i ceti borghesi in particolare \\
Vedere definizione che da della scapigliatura \\
Poetica contro i modelli, parola chiave è la ribellione \t inoltre la vita e l'arte coincidono, fondamentale perche questa identita è espressione della follia fuori i manicomi \\
Pirandello dice che soltanto chi è fuori di testa puo dire tutto cio che vuole \t puo parlare male del tiranno (alfieri) perche non ha filtri \\
Si coglie dualismo, per cui gli sacpigliati come i matti possono dire tutti \t perche hanno una mente che vuole gettare lucidita, ma non è lucida, e inoltre spesso scrivono sotto l'effetto di altro \\
Infatti hanno vita dissoluta, che pratica un'esaltazione di questa sregolatezza \t questo permette di legarsi a seneca: rifiuto di ogni sentimentalismo \\
Rifiutano le manifestazioni di dolcezza e di tenerezza, e i sentimenti teneri = anticonformismo viscerale \\
Denominatore comune è il rifiuto di Manzoni \t che pero non ha mostrato i suoi personaggi mai nei sentimenti sdolcinati (renzo e lucia non si scambiano mai un bacio)

\v

Rimbaud, Hoffman, Poe per la prosa \\
I contenuti sono simili: il turpe, il misterioso, il malato \t il mistero del soprannaturale, di cio che non è spiegabile con la ragione \\
Bohem è il termine cardine \t si trovavano sul Rig Gausch e pubblicavano sul Chat Noir \\
Bohem viene da boemia \t sembra ricondurre a un'abitudine di vita errabonda, e nomade \t i bohemienn rappresentano la volonta di trasgredire le regole \\
Tante vittime di giovanissima eta \\
Nella letteratura latina, il vagabondare si trova in Lucio Asino nell asino d'oro, il libertinaggio in ?? e in Catullo 

\ss[Preludio - Emilio Praga]
Documento che presenta tutto cio che viene rifiutato \t è tratto dalla raccolta "Penombre" del 64, ed è all'insegna dell'antiromanticismo e dell'antimanzonismo, e antivalori \\
Questo testo viene considerato il manifesto della scapigliatura \\
Il lettore è invece fratello / nemico \\
Strofe di 4 versi \t ripresa continua che scandisce un ritmo \\
Sono paradigmatici i primi versi \t mostrano la difficolta delle aquile a volare \t rappresenta la maestosita \\
L'aquila è depotenziata dal valore politico \t deve prendere un altro corso, ma noi siamo muti attoniti e affamati \t climax ascendente che ricorda il 5 maggio di Manzoni \\
Numi sono divinita, ma qua è agonizzante e moribondo \\
Paradigma \t arcano senso delle cose ?? \\
"Splende sull'arca" nebbia remota evoca lo splendore dell'arca \\
Patriarca e idolo sono i soggetti \t gioco verbale che si esprime in anastrofe/iperbato molto forte \\
Testo della dissacrazione dei valori religiosi \\
Evocazione della passione di cristo, e la vergine è maria \t colta nel momento della morte di cristo, il cui corpo viene avvolto nel lenzuolo \\
Sudario è con la maiuscola, inteso come se fosse uomo, mentre vergine con la minuscola, disvalorizzata \\
"Casto poeta" = manzoni \\
Gli antecristi sono i diavoli \t anche Nietzsche, dio è morto, e "De finibus bonorum et malorum" (circa i parametri del bene e del male) di Cicerone \\
Cicerone parla dell'importanza della patria \t in nome della patria l'amicizia si puo cancellare \\
Pluralita dei nemici di dio \t gli antecristi \\
"Io canto la noja" \t tedium vitae, anche in Boudelaire, in Leopardi \\
Il loto è il fiore che nasce nel fango \t equivale a dire gli scarti (nella beletta, D'Annunzio) \\
Verso 19: climax con anafora \\
I peccati inginocchiati sono una sinestesia \\
Ideale con la maiuscola che declina nel fango \\
Lettore viene chimato fratello con un imperativo categorico \t sottolineare: la maschera al pensiero verso 31 \\
Canto una misera canzone, ma canto il vero \t manzoni cantava in vece il bello e il buono, mentre qua il turpe, il malato e il deforme \t tutto cio che è anti \\
Il minio serviva per il trucco di colori esagerati, perche nel teatro devono essere visti a distanza \\
La maschera invece è quella del pensiero \t bisogna toglierla per raggiungere la liberta del pensiero \t il minio è di colore rosso \\
Rimando a pirandello \t parla di una signora imbellettata, e bisogna ricordare che era abitudine truccarsi comunemente
\end{document}
