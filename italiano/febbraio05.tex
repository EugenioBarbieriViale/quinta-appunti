\documentclass[12pt]{article}

\usepackage[a4paper, total={6in, 8in}]{geometry}
\usepackage{textcomp}

\begin{document}
\setlength{\parindent}{0pt}

\def \t {\textrightarrow}
\def \v {\vspace{1em}}
\def \bi {\begin{itemize}}
\def \ei {\end{itemize}}
\def \s[#1] {\section*{#1}}
\def \ss[#1] {\subsection*{#1}}
\def \sss[#1] {\subsubsection*{#1}}

\s[Pascoli]
"Mirice non omnes iuvant" = "non ha tutti piacciono le umili tamerici" \t preso dalle bucoliche, taglio verso il mondo campestre delle piccole cose \t non gli alberti altisonanti del mondo d'annunziano, ma le tamerici sono umili \\
Sono dei cespugli semplici \\
C'era un confronto tra d annunzio e pascoli \t ma era normale \t lo scambio di liriche era frequente \\
Ma nella lirica di novembre pascoli sembrava aver imitato qualcosa di d'annunzio, che lo accusa \t ma poi esce un altra occasione di confronto \\
C'è un disegno di pascoli che risponde a un atteggiamento di d'annunzio \t e d'annunzio dice di pascoli che era un pantofolaio, mentre pascoli risponde pubblica il ritratto di d'annunzio ironico \\
Jammes, Rodenbach e Metterlinch \t i poeti decadentisti fiamminghi \\
La ripresa di Leopardi è presente anche nei "Canti di Castelvecchio" \t componimenti + estesi, ma con natura allusiva \t quindi analogia, sinestesia e similitudine (figure retoriche che caratterizzano il testo pascoliano) \\
La nebbia sfoca i contorni \t che si trova in diversi testi, come in Mirice, e poi la nebbia di Carducci \\
Il non vedere i contorni delle cose \t la nebbia rappresenta questa

\v

Da mirice ci si aspetta una forte allusivita, una poesia che mette al centro la natura e l'io del poeta, un linguaggio elettivo, che fa dire a \textbf{Gian Franco Contini} lo specifismo del linguaggio pascoliano \t nel saggio \textbf{"Lo specifismo del linguaggio pascoliano"} \\
Anche \textbf{Pier Vincenzo Mengaldo} \t ne parla in \textbf{"Varianti ed altra linguistica"} \t tratta anche lui lo specifismo \\
Specifismo perchè? \t "La poetica del fanciullino" \t è colui che avvicina le cose lontane, ed è la condizione necessaria della poesia \\
Se è dentro ciascuno di noi un fanciullino, non significa che siamo tutti poeti \t solo chi lo sa ascoltare, perche il poeta sa dire cio che gli altro vedono e non sanno dire \t colui che sa far parlare il fanciullino \\
Il fanciullino si puo trovare anche nell masnadiero = uno che raggira, che prende in giro \t non è l'etichetta ma l'anima a parlare \\
Se l'anima scaturisce da quello, la poesia è fatta di queste cose \t le analogie rendono questo, che collegano \\
Le sinestesia invece accorpano sfere sensoriali diverse, mentre le similutidine sono le piu manifeste \t onomatopee riproducono il suono \\
Tutte queste figure sono accomunate dal mettere vicino il lontano ??? o il contrario \\
Collega ciò che è inafferrabile \t leopardi, in passero solitario, parla di assiuolo \t anche qua specifismo ?? \\
Il ritrarre una natura che in se ha un paradigma \t questo fa pascoli \t ritrae un senso di finitezza e di morte

\v

Gli oggetti della poesia \t l'oggettistica montaliana parte da qui \\
Qua siamo fermi all'analogia, mentre in montale l'oggetto incarna un sentimento \t qua capiamo il sentimento, mentre in montale il sentimento viene taciuto \\
\textbf{Montale si correla a Elliot \t Joyce e Italo Svevo} \t ma anche Italo Svevo e Montale, perche montale valorizza svevo \\
Nella natura pascoliana c'è un materialismo \t causa ed effetto si determinano a vicenda nel loro susseguirsi \\
La poetica del fanciullino ci porta a ravvisare una poesia frammentaria = segue il flusso dei pensieri \\
Le poesie di Pascoli hanno anche un altro aspetto \t come l'emigrazione, in cui gli emigranti cercavano fortuna altrove \\
La grande proletaria si è mossa \t lo spirito socialista filo-sorelliano (Sorel) si traducono in alcuni testi \\
Produce uno scritto teorico \t e "Il saggio sul morismo" di Pirandello \\
Fornisce delle linee programmatiche di poetica \t cosa che leopardi non fa \\
Nle frammentario inoltre non si coglie una logica \\
Il fanciullo è in noi \t gli adulti hanno messo una benda alla bocca del fanciullino \\
I bambini trovano il noto in qualcosa di nuovo \\
"Il fanciullino è un musico" \t musicalita è la quintessenza della poesia \\
Similitudine = "questo è come", mentre analogia = "questo è", mentre nella metafora è sostituzione totale

\v

Novembre \t lavandare \t 10 agosto \t arano \t l assiuolo \t peranze e memorie \t nebbia \t lampo tuono temporale \t mare 

\s[Lavandare]
La lirica fa entrare nel costume \t e il profilo filologico \t a milano, nelle sponde dei navigli, venivano lavati i panni \\
Le signore si recavano in una ritualita di comunanza e fraternita, con canti, che trovavano come strumento lo sciabordare (onomatopea), ovvero il rumore dei panni sbattuti \\
\textbf{I testi pascolinani vengono raccolti da Nava} \t Giuseppe Nava è il filologo che propone il pascoli di queste liriche \\
L'edizione critica è la sua \t lui è filologo e fa edizione ciritca \t non bisogna dimentricare neance Ferrari 

\v

Immagine campestre a temporale, indefinito con sospensione \t meta aratro e meta no \\
La coltivazione voleva la rotazione per rendere + fertile i campi \t l'aratro è uno strumento che ci avvicina al mondo verghiano, nel quale si parlava di una societa campestre oltre che marinara \t e la forza lavoro sfruttata e strascurata \\
L'aratro è la componente unmana \t che poi verra sostituita nella lavorazione dei campi \\
Mezzo e mezzo è un anafora, ed + anche un chiasmo: mezzo nero mezzo grigio \\
Atmosfera autunnale, in cui l'abbandono viene descritto dall enjambement tra v. 2 e 3 \t che lascia in sospensione il termine dimenticato \\
Il vapor leggero ci fa capire che è autunno \t è la nebbia mattutina \\
"Gora" sta per canale \t specifismo pascoliano, il canale che fa parte della descrizione paesaggistica \\
Altro enjambement tra viene/lo sc. \t sciabordare è onomatopeico \\
Cantilene \t mondo dell'infanzia \t ai bambini si cantano le cantilene cadenzate e ripetitive, come i canti dei lavandai \\
Tonfi spessi \t sinestesia, e anche lunghe cantilene \\
I versi + intensi sono ultima parte \\
Nevica la frasca \t anastrofe e sinestesia ?? \t e forse chiasmo \\
Poetica della distanza, degli oggetti, è gia qui \t entra quindi il tema dell'emigrazione, che riporta all'addio ai monti nei promessi sposi \\
Partisti è un perfetto, mentre son rimasta è un passato prossimo che rappresenta la continuita \t mentre il perfetto è finito nel passato \\
Ultima è un analogia \t maggese è termine specifico, che indica lo specifismo
\end{document}
