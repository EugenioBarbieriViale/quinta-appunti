\documentclass[12pt]{article}

\usepackage[a4paper, total={6in, 8in}]{geometry}
\usepackage{textcomp}

\begin{document}
\setlength{\parindent}{0pt}

\def \t {\textrightarrow}
\def \v {\vspace{1em}}
\def \bi {\begin{itemize}}
\def \ei {\end{itemize}}
\def \s[#1] {\section*{#1}}
\def \ss[#1] {\subsection*{#1}}
\def \sss[#1] {\subsubsection*{#1}}

\s[Recuperare 17 e 18 dicembre]
\s[Operette morali]
Composte per la maggior parte nel 1824, a Roma (citta che lo ha deluso) \t filosofia, prosa, utilizzo degli animali e figure mitologiche, non cambiano pero i temi \t cambia lo sguardo sui temi della vita \t ma rimangono gli stessi, cambia solo lo sguardo \\
Nelle operette morali subentra il discovrir del vero = drammaticità, sconcertazione, sovrapposizione tra vissuto e immaginato \\
Novita della prosa \t era presente anche nello zibaldone (ma lui è un poeta di fatto) \\
IN quest'opera ritorna la prosa, che è consapevolezza e corrisponde al discovrir del vero, il crollo delle illusioni \\
Nelle operette c'è una struttura complessa \\
A roma, viene deluso dall ambiente orientato all utile e all opportunismo \t delusione coinvolge 27, 24 e 23 di composizione, è una revisione delusa della teoria del piacere, del tema dell'infelicita umana, e il distacco del non voler piu soffrire \\
Attraverso l'introduzione del sarcasmo (diverso da ironia: guarda le debolezze e le alleggerisce, sarcasmo amplifica le debolezze) \\
È deluso sia dalle persone, sia dai luoghi \\
Qui si vede l'animo risentito di leopardi, che cera un distacco sarcastico \t l'uomo è penalizzato rispetto agli animali, che non soffrono

\v

Luciano di Samosata è il modello \t "Apuleio e l'asino d'oro" (Lucio è il protagonista) \\
Luciano ha composto dei dialoghi, in cui c'erano dei personaggi fantastici con tratti comici e con una grande presenza del dialogo \t il "Dialogo di un folletto e di uno gnomo" si svolge con la consapevolezza che gli dei osservano \\
Il dialogo ha il punto di vista superiore della ragione \\
Anche amore e psiche è un rinforzo per Leopardi

\v

Inclinazione satirica delle operette portano spesso a malinconia \t si fa satira di solit su aspetti difficili della realta \\
L'essere umano è portatore di sofferenza, e di dolori non sempre sopportabili \t e dell'avere colpa di una natura matrigna e nemica insieme \\
Il dialogo è molto presente nella storia della letteratura, e molto nelle satire \t dialogo come anticipatore della teatralita \t come in pirandello: nella sua prosa, la presenza dei dialoghi è fitta \t attua la riforma del teatro \\
Dialoghi in realta erano presenti anche nei versi \t come verso la luna, pastore errante e a se stesso \\
Lui aveva letto i testi di Luciano, e anche dei testi fornitogli da Giordani \t aveva sottolineato la volonta di denuciare la sofferenza, gli ideali a cui aveva creduto e che l'avevano condannato \\
Vengono calate le riflessioni nei dialgohi di leopardi \t le opere nel 23, e poi 24, vengono scritte in velocita \\
Individuo-società, natura-individuo e perdita delle illusioni \t importante \\
Gli idilli di Teocrito \t imp. per leopardi e altri greci ????

\sss[Venditore di almanacchi]
Riflessione del anno passato e di quello che verra \t nell'anno passato, c'è l idea dell uomo adulto che si gira a guardare, mentre nell anno che verra c'è lo sguardo speranzoso \\
Nell anno passato c'è la vecchierella, mentre in quello nuovo la donzelletta

\s[Dopo Leopardi]
Dopo manzoni c'è il rilancio del romanzo e c'è un pubblico + consapevole (mentre Leopardi non aveva la percezione di pubblico) \\
Dopo Leopardi c'è quindi un nuovo silenzio della poesia \t che pero porta a considerare il clima risorgimentale, e i moti del 30 etc. \\
La cultura italiana si sta aprendo alla mentalita oltr'Alpe \t ad altre culture ed altre voci letterarie quindi \\
La letteratura manifesta interesse per temi risorgimentali \t l'obbiettivo era l'unita d italia, ma la poesia di leopardi è lontana da questa consapevolezza \\
Si verifica un meccanismo particolare \t il romanzo filomanzoniano e la ripresa di Manzoni, letteratura didattica regionale (Emilio De Marchi)  \\
Ippolito Nievola con "Confessioni di un italiano": il romanzo risorgimentale \t carlino altoviti, che vive le fasi della sua vita parallelamente all unita d italia  \\
Nicolo Tommaseo \t rappresenta uno spartiacque: crea il primo dizionario della lingua italiana, e scrive fede e bellezza \t corrisponde alla dualita dell essere umano \t romanzo mette a fuoco i due opposti: la ricerca di una perfezione spirituale, e la bellezza che impedisce questa via verso dio \\
Introduce un elemento sensualistico nell approccio alla realta \t percezione con i 5 sensi, ed è quindi anticipatore di D'Annunzio \\
Era filomanzoniano, ma il livello del suo romanzo non è mai stato raggiunto 

\v

1861, unita d italia, statuto albertino viene esteso a tutto il Paese \t ma frammentarieta è ancora presente \\
Inoltre è presente il blocco del centro, con lo stato pontificio, dove c'era poverta, malaria, brigantaggio, mentre il sud era completamente disagiato e arretrato (con latifondi e baronati) \\
Inoltre l'analfabetismo è ancora molto diffuso \t italia è in una condizione difficile \\
Decisioni sabaude vengono estese a tutta la penisola, anche se non adatte \\
Il focus è pero su Milano, dove ci sono giovani intellettuali che vivono non la milano manzoniana e scapigliata \\
Intellettuali sono giovani e ribelli, e condividono gli stessi ideali (o anti-ideali, perche rifiutano il manzonismo e la borghesia) \\
Manzoni è il fautore dell unita di itali \t ma questi intellettuali non creano una scuola, perche vivevano molto liberamente \\
Sono antesignani dell'aperuta di tutte le porte al mondo francese \t la letteratura e poesia francese \\
La morte della poesia è ancora presente \t pero riescono a cambiare la cultura e la mentalita
\end{document}
