\documentclass[12pt]{article}

\usepackage[a4paper, total={6in, 8in}]{geometry}
\usepackage{textcomp}

\begin{document}
\setlength{\parindent}{0pt}

\def \t {\textrightarrow}
\def \v {\vspace{1em}}
\def \bi {\begin{itemize}}
\def \ei {\end{itemize}}
\def \s[#1] {\section*{#1}}
\def \ss[#1] {\subsection*{#1}}
\def \sss[#1] {\subsubsection*{#1}}

\s[L'infinito]
1818/19 \t fa parte della raccolta "I canti" (uscito definitivamente nel 31, curato da Ranieri) \t a Recanati \\
Romanticismo \\
Paradigma del limite \t testo dell'astrazione, che non è tangibile ma esiste \\
Leopardi è il poeta dell'infinito \t l'infinità del male non lo fa soccombere, non lo schiaccia \t ma lo fa trovare nel limbo, nell'ovattamento, protezione che solo il ventre materno sa riprodurre \\
Critica psicanalistica \t mare, liquido amniotico \\
L'io solo davanti all'infinito \\
Un colle, rappresenta solitudine \t monte Tavor, nel quale l'autore andava per riflettere \\
Siepe \t rappresenta il limite \t l'osservazione di ciò che è altro e oltre il limite \\
Strategia testuale \t corrisponde alle sue emozioni interiori \t la natura qui non è ancora matrigna, ma avvolge, è possibilita di altro \\
Un altro che non centra con il superuomo di Nietzsche \\
Ermo colle \t provocazione \t ermo è difficile da raggiungere, astrazione è in cio che non possiamo toccare con mano \\
Astrazione è oltre, e paradossalmente dentro di noi \t senso di totalita, che azzera la propria insignificanza \\
Nella sua unicità è carico di significato, ma nel sistema è un punto

\v

Sempre caro mi fu \t iperbato \\
Caro e ermo \t chiasmo \\
Evidenza concreta che apre un varco verso l assoluto \t questa siepe, che inoltre rappresenta un limite \\
Ma dove la vista viene meno, il cuore legge con profondita \t cio che descrive il mistero umano è cio che è invisibile agli occhi \\
Ultimo orizzonte \t quella linea che si percepisce come nitida tra cielo e mare \t linea in se non esiste, ma si vede \\
Avversativa al verso 4 \t permette di trovare congruenza tra l infinito e la descrizione di esso \\
Sedendo e mirando \t gerundi \t quasi a climax, interminati \t la percezione dell'assoluto \\
Interminati \ spazi \t enjambement \t profondissima quiete \t triade (sovrumani spazi, silenzi, profondissima quiete) di immagini che evocano sensazioni uditive \t infinito supera la parola \\
Quiete da "quiesco", o "requiesco" \t la morte \\
L'assertivita massima giunge al verso 7 \t "io nel pensiero mi fingo (immagino, latino)" \t io sprofondo nel pensiero e divento tuttuno con un mondo che posso raggiungere solo con l immaginazione  \\
"Ove per dove lo cor non si spaura" \t l'essere su un filo immaginario, tra la vita e la morte \\
La sensazione di un momento in cui siamo colti da una mancanza di respiro \t battito di ciglia in cui si percepisce di essere in un punto di non ritorno \\
Momento di smarrimento davanti a una natura sublime \t es. Didone che maledice Enea dagli scogli \\
Sublime secondo Edmond Burk \t oppure pittoresco \\
Lo stormire tra queste piante paragonato al silenzio dell'infinito \\
Smarrimento di rendersi conto di essere di fronte all infinito \\
L'io limitato dell uomo \\
Nell immensita l eterno è descritta anche attraverso la successione temporale \\
Presenta e viva \t sinestesia \t dove è l'infinito? \t nell'immaginazione \t essa permette di superare il limite e fluire nell infinito \\
La sostanza è l'interiorrizzazione del'assoluto \t il cielo non è in alto, ma è dentro \t l'assoluto non si puo toccare ma è dentro di noi \\
Se l'assoluto è dentro, il limite dove è? \t è fuori di noi, quindi non ci appartiene \t non dipende da me \t io devo sentire la compiutezza del mio essere

\v

Canto dell'infinito \t i canti sono la raccolta in versi \\
Permette di individuare Leopardi come il poeta dell'infinito \\
I temi principalemente hanno un paradigma esistenziale gia fissato \t teoria del piacere \\
Lo stordimento del sublime \t permette di abbracciare l'infinito \\
Nel testo si ha una struttura evocativa dell'infinito \t e anche dei tempi verbali: con il gerundio e i paritici, si descrive qualcosa che va oltre la temporalita \\
Immensita dell'orizzonte che leopardi descrive non rappresenta un traguardo raggiungibile con le risorse materiali umane, ma solo con l'immaginazione \\
Irraro presente \t ai versi 12 e 13: le morte stagioni \\
Dimensioni temporali, indicazione di un tempo non tempo, limite è valicabile \\
Il muro è proibitivo, respingente, è un limite evidente \t mentre qua il limite non si vede \\
Muro in mentale \t pero con realismo sferzante, in cima al muro ci sono cozzi aguzzi di bottiglia \\
"Horuts conclusus" \t orto perimetrato \t espressione di d'annunzio \\
Dove è il limite \t anche Nella poesia di novembre di pascoli \t il cielo si osserva da sprazzi, la vista del cielo è limitata dai rami \t cielo si vede da un rete \\
Limiti sono anche quelli della societa \t pregiudizi e preconcetti \t come la societa che Leopardi vive a Recanati \t non trova neanche la scappatoia a Roma \\
Anche da Roma viene deluso dalla falsità \\
L'amicizia con Ranieri è imporante, come quella con Giordani \\
Tra i percorsi dell'infinito \t è presente il mare \t "Il vecchio e il mare" di Hemingway \\
Psicanalisi Freudiana \t gli inganni della coscienza di zeno \t illudersi per sopravvivere \\
Leopardi è antisegninano di un scavo interiore \t e di un meccanicismo che portera al positivismo di Compt (ultimi dell 800) \\
Visione della realta deterministica, di cause ed effetti \t che porta alla risoluzione dei mali della umanita \\
Nella filosofia Comptiana sono presenti diverse sfaccettature \t uomo si vede come prodotto di una serie di fattori \\
Freud \t "Psicoanalisi della vita quotidiana????" \t chiedere cosa ha detto qua \\
Il castello di franz kafka è una descrizionie di limite \t come l'assaggio dell'acqua in sotto la ruota ripropone l'aggancio con la mdare \\
Leopardi ebbe invece una madre totalemtene anafettiva \t l'uomo cerca quello che gli manca \\
Interiorizzazione dello spazio, infinito spaziale, uditivo e ontologico \t che pero esiste solo nella immaginazione \\
Viandante sul mare di nebbia

\s[La sera del di di festa]
Chiedere info. generali \\
Il testo è spezzato in due \\
Prima parte referinziale: osservazioine della natura, paesaggio pittoresco e allineato al romanticismo straniero \t squarcio paesaggistico dalla finestra \t montagna fa da limite che protegge \t invocazione alla donna \\
Apostrofe alla donna amata \\
Sinestesia \t i sentieri tacciono \\
Continuo alternarsi di io e tu \t piani disallieati \\
Non ti morde \t suono della R \\
Piaga in mezzo al petto \t ricorda un iconografia cristiana \t anche se lui non era cristiano \\
Lei non sa di averlo ferito \t lui soffre \t verso 11 emblematico, tu/io nello stesso verso \\
Cielo appare benigno, "l'antica natura ognipossente" \t altro verso importante \\
"Mi fece" \t linguaggio quotidiano \\
Mi e a te \t discorso libero \t anticipazione di Joyce con stream of consciousness \t discorso diretto libero \\
"Nego mi disse anche la speme" \t chiasmo \\
Verso dopo \t vittimismo leopardiano \t guardare dentro di se, rimando a seneca \\
"Mi getto, e grido, e fremo" \t perdita di controllo \\
Eta verde \t sinestesia \t "fiore degli anni caduto", espressione carducciana \\
Dal verso 21 continua riflessione filosofica \\
"Come tutto al mando passa" \t riflessione sul tempo, seneca \\
"Orma" \t termine manzoniano \t l'orma che Napoleone ha voluto lasciare, 5 maggio \\
Ciclicita del tempo \t il non essere per sempre, il tempo che fugge, il tempo che gia è oltre \\
Elegia per le fasi storiche passate \t gli antichi che sapevano illudersi \\
Allitterazione della R \t roma, armi, fragorio \\
Dopo queste domande torna la riflessione filosofica \\
Premea le piume \t materasso

\v

Una parte di rappresentazione, poi rimbalzo che anticipa la riflessione, poi rimando al tema che ha scatenato la composizioine \\
Poi di nuovo il blocco conclusivo filosofico \\
È presente cicerone nelle domande retoriche \\
Endecasillabi sciolti \t mette in campo anche il monologo (donna non risponde) \t anche la luna compare \\
Luna è femminilita, mistero (ariosto) \t luna leopardiana è la interlocutrice privilegiata \\
La luna è puramente ascolto, abbraccio \t questo distacco consente all autore la verbalizzazione \t valore catartico che porta alla maieutica \\
Il dottor S nella coscienza di Zeno \t lascia trasparire un minimo di giudizio \\
Debussy \t Claire di Lune \t futurismo che distrugge la luna \t "Falce di luna calante" d'annunzio \\
"Ciaula scopre la Luna" \t pirandello \t "La luna e il falo" di Pavese \\
"La volta degrignata al plenilunio" di Ungaretti

\s[Il sabato del villaggio]
È un testo speculare a "La sera dei nidi di festa" \\
Parte dall' hic et nunc \\
Idillio campestre \t si intravedono le parti all'interno dell'opera \t si trova il fatto filosofico \\
Anche in questo componimento (si configura negli idilli) \t in conclusione si ha la "cogitatio" \t ovvero la riflessione sulla vita e sul tempo 

\v

Tramonto \t dimensione campestre \t figura anonima addolcita da un avvezzativo, tipico medievale \t "donzella" evoca anche atmosfera tassesca \\
Operosita \t reca un fascio d'erba \t in mano ha anche un mazzolino di rose \t Pascoli: contesta a leopardi la mancanza di uno specifismo, l'approssimazione \t rose e viole non fioriscono nello stesso periodo \\
Ma lui è il poeta dell'infinto \t non gli interessa, vede oltre \\
Alter ego della donzella sara la vecchia \\
Termini elettivi e quotidiani vengono mischiati \\
Attesa, aspettative, speranza \t prima fase: la memoria abbrevia il corso \\
Spazialita \t donzelletta viene dai campi, che sono in basso \\
La vecchia siede in alto \t lei ha gia salito la scala della vita \t è gia al vertice della vita \\
Ha gia varcato la soglia del "discovrir del vero" \\
Operosita campestre \t lei lavora e fila \\
Ricorda di come si ornava la testa per la festa \\
Insieme aspetti ludici, convivialita, gioiosita e condivisione \t il vivere in una dimensione simmetrica \\
Poi l'attenzione si sposta all'esterno \t atmosfera idilliaca di interazione con la natura \t individuo sperimenta una visione edenica del mondo, ecorapporto con la natura (che non è matrigna qua) \\
Luna ricopre un ruolo importante \\
Suono delle campane \t la festa che si avvicinando \t tono colloquiale \t "il cor si riconforta" \t cuore aveva gia seguito il ciclo dello sconforto, e poi del riconforto \\
Montale \t "I limoni" \\
Piazzuola \t luogo limitato, ludico, circoscritto \t non conosce il dolore, è un limite di protezione \\
Dimensione di gioia viene descritta anche con il danzare e i salti di gioco dei bambini \\
Operosita \t il zappatore che fisichia \\
"Tutto tace" \t personificazione e sinestesia \\
"odi la sega / del legnaiuol" \t enjambmement \t lucerna, energia elettrica \\
Operosita precede il giorno festivo \\
Esterno - esterno - interno ?? \\
Manca ora la parte riflessiva, cogitativa \\
"Diman" \\
Il giorno in cui si sperimenta la felicita \t che è correlata alla convivialita???
\end{document}
