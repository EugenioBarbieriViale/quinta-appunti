\documentclass[12pt]{article}

\usepackage[a4paper, total={6in, 8in}]{geometry}
\usepackage{textcomp}

\begin{document}
\setlength{\parindent}{0pt}

\def \t {\textrightarrow}
\def \v {\vspace{1em}}
\def \bi {\begin{itemize}}
\def \ei {\end{itemize}}
\def \s[#1] {\section*{#1}}
\def \ss[#1] {\subsection*{#1}}
\def \sss[#1] {\subsubsection*{#1}}

\s[Futurismo]
Movimento d'avanguardia italiano \\
Ultimo autore di prestigio italiano era stato canova \t italia rimane anche esclusa dalle innovazioni artistiche \t per esempio, no esperienze impressionista \\
Macchiaioi toscani \t anticipano un po il futurismo \t e anche la Scapigliatura \t ma non sono molto rilevanti 

\ss[Divisionismo]
Unico movimento di rilievo è stato il divisionismo \t portato a Milano da un gallerista e pittore (Vittore Grubicy), che raccoglie intorno a se degli studenti dell'accademia di brera \\
Si presentano al pubblico con una mostra nel 1895 \\
Divisionismo (post-impressionismo) \t recipisce il puntinimso francese = movimento artistico post-impressionista che non rifiuta l'impressionismo (come Van Gogh, Munch), ma consolida l'aspetto scientifoc del imp. e si concentra sulla teoria del colore \\
Il puntinismo fra proprio la scoperta del chimico Chevreul \t scopre che due colori primari accostati (come in due punti), da vicino si distinguono, da lontano si mischiano \\
Fenomeno della "mescolance optique" \t colore risultante si definisce direttamente nella retina dell'uomo \\
Il puntinismo non ha dinamismo, velocità \t richiedono un lavoro molto lungo, non come impressionisti \\
Il puntinismo arriva a Milano da Grubicy \t pittori come Pellizza da Volpedo, Segantini non usano i puntini, ma delle sottili pennellate \t lavoro molto lungo e complesso \\
I divisionisti usano una tecnica nuova \t ma i contenuti sono ancora tradizionali \t sono simbolisti o di denuncia sociale \t non si è ancora fatto il passo della rappresentazione della realta in italia \\
C'è ancora necessità di inserire dei simboli \t manca innovazione iconografica \\
In italia infatti prevale ancora l'accademia \t principalmente arte religiosa \t e sono anche le opere che vengono + comprate

\v

Futurismo è un movimento d'avanguardia e ha riconoscimento internazionale importante \t incontrano anche pittori di altre avanguardie \t e si influenzano a vicenda \\
Nasce a Milano \t città + avanzata d'italia in questo momento \\
Non è solo un movimento artistico, ma tocca tutti gli aspetti creativi \\
Primo protagonista è Marinetti \t che è un poeta, scrittore \t anche musica e poesia futurista \t poi si cercherà una ricostruzione della realtà futurista \\
3 punti:
\bi
    \item rifiuto radicale del passato
    \item esaltazione della modernità
    \item aggressività della rappresentazione artistica
\ei
Si presentano al pubblico con manifesti (non sono trattati, che si usano nella teoria artistica) \t testi molto sintetici \\
L'altra modalità sono le serate futuriste \t prima nel 1909, a Torino \t leggono poesie, si suona, si spiegano le idee futuriste \t è una performance \t la serata in se è un prodotto artistico \t anticipa l'arte contemporanea \\
Le serate vengono organizzate in modo che finiscano sempre in rissa \t tra i futuristi e gli spettatori, che sono i giovani borghesi \t poi arriva la polizia che arresta gli organizzatori \\
Futuristi infatti durante questi spettacoli (improvvisati) provocano il pubblico \t per esempio vendono lo stesso biglietto a + persone, e altri trucchi \\
Così da creare una tensione iniziale

\ss[Il manifesto di Marinetti]
Viene pubblicato in francese sul quotidiano "Figaro" nel 1909 \\
Presentazione di una nuova idea di bellezza \t il mondo della modernita contrapposto alla bellezza tradizionale \t la Nike di Samotracia \\
Il rifiuto del passato viene espresso in modo aggressivo \t "vogliamo distruggere i musei e combatteri contro il moralismo" \t il passato è da distruggere \\
Esaltazione della modernità anche nei suoi prodotti \\
La violenza e il vitalismo sono insiti nella bellezza \\
La guerra viene vista come la ?? \t "Igene della guerra", marinetti parla della necessità della guerra per generare un mondo nuovo \\
Vanno anche in guerra \t e Boccioni muore a Verona, Santeria muore in battaglia

\ss[Manifesto dei pittori futuristi]
Riprende i punti espressi da Marinetti e li applica alla pittura \t modernità è il soggetto dell'opera \\
Nausea che nasce dal rimanere ancorato al passato del contesto culturale italiano \\
Il contenuto è al punto 8 \\
L'arte deve essere legata alla contemporaneità, al contesto in cui si vive \t l'arte + significativa è questa

\ss[Manifesto tecnico]
Aspetto iconografico e come si deve dipingere \\
Probabilmente scritto da Boccioni \t teorico del futurismo, che tenta di definire il futurismo e formulare un pensiero teoirco futurista \\
Nel futurismo \t librazione, dinamismo universale \t oggetto di rappresentazione è ancora la realtà concreta \\
Futuristi non si concentrano tano sull'oggetto, ma sul suo rapporto con lo spazio \t oggetti si compenetrano in un'unica realtà dinamica \\
Spazio e oggetto vengono rappresentati in una realtà dinamica \t a cui si aggiunge l'aggressività del colore \\
Uso violento del colore puro \\
"Non può esistere pittura senza divisionismo" \t unica cosa che salvano della tradizione artistica italiana

\v

Futurismo è influenzato da Bergson, come cubismo \t ma il tempo cubista è un tempo lento, che attraverso la memoria rappresenta il passato e il presente \\
Il tempo futurista rappresenta passato e presente, ma anche il futuro \t futurismo rappresenta la continuità dinamica dell'oggetto \\
Cubismo si concentra invece sulla conoscenza dell'oggetto in se \\
Critici hanno individuato nel futurismo una fusione di:
\bi
    \item espressionismo: visione globale e dinamica
    \item cubismo: frammentazione della realtà stessa
\ei
\end{document}
