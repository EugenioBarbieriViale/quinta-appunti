\documentclass[12pt]{article}

\usepackage[a4paper, total={6in, 8in}]{geometry}
\usepackage{textcomp}

\begin{document}
\setlength{\parindent}{0pt}

\def \t {\textrightarrow}
\def \v {\vspace{1em}}
\def \bi {\begin{itemize}}
\def \ei {\end{itemize}}
\def \s[#1] {\section*{#1}}
\def \ss[#1] {\subsection*{#1}}
\def \sss[#1] {\subsubsection*{#1}}

\s[Avanguardie storiche]
Tra il 1905 e il 1924 \t alcune nascono dopo, come dadaismo \\

\ss[Inquadramento storico]
Periodo dal 1890 al 1914 viene definito Belle Epoque \t termine viene coniato dopo la prima guerra mondiale, con sentimento nostalgico per descrivere un periodo passato ma felice \\
È un'epoca bella perchè per molto tempo non ci sono guerre in euorpa (dal 1870 al 1914) \\
Poi c'è grande sviluppo commerciale ed economico \t si usa elettricita e petrolio \\
Inoltre ci sono delle innovazioni tecnologiche come la radio e la macchina, diffusione del gas e elettricità \\
Anche in campo medico: vaccino per la tubercolosi \\
La popolazione europea migliora il suo stato di benessere (non tutta la popolazione) \t ma il benessere si diffonde anche alla media e piccola borghesi \\
C'è pace pero con tensioni \t come tra francia e germania, russia e austria \t che si accumulano e si uniscono a:
\bi
    \item forte nazionalismo
    \item corsa agli armamenti
\ei
Tutto questo a un certo punto scoppia con la prima guerra mondiale \t inoltre c'erano anche pressioni sociali, perchè contadini e operai vivono ancora in condizioni misere 

\v

Prevale l'ottimismo positivistico di origine ottocentesca \t prevale la convinzione che ci sia un progresso inarrestabile e che quindi la scienza porterà a un maggiore sempre crescente \\
Questa è una visione ottocentesca che rimane \\
Ci sono innovazioni scientifiche/filosofiche che portano in realtà alla "crisi delle certezze" \t ci sono nuove interpretazioni dello spazio e della psiche umana \\
La realta si considera in modo diverso ora \t crollano le certezze sull'interpretazione di se stessi e della realta \\
Questi eventi sono:
\bi
    \item teoria della relatività di Einstein \t cambia rapporti tra spazio e tempo
    \item Freud \t 1901 pubblica interpretazione dei sogni \t legge psiche in modo diverso
\ei
Tradizionalmente la psiche era dove ragione e sentimento combattevano \t Freud introduce il concetto di inconscio \\
Inconscio = luogo mentale dove ci sono ricordi repressi ed esperienze di vita \t e determina le nostre scelte in modo inconsapevole \\
Bergson poi \t si pone il problema della conoscenza \t e conclude che la conoscenza è soggettiva e non oggettiva, perchè si inserisce la memoria \\
Filosofia di Nietzsche influenza molto \t artisti colgono alcuni aspetti:
\bi
    \item concetto di bellezza dionisiaca, contrapposta a quella apollinia \t bellezza libera da qualsiasi vincolo 
    \item idea che ci sia la necessita di costruire un mondo nuovo \t pero prima bisogna distruggere il mondo presente
\ei
Queste nuove idee si diffono in un contesto ristretto ed elitario \\

\ss[Arte]
L'accademia è sempre dominante \t chi compra opere si rivolge sempre li \\
L'impressionismo inizia ad essere apprezzato \\
Periodo è caratterizzato da due anime artistiche (non in contrapposizione) \t entrambe rifiutano l'accademia ma in modo diverso:

\sss[Art nouveau]
È l'espressione artistica della borghesia europea e si esprime nell'arte applicata (ovvero l'ordierno design) \\
Abbandona finalmente lo storicismo \t creando un nuovo mondo di forme, ispirate alla natura \\
Si sperimentano nuovi materiale e si comincia a progettare per l'industria \t nascono i primi "designer" \\

\sss[Avanguardi storiche]
Propono un'arte assolutamente rivoluzionaria \t mentre l'art nuoveau diffonde un gusto, le avanguardie sono dei gruppi elitari e poco conosciuti \\
Fanno però un lavoro radicale \\
Hanno dei caratteri comuni:
\bi
    \item ideologico: gli artisti si pongono in radicale opposizione alla societa occidentale e dei valori borghesi \t culto del denaro, ipocrisia morale, ...
    \item rifiuto di tutta la tradizione artistica occidentale \t quindi l'arte accademica, ma anche quella classica/rinascimentale/neoclassica...
\ei
Trovano nuove ispirazioni, che non hanno nulla di classico \t si ispirano all'arte preclassica, barbarica, al folklore contadino, all'arte dei bambini e dei pazzi (perchè sono "vergini culturalmente", cercano un'assenza di tradizioni) \\
Alcuni artisti del postimpressionismo (Van Gogh, Munch, Gaugin, Cezanne) \\
Una delle fonti è anche il primitivismo \t riprende arte dell'africa subshariana e dell'oceania \t con la conolizzazione dell'africa arrivano in Europa statue e maschere delle popolazioni \\
Avevano una funzione religiosa \t ma vengono ammirati per l'estrema sintesi formale e mancanza di naturalismo \t diventa una forte fonte di ispirazione \\
VEDERE FILE 

\v

Hanno aspetti in comune:
\bi
\item rifiuto della mimesis, ovvero del naturalismo: si rifiutano le proporzioni, l'anatomia e anche la prospettiva \t l'arte non deve imitare, ma deve creare una nuova realtà \t artista = creatore di mondi (1)
\item anticipitare il futuro: avanguardia è un termine militare (gruppo di soldati che esplorano territori nuovi prima dell'esercito) (2)
\ei

1) Artista è come un tronco di un albero  \t conferenza di Paul Klee a Jena 1924:
\bi
    \item le radici sono l'esperienza visiva e personale \t tutto ciò che determina l'artista come uomo
    \item dalle radici passa la linfa attraverso il tronco, che è l'artista
    \item dalla linfa, che viene dalle radici, produce la chioma che è qualcosa di completamente diverso (chioma = opera d'arte)
\ei

2) Arte si fa con tutto \t vengono inseriti degli oggetti reali nelle opere d'arte \t come i cubisti e dadaisti \\
Viene sperimentato anche il cinema, non per raccontare, ma come arte \t nasce così la videoarte \\
Nuove categorie di prodotto artistico \t può essere un'idea o un concetto \t si può anche separare chi realizza e chi ha l'idea (esempio cattalan, la statua davanti alla borsa) 

\v

Espressinismo, cubismo, astrattismo, dadaismo, futurismo, ... \\
Questi movimenti durano pochi anni \t e spesso gli artisti passano da un movimento all'altro \\
A volte spiegano anche la loro arte in dei manifesti (non in trattati teorici) \\
Artisti di diversi movimenti si incontrano con altri di altre avanguardie
\end{document}
