\documentclass[12pt]{article}

\usepackage[a4paper, total={6in, 8in}]{geometry}
\usepackage{textcomp}

\begin{document}
\setlength{\parindent}{0pt}

\def \t {\textrightarrow}
\def \v {\vspace{1em}}
\def \bi {\begin{itemize}}
\def \ei {\end{itemize}}
\def \s[#1] {\section*{#1}}
\def \ss[#1] {\subsection*{#1}}
\def \sss[#1] {\subsubsection*{#1}}

\s[Movimento moderno]
Realizza tra le 2 guerre un radicale rinnovamento dell'architettura \\
Si sviluppa in Europa, in particolare in Germania intorno alla scuola del Bauhaus, e la Francia, con Lecorbusier \\
Ma tutti artisti si rifanno a queste novita \t che vanno dal 1919 al 1933, anno in cui Hitler prende il potere e la scuola del Bauhaus viene chiusa \\
Rivoluzione architettonica nasce da una serie di esperienze che si sono svolte nell 800 e nella prima parte del 900 \\
Queste novita si differenziano dall'architettura del periodo, che è storicista ed eclettica (e quindi impostata sulla decorazione, legata al recupero degli stili del passato) \\
Sant elia nel suo manifesto critica molto storicismo ed eclettismo, e anche i critici in seguito definiscono il ballo in maschera dell'architettura \t l'architetto riduce la sua funzione a quella del decoratore \\
Esperienze precedenti:

\ss[L'architettura del ferro]
Si usa l'acciaccio e la ghisa e il vetro, prodotti industrialmente \t nasce la figura dell'ingegnere \\
Inoltre manca decorazioni \t ingegneri non hanno formazione artistica, quindi non decorano \\
Novita non incidono sulla produzione comune \t perche il ferro si sua per le tipologie nuove (padiglioni espositivi, stazioni ferroviarie, etc...)

\ss[Scuola di Chicago]
Gruppo di architetti che ricostruisce il centro della citta di Chicago negli ultimi 20 anni dell 800 \t citta era distrutta in incendio nel 1871 \\
Si inventano il grattacielo, per cui si usano i materiali nuovi, e mancano decorazioni (questo perche è difficile decorare un grattacielo, e gli architetti non riflettono su questo, anche perche la ricostruzione deve essere veloce) \\
Anche qua mancanza di decorazione deriva da una necessita / scelta incoscia \\
Non c'è una diffusione dell'idea di smettere con l'approccio decorativo \\
Architetti, come Sullivan, poi applicano l'approccio storicistico ai grattacieli 

\ss[Arts and Crafts]
Dopo esposizione universale del 1851 a Londra (la prima), si nota che i prodotti delle industrie europee sono bruttissimi \\
Si tenta di applicare al prodotto industriale, l'apporccio alla moda del momento (quindi stile storicista) \\
Quindi vengono applicate delle decorazioni ai prodotti (es. lampade gotiche) \t e quindi inizia campagna per miglioramento estetico dei prodotti industriali \\
Si sceglie di abbandonare lo storicismo \t ma applicazione non funzione \t es. Morris e altri organizzano ?? dove espongono oggetti come proposte per la produzione industriale \\
A volte usano decorazioni, ma non storiciste \t inoltre si aspirano al minimalismo giapponese ?? \\
Ma questi oggetti sono prodotti artigianalmente, e quindi non sono producibili industrialmente, che funzione diversamente \t è necessaria una progettazione per l'industria

\ss[Art Nouveau]
L'Arts and crafts viene ripresa nell Art Novuea, che si esprime nelle arti applicate ed elimina lo storicisimo \\
Si ispira al mondo naturale e lo ripropone in modo stilizzato \\
Architetti sono innovativi, ed sperimentano con il calcestruzzo armato \\
Innovano dal punto di vista progettuale (es. Gaudi, che progetta piante alveolari), e considerano l'edificio come un organismo da progettare in modo completo \\
Creano anche l'arredo, le maniglie, etc. \t rifiutano lo storicismo, a volte sostituendolo con decorazioni naturali o elimando completamente \\
Es. Adolf Loos afferma che l'ornamento è un delitto, e che la decorazione è un residuo di abitudini barbariche \\
Inoltre le costruzioni sono molto costose, e i loro committenti sono gli industriali (che si possono permettere dei costi di questo tipo) \\
Quindi le novita non si possono diffondere, e sono limitate in una committenza molto facoltosa

\v

Dopo la prima guerra mondiale, architetti riflettono sullo scopo e la funzione dell'architettura \t che non puo avere uno scopo ideale, ma deve essere concreto \\
Non si puo dire che lo scopo è raggiungere la bellezza o la perfezione, che sono astoriche ed apolitiche \t ma arch. deve risolvere i problemi della modernita \\
È quindi molto legata al contesto in cui si trova, che deve essere quello contemporaneo \t questo era anche stato espresso nel futurismo \\
Quindi architetti analizzano quali problemi l'architettura puo risolvere \t europa esce dalla 1WW, e c'è necessita di costruire \\
In germania soprattutto, c'è grande spostamento della popolazione \t e questo comporta la nascita delle periferie delle citta industriali \\
Gli operai vivono quindi in condizioni terrificanti \t le condizioni sono simili a quelle della prima riv. industriale in inghilterra \\
Si vuole quindi regolare lo sviluppo della citta \t e questo è il problema principale: bisogna concentrarsi su questo problema, ovvero dare un alloggio a tutta questa popolazione \\
Nella costituzione di Weimar, articolo 51, c'è scritto che lo stato deve garantire una abitazione decorosa a tutte le famiglie \\
Nasce quindi la committenza pubblica dello stato \t e questo comporta:
\bi
    \item bisogna dare alloggio a tutte queste persone
    \item deve essere decoroso ed avere degli standard
    \item deve essere economico per lo stato e per chi paga l'affitto
\ei
È necessario quindi un accordo con l'industria, che deve produrre degli elementi edilizi in serie \t che abbatte i prezzi  \\
E cosi gli architetti si devono interessare anche di urbanistica 

\v

La grande novita del movimento moderno è il nuovo metodo progettuale proposto, che ribalta il modo in cui si progetta \\
\textbf{La forma è il risultato dell'organizzazione razionale delle funzioni} \t nuovo metodo si puo esprime in questa frase \\
Avviene quindi al contrario \t prima: si parte dal terreno, e si costruiscono gli alloggi all'interno \\
Qua invece si analizzano le funzioni a cui l'edificio deve rispondere (in questo caso l'alloggio) \t l'appartamento deve servire per lavarsi, dormire, mangiare, soggiornare, etc. \\
Una volta definite le funzioni, bisogna tenere conto che si stanno costruendo delle case "popolari" \t quindi bisogna ridurre le metrature, ma garantendo uno standard qualitativo \\
Quindi si definiscono gli spazi minimi, ma adeguati, e li collego razionalmente, in modo da sprecare meno spazio possibile nei collegamenti (come corridoi etc..) \\
E poi aggrego gli appartamenti nel modo piu efficiente possibile \t e alla fine risulta la forma dell'edificio, quindi il percorso classico viene invertito \\
Si applica in realta anche all'urbanistica, a quello che sara il design, etc.

\s[Alexander Klein]
Architetto russo ed ebreo \t negli anni 20, fino al 33, lavora in Germania \t è uno dei protagonisti del movimento moderno \\
Lavora sugli insediamenti popolari che vengono costruiti ora \t ma fa uno studio + teoirco: definisce delle tipologie standard di alloggi, partendo dall'analisi funzionale \\
(leggere testo su classroom) \\
Ogni stanza deve avere una finestra, si cercano i rapporti minimi ottimali tra la superficie illuminante e la cubatura di una stanza, etc... \\
È importante che l'alloggio deve rispondere ai requisiti fisici ma anche psicologici \t alloggio deve garantire anche benessere psicologico, non solo igene etc. \\
Si studia quindi:
\bi
    \item l'antropometria (lo studio delle misure dell'uomo) 
    \item dell'ergonomia (che nasce in questi anni, per l'industria) = disciplina che studia il sistema uomo-macchina-ambiente\t se devo creare una catena di montaggio in una fabbrica, gli operai devono potersi muovere in modo efficiente, etc.
\ei
Klein studia anche l'orientamento ottimale delle stanze all'interno dell'edificio, tenendo conto dell'ambiente \t es. dire che soggiorno deve stare ad ovest e camere da letto ad est, cosi la mattina si ha il sole e nel pomeriggio anche \t per ottimizzare riscaldamento etc. (in Germania) \\
Si cerca il benessere, il risparmio, l'ottimizzazione

\v

Nel 33 va in Palestina, dove lavora e porta lì il linguaggio del movimento moderno \t a Tel Aviv, dove lavora di più, le case vengono costruite secondo il movimento moderno e lui partecipa molto alla costruzione \\
Poi, non d'accordo con le politiche di Israele, migrerà di nuovo negli USA

\s[La cucina di Francoforte]
Margarethe Schutte-Lihotzky è austriaca, lavora in Austria, Germania, poi sarà anche deportata in un lager \\
La sua progettazione + importante è la \textbf{cucina di Francoforte} \\
Progetta, per un grande insediamento di case popolari, una cucina (di cui tutte le case sono dotate) \t precede le cucine moderne \t in Italia arriva denominata come "la cucina americana", ma è tedesca 

\v

Le cucine dell'epoca \t erano polifunzionali: si mangiava, si soggiornva, si dormiva \\
Lei separa lo spazio di lavoro \t nella cucina si lavora, e poi il tavolo è fuori dalla cucina, che è uno spazio ristretto (con finestra) \\
Nella cucina ci devono essere tutti gli attrezzi dei lavori domestici \t per cui definisce gli elementi d'arredo (che sono standardizzati, con delle misure definite) \\
Parte quindi dall'antropometria, e quindi dalle misure della donna tedesca media \\
Posiziona gli arredi sulla base di studi ergonomici \t per rendere + efficiente il lavoro, e meno stancante \\
Crea quindi la prima cucina componibile? \t lavello, sgabello, ghiacciaia in cucina \t e gli elementi sono di legno, e tutti gli alloggi avranno questa cucina = prezzi bassi (per la produzione in serie) \\
Inoltre sono tutti verniciati di azzurro

\v

Il movimento moderno va dal 19 al 33 (quando Hitler sale al potere) \t e Hitler fa chiudere Bauhaus, e i vari architetti quindi vanno via dalla Germania (la maggior parte va negli USA) \\
Questa emigraizone è un passaggio importante \t il centro dell'elaborazione artistica occidentale passa da essere europea ad essere americana \\
Molti aristi, negli USA, lavorano molto \t e modernizzano gli USA

\ss[La questione formale]
Per ora analisi funzionale \t ma anche formale è importante \\
Gli edifici progettati, dal punto di vista formale, non hanno decorazioni, sono semplici ed essenziali \\
La qualità estetica deriva dai materiali (calcestruzzo, vetro, ferro) e l'articolazione volumetrica (e da quanto complessa è) \\
Gli architetti del mov. moderno si rendono conto del fatto che le novita che introducono non vengono colte, ma si colgono soltanto gli aspetti esterni \t senza capire il metodo progettuale \\
La ? dell'edificio deriva dalla qualita progettuale non dall'aspetto? \\
Quindi sensibilizzano, fanno mostre, scrivono su riviste \t e organizzano i CIAM (congressi, vedere acronimo) \\
Nei CIAM di solito scelgono un tema da approfondire, e poi producono un documento \t l'ultimo è del 33, dove si sviluppa la "Carta d'Aetene", in cui sono definiti i principi dell'urbanistica \\
Edifici senza decorazioni e semplici non sono proprio del movimento moderno, perchè manca tutto l'aspetto della progettualita che definisce la qualità ?

\v

La ZEN (Zona di Espansione Nord) di Palermo è un quartiere \t è molto bello dal punto di vista progettuale, ma poi è costruito con materiali scadenti, non è ben collegato alla città \t quindi diventa un ghetto \\
Questo provoca il degrado
\end{document}
