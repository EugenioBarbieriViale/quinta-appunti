\documentclass[12pt]{article}

\usepackage[a4paper, total={6in, 8in}]{geometry}
\usepackage{textcomp}

\begin{document}
\setlength{\parindent}{0pt}

\def \t {\textrightarrow}
\def \v {\vspace{1em}}
\def \bi {\begin{itemize}}
\def \ei {\end{itemize}}
\def \s[#1] {\section*{#1}}
\def \ss[#1] {\subsection*{#1}}
\def \sss[#1] {\subsubsection*{#1}}

\s[Espressionismo]
Raggruppamento di movimenti che operano in tutta Europa \t in francia ci sono i "Fauves" (belve) e in germania "Die Brucke" (il ponte) \\
Rfiuto della visione ottimistica della realta e della societa, e dell'ipocrisia morale borghese \\
I tedeschi vivono in realta una societa + rigida, quindi sentono piu bisogno di liberarsene \\
Voglio creare un espressione artistica spontanea e istintiva \t vogliono rivelare l'identita piu profonda, che attraverso l'arte si libera \\
Altri movimenti si concentrano sulla rappresentazione della realta \t ma non si vuole descrivere quello che si rappresenta, ma attraverso la realta si vuole mostrare la profondita del proprio essere \t è un mezzo \\
Per espressionismo arte oceanica e africana \\
Gaugin e Van Gogh per la francia, in germania Van Gogh e Munch per germania \t all'epoca poco conosciuti \\
Post impressionismo \t artisti partono da esperienze impressioniste, e poi prendono direzioni diverse \\
Impressionismo è opposto all espressionismo \t realta si imprime sul soggetto (rappresenta la propria percezione della realta), mentre espressionismo va nel senso opposto \t è il soggetto che esprime la realta

\s[Van Gogh]
\ss[Vita]
È olandese e figlio di un pastore protestante \\
Dipinge dal 1880 al 1890 circa \t arriva alla pittura, che diventa la sua ragione di vita, dopo una serie di esperienze delusorie (in campo sentimentale) \\
Poi prova ad entrare nella facolta di teologia, ma non viene ammesso \t vuole diventare evangelizzatore laico, ma è considerato fanatico e viene allontanato \\
Ha problemi psichiatrici \t probabilmente bipolare, ma non si sa \\
Ha un interiorita molto ricca che cerca di far uscire \t ma non riesce a relazionarsi col mondo \t l'unica persona che lo aiuta e sostiene economicamente sara suo fratello \\
La pittura per lui è l unico mezzo per relazionarsi col mondo 

\v

All'inizio si rifa al realismo (per i soggetti, ma non dal punto di vista formale) \t poi va a Parigi dove c'è suo fratello \\
Qui conosce l'ambiente impressionista \t momento importante dal punto di vista pittorico \\
Cosi scopre il colore, e comincia a dipingere con il colore puro \\
Cambia anche la tecnica pittorica, usando delle pennellate libere \\
Poi va nel sud della francia \t vive un periodo con Gaugin, che però è difficile \\
Poi si rende conto che sta male, e si fa ricoverare in una clinica psichiatrica \\
Nel 90 va a vivere in un paese della campagna parigina, presso un medico \t ma si suicida e muore a 40 anni \\

\ss[Pittura]
Rappresenta luoghi, anche la sua stanza e autorittratti \t è un modo per comincare se stesso, cosa che non riesce a fare direttamente \\
Usa modi formali come dinamismo, uso violento e innaturale dei colori, e la deformazione delle figure \\
Usa il colore in modo intenso \t colore ha una connotazione non descrittiva, ma più psicologica

\v

Autoritratto \t ha il viso quasi deformato, molto scavato \\
Poi la barba tende al rosso che in realta è biondo, occhi molto blu \\
Le grandi pennellate creano grande dinamismo, perche fa percepire il movimento delle pennellate \t sono disposte a raggiera verso l'esterno, mentre nello sfondo sono oblique 

\ss[Il grido di Munch]
Realizzato in serie, anche con tecniche diverse (oli, pastelli, stampe) \\
Anche questo è un autoritratto \\
Munch è norvegese e soggiorna per un periodo a parigi \t frequenta piu l'ambiente tedesco \\
Anche lui va in una clinica psichiatrica \t ma all inizio del 900 si isola in un fiordo norvegese, dove dipinge \\
Fa anche fatica a separasi dai suoi dipinti \t li considera come una parte di se 

\v

Figura in primo piano che sta urlando, tenendo le mani sulle orecchie per non percepire il proprio urlo \\
Si trova su un ponte, dove ci sono altre due figure \\
Il blu rimanda al fiordo, c'è una barca, ma sfondo non è descritto molto bene \\
Questo dipinto parte da un suo attacco di panico insieme a degli amici che non se ne erano accorti \t ma non è un racconto di un esperienza personale \\
Vuole raccontare un angoscia esistenziale \\
Dal punto di vista formale non c'è naturalismo \t la testa ricorda quasi un teschio, gli occhi sbarrati, la bocca un ovale, sproporzionato \\
Corpo sembra senza scheletro, deformato come un serpente \\
Prospettiva accidentale intuitiva, che fa percepire la profondita del ponte \t viene pero contradetta dall ambientazione che non ha rappresentazione spaziale \\
Movimenti ondulatori del fiordo e del cielo \t che smebrano propagare l'urlo \\
Uso del colore violento \t rosso giallo e blu, contrasto dei primari = contrasto + aggressivo realizzabile 

\s[Die Brucke]
Dal 1905 al 1913 \\
Protagonisti sono 4 studenti di architettura dell'accademia di Dresda, che decidono di darsi alla pittura \\
Architettura si faceva nelle accademie, non nei politecnici \t quindi aveva un taglio + artistico \\
Il ponte è un riferimento diretto a Nietzsche \t il ponte è un cavo teso vero un futuro completamente nuovo \\

\ss[Nudo di un quarto d'ora]
Si voleva creare in se stessi una verginita culturale, cancellando la propria formazione \\
Fanno quindi degli esercizi per rimuovere gli automatismo che avevano imparato \\
Riprendono il nudo \t che in realta è il tema classico per eccellenza \t e si studia molto all'accademia \\
Si danno quindi un tempo per dipingere e senza che i modelli stiano fermi \t si muovono \\
Poi creano l'ambientazione con l'acquarello \\
Le figure sono in movimento, sproporzionate e esemplificate \\
Questo serve proprio a cancellare la formazione accademica \\
Facevano questi esecizi di dipengere in un quarto d'ora per mettersi in una condizione mentale di rimuovere 

\ss[Marcella]
Olio su tela di Krichner \\
È ancora un nudo di una ragazza di nome Marcella \\
Queste ragazze abbandonate, che facevano una vita di strada, rappresentano la contraddizione tra l'ambiente e ancora segni di innocenza \\
Si copre, sembra sporgesi in avanti \t è ancora presente il pudore e imbarazzo della nudita \\
Il viso: ha gli occhi grandi forse truccati, la bocca grande \t non è un viso infantile, mentre ha un fiocco bianco da bambina \\
Non si capisce i limiti della stanza, non si capisce dove è seduta (semrba un tappetto ma se ha le gambe accavallate deve essere rialzato) 

\v

Formale: è un nudo con il corpo sproporzionato (es. la spalla), non è definita l'anatomia \t vediamo il contorno segnato da pennellate spesse, verdi e rosse \\
Accentuano gli spigoli e le parti rette, piu che quelle curvilinee \\
Viso schematico e quasi triangolare \t circondato da pennellate verdastre e pesnati, inoltre il verde crea una macchia tra i capelli \\
Nessun naturalismo, che si vede anche nello spazio \t non c'è prospettiva e non si capisce dove si trova \\
Colori aggressivi con contrasti cromatici non gradevoli (es. giallo senape acceso che abbate il rosino cipria) \\
Kirchner è solito cercare contrasti cromatici sgradevoli \t non cerca il bello e l'equilibrio 

\v

Centro percettivo è il viso \\
Linee compositive non si riescono a definire
\end{document}
