\documentclass[12pt]{article}

\usepackage[a4paper, total={6in, 8in}]{geometry}
\usepackage{textcomp}

\begin{document}
\setlength{\parindent}{0pt}

\def \t {\textrightarrow}
\def \v {\vspace{1em}}
\def \bi {\begin{itemize}}
\def \ei {\end{itemize}}
\def \s[#1] {\section*{#1}}
\def \ss[#1] {\subsection*{#1}}
\def \sss[#1] {\subsubsection*{#1}}

\s[Cubismo]
Si sviluppa in francia tra il ?? e ?? \t per 7 anni \\
Gli inventori sono Picasso e Braque \t lavorano insieme in perfetta sintonia \\
L'oggetto della ricerca pittorica è la realtà concreta delle cose \t vuole rappresentare l'oggetto, non sua percezione e non il proprio stato d'animo \\
Si differenzia dall'astrattismo \t che si esprime con forme e colori, si allontana dalla rappresentazione del reale \\
Fanno una critica sia all'espressionismo e dell attrattismo \t anche se riconoscono l'importanza dell'espressionismo, che è andato contro l'arte accademica e di cercare una struttura cromatica diversa dalla realtà \\
Criticano espressionismo perche fa dipendere la forma da aspetti contigenti (come lo stato d'animo, il tempo, ...) e dicono che è povero di contenuto, superficiale \\
Espressionismo fa dipendere la forma dall'identita dell'artista \t alcuni espressionisti poi passano anche al cubismo 

\v

Premesse del cubismo sono 3:

\ss[1. Primitivismo]
L'arte africana \t che vedono e riprendono (cone le maschere africane) 

\ss[2. Cezanne]
Cezanne \t è un artista post-impressionista, che è il precedente del cubismo \\
Nasce da una famiglia benestante in provenza e va a parigi \t si dedica alla pittura ma senza frequentare l'accademia \\
Entra a far parte degli impressionisti \t ma presto si differenzia \\
Lascia parigi e torna in provenza, dove rimane tutta la vita \\
Coglie alcuni aspetti fondamentali dell'impressionismo \t rifiuta accademia, niente disegno ma colore definisce la forma, pittura an-plen-air \\
Pero poi ritiene che la realta da rappresentare c'è bisogno di conoscenza + profonda \t bisogna cercare le forme originarie \\
Tutta la realta è riconducibile al cilindro, sfera e cono \\
Inoltre ritiene che bisogna usare anche gli altri sensi oltre la vista per rappresentare la realta \t bisogna dipingere anche gli odori \\
Fa anche delle lunghe passeggiate con un geologo \t per conoscere meglio la realta ??? \\
Fa un processo + razionale e + profondo \\
Non dipinge con le pennellate veloci, ma con le "pezzature di colore" \t sono come macchie di colore abbastanza grandi \\
Nel 1907 viene organizzata una mostra su di lui \t picasso e braque la guardano e rimangono affascinati dalle nature morte

\ss[Tavola di cucina]
Natura morta: tavolo con panno, della frutta, una brocca, cesto, caffettiera, in un interno di una stanza con una sedia e una credenza \\
Prevale la forma geometrica della sfera \t nella brocca e nella frutta \\
Inoltre le proporzioni non sono corrette \t una pera è enorme rispetto alle altre \\
Cezanne unifica punti di vista diversi nella stessa immagine \t ci sono punti di vista diversi \\
Non c'è corrispondenza tra il punto di vista della stanza e quello del tavolo \t non è coerente \\
Anche la brocca è vista dall'alto, cesto di fronte, tavolo dal basso \\
Ritiene che il dipinto non deve rappresentare la percezione dell uomo, ma deve seguire delle sue regole interne 

\v

Viene ripreso nel cubismo
\bi
    \item approccio razionale alla rappresentazione e alla volonta di conoscere la realta
    \item unificazione delle viste 
\ei

\ss[3. Bergson]
Filosofo e si pone il problema della conoscenza \t ritiene che per la conoscenza della realta non sono sufficienti le 3 dimensioni spaziali \\
Bisogna inserire fattore spaziale, ovvero la memoria \t significa inserire un aspetto soggettivo \\
4 dimensione che integra le 3 geometriche e la memoria \t la conoscenza si fa attraverso questa 4 dimensione \\
Questa visione influenza il cubismo

\ss[Ricerca della verita e non della somiglianza]
La prospettiva deforma sia le forme sia le dimensioni \t le proiezioni rappresentano oggettivamente \\
L'oggetto pero prima deve essere conosciuto \t e attraverso la 4 dimensione \\
Per conoscere l'oggetto bisogna osservalo in tutte i suoi aspetti \t non solo da un punto di vista \\
Inoltre oggetto deve subire un processo \t destrutturazione e poi libera ricomposizione

\v

3 fasi:
\bi
    \item protocubismo: 1907-8 \t influenza di cezanne, punto di partenza
    \item cubismo analitico: 9-11 \t destrutturazione
    \item cubismo sintetico: 11-14 \t libera ricomposizione
\ei

\s[Picasso]
+ grande artista del 900 \t stessa importanza di michelangelo nel 500 ed è una figura dominante \\
Vive 90 anni \t figura di confronto per tutti gli artisti del 900 \\
È spagnolo ed è un talento precoce \t entra a 14 anni nell accademia e va all accademia superiore di madrid \t a 16 anni dipingeva come raffaello \\
Poi va a parigi \t centro dell'arte occidentale \\
Nella sua vita ha diversi stili \t quasi eclettico \\
Dopo che franco sale al potere non torna piu in spagna 

\ss[Le Demoiselles d'Avignon]
Prostitute in un bordello \t ci dovevano essere anche degli uomini, poi cancellati \\
La sua non è una pittura veloce, ma molto lenta \t fase preparatoria e poi cambiamenti in corso d'opera \t processo molto meditato \\
Mentre impressionismo era istintivo \t qua no

\v

5 donne in un interno \t tavolino con natura morta, omaggio a Cezanne \\
Primitivsmo \t estrema semplificazione dei corpi, mancanza di proporzioni e nei visi che sembrano rifarsi alle maschere africane \\
Anche il colore \t alcune color legno \\
Molte parti del corpo sono rappresentate sotto punti di vista diversi unificati \\
Gli occhi non sono simmetrici, il naso ribaltato \t è di schiena ma il viso è di fronte \\
Anche ribaltamento dei piani \t una parte del corpo è aperta e distesa sulla tela \t gamba della donna a sinistra o il seno in alto a destra \\
Natura morta vista dall'alto, ambientazione che manca \t le tende chiudono tutto e non hanno un panneggio morbido, sembrano rigide \\
Rosso con rosa e giallo e il blu, ma i colori non sono particolarmente importanti \t il cubismo vuole un approccio razionale e non emotivo 

\ss[Ritratto di Ambrese Vollard]
\textbf{Analisi iconografica}: si vede il volto di una persona, e le spalle, è vestito con il colletto e la cravatta, taschino con la giacca \\
Sembra seduto \t sta leggendo e guarda verso il basso \\
Si vede anche una bottiglia a sinistra, e a destra una libreria \t forse si trova in uno studio \\
Lentamente si puo riconoscere tutto \t fase analitica, l'oggetto può essere destrutturato completamente \\
C'è tempo necessario per l artista di conoscere, ma anche da parte dell'osservatore \\
Serve un approccio razionale sia dall artista sia dall osservatore \\
Centro percettivo è la testa per il contrasto cromatico \\
Non ci sono linee compositive \t occhi si muove liberamente, non è guidato \\
La scelta dei colori \t si usano i grigi e le terne \t si punta alla razionalita \\
Inoltre in questa fase non firmano le opere \t non ci deve essere nulla di personale, solo razionale
\end{document}
