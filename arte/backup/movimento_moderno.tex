\documentclass[12pt]{article}

\usepackage[a4paper, total={6in, 8in}]{geometry}
\usepackage{textcomp}

\begin{document}
\setlength{\parindent}{0pt}

\def \t {\textrightarrow}
\def \v {\vspace{1em}}
\def \bi {\begin{itemize}}
\def \ei {\end{itemize}}
\def \s[#1] {\section*{#1}}
\def \ss[#1] {\subsection*{#1}}
\def \sss[#1] {\subsubsection*{#1}}

\s[Movimento moderno]
Realizza tra le 2 guerre un radicale rinnovamento dell'architettura \\
Si sviluppa in Europa, in particolare in Germania intorno alla scuola del Bauhaus, e la Francia, con Lecorbusier \\
Ma tutti artisti si rifanno a queste novita \t che vanno dal 1919 al 1933, anno in cui Hitler prende il potere e la scuola del Bauhaus viene chiusa \\
Rivoluzione architettonica nasce da una serie di esperienze che si sono svolte nell 800 e nella prima parte del 900 \\
Queste novita si differenziano dall'architettura del periodo, che è storicista ed eclettica (e quindi impostata sulla decorazione, legata al recupero degli stili del passato) \\
Sant elia nel suo manifesto critica molto storicismo ed eclettismo, e anche i critici in seguito definiscono il ballo in maschera dell'architettura \t l'architetto riduce la sua funzione a quella del decoratore \\
Esperienze precedenti:

\ss[L'architettura del ferro]
Si usa l'acciaccio e la ghisa e il vetro, prodotti industrialmente \t nasce la figura dell'ingegnere \\
Inoltre manca decorazioni \t ingegneri non hanno formazione artistica, quindi non decorano \\
Novita non incidono sulla produzione comune \t perche il ferro si sua per le tipologie nuove (padiglioni espositivi, stazioni ferroviarie, etc...)

\ss[Scuola di Chicago]
Gruppo di architetti che ricostruisce il centro della citta di Chicago negli ultimi 20 anni dell 800 \t citta era distrutta in incendio nel 1871 \\
Si inventano il grattacielo, per cui si usano i materiali nuovi, e mancano decorazioni (questo perche è difficile decorare un grattacielo, e gli architetti non riflettono su questo, anche perche la ricostruzione deve essere veloce) \\
Anche qua mancanza di decorazione deriva da una necessita / scelta incoscia \\
Non c'è una diffusione dell'idea di smettere con l'approccio decorativo \\
Architetti, come Sullivan, poi applicano l'approccio storicistico ai grattacieli 

\ss[Arts and Crafts]
Dopo esposizione universale del 1851 a Londra (la prima), si nota che i prodotti delle industrie europee sono bruttissimi \\
Si tenta di applicare al prodotto industriale, l'apporccio alla moda del momento (quindi stile storicista) \\
Quindi vengono applicate delle decorazioni ai prodotti (es. lampade gotiche) \t e quindi inizia campagna per miglioramento estetico dei prodotti industriali \\
Si sceglie di abbandonare lo storicismo \t ma applicazione non funzione \t es. Morris e altri organizzano ?? dove espongono oggetti come proposte per la produzione industriale \\
A volte usano decorazioni, ma non storiciste \t inoltre si aspirano al minimalismo giapponese ?? \\
Ma questi oggetti sono prodotti artigianalmente, e quindi non sono producibili industrialmente, che funzione diversamente \t è necessaria una progettazione per l'industria

\ss[Art Nouveau]
L'Arts and crafts viene ripresa nell Art Novuea, che si esprime nelle arti applicate ed elimina lo storicisimo \\
Si ispira al mondo naturale e lo ripropone in modo stilizzato \\
Architetti sono innovativi, ed sperimentano con il calcestruzzo armato \\
Innovano dal punto di vista progettuale (es. Gaudi, che progetta piante alveolari), e considerano l'edificio come un organismo da progettare in modo completo \\
Creano anche l'arredo, le maniglie, etc. \t rifiutano lo storicismo, a volte sostituendolo con decorazioni naturali o elimando completamente \\
Es. Adolf Loos afferma che l'ornamento è un delitto, e che la decorazione è un residuo di abitudini barbariche \\
Inoltre le costruzioni sono molto costose, e i loro committenti sono gli industriali (che si possono permettere dei costi di questo tipo) \\
Quindi le novita non si possono diffondere, e sono limitate in una committenza molto facoltosa

\v

Dopo la prima guerra mondiale, architetti riflettono sullo scopo e la funzione dell'architettura \t che non puo avere uno scopo ideale, ma deve essere concreto \\
Non si puo dire che lo scopo è raggiungere la bellezza o la perfezione, che sono astoriche ed apolitiche \t ma arch. deve risolvere i problemi della modernita \\
È quindi molto legata al contesto in cui si trova, che deve essere quello contemporaneo \t questo era anche stato espresso nel futurismo \\
Quindi architetti analizzano quali problemi l'architettura puo risolvere \t europa esce dalla 1WW, e c'è necessita di costruire \\
In germania soprattutto, c'è grande spostamento della popolazione \t e questo comporta la nascita delle periferie delle citta industriali \\
Gli operai vivono quindi in condizioni terrificanti \t le condizioni sono simili a quelle della prima riv. industriale in inghilterra \\
Si vuole quindi regolare lo sviluppo della citta \t e questo è il problema principale: bisogna concentrarsi su questo problema, ovvero dare un alloggio a tutta questa popolazione \\
Nella costituzione di Weimar, articolo 51, c'è scritto che lo stato deve garantire una abitazione decorosa a tutte le famiglie \\
Nasce quindi la committenza pubblica dello stato \t e questo comporta:
\bi
    \item bisogna dare alloggio a tutte queste persone
    \item deve essere decoroso ed avere degli standard
    \item deve essere economico per lo stato e per chi paga l'affitto
\ei
È necessario quindi un accordo con l'industria, che deve produrre degli elementi edilizi in serie \t che abbatte i prezzi  \\
E cosi gli architetti si devono interessare anche di urbanistica 

\v

La grande novita del movimento moderno è il nuovo metodo progettuale proposto, che ribalta il modo in cui si progetta \\
\textbf{La forma è il risultato dell'organizzazione razionale delle funzioni} \t nuovo metodo si puo esprime in questa frase \\
Avviene quindi al contrario \t prima: si parte dal terreno, e si costruiscono gli alloggi all'interno \\
Qua invece si analizzano le funzioni a cui l'edificio deve rispondere (in questo caso l'alloggio) \t l'appartamento deve servire per lavarsi, dormire, mangiare, soggiornare, etc. \\
Una volta definite le funzioni, bisogna tenere conto che si stanno costruendo delle case "popolari" \t quindi bisogna ridurre le metrature, ma garantendo uno standard qualitativo \\
Quindi si definiscono gli spazi minimi, ma adeguati, e li collego razionalmente, in modo da sprecare meno spazio possibile nei collegamenti (come corridoi etc..) \\
E poi aggrego gli appartamenti nel modo piu efficiente possibile \t e alla fine risulta la forma dell'edificio, quindi il percorso classico viene invertito \\
Si applica in realta anche all'urbanistica, a quello che sara il design, etc.

\newpage
\s[Alexander Klein]
Architetto russo ed ebreo \t negli anni 20, fino al 33, lavora in Germania \t è uno dei protagonisti del movimento moderno \\
Lavora sugli insediamenti popolari che vengono costruiti ora \t ma fa uno studio + teoirco: definisce delle tipologie standard di alloggi, partendo dall'analisi funzionale \\
(leggere testo su classroom) \\
Ogni stanza deve avere una finestra, si cercano i rapporti minimi ottimali tra la superficie illuminante e la cubatura di una stanza, etc... \\
È importante che l'alloggio deve rispondere ai requisiti fisici ma anche psicologici \t alloggio deve garantire anche benessere psicologico, non solo igene etc. \\
Si studia quindi:
\bi
    \item l'antropometria (lo studio delle misure dell'uomo) 
    \item dell'ergonomia (che nasce in questi anni, per l'industria) = disciplina che studia il sistema uomo-macchina-ambiente\t se devo creare una catena di montaggio in una fabbrica, gli operai devono potersi muovere in modo efficiente, etc.
\ei
Klein studia anche l'orientamento ottimale delle stanze all'interno dell'edificio, tenendo conto dell'ambiente \t es. dire che soggiorno deve stare ad ovest e camere da letto ad est, cosi la mattina si ha il sole e nel pomeriggio anche \t per ottimizzare riscaldamento etc. (in Germania) \\
Si cerca il benessere, il risparmio, l'ottimizzazione

\v

Nel 33 va in Palestina, dove lavora e porta lì il linguaggio del movimento moderno \t a Tel Aviv, dove lavora di più, le case vengono costruite secondo il movimento moderno e lui partecipa molto alla costruzione \\
Poi, non d'accordo con le politiche di Israele, migrerà di nuovo negli USA

\s[La cucina di Francoforte]
Margarethe Schutte-Lihotzky è austriaca, lavora in Austria, Germania, poi sarà anche deportata in un lager \\
La sua progettazione + importante è la \textbf{cucina di Francoforte} \\
Progetta, per un grande insediamento di case popolari, una cucina (di cui tutte le case sono dotate) \t precede le cucine moderne \t in Italia arriva denominata come "la cucina americana", ma è tedesca 

\v

Le cucine dell'epoca \t erano polifunzionali: si mangiava, si soggiornva, si dormiva \\
Lei separa lo spazio di lavoro \t nella cucina si lavora, e poi il tavolo è fuori dalla cucina, che è uno spazio ristretto (con finestra) \\
Nella cucina ci devono essere tutti gli attrezzi dei lavori domestici \t per cui definisce gli elementi d'arredo (che sono standardizzati, con delle misure definite) \\
Parte quindi dall'antropometria, e quindi dalle misure della donna tedesca media \\
Posiziona gli arredi sulla base di studi ergonomici \t per rendere + efficiente il lavoro, e meno stancante \\
Crea quindi la prima cucina componibile? \t lavello, sgabello, ghiacciaia in cucina \t e gli elementi sono di legno, e tutti gli alloggi avranno questa cucina = prezzi bassi (per la produzione in serie) \\
Inoltre sono tutti verniciati di azzurro

\v

Il movimento moderno va dal 19 al 33 (quando Hitler sale al potere) \t e Hitler fa chiudere Bauhaus, e i vari architetti quindi vanno via dalla Germania (la maggior parte va negli USA) \\
Questa emigraizone è un passaggio importante \t il centro dell'elaborazione artistica occidentale passa da essere europea ad essere americana \\
Molti aristi, negli USA, lavorano molto \t e modernizzano gli USA

\ss[La questione formale]
Per ora analisi funzionale \t ma anche formale è importante \\
Gli edifici progettati, dal punto di vista formale, non hanno decorazioni, sono semplici ed essenziali \\
La qualità estetica deriva dai materiali (calcestruzzo, vetro, ferro) e l'articolazione volumetrica (e da quanto complessa è) \\
Gli architetti del mov. moderno si rendono conto del fatto che le novita che introducono non vengono colte, ma si colgono soltanto gli aspetti esterni \t senza capire il metodo progettuale \\
La ? dell'edificio deriva dalla qualita progettuale non dall'aspetto? \\
Quindi sensibilizzano, fanno mostre, scrivono su riviste \t e organizzano i CIAM (congressi, vedere acronimo) \\
Nei CIAM di solito scelgono un tema da approfondire, e poi producono un documento \t l'ultimo è del 33, dove si sviluppa la "Carta d'Aetene", in cui sono definiti i principi dell'urbanistica \\
Edifici senza decorazioni e semplici non sono proprio del movimento moderno, perchè manca tutto l'aspetto della progettualita che definisce la qualità ?

\v

La ZEN (Zona di Espansione Nord) di Palermo è un quartiere \t è molto bello dal punto di vista progettuale, ma poi è costruito con materiali scadenti, non è ben collegato alla città \t quindi diventa un ghetto \t questo provoca il degrado

\newpage
\s[Bauhaus]
Il movimento moderno si compie nel Bauhaus e in Corbusier \\
Bauhas è una scuola di architetti fondata nel 19 da Walter Gropius \\
È molto innovativa a livello didattico \t all'epoca gli architetti e gli artigiani erano formati nell'accademia \\
La formazione era quindi di tipo artistico \t e nello specifico dell'architettura, si lavorava sulla copia \\
Gli artigiani invece si formavano nelle scuole di arti e mestieri, in cui la formazione era solo pratica \t imparavano a realizzare degli oggetti \\
Bauhaus è una scuola sia per architetti sia per artigiani \t e parte con il semestre propedeutico, in cui tutti gli studenti sono insieme e studiano:
\bi
    \item teoria della forma \t effetti delle forme a partire dalle forme primarie (cerchio, quadrato, triangolo)
    \item la teoria del colore
    \item la teoria dei contrasti \t dei contrasti tra colori, superifici, etc.
\ei
Questo si fa nella pratica \t gli studenti realizzano dei progetti di tutti i tipi \\
Si studiano queste 3 materie perche sono una presa di consapevolezza di fattori che non sono solo estetici, che poi sono alla base della progettazione di qualsiasi oggetto \\
A questi corsi sono associati altri corsi come storia dell'arte, disegno geometrico, studio ritmico del nudo (non come in accademia, in cui c'è il modello che viene riprodotto con attenzione all'anatomia, ma è ritmico \t uomo studiato nel movimento) \\
Poi c'è esame a fine del semestre \t passano il 50 percento e continuano studio o nella scuola di architettura, o laboratori, che intorno al 26 si stabilizzano \t sono 6:
\bi
    \item laboratorio della tessitura
    \item della tipografia
    \item del legno \t poi del mobile
    \item  del metallo
    \item di scultura
    \item di decorazione parietale
\ei
Ogni laboratorio ha due maestri: 
\bi
    \item uno è il maestro d'arte \t si occupa della parte teorica
    \item l'altro è il maestro artigiano \t parte pratica
\ei
Lavoro è collaborativo e si realizzano oggetti reali \t collaborativo perche studenti e maestri lavorano insieme \\
Oggetti sono per la produzione industriali, che poi sono brevettati e poi venduti \t cosi anche per autofinanziarsi \\
Bauhaus è scuola pubblica pero \\
Alcune industrie comprano i brevetti di questi oggetti, e poi li producono \\
Questa scuola viene osteggiata dagli ambienti accademici locali \t è troppo innovativa, e la vicinanza tra insegnanti e studenti era troppa, e vengono accettate anche le donne (anche se finiscono sempre nei laboratori di tessitura) \\
Alcune riescono ad entrare anche negli altri laboratori \\
La qualità dell'insegnamento è altissima \t insegnano gli artisti migliori dell'epoca in Germania \t anche Kandinsky insegna qua, Mondrian tiene delle conferenze \\
Infatti è un ambiente molto vivo, in cui ci sono spettacoli e conferenze \\
Tutto questo non piace \t e a un certo punto viene chiusa nel 24/25, e riapre a Dessau

\v

Dessau è una zona industriale \t e la scuola viene praticamente chiamata dall'industria, che vuole personale formato \\
La scuola costruira anche l'edificio proprio \\
Verra chiusa nel 33 definitivamente da Hitler \t perche la considera un covo di bolscevichi \t anche se Gropius cerca di non dare connotazione politica alla scuola, ma alla fine proprio la struttura della scuola è troppo democratica ed aperta \\
I vari artisti migrano poi negli stati uniti

\v

Ci sono due luoghi nei quali, secondo i critici, nasce il design moderno \\
Design = contrazione di disegno industriale \t in sè vuol dire progettazione per l'industria \\
I luoghi sono due:
\bi
    \item Bauhaus
    \item Ford \t industria automobilistica americana
\ei
Le automobili che produceva, prima del lavoro di Ford, erano oggetti per ricchi amatori \t prezzi enormi, e li poteva comprare chi aveva capacita meccanica \t si rompevano spesso \\
Erano oggetti di lusso per pochi con capacita meccanica \\
Ford vuole invece produrre automobili che i suoi operai possano permettersi \t vuole produrre qualcosa di completamente diverso, a basso costo, e che non fossero necessarie delle competenze da meccanico per usarle \\
Crea quindi la catena di montaggio \t e semplifica al massimo l'oggetto, che viene prodotto in serie \t e si progettano delle automobili di alta qualita che durano nel tempo \\
Produce all'inizio un solo modello, il modello T \t ogni variazione porterebbe costi aggiuntivi \t questo perche ha un obbiettivo commerciale di venderne tantissime (non scopo ideale)

\v

In realta queste sono le stesse qualita degli oggetti di Bauhaus \t questi 3 elementi sono in realtà legati tra loro
\bi
    \item qualita del prodtto, che deve durare nel tempo \t il contrario dell'obsolescenza programmata
    \item basso costo, e quindi produzione in serie
    \item essenzialità formale
\ei
Qua pero non c'è obbiettivo di commercializzazione \t piu scopo democratico di dare a tutti oggetti validi che costano poco \t i risultati pero rispondo alle stesse caratteristiche

\ss[Laboratorio del mobile]
\sss[Sedia di breuer 1]
Parte come laboratorio del legno \\
Sedie realizzate con tubi di metallo \t quelli delle biciclette \t tubo d'acciaio curvato e saldato a caldo o a freddo (con delle viti) \\
Qua c'è grande innovazione tipologica \t non ha le quattro gambe, ma è sospesa \t nelle foto sono in pelle, mentre originariamente il tessuto era resistente di canapa \\
C'è anche studio per renderla il più confortevole possibile 

\sss[Sedia Wassily (sempre di Breuer)]
Si chiama cosi per omaggio a Kandisky (che si chiama Wassily) \\
Questi sono 6 pezzi saldati a freddo \t i braccioli, lo schienale etc. sono di tessuto \\
Interessante pensare anche al trasporto dell'oggetto \t si può smontare e quindi agevola il trasporto \\
La progettazione del design non riguarda sempre solo l'oggetto in se, ma anche la sua commercializzazione 

\sss[Poltrona Barcellona]
Realizzata da van der Rohe \t usa due pezzi di metallo che si intersecano \t sono curvilinei e a croce, dove poi si appoggiano i cuscini

\ss[Laboratorio del metallo]
Viene diretto da Marianne Brandt \t donna che riesce a diventare maestra \\
Si concentra sulle lampade \\
Inventa la prima lampada orientabile, oppure la sali-scendi \t lampada da soffitto con lungezza variabile \\
È presente una grande semplificazione, e utilizzati solidi di rotazione compenetrati tra loro \\
Teiera \t che non ha innovazione tipologica, ma molto estetica

\newpage
\s[Nuova sede del Bauhaus a Dessau]
È il manifesto dell'architettura moderna \t ci sono tutte le caratteristiche tipiche \\
Il metodo progettuale è legato alla progettuazione funzionale \t la forma viene descritta dalla funzione, quindi ci deve essere uno studio delle funzioni \\
Questa è una scuola \t ha bisogno degli spazi necessari: aule, laboratori, uffici, alloggi per gli studenti, la mensa, spazi per la vita comune e un teatro \\
Si organizzavano al di la del percorso didattico degli spettacoli realizzati dagli studenti, che progettavano tutto, dalla sceneggiatura ai costumi \\
E invitavano la popolazione locale ad assistere

\ss[Aspetto distributivo]
Dall'assonometria si possono riconoscere i diversi volumi \t si nota subito l'originalità perche per la prima volta si us a un lotto di terra in modo libero \\
Solitamente si decide di definire il perimetro e di organizzarci dentro gli spazi \t qua invece c'è uso libero dello spazio \\
Le posizioni vengono studiate a partire dalle diverse funzioni che devono essere svolte e degli orientamenti \\
Qua la struttura è come se fosse una doppia L con un perno nel punto di intersezione \\
Ci sono 5 volumi in realta, che sono suddivisi per funzioni:
\bi
    \item quello in fondo è il volume dei laboratori
    \item a destra ci sono le aule
    \item nella struttura di raccordo a ponte (sotto c'è la strada di accesso alla scuola) ci sono gli uffici
    \item a sinistra c'è un blocco basso dove ci sono la cucina, la mensa e il teatro \t che prosegue con un altro volume con gli spazi di soggiorno
    \item il volume piu alto è lo studentato
\ei
I volumi sono di diverse forme e altezza e sono collegati razionalmente \\
Nelle piante si possono vedere che ci sono diverse altezze

\ss[Il calcestruzzo armato]
Viene usato il calcestruzzo armato come materiale fondamentale, e il vetro e gli infissi in acciaio \\
Leggere parte sul calcestruzzo armato \\
Questo materiale viene messo a punto nella fine dell 800 e sperimentato all inizio del 900, per poi diventare il materiale d'eccellenza per costruzione \\
Materiale composto con calcestruzzo e e acciaio \\
Calcestruzzo era stato inventato dai romani \t era flessibile formalmente ed è composto \\
Il calcestruzzo resiste alle forze di compressione, mentre acciaio di torsione ed estensione \t quindi insieme molto resistente \\
Il calcestruzzo viene miscelato industrialmente \t nella bettoniera viene mischiato insieme all'acqua, viene calato nelle casseforme e poi si soldifica \\
A questo si aggiunge l'armatura, costituita da tondini d'acciaio di senzione variabile (fra i 3 e 8 mm) a seconda delle esigenze \\
Ha elementi verticali e delle scaffe che li tengono insieme \t cassaforma, in cui vengono inseriti elementi verticali e scaffe, e poi calcestruzzo \\
All'inizio i tondini erano lisci \t poi vengono fatti zigrinati per migliorare l'aderenza \\
Questo sistema permette di costruire edifici alti ed ampi con la garanzia di grande liberta \t si torna a un sistema architravato, in cui elementi verticali sostengono quelli orizzontali in un sistema solidale \\
Dal sistema architravato a quello archivoltato romano, si aveva il problema che non si riesce a creare spazi ampi perche se no servivano motli elementi verticali \\
Quindi avevano creato le volte \t qua invece, grazie al materiale, per grandi spazi servono meno pilastri \\
Per esempio, arretrando i pilastri rispetto al perimetro, si puo trattare il perimetro senza vincoli \\
Calcestruzzo armato garantisce liberta progettuale

\v

Prima foto è il volume dei labroatori \t con una grande parte in vetro, che si ottiene arretrando i pilastri rispetto al perimetro \\
Foto sopra i materiali: si vede lo studentato a sinistra, in cui le stanze degli studenti sono abbastanza grandi e hanno una grande finesra, un porta finestra e un balcone \\
Dai prospetti si vede che non c'è una facciata principale \t non c'è facciata trattata specialmente

\ss[Aspetto formale]
La qualita formale deriva dall'articolazione volumetrica \t gli edifici hanno volumi diversi, le facciate sono variegate e tutte diverse, dal dinamismo dell'insieme (girando attorno all'edificio, si hanno continue visuali diverse) \\
Non c'è alcun tipo di decorazione \t le pareti vengono trattate con l'intonaco bianco/in toni grigi \t molto essenziale \\
L'unica "decorazione" è la scritta "Bauhaus" \\
Ancora adesso è usato come sedie universitaria per Master e PhD architettonici \\
Ci sono anche le case dei professori \t piu vicine al bosco, e sono come delle villette \t in una vivevano Klee e Kandisky

\newpage
\s[Le Corbusier]
Le C. è uno delle figure fondamentali del movimento moderno \t in realta ha un altro nome, ma apre un architetto con suo cugino e quindi per differenziarsi cambia \\
Si ritrovano nei suoi lavori l'accordo con l'industria (che è molto stretto in lui, che definisce la casa una macchina da abitare), razionalita funzionale, e ragionamento ??

\ss[Vita]
Lui è anche pittore, urbanista, designer, ma anche teorico e scrive dei trattati \t ed è il principale animatore dei CIAM \\
È svizzero, ma poi si trasferisce in Francia \t lavora e studia a Parigi etc. \\
Non ha laurea in architettura ma ha una formazione superiore di tipo artistico \t come se avesse frequentato il liceo artistico \\
Negli studi superiori viene avviato all'architettura \t e una volta diplomato fa un lungo viaggio in europa, e va anche in italia \\
Soprattutto va nell'est europa, per fermarsi a berlino per 6 mesi \t qua fa uno stage presso uno studio di architettura presso Peter Behrens \\
Behrens è considerato uno degli immediati precedenti del movimento moderno \t usa il calcestruzzo armato, elimina le decorazioni e inizia un approccio allo studio funzionale \\
Vive quindi 6 mesi in uno studio di architettura estremamente innovativo \t e nello stesso periodo erano in quello studio Walter Gropius e ??? \t maggiori esponenti del movimento moderno \\
Condividono quindi una parte piccola ma importante della loro formazione \\
Poi a Parigi va presso lo studio di Auguste Perret \t che è un architetto di ambito Art Noveau \t ed è uno dei + importanti sperimentatori del calcestruzzo armato \\
Poi a 30 anni a Parigi apre studio con il cugino

\ss[Verso una architettura]
Lui pubblica il suo primo testo nel 23 e si intotla "Verso una architettura" \t pubblicazione molto contenuta, che non è proprio un trattato \t i trattati (come quelli di Leon Battista Alberti) hanno una struttra che parte sempre da riflessioni culturali e filosofiche \\
Lui invece fa un discorso estremamente pratico \t definisce che l'alloggio deve essere prodotto in serie come l'automobile \t e poi definisce i 5 punti dell'architettura e le caratteristiche che un edificio deve avere:
\bi
    \item "pilotis" \t pilastrini, deve essere sollevato da terra per lasciare spazio libero + possibile da sotto, e si protegge l'edificio dall'umidità
    \item pianta libera \t il calcestruzzo armato permette di poter avere pochi pilastri, e quindi di poter progettare liberamente le piante e di avere piante diverse ai diversi piani
    \item facciata libera \t muri non sono + importanti e i pilastri sono portanti \t possono quindi essere arretrati e si puo fare quello che si fa
    \item finestre a nastro \t le abbiamo viste nel bauhaus (che pero viene dopo, questo 23 bauhaus 26) \t finestre continue per avere illuminazione
    \item tetto giardino \t tetto deve essere piano per essere sfruttato a terrazza e a giardino
\ei
Tutti qeusti punti derivano dalla tecnologia del calcestruzzo armato

\ss[Villa Savoye]
A Poiussi, e Savoye è il cognome del committente \t Villa vuol dire villa, ville invece città \\
Savoye si fa costruire una casa in campagna per la sua famiglia e sono in tre (sposa e figlio) \\
La prima cosa che si nota all'esterno è la razionalita geometrica (quasi rinascimentale) \t parallelepipedo sollevata su pilastri \\
Ma poi geometria viene interrotta da grande articolazione spaziale negli interni e contrapposizione tra linee rette del parallelepidpedo e muri curvilinei interni \\
Non c'è raccordo con l'ambiente che lo circonda \t non viene integrato, ma una struttura quasi astratta che si posiziona in mezzo al verde \\
Ci sono i pilotis

\sss[Analisi distributiva]
La pianta è  libera \\
Al piano terra c'è un garage (con 3 auto), lavanderia, appartamento per l'autista e collegamenti verticali \\
Elementi verticali con scala elicoidale e una rampa \\
Primo piano è quello dell'appartamento \t ma non è tutta occupata e imparte c'è un patio (terrazza interna) \t dalla disposizione degli ambienti è libera e i pilotis \\
La parte chiusa è studiata in modo razionale con divisione rigorosa tra zona giorno e notte \\
C'è una sala enorme che si affaccia sul patio con delle porte a vetri, e c'è la cucina, balconcino di servizio, poi le stanze da letto e l'abiente di servizio \\
Il tetto giardino viene usato come solarium e come stenditoio, e poi ci sono i vasi con le piante \t sopra presenta una struttura curvilinea in calcestruzzo, che serve a creare privacy e per creare delle ombre \t se no tetto è completamente al sole \\
Le facciate sono libere, perchei pilotis sono arretrati e vengono trattate con la finestra nastro \t le quattrofacciate del parallelepipedo sembrano tutte uguali, ma la parte del patio non ha finestre \\
I collegamenti verticali sono una scala elicoidale, che è interna, mentre la rampa si sviluppa dall'interno al patio e poi all'esterno \\
Il salone ha porte a vetro che danno sul patio \t ma cosa è una rampa? è un collegamento verticale che ha una pendenza ma non ha gradini \\
Il percorso su una rampa è piu lungo e lento rispetto a una scala \t ma Le C. ama le rampe, attraverso le quali puo creare delle passeggiate architettoniche \t effettivamente cosi si creano molte visuali diverse

\v

Dal punto di vista formale \t tutto è intonacato di bianco senza nessuna decorazione \t e ci sono i contrasti tra le strutture curvilinee al piano terra e la terrazza con geometria razionale, e con semplicita esterna vs. complicatezza interna

\ss[Unitè d'habitation]
Realizzato da Le Corbusier tra il 1947 e il 1952 \t dopo la seconda guerra mondiale \\
La francia è distrutta è c'è una necessita urgente di costruire per dare un alloggio a tutti \t e si crea il ministro della ricostruzione, che si rivolge a Le Corbusier \\
Bisogna costruire in fretta e i costi dei materiali, dei terreni, etc. stanno aumentando \t bisogna risparmiare \\
Le Corbusier mette a punto un progetto che è il risultato di anni di studio e di lavoro \t e presenta "l'unità di abitazione", che puo essere replicata in tutti i luoghi \t porpone un modello di edificio replicabile \\
Il primo è quello di marsiglia \t poi altri edifici vengono costruiti in altre citta \\
È un edificio enorme, ma non è un normale condominio \t viene pensato come unita abitativa della citta contermporanea

\v

Urbanistica di Le Corbusier \t la citta deve:
\bi
    \item essere verticale: deve svilupparsi in verticale e non orizzontale \t per lasciare il territorio al giardino
    \item essere giardini: molto verde
\ei
Non ci sono qui solo degli alloggi \t ma è un intero isolato che si sviluppa verticalmente, e all'interno ci sono "i prolungamenti dell'alloggio", ovvero una serie di servizi (dal mercato alla scuola dell'infanzia)

\sss[Struttura]
Ci sono 17 piani + tetto-giardini e 7 strade interne con tutti i servizi \t è pensata per 1600 persone \\
Questo ricorda i falansteri (che appartengono alla tradizione francese, Fourier) \t dovevano ospitare circa 1600 persone e dovevano avere servizi, ma non c'è abolizione della prop. privata etc. \\
Il target era la classe media \\
Il materiale è il calcestruzzo armato, lasciato però grezzo \t non viene intonacato, quindi si vedono i segni della cassaforma \t questa è una scelta estetica \\
"Beton brut" \t calcestruzzo lasciato grezzo \t e a partire dagli anni 50 nasce il brutalismo, che ha origine in questa scelta di Le Corbusier e dal nome del materiale lasciato grezzo in francese \\
Ci sono i 5 punti \t non ci sono i pilotis, ma i pilastri comuni (che sono di grandi dimensioni, a tronco di cono rovesciato) \t e sono arrtrati ripsetto le facciate \\
Quindi i muri perimetrali non sono portanti, e la facciata è trattabile liberamente e non vincolata agli elementi costruttivi \\
C'è una finestratura continua, non proprio a nastro \t continuita di enormi vetrate assimilabili a finestre a nastro \\
Tetto è piano ed è un giardino \t e c'è la pianta libera \t qui vuol dire studio complesso di quella che è la distribuzione degli alloggi \\
Le Corbusier studia l'alloggio nel corso degli anni della guerra, in cui non lavora \t e scriverà anche un testo 

\sss[Gli alloggi]
Li pensa come case unifamiliari a due piani \t sono dei duplex, e senza muri in comune \t solo intercapedini con materiale assorbente, che risulta in isolamento acustico \\
Spesso hanno 1 o 2 logge esterne \t si ha spazio aperto \\
Per progettare la pianta di questi alloggi definisce dei blocchi funzionali:
\bi
    \item blocco cucina-soggiorno
    \item con la scala si arriva alla zona notte, dove: blocco della camera matrimoniale con bagno, e blocco della camera singola raddoppiabile con bagno \t entrambi con armadiature
\ei
All'epoca non c'erano i bagni interni \t quindi è estremamente avanzato \\
C'è chiara divisione zona notte e giorno \t e ci sono due logge esterne \\
Altezza cucina e zona notte sono 226, per un tot. di 480 \t oggi l'altezza minima è 270 \t ma non ci si sente oppressi, perche c'è anche una zona alta 480 con una vetrata \t quindi non c'è percezione di spazio chiuso \\
Questa misura però non è casuale \t ci arriva con una serie di studi, che spiega nel testo "Modulor" (modul e section d'or = modulo + sezione aurea) \\
Qui studia una serie di misure adattate al corpo umano, basandosi sulla sezione aurea \t su cui si basano dall'antichita i rapporti delle misure umane \\
Prende un uomo altro 183, che parte con la mano alzata e si arriva a 226 \t poi prende misure dai piedi a ombelico, da li alla testa e da li alla mano alzata \t e se poi si dividono, esce il rapporto aureo \\
Da qui mette a punto un sistema di misure, che fa anche per l'uomo anche in diverse posizioni, e crea un sistema dimensionale che poi applica a tutti gli aspetti dell'alloggio (altezza corrimano, mobili etc.)

\sss[Cucina]
È progettata in dettagli da Charlotte Periandj \t è come la cucina di Francoforte, per cui si basa su elementi componibili distribuiti in modo razionale ed efficiente \\
Ci sono tutti gli utensili, ed è avanzata (ha cestino per rifiuti per es.) 

\sss[Caratteristiche dell'alloggio]
Ha parque, doppi vetri (isolano dal rumore e mantengono temperatura), infissi di legno \\
C'è riscaldamento sia centralizzato (con tubi nelle pareti), sia regolazione personale sotto la loggia

\sss[Esterno]
La logga è profonda 1 metro e mezzo ed è quindi utilizzabile \t questo è possibile grazie al calcestruzzo armato, mentre in un sistema costruttivo tradizionale no \\
C'è anche ripiano in calcestruzzo, con degli inserti di porcellana colorata \t e ci sono gli infissi di legno \\
"Brise soleil" è il frangi sole \t infatti lui si pone il problema climatico del caldo (siamo a Marsiglia) \\
Le logge hanno funzione frangi-sole, per cui i raggi vengono deviati e non c'è soleggiamento diretto sulla parete \\
Poi ci sono dei frangi-sole, che sono elementi obliqui in calcestruzzo, posti al di fuori delle finestre \\
Inoltre la parete nord è completamente nord: non ci sono alloggi che si affacciano, ne logge \t ma solo scale e locali di servizio, perchè a marsiglia c'è il maestrale che un vento molto freddo che viene da nord \\
Quindi tiene conto anche degli aspetti climatici

\sss[Spazi comuni]
L'edificio era posto alla periferia di Marsiglia, in una zona quasi campestre, mentre ora è all'interno della citta \\
Quindi non c'erano negozi di base \t che quindi sono dentro l'edificio, come il parruchiere, macellaio, panettiere etc. \\
Inoltre qua c'è il primo supermercato francese \t e c'è un servizio di spedizione nelle case dei prodotti \\
C'è un sistema di interfono interno, per cui si poteva chiamare il negozio e farsi arrivare la spesa nell'alloggio \t e veniva posta nelle cassette davanti alla porta \\
La strada interna è in parte di linomium, in parte in calcestruzzo con inserti colorati \t e si affacciano le porte \\
C'erano anche delle sale per la vita comune \t e anche una foresteria, con dei locali dotati anche di cucina \\
Adesso non ci sono piu questi negozi, e la foresteria è diventata un albergo con anche ristorante

\sss[L'atrio]
Parte della parete è trattata con rettangoli aperti \t che rimandano a certi dipinti astratti di Mondrian \\
Inoltre ci sono ascensori

\sss[Terrazza]
Qua c'è la scuola materna con anche una piscina \\
Poi c'era anche una palestra, e sempre per lo sport c'era anche una pista da corsa che circondava tutto il perimetro \\
Estremamente innovativo \\
Inoltre c'è il teatro all'aperto, che in realta è solo una gradinata \\
Ci sono due strutture per gli impianti, uno è l'acqua potabile e l'altro è l'impainto di riscaldamento

\sss[Punto di vista formale]
Il calcestruzzo a vista, e anche l'uso del colore è particolare \t caratterizza tutte le porte e ci sono gli inserti di ceramica anche nelle logge \\
Le pareti interne delle logge sono intonacate con dei colori primari \t questo da anche una connotazione estetica formale

\newpage
\s[Urbanistica moderna - Le Corbusier]
"L'urbanistica è lo specchio di una civiltà" \t e le scelte urbanistiche possono indurre delle trasformazioni nella società stessa \\
Le Corbusier riflette sulla città contemporanea, e i punti fondamentali sono la città verticale e la città giardino \\
La citta si sviluppa verticalemente e lo spazio libero deve essere dedicato al verde pubblico \\
Lui fa anche delle proposte di citta \t che pero sono ideali, non sono applicabili ne pensate per un luogo specifico \\
Prevede edifici alti immersi in un grande giardino, tutto studiato su base funzionale secondo il principio progettuale del movimento moderno \\
Poi fa proposte ideali anche per intervenire su città 

\ss[Plan Voisin]
Proposta per il centro di Parigi \t spostando dei monumenti e creando un grande giardino con edifici

\ss[Carta d'Atene]
Al CIAM del 33 si arriva al documento della "Carta d'Atene", in cui sono definiti i principi dell'urbanistica moderna \\
Ai CIAM, che sono congressi, partecipano architetti e discutono di un tema \t quello del 33 si svolge su una crocera che va da Marsiglia ad Atene \\
Urbanistica = "deve assicurare ai cittadini condizioni di vita che salvaguardino la loro salute fisica, psicologica e la gioia di vivere che ne deriva" \\
Nella carta ci sono 95 punti \t che derivano anche dall'analisi di città europee e nord-americane, con la quale si conclude che non c'è alcuna regola e si sviluppano nel caos \\
Si definisicono le aree funzionali \t che sono:
\bi 
    \item abitare \t punto di partenza
    \item lavorare
    \item ricrearsi
    \item circolare \t separazione dei percorsi (ideata gia da Leonardo) per cui si distinguono strade pedonali, ciclabili, etc. \t che devono essere rigidamente separato
\ei

\ss[Chandigarh]
A Le Corbusier  viene commissionata la costruzione di una nuova citta \\
Dopo la colonizzazione dell'india, uno stato si separa dall'India e diventa Pakistan (si separa dal Punjab) \\
La capitale del Punjab, ovvero Laore, rimane nel territorio pakistano \t serve quindi una nuova citta per il Punjab \\
Il lavoro viene affidato da Nehru (successore di Gandhi) a Le Corbusier \\
Anche se la città è nuova, Le Corbusier tiene conto della tradizione del luogo \t non applica la teoria senza applicarla al contesto \\
Inoltre il risultato è una delle città + gradevoli e vivibili dell'India \t e si vede un tentativo che porta a scelte urbanistiche, che mirano a una trasformazione della società

\sss[Il piano urbanistico]
Strada che porta a nuova Delhi è orizzontale \t va da ovest a est \t mentre l'altra è quella di penetrazione nella citta \t sono strade di percorrenza veloci \\
Poi cè il campidoglio e gli edifici governativi, che sono separati dalla citta \t analisi funzionale: deve essere una capitale con sede del governo \\
All'intersezione delle due strade e altre parti, ci sono gli edifici che servono a tutta la città (es. universita, musei, stadio) = strutture per tutta la citta \\
Progettata per 150.000 abitanti ed espandibili a 500.000 (in realta si è espansa molto di piu) \\
La parte residenziale è definita da due maglie ortogonali sovrapposte:
\bi
    \item la prima divide la città in settori rettangolari, con strade secondarie che circondano i settori e servono per spostarsi
    \item la seconda è sovrapposta ed è ortogonale ad essa:
        \bi
            \item nella direzione nord sud prevede un parco, e ci sono i percorsi pedonali \t e ci sono tutti i serivizi di quartiere: bar, ristoranti, centri sportivi, scuole, etc.
            \item nella direzione est ovest: vie commerciali con tutti i negozi tipici indiani \t la distribuzione locale rimane quella tradizionale = mediazione con la tradizione
        \ei
    \item nei settori, delimitate dalle strade, ci sono gli edifici residenziali, accessibili solo a piedi
\ei

\v

La socità indiana era ancora divisa in caste, e non sono assimilabili a diverse classi sociali \t ma sono legate a (oltre a una gerarchia) a lavori, e non ci si puo sposare in caste diverse etc. \\
Quindi è una societa rigidissima \t e Gandhi voleva l'eliminazione delle caste, e questo viene portato avanti da Nehru \\
Pero questo deve succedere gradualmente e deve essere assimilato dalla vita quotidiana \t non basta una legge che scardina una organizzazione sociale millenaria \t percorso richiede tempo \\
Si vuole aiutare questo processo nella progettazione della citta \t per cui diverse caste vengono integrate nello stesso settore \\
Le abitazioni sono pero case basse (attenzione agli usi abitativi indiani) e si definiscono tipologie edilizie diverse per le diverse caste \\
Nonostante ciò, abitazioni vengono mescolate \t e gli stessi serivizi vengono usati, e i figli vanno nelle stesse scuole \t che è la parte + importante \\
Per i bambini integrazione è piu naturale che negli adulti \t cosi si cerca di eliminare queste suddivisioni \\
Si vede come scelte progettuali urbanistiche cercano di incidere sull'evoluzione della socità 

\sss[Palazzo del governo]
Per gli edifici del campidoglio non media con la tradizione indiana e applica le sue idee \t pero fa conto del clima, che è caldo e umido \\
Cerca quindi di alleviare il piu possibile il calore \\
Calstruzzo a vista grezzo, e ci sono dei volumi posti sulla terrazza \t che hanno forme geometriche o libere, si impongono visivamente e sono funzionali \\
La facciata principale presenta un portico molto profondo, e i setti fanno da frangi sole (e anche sulle pareti con le finestre) \t si evita soleggiamento diretto \\
Anche i setti sono forati \t e ci sono delle aprture di forma libera, che servono alla circolazione dell'aria

\sss[Corte di giustizia]
C'è la stessa logica \t ma la copertura è staccata, per il circolo dell'aria \t e le scale sono aperti (anche se coperti), sono esterni \\
E poi usa il colore \t tipico di Le Corbusier \t e ci sono le forme libere delle aperture, ed elementi curvilinei che danno articolazione all'edificio

\sss[Segretariato]
Dove ci sono i ministeri \\
Usa il colore, e le aperture hanno formi ovali ma libere e non geometriche

\sss[Simbolo della città]
Mano del Modulor \t diventa il simbolo della citta e viene messa su un asta d'acciaio che si muove col vento \\
E il simbolo del partito del congresso di Nehru è una mano aperta

\vspace{3em}

\s[Analisi architettonica]
\bi
    \item contestualizzazione
    \item analisi distributiva \t distribuzione degli spazi \t nel movimento moderno bisogna parlare del metodo progettuale, in cosa consiste e la sua importanza
    \item analisi strutturale \t materiali e sistema costruttivo (facendo riferimento tipicamente ai 5 punti di Le Corbusier)
    \item analisi formale \t quali sono gli aspetti che determinano le qualità formali dell'edificio
\ei

\end{document}
