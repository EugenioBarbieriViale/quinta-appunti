\documentclass[12pt]{article}

\usepackage[a4paper, total={6in, 8in}]{geometry}
\usepackage{textcomp}

\begin{document}
\setlength{\parindent}{0pt}

\def \t {\textrightarrow}
\def \v {\vspace{1em}}
\def \bi {\begin{itemize}}
\def \ei {\end{itemize}}
\def \s[#1] {\section*{#1}}
\def \ss[#1] {\subsection*{#1}}
\def \sss[#1] {\subsubsection*{#1}}

\s[Futurismo]
Movimento d'avanguardia italiano \\
Ultimo autore di prestigio italiano era stato canova \t italia rimane anche esclusa dalle innovazioni artistiche \t per esempio, no esperienze impressionista \\
Macchiaioi toscani \t anticipano un po il futurismo \t e anche la Scapigliatura \t ma non sono molto rilevanti 

\ss[Divisionismo]
Unico movimento di rilievo è stato il divisionismo \t portato a Milano da un gallerista e pittore (Vittore Grubicy), che raccoglie intorno a se degli studenti dell'accademia di brera \\
Si presentano al pubblico con una mostra nel 1895 \\
Divisionismo (post-impressionismo) \t recipisce il puntinimso francese = movimento artistico post-impressionista che non rifiuta l'impressionismo (come Van Gogh, Munch), ma consolida l'aspetto scientifoc del imp. e si concentra sulla teoria del colore \\
Il puntinismo fra proprio la scoperta del chimico Chevreul \t scopre che due colori primari accostati (come in due punti), da vicino si distinguono, da lontano si mischiano \\
Fenomeno della "mescolance optique" \t colore risultante si definisce direttamente nella retina dell'uomo \\
Il puntinismo non ha dinamismo, velocità \t richiedono un lavoro molto lungo, non come impressionisti \\
Il puntinismo arriva a Milano da Grubicy \t pittori come Pellizza da Volpedo, Segantini non usano i puntini, ma delle sottili pennellate \t lavoro molto lungo e complesso \\
I divisionisti usano una tecnica nuova \t ma i contenuti sono ancora tradizionali \t sono simbolisti o di denuncia sociale \t non si è ancora fatto il passo della rappresentazione della realta in italia \\
C'è ancora necessità di inserire dei simboli \t manca innovazione iconografica \\
In italia infatti prevale ancora l'accademia \t principalmente arte religiosa \t e sono anche le opere che vengono + comprate

\v

Futurismo è un movimento d'avanguardia e ha riconoscimento internazionale importante \t incontrano anche pittori di altre avanguardie \t e si influenzano a vicenda \\
Nasce a Milano \t città + avanzata d'italia in questo momento \\
Non è solo un movimento artistico, ma tocca tutti gli aspetti creativi \\
Primo protagonista è Marinetti \t che è un poeta, scrittore \t anche musica e poesia futurista \t poi si cercherà una ricostruzione della realtà futurista \\
3 punti:
\bi
    \item rifiuto radicale del passato
    \item esaltazione della modernità
    \item aggressività della rappresentazione artistica
\ei
Si presentano al pubblico con manifesti (non sono trattati, che si usano nella teoria artistica) \t testi molto sintetici \\
L'altra modalità sono le serate futuriste \t prima nel 1909, a Torino \t leggono poesie, si suona, si spiegano le idee futuriste \t è una performance \t la serata in se è un prodotto artistico \t anticipa l'arte contemporanea \\
Le serate vengono organizzate in modo che finiscano sempre in rissa \t tra i futuristi e gli spettatori, che sono i giovani borghesi \t poi arriva la polizia che arresta gli organizzatori \\
Futuristi infatti durante questi spettacoli (improvvisati) provocano il pubblico \t per esempio vendono lo stesso biglietto a + persone, e altri trucchi \\
Così da creare una tensione iniziale

\ss[Il manifesto di Marinetti]
Viene pubblicato in francese sul quotidiano "Figaro" nel 1909 \\
Presentazione di una nuova idea di bellezza \t il mondo della modernita contrapposto alla bellezza tradizionale \t la Nike di Samotracia \\
Il rifiuto del passato viene espresso in modo aggressivo \t "vogliamo distruggere i musei e combatteri contro il moralismo" \t il passato è da distruggere \\
Esaltazione della modernità anche nei suoi prodotti \\
La violenza e il vitalismo sono insiti nella bellezza \\
La guerra viene vista come la ?? \t "Igene della guerra", marinetti parla della necessità della guerra per generare un mondo nuovo \\
Vanno anche in guerra \t e Boccioni muore a Verona, Santeria muore in battaglia

\ss[Manifesto dei pittori futuristi]
Riprende i punti espressi da Marinetti e li applica alla pittura \t modernità è il soggetto dell'opera \\
Nausea che nasce dal rimanere ancorato al passato del contesto culturale italiano \\
Il contenuto è al punto 8 \\
L'arte deve essere legata alla contemporaneità, al contesto in cui si vive \t l'arte + significativa è questa

\ss[Manifesto tecnico]
Aspetto iconografico e come si deve dipingere \\
Probabilmente scritto da Boccioni \t teorico del futurismo, che tenta di definire il futurismo e formulare un pensiero teoirco futurista \\
Nel futurismo \t librazione, dinamismo universale \t oggetto di rappresentazione è ancora la realtà concreta \\
Futuristi non si concentrano tano sull'oggetto, ma sul suo rapporto con lo spazio \t oggetti si compenetrano in un'unica realtà dinamica \\
Spazio e oggetto vengono rappresentati in una realtà dinamica \t a cui si aggiunge l'aggressività del colore \\
Uso violento del colore puro \\
"Non può esistere pittura senza divisionismo" \t unica cosa che salvano della tradizione artistica italiana

\v

Futurismo è influenzato da Bergson, come cubismo \t ma il tempo cubista è un tempo lento, che attraverso la memoria rappresenta il passato e il presente \\
Il tempo futurista rappresenta passato e presente, ma anche il futuro \t futurismo rappresenta la continuità dinamica dell'oggetto \\
Cubismo si concentra invece sulla conoscenza dell'oggetto in se \\
Critici hanno individuato nel futurismo una fusione di:
\bi
    \item espressionismo: visione globale e dinamica
    \item cubismo: frammentazione della realtà stessa
\ei

\ss[Dinamismo di un automobile - Russolo]
1912/13 \\
Ritrae il dinamismo universale \\
Automobile in corsa, e si possono immaginare le diverse posizioni che assume \t unificate in una continuità formale \\
Dinamismo viene definito da una serie di cuspidi \t tipico di questo periodo è lo studio della psicologia della forma (approfondito nell'astrattismo) \\
La percezione è che l'automobile arrivi molto veloce, ma poi brusca frenata (perchè lu cuspidi li allargano andando a sinistra) \\
Indietro si vedono case \\
Contrasto di primari (giallo, rosso, blu) \t porta a uso dei colori aggressivo

\ss[Bambina che corre sul balcone - Balla]
Balla \t è in realta un fotografo \t poi si trasferisce da Milano a Roma \\
È + vecchio degli altri futuristi \t è sposato e ha dei figli, quindi non va in guerra \\
Ha due figlie, Elica e Luce \\
Usa tecnica divisionista con puntini di grandi dimensioni \t rappresenta il movimento della bambina come una sequenza di posizioni \\
Testa in alto, scarpe in basso, mentre al centro non si distingue il movimento \t rappresenta la continuità formale nel dinamismo \\

\s[Boccioni]
Muore durante la guerra \\
Nasce a reggio calabria e poi si trasferisce a roma \\
Fa percorso di formazione tecnico (geometra) ma ha passione per la pittura \t segue i corsi dell'accademia, e frequenta lo studio di balla \t e qui impara la tecnica divisionista \\
Poi si trasferisce a milano \t ma prima si muove molto \t ma qua a milano entra in contatto con il futurismo \\
Boccioni è il più teorico del gruppo \t e cerca di spiegare gli aspetti teorici del futurismo \\

\ss[La città che sale]
La definisce un'opera di transizione \t non è ancora del tutto futurista \t è ancora legata alla tradizione formale e anche nei contenuti ci sono ancora simboli 

\v

Grande piazza \t dei cavalli che corrono e degli uomini cercano di fermarli \\
Sulla destra c'è un palazzo in costruzione e delle ciminiere, a sinistra si vede un filobus \\
La città è milano \t in questo periodo con piano Beruto etc. \t si sta sviluppando molto \\
Viene presentata una città moderna che si sviluppa \\
I cavalli hanno un contenuto simbolico \t rappresentano il progresso, che non può essere fermato \t alcuni uomini cercano di fermarlo \\
Ma non è futurista in questo \t futuristi non sono simbolisti 

\v

Tecnica divisionista, ma anche elementi tradizionali \t spazio è in parte costruito prospetticamente \\
Palazzi sulla destra sono in prospettiva accidentale \\
Il corpo del cavallo e degli uomini sono compenetrati in alcuni punti \t molto futurista, e anche nell'estremo dinamismo \\
Linea di forza del cavallo, muovendosi a spirale, e poi altro cavallo che sembra scontrarsi \\
Centro percettivo è attorno alla testa del cavallo rosso \t e c'è anche percezione dello scontro \\
I colori sono principalmente primari \t rosso e il blu, scontro tra contrasto di primari e dinamismo estremo fanno percepire la violenza

\ss[Trittici degli stati d'animo]
Sono 3 coppie di 2 dipinti \\
Pieno futurismo \t no simbolismo, no prospettiva \\
I primi tre li realizza prima del viaggio a parigi, i secondi tre dopo \t viaggio a parigi è importante per i futuristi \t hanno confronto internazionale e conoscono cubismo \\
Lui prende un evento, che è una partenza \t è un evento imporante, perchè in questo periodo c'è forte migrazione \\
Partenza importante, anche definitiva \\
Suddivide in tre momenti \t rappresenta le emozioni, che non sono le sue, ma fa un analisi \t distingue tre momenti: gli addi, quelli che vanno, quelli che restano \\
Indaga così gli stati d'animo in questi momenti diversi \\
Lui indiviuda le linee forza, che meglio esprimono uno stato d'animo

\sss[Gli addii I]
Linee forza sono ondulate ed ellettiche \t andamento è dinamico e presenta uno stato d'animo variegato \t l'addio è un momento fortemente emotivo che coinvolge diversi sentimenti \\
Ovale con le teste è una vista dall'alto di un abbraccio \t controno è definito da una linea ondulata e obliqua \\
Usa pennellate di colore puro che definiscono la forma

\sss[Quelli che vanno I]
La persona parita si vede dal finestrino del treno \t si vedono biciclette di strada parallela, delle case, un palo \t il tutto integrato in linee rettilinee oblique \\
Linee portano a movimento veloce da destra a sinistra \t soggetto e oggetto fanno parte di un'unità dinamica \\
Si percepisce dinamismo ed emozione, che è positiva \t stato d'animo è rivolto verso il futuro, c'è un'aspettativa \\

\sss[Quelli che restano I]
Stanno tornando a casa dopo l'addio \\
Usati colori freddi \t mentre prima contrasti cromatici \\
Pennellate verticali, e sagome delle persone (create da pennellate con orientamenti leggermente diversi) \\
No dinamismo \t si percepisce la lentezza delle persone, che hanno stato d'animo triste 

\v

Nei dipinti dopo l'esperienza a parigi, ci sono le stesse linee forza e colori \\
Anche Kandisky scrive che determinate linee e colori hanno una riposta diversa nella percezione umana

\sss[Gli addii II]
Inserimento di alcuni elementi cubisti \t le persone si abbracciano, e sono sempre inserite in forme ovali = dinamsimo \\
Ma c'è anche scomposizione cubista \\
Inoltre spiegazione del contesto + dettagliata \t locomotiva con numero (che rimanda a cubismo sintetico) e si vedono anche ruote e rotaie \\
La linea curva e variegata è quella che determina l'immagine

\sss[Quelli che vanno II]
Linea è la stessa \\
Ma qua si vedono anche i visi, scomposti, delle persone sul treno

\sss[Quelli che restano II]
Quelli che rimangono sono scomposti secondo la logica cubista \t sempre colori freddi 

\v

Tutto questo completamente separato da espressionismo però \t non riguarda i sentimenti dell'autore, ma una analisi oggettiva

\ss[Forme uniche della continuita nello spazio]
Pag 159 \\
Anche nella scultura si vuole ritrarre il dinamismo universale \\
Boccioni non ha tecnica scultorea \t non ha formazione \t quindi realizza le sculture in bronzo, che è + semplice \t si realizza con stampa d'argilla \t ciò che realizza effettivamente l'artista \\
Mentre i greci realizzavano tutte le fasi della produzione, col tempo queste fasi si sono separate \t l'artista realizza lo stampo, che poi si porta dal bronzista \t che fa il lavoro artigianale

\v

Viene rappresentato un uomo che corre \t no naturalismo, pero si riconosce la linea umana grazie alle linee di forza che definiscono il corpo umano \\
Studio del corpo umano e del suo movimento \t rappresentazione del dinamismo, e l uomo che è conmprenetato con lo spazio che lo circonda \\
Fattore temporale \t boccioni identifica come un movimento veloce \t proiettato anche verso il futuro \\
No naturalismo \\
L'arista scompone il corpo in diverse parti \t e di ognuna ne rappresenta la continuita del movimento nello spazio, creando delle scie, per poi assemblarle \\
Le scie creano dinamismo e assemblate creano il corpo nel suo movimento veloce \\
Il titolo descrive gia questo

\ss[La strada che entra nella casa]
Titolo alternativo \t "Visioni simultanee" \\
In primo piano donna che si affaccia al balcone, girata di schiena, che si appoggia a una ringhiera \t è la madre di Boccioni \\ 
La sua posizione coincide con quella dello spettatore \t mostra quello che lei vede \\
Poi palazzi circondano la piazza \t ci sono altre donne affacciate, che guardano \\
In basso c'è un insieme di operai che stanno lavorando in un cantiere, con i pali \\
A destra c'è anche un cavallo, e in fono una astrada \\
Periferia di Milano \t infatti Boccioni abitava nella perifieria, nella parte di espansione del piano Beruto \t via Adige ?? \\
Rappresenta ancora la città che sale \t ma in modo completamente futurista \t no prospettiva e no simbolismo \\
Dinamismo universale \t i palazzi, che dovebbero essere verticali \t sembrano cadere verso il centro \t e sono scomposti (a destra) \t rappresenta delle visioni simultanee \t nei diversi punti di vista i palazzi si vedono in forme diverse \\
C'è anche la compenetrazione \\
Dal punto di vista compositivo \t centro percettivo è la piazza, che è inserita in una circonferenza a fondo giallo \\
Nell'andamento compositivo si percepisce un movimento a spirale degli operai

\v

Tra le due guerre c'è in Italia il fascismo \t tutti gli artisti italiani cercano di essere accettati dal regime \\
E artisti anche molti diversi tra loro \t si propongono come arte di regime \t lo fanno anche i futuristi (rimasti) \\
Carra, Sioni e altri in realta cambiano avanguardia \t altri invece continuano l'esperienza futurista, proponendosi al fascismo \\
Fascismo all inizio si presenta come qualcosa di nuovo, giovane, che vuole trasformare \t vicino agli ideali futuristi \\
Nel secondo futurismo ci sono tanti centri provinciali \t diversi artisti rappresentano in modalita futurista le nuove piazza costruite con il fascismo etc. \\
Prende piede la aereopittura \t sempre futurista, ma si rappresenta il dinamismo universale come se vista da un aereo in volo \\
Per esempio Gerardo Dottori \t rappresenta viste dall'alto di paesaggi umbri \\
Balla \t redime un manifesto con la natura futurista \t crea i fiori futuristi
\end{document}
