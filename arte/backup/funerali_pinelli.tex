\documentclass[12pt]{article}

\usepackage[a4paper, total={6in, 8in}]{geometry}
\usepackage{textcomp}

\begin{document}
\setlength{\parindent}{0pt}

\def \t {\textrightarrow}
\def \v {\vspace{1em}}
\def \bi {\begin{itemize}}
\def \ei {\end{itemize}}
\def \s[#1] {\section*{#1}}
\def \ss[#1] {\subsection*{#1}}
\def \sss[#1] {\subsubsection*{#1}}

\s[I funerali dell'anarchico Pinelli]

\ss[Autore - Enerico Baj]
Enerico Baj \t 1924-2003 \t artista milanese \\
Ha un prestigio internazionale, ed espone anche al MOMA di New York \t non puo essere inserito in un movimento, perche ha modalita espressiva individuale \\
Il significato delle sue opere è legato al suo pacifisimo e antimilitarismo \t alla denuncia della violenza del potere \t e spesso usa anche l'ironia \\
Dal punto di vista formale, c'è un ricordo delle avanguardie per mancanza di naturalismo e per l'utilizzo di materiali nuovi \\
Suo specifico è l'utilizzo del collage di materiali di scato \t in particolare usa ingranaggi degli orologi

\ss[Aspetti tecnici]
Chiari riferimenti a guernica \\
È un installazione \t ci sono una serie di forme, unite come un puzzle \\
Oltre al pannello ci sono 4 sagome separate \t le due bambine e la moglie di pinelli \\
Poi c'è il pavimento coperto di passamanerie, che sono scarti

\ss[Contesto]
Anni di Piombo \t dal 69 ai primi anni 80 \t chiamati così perchè c'erano continui scontri armati con forze di estrema destra e sinistra \\
\bi
    \item estrema destra \t utilizza la tecnica della tensione e vuole creare la paura nella popolazione, per favorire una presa di potere autoritaria \t fanno attentati
    \item estrema sinistra \t gruppo + famoso è l'estrema sinistra, che vuole creare le condizioni per una rivoluzione \t attaccano persone singole \t per esempio Aldo Moro \t comunque chi voleva portare il cambiamento per via istituzionale
\ei
Aldo Moro stava collaborando con Berlinguer (capo del parito comunista) ed era demo-cristiano \\
12 dicembre del 60 \t attentato in piazza fontana, scoppia una bomba \\
Calabresi \t era chi investigava \t Pinelli viene chiamato in questura per 3 giorni (illegalmente) \t non ci sono prove \\
Il questore è Marcello Guida \t il direttore del carcere di Ventotene \t che si rifiuta di dare la mano a Bertini, che era stato confinato a Ventotene

\v

Baj realizza quest'opera perche \t il sindaco di milano, cognato di Gramsci \t organizza ogni anno una mostra \t Baj decide di realizzare opera

\ss[L'opera]
Sulla sinistra c'è il funerale, mentre sulla destra carica della polizia \t non era accaduta, ma rimanda al clima \\
Pinelli è al centro ed è una sagoma staccata \t sopra c'è un pannello, con l'occhio (citazione alla Guernica) \\
Lui sta cadendo ed è il centro percettivo \t ed è la figura che nella Guernica esce dal palazzo in fiamme \\
A sinistra e a destra c'è il gruppo anarchico che è il corteo funebre e i polizziotti che stanno venendo caricati \\
Le due bambine sono le sue figlie \t una guarda il padre e sporge le mani, l'altra si copre le mani \\
Dietro c'è il gruppo degli anarchici \t e sono identificabili, sono quelli che fanno parte del circolo anarchico \t per esempio la donna è Camilla \\
Tutto rimanda al contrasto politico dell'epoca \\
Utilizzate le passamanerie per decorare \t stanno piangendo \\
La polizia \t c'è il capo che fa il saluto fascista, e gli altri sono armati \t a destra c'è la moglie, ed è citata da Guernica \t è la donna stupita \\
I poliziotti sono disumanizzati \t sono delle sagome informi \t gli occhi sono meccanismi di orologi, hanno le medaglie, e hanno un ghigno \\
Utilizzati colori saturi, che aumentano la violenza \\
Il pannello in alto ha una decorazione dipinta \t che rimanda ai palazzi milanesi ? 

\v

Linee di forza vengono dalle mani alzate del corteo e dei polizziotti \t e indicano su pinelli, che è il centro \\
Tutto questo crea molto dinamismo, dall'alto vero il basso, destra sinistra, ... 

\v

Nel 72 doveva essere esposta l'opera \t ma quella mattina del 17 maggio viene ucciso Calabresi \\
Infatti calabresi era stato ritenuto il responsabile \t e viene assassinato \\
Indagini senza risultato \t poi 10 anni dopo, un militante di lotta (Marino) continua ammette il delitto \t Soffi, che era il capo di lota continua, va in carcere ma nega \\
Bompressi chiede la grazia, e ? \t erano in Francia \\
Quelli di destra, per non essere catturati, vanno in sud america \t mentre quelli di sinistra vanno in francia \t c'era dottrina in francia che accoglieva questi terroristi di sinistra \\
Li accoglieva e non li estradava in italia \t estraneazione si puo fare se lo stato ritiene che vengano garantiti i diritti nei processi \t francia non lo considerava \\
Quindi Calabresi viene ucciso \t quindi l'opera non viene esposta \t e alla fine l'esposizione non viene mai fatta \\
L'opera viene regalata alla moglie di pinelli \t e quindi viene acquistata da Marconi, e la tiene in galleria (smontata) \t nel 2012 viene esposta nella sala delle cariatidi \\
Sala della cariatidi è ancora rovinata dai bombardamenti della WW2 \t quando Pisapia era sindaco
\end{document}
