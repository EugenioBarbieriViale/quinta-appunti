\documentclass[12pt]{article}

\usepackage[a4paper, total={6in, 8in}]{geometry}
\usepackage{textcomp}

\begin{document}
\setlength{\parindent}{0pt}

\def \t {\textrightarrow}
\def \v {\vspace{1em}}
\def \bi {\begin{itemize}}
\def \ei {\end{itemize}}
\def \s[#1] {\section*{#1}}
\def \ss[#1] {\subsection*{#1}}
\def \sss[#1] {\subsubsection*{#1}}

\s[Cubismo]
Si sviluppa in francia tra il ?? e ?? \t per 7 anni \\
Gli inventori sono Picasso e Braque \t lavorano insieme in perfetta sintonia \\
L'oggetto della ricerca pittorica è la realtà concreta delle cose \t vuole rappresentare l'oggetto, non sua percezione e non il proprio stato d'animo \\
Si differenzia dall'astrattismo \t che si esprime con forme e colori, si allontana dalla rappresentazione del reale \\
Fanno una critica sia all'espressionismo e dell attrattismo \t anche se riconoscono l'importanza dell'espressionismo, che è andato contro l'arte accademica e di cercare una struttura cromatica diversa dalla realtà \\
Criticano espressionismo perche fa dipendere la forma da aspetti contigenti (come lo stato d'animo, il tempo, ...) e dicono che è povero di contenuto, superficiale \\
Espressionismo fa dipendere la forma dall'identita dell'artista \t alcuni espressionisti poi passano anche al cubismo 

\v

Premesse del cubismo sono 3:

\ss[1. Primitivismo]
L'arte africana \t che vedono e riprendono (cone le maschere africane) 

\ss[2. Cezanne]
Cezanne \t è un artista post-impressionista, che è il precedente del cubismo \\
Nasce da una famiglia benestante in provenza e va a parigi \t si dedica alla pittura ma senza frequentare l'accademia \\
Entra a far parte degli impressionisti \t ma presto si differenzia \\
Lascia parigi e torna in provenza, dove rimane tutta la vita \\
Coglie alcuni aspetti fondamentali dell'impressionismo \t rifiuta accademia, niente disegno ma colore definisce la forma, pittura an-plen-air \\
Pero poi ritiene che la realta da rappresentare c'è bisogno di conoscenza + profonda \t bisogna cercare le forme originarie \\
Tutta la realta è riconducibile al cilindro, sfera e cono \\
Inoltre ritiene che bisogna usare anche gli altri sensi oltre la vista per rappresentare la realta \t bisogna dipingere anche gli odori \\
Fa anche delle lunghe passeggiate con un geologo \t per conoscere meglio la realta ??? \\
Fa un processo + razionale e + profondo \\
Non dipinge con le pennellate veloci, ma con le "pezzature di colore" \t sono come macchie di colore abbastanza grandi \\
Nel 1907 viene organizzata una mostra su di lui \t picasso e braque la guardano e rimangono affascinati dalle nature morte

\ss[Tavola di cucina]
Natura morta: tavolo con panno, della frutta, una brocca, cesto, caffettiera, in un interno di una stanza con una sedia e una credenza \\
Prevale la forma geometrica della sfera \t nella brocca e nella frutta \\
Inoltre le proporzioni non sono corrette \t una pera è enorme rispetto alle altre \\
Cezanne unifica punti di vista diversi nella stessa immagine \t ci sono punti di vista diversi \\
Non c'è corrispondenza tra il punto di vista della stanza e quello del tavolo \t non è coerente \\
Anche la brocca è vista dall'alto, cesto di fronte, tavolo dal basso \\
Ritiene che il dipinto non deve rappresentare la percezione dell uomo, ma deve seguire delle sue regole interne 

\v

Viene ripreso nel cubismo
\bi
    \item approccio razionale alla rappresentazione e alla volonta di conoscere la realta
    \item unificazione delle viste 
\ei

\ss[3. Bergson]
Filosofo e si pone il problema della conoscenza \t ritiene che per la conoscenza della realta non sono sufficienti le 3 dimensioni spaziali \\
Bisogna inserire fattore spaziale, ovvero la memoria \t significa inserire un aspetto soggettivo \\
4 dimensione che integra le 3 geometriche e la memoria \t la conoscenza si fa attraverso questa 4 dimensione \\
Questa visione influenza il cubismo

\ss[Ricerca della verita e non della somiglianza]
La prospettiva deforma sia le forme sia le dimensioni \t le proiezioni rappresentano oggettivamente \\
L'oggetto pero prima deve essere conosciuto \t e attraverso la 4 dimensione \\
Per conoscere l'oggetto bisogna osservalo in tutte i suoi aspetti \t non solo da un punto di vista \\
Inoltre oggetto deve subire un processo \t destrutturazione e poi libera ricomposizione

\v

3 fasi:
\bi
    \item protocubismo: 1907-8 \t influenza di cezanne, punto di partenza
    \item cubismo analitico: 9-11 \t destrutturazione
    \item cubismo sintetico: 11-14 \t libera ricomposizione
\ei

\s[Picasso]
+ grande artista del 900 \t stessa importanza di michelangelo nel 500 ed è una figura dominante \\
Vive 90 anni \t figura di confronto per tutti gli artisti del 900 \\
È spagnolo ed è un talento precoce \t entra a 14 anni nell accademia e va all accademia superiore di madrid \t a 16 anni dipingeva come raffaello \\
Poi va a parigi \t centro dell'arte occidentale \\
Nella sua vita ha diversi stili \t quasi eclettico \\
Dopo che franco sale al potere non torna piu in spagna 

\ss[Le Demoiselles d'Avignon]
Prostitute in un bordello \t ci dovevano essere anche degli uomini, poi cancellati \\
La sua non è una pittura veloce, ma molto lenta \t fase preparatoria e poi cambiamenti in corso d'opera \t processo molto meditato \\
Mentre impressionismo era istintivo \t qua no

\v

5 donne in un interno \t tavolino con natura morta, omaggio a Cezanne \\
Primitivsmo \t estrema semplificazione dei corpi, mancanza di proporzioni e nei visi che sembrano rifarsi alle maschere africane \\
Anche il colore \t alcune color legno \\
Molte parti del corpo sono rappresentate sotto punti di vista diversi unificati \\
Gli occhi non sono simmetrici, il naso ribaltato \t è di schiena ma il viso è di fronte \\
Anche ribaltamento dei piani \t una parte del corpo è aperta e distesa sulla tela \t gamba della donna a sinistra o il seno in alto a destra \\
Natura morta vista dall'alto, ambientazione che manca \t le tende chiudono tutto e non hanno un panneggio morbido, sembrano rigide \\
Rosso con rosa e giallo e il blu, ma i colori non sono particolarmente importanti \t il cubismo vuole un approccio razionale e non emotivo 

\ss[Ritratto di Ambrese Vollard]
\textbf{Analisi iconografica}: si vede il volto di una persona, e le spalle, è vestito con il colletto e la cravatta, taschino con la giacca \\
Sembra seduto \t sta leggendo e guarda verso il basso \\
Si vede anche una bottiglia a sinistra, e a destra una libreria \t forse si trova in uno studio \\
Lentamente si puo riconoscere tutto \t fase analitica, l'oggetto può essere destrutturato completamente \\
C'è tempo necessario per l artista di conoscere, ma anche da parte dell'osservatore \\
Serve un approccio razionale sia dall artista sia dall osservatore \\
Centro percettivo è la testa per il contrasto cromatico \\
Non ci sono linee compositive \t occhi si muove liberamente, non è guidato \\
La scelta dei colori \t si usano i grigi e le terne \t si punta alla razionalita \\
Inoltre in questa fase non firmano le opere \t non ci deve essere nulla di personale, solo razionale

\v

Cubismo analitico crea pero un problema \t quello di sforciare nell'attratismo \t la comprensibilita dell opera non è sempre fattibile \\
Si arriva all'ermetismo e all'attratismo \t l'oggetto di partenza non si riesce a percepire 

\s[Cubismo sintetico]
Dall'11 al 14 \\
Si cerca un maggiore contatto con la realta \t inserendo nell'opera degli elementi reali \t si parla di papier collé \\
Inoltre si usano lettere e numeri \t che sono immediatamente riconoscibili \\
Si cerca di indurre una riflessione sull'oggetto e sulla sua rappresentazione \t cubismo sintetico è infatti il precedente di dell'arte moderna \\
La riflessione sul signifcato dell'immagine e del suo rapporto con la realta \t costituisce l'origine dell'arte concettuale 

\ss[Natura morta con sedia impagliata - Picasso]
Formato particolare ovale \t la cornice è una corda, scelta da picasso \t fa parte dell'opera \\
Si vede impagliatura di una sedia \t è un adesivo dove è stampata l'impagliatura della sedia \t erano usati sulle sedie vere per simulare l'impagliatura \\
A sinistra si vede JOU \t journal (in francese) \t oggetto riconoscibile \\
Altri elementi sono dipinti e scomposti analiticamente \t bicchiere e limone \\
Quindi sembra essere in un bar \t si riesce a ricostruire la scena, non soltanto attraverso la parte dipinta, ma anche con le lettere e il collage (adesivo) \\
Come nel cubismo analitico \t i colori sono neutri: terrei, grigi \t per indurre a una riflessione razionale \\
Adesivo è un oggetto vero \t ma è un falso \t inoltre giornale viene rappresentato col suo nome 

\v

Kosuth \t "Una e tre sedie" \t primo esempio di arte concettuale \t il soggetto è una sedia \\
Al centro c'è una sedia concreta \t a sinistra una foto della stessa sedia \t poi a destra la definizione di sedia \\
Ci si chiede \t qual'è la vera sedia? \t fa riflettere sull'oggetto e sul suo rapporto con la realtà \t questo ha le origini nel cubismo sintetico

\ss[La Ville - Fernand Leger]
Si forma come architetto \t ma si avvicina al cubismo dando una sua interpretazione \\
Il tema è la ville = la città \t spesso usa questo soggetto \t è affascinato dalla città moderna \\
Dimensioni enormi \t 2x3 metri \\
Elementi della città: uomini, traliccio che rimanda alle gru (città dinamica, in costruzione), strada, palo viola, lampione giallo \\
Elementi vanno a descrivere un ambiente urbano, ma dinamico e moderno \t con l osservazione si riescono a trovare \\
Gli uomini non sono i protagonisti \t ma ci sono \t sono pero anonimi, come dei robot, molto semplificati \\
Spazio non è prospettico \t visione bidimensionale, ma uomini e palo hanno un chiaroscuro che determina una tridimensionalita \\
Qui i colori sono presenti \t primari, secondari, bianco e nero, stesi per lo più piatti (senza variazioni di tono) \\
Centro percettivo inesistente \t il palo e la cornice di un palazzo sembrano essere in primissimo piano, ma l'occhio non è proprio guidato \\
Linee compositive \t non delineate, ma prevalenza di linee verticali \\
Rapporto dell'uomo con la città \t è neutro \t uomini sono inseriti nell'ambiente urbano, senza ne angoscia (kirchner) ne esaltazione

\ss[Foto cartina]
È una donna, probabilmente giovane \t ha i capelli ricci lunghi, tra i capelli ha una margherita \t forma del viso triangolare e allungata \\
Ha una collana con la croce, ha un cappotto \t ambientazione esterna \\
È la futura moglie di Braque

\s[Guernica - Picasso]
Quest'opera testimonia anche l'impegno civile dell'artista \\
Nel 1937 a parigi si sta organizzando l'esposizione universale \t a picasso era stata affidata una parete del padiglione spagnolo \t ma non ha idee particolari e lavora a malavoglia \\
Ma nell'aprile la cittadina basca di Guernica viene bombardata dai tedeschi \t perche in spagna c'è la guerra civile tra il governo eletto e francisco franco \t franco vince nel 39 e instaura una dittatura \\
La germania supporta franco \t l'aviazione tedesca bombarda la cittadina basca, che era fedele al governo democratico \t è il primo bombardamento a tappeto su civili della storia \\
Era scoinvolgente \\
I tedeschi lo fanno anche per dimostrare la loro potenza militare \\
Cosi picasso prende spunto per la sua opera da questo evento \t ha anche scopo politico, per chiedere aiuto alle potenze europee (che pero assumono posizione neutrale) \\
Lavora di fretta \t ma seguendo sempre il suo processo lungo e razionale \t viene esposta a maggio/giugno \\
In realta picasso non fa cronaca \t anche se opera è dedicata alla citta \t non si vede neanche il bombardamento in se \\
Mostra la tragedia della guerra \t da un evento si va a una considerazione piu alta e generale

\v

Dimensioni: 350x380 \t grande quanto una parete \\
Dal punto di vista iconografico ci sono molti simboli \\
È un trittico \t è divisa in tre parti \t una centrale grande e due laterali piccole

\ss[Analisi iconografica]
\sss[Parte sinistra]
A sinistra donna che urla, con un bambino morto \t chiaro riferimento alla pieta \\
Toro si volta, sembra indifferente a quello che sta accadendo \t simboleggia la violenza e la brutalità \\
Dietro c'è un tavolo \t sopra c'è un uccello con il becco aperto e sembra urlare \t sembra ribadire l'urlo della donna e rappresneta il lamento dell anima del popolo spagnolo

\sss[Parte destra]
È un esterno e si vede un edificio in fiamme \\
Donna sta scappando con testa rovesciata e urla \\
C'è un altra figura femminile che si butta dall finestra, sta urlando e porta una lampada a petrolio \t che si confronta con la lampadina elettrica con l'occhio \\
È un riferimento alla tecnologia, che rimane indifferente \t necessita di porsi delle questioni etiche

\sss[Parte centrale]
Al centro c'è un cavallo \t simbolo del popolo spagnolo \t posizione contorta e sembra che sta crollando \\
Ha un nitrito \\
Le figure in basso fanno da collegamento con quelle laterali \\
A destra c'è la "donna stupita" \t sta correndo e ha espressione spaesata \\
A sinistra c'è soldato steso a terra \t sta morendo e sta urlando \t tiene in mano una spada spezzata, che in teoria dovrebbe difendere gli innocenti \\
C'è un fiorellino vicino alla sua mano, che rappresenta comunque la speranza tra la devastazione \\

\ss[Analisi formale]
Esperienza cubista classica era finita nel 14 \\
Ma quando deve esprimere qualcosa di forte torna al cubismo \t si vede nel
\bi
    \item diversi punti di vista
    \item deformazione e semplificazione (no proporzioni, no nat.) tra corpo del cavallo e destra c'è scomposizione analitica
    \item spazio è poco naturalistico \t a sinistra un interno (tavolo), a destra un esterno (dei palazzi), figure umane completamente sproporzionate rispetto a palazzi
\ei
Ma in realta confusione spaziale è perfetta per descrivere la tragicita di un bombardamento \t modalita che crea caos \\
Bianco e nero \t prima fa schizzi colorati, ma poi fa scelta monocroma \t scelta viene forse ispirata dalle immagini che escono sui giornali, che erano senza colori \\
In realta pero Picasso ha periodi monocromi \t es. cubismo blu, ... \\

\ss[Analisi compositiva]
Estremamente dinamico \t linee di forza sono per ogni figura e hanno direzioni diverse \\
Si possono individuare linee di forza che vanno verso l'alto \t ma generale \\
No centro percettivo \t occhio gira senza un riferimento \\
Le due linee compositive che separano il trittico, poi asse di simmetria dove si trova il fiore \\
Inoltre triangolo che ha vertice la fiamma della lampada, poi mano del soldato, e poi piede della donna \t struttura compositiva triangolare 

\v

Questo dipinto poi da parigi viene spostato \t ma non torna in spagna, per volere di picasso \\
Rimane al MOMA per molto tempo \t torna in Spagna nel 81 \t Franco muore nel 75, dipinto arriva dopo \\
Viene esposto prima al Prado in un suo padiglione \t poi l'hanno spostato \t Picasso voleva che venisse esposto al Prado \\
Picasso muore pero nel 73 \t non riesce a vedere la fine della dittatura, alla quale si era sempre opposto \t non riesce piu a tornare in spagna \\
Guernica nel 53 arriva anche a Milano \t venne esposto nella Sala delle Cariatidi, nel palazzo reale di milano \\
Sala delle Cariatidi era sala da ballo \t affreschi dei fasti napoleonici e cariatidi (figure femminili) che sostengono le cornici \\
Ma Palazzo Reale venne bombardato \t sala venne parzialmente distrutta \t quando viene restaurata, si fa la scelta di non restaurare le pareti \\
È stata lasciata per conservare la memoria dei bombardamenti \t luogo espositivo era giusto per la Guernica
\end{document}
