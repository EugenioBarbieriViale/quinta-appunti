\documentclass[12pt]{article}

\usepackage[a4paper, total={6in, 8in}]{geometry}
\usepackage{textcomp}

\begin{document}
\setlength{\parindent}{0pt}

\def \t {\textrightarrow}
\def \v {\vspace{1em}}
\def \bi {\begin{itemize}}
\def \ei {\end{itemize}}
\def \s[#1] {\section*{#1}}
\def \ss[#1] {\subsection*{#1}}
\def \sss[#1] {\subsubsection*{#1}}

\s[Architettura futurista]
Non è un'architettura costruita, solo pensata ma non realizzata \t l'architettura ha tempi diversi rispetto a pittura e scultura

\ss[Manifesto dell'architettura futurista]
Inoltre il tipo di architettura non è solo l'edificio, ma propone un'idea nuova di città \t prevedeva trasformazioni radicali \\
Manifesto dell'architettura futurista di Antonio Sant'Elia \t prima parte \t esaltazione della modernita e rifiuto del passato \\
Posizione contro l'architettura contemporanea, ovvero storicismo ed eclettismo (vedere definizioni) che definiscono l'architettura comune in europa \\
Questo tipo di arc. riduce l'architetto a un decoratore \\
Era stato definito il "ballo in maschera dell'architettura" \\
Necessita di una nuova archiettura \t che risponda alle esigenze dell'uomo moderno \\
La nuova architettura \t non significa un cambiamento formale \t viene criticato anche l'art nouveau (rane mosconi etc..) \t è innovativa, ma in italia (liberty) è solo decorativo \\
Ma architettura deve essere rinnovata completamente, non solo negli aspetti formali \\
Poi descritta la città futurista \t ma edificio e contesto urbano si compenetrano, infatti non viene fatta una distinzione nella descrizione \\
Il nuovo materiale è il calcestruzzo armato (e non solo ferro e vetro) \t prima sperimentale, poi diventa il materiale dell'edilizia per eccellenza \\
Inoltre la decorazione va abolita, e la bellezza non esiste ? \\
Punto 4 \t la decorazione è un assurdo \t rifiuto totale, ma c'è lo stesso un valore formale, che pero deriva dal materiale (che puo essere grezzo, nudo, o violentemente colorato) \\
Punto 8 \t caducita e la transitorieta sono propri della città futurista \t le necessita dell'uomo cambiano continuamente, ogni generazione è diversa e quindi deve avere un contesto sempre nuovo in cui vivere

\s[Antono Sant'Elia]
Ha formazione tecnica in Lombardia e a Milano, dove partecipa all'esperienza futurista \t viene realizzato solo il cimitero di guerra di Monfalcone \\
1888-1916 \t muore in battaglia sul Carso \\
Fa dei disegni architettonici in prospettiva geometrica a quota zero (linea di terra e linea dell'orizzonte coincidono, quindi viene esaltata la verticalità) \\
Non sono dipinti ma neanche progetti e neanche dei piani urbanistici \t sono prospettive che danno visione di una porzione di citta, dove si notano gli edifici ma anche la strada o la ferrovia, che sono compenetrati \t città continua

\ss[Città nuova 1]
Edificio di grandi dimensioni \\
Si percepisce l'utilizzo del calcistruzzo, e il ferro e il vetro \\
Elementi che sporgono dall'edificio, che probabilmente contengono gli ascensori \t e assoluta mancanza di decorazioni \\
La ferrovia inoltre entra nell'edificio \t compenetrazione dello spazio nell'oggetto

\ss[Città nuova 2]
Questo è inchiostrato \\
Si vogliono rappresentare degli edifici enormi, dei grattacieli

\ss[Stazione aereoferroviaria]
Si inventa una nuova struttura \t inizia ad esserci gli aerei, quindi si immagina una combinazione per treni e aerei \\
Concetto di avanguardia \t prefigurazione del futuro

\ss[Stazione di Milano]
Concorso viene vinto da Stacchini, che negli anni 30 applica il suo progetto eclettico \\
Sant'Elia non partecipa al concorso, pero aveva realizzato uno schizzo \\

\ss[Cattedrale - Chiattone]
Assenza di decorazioni, articolazione volumetrica complessa, uso del calcestruzzo \t riprende pero le torri delle cattedrali gotiche francesi

\ss[Edifico visto da un aereo - Marchi]
Dopo prima guerra mondiale nasce versione del futurismo: aereopittura = rappresentazione di paesaggi naturali/urbani da un aereo in volo \\
Questo è un dipinto \t si percepisce l'idea del movimento dell'aereo \t immagina degli edifici in estrema continuita \t qui non c'è distinzione tra edifici, solo un'entita continua \\
Edifici sono impostati su linee curve \t movimento dell'aereo \\
Marchi dopo la seconda guerra mondiale lavora in realta come scenografo \\
La citta viene considerato come un luogo di espazione dell'attivita percettiva e sensoriale dell'uomo \t attraverso una molteplicita di stimoli di tutti i tipi \t come un luna park \\
Luci, rumori, che investono l'uomo, da cui ne viene stimolato \\
Dall'angoscia di Kirchner, alla neutralita di Leger, all'esaltazione futurista della città

\v

Umberto bonetti

\end{document}
