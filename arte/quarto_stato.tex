\documentclass[12pt]{article}

\usepackage[a4paper, total={6in, 8in}]{geometry}
\usepackage{textcomp}

\begin{document}
\setlength{\parindent}{0pt}

\def \t {\textrightarrow}
\def \v {\vspace{1em}}
\def \bi {\begin{itemize}}
\def \ei {\end{itemize}}
\def \s[#1] {\section*{#1}}
\def \ss[#1] {\subsection*{#1}}
\def \sss[#1] {\subsubsection*{#1}}

\s[Pellizza da Volpedo]
Conclude l'opera nel 1901 \\
Lui viene da Volpedo, in Piemonte \t si forma fra Brera e Roma ed è uno dei protagonisti del divisionismo italiano \\
Il divisionismo italiano usa una tecnica molto moderna, ma i contenuti sono ancora tradizionali \t simbolisti o di denuncia sociale \t non c'è ancora rappresentazione del banale quotidiano, come nella riv. impressionista \\
Pellizza è socialista \t assegna all'arte una funzione educativa \t è uno dei mezzi per elevare culturalmente le masse \\
Questo definisce il suo percorso artistico \t e il suo dipindo si colloca nella crisi agraria degli anni 80 \t il prezzo del pane aumenta, il popolo minuto è alla fame e nel 98 ci sono rivolte \\
Umberto I manda Bava Beccaris a fermare la folla \t e apre fuoco sulla folla \t poi Umberto I ringrazia Beccaris, e poi Gaetano Bresci (un anarchico) assassina il re

\ss[Ambasciatori della fame]
Questo porta Pellizza ad elaborare il tema delle proteste \t che fa gia prima del "Quarto stato" \t "Ambasciatori della fame" \\
Assiste a una scena nel suo paese, in cui dei braccianti parlano con il padrone per chiedere condizioni di vita migliori \\
Le tre figure sono in secondo piano \t in primo c'è l'ombra, che fa vedere come la luce è frontale \t ed è l'ombra del palazzo Malaspina (dei signori locali) = localizzazione specifica \\
La figura centrale \t ha un cappello e gilet rosso \t sta camminando \t questa figura centrale rimarrà anche negli altri dipinti \\
Ai suoi lati altre persone \t mentre dietro c'è la folla \t non è una protesta però, quelli dietro sono fermi \t stanno aspettando \\
Questa è la prima traccia del "Quarto stato"

\ss[La fiumana]
È uno studio preparatorio del "Quarto stato" \t ha dimensioni simili e si trova a Brera \\
È infatti anche nella forma di un bozzetto, nonostante le dimensioni \t infatti: non è divisionista, le persone davanti sono distinte mentre dietro mancano i dettagli \\
Scompare pero la localizzazione specifiica \t non è cronaca di un evento come prima, ma rappresenta la protesta \\
Le persone sono + vicine allo spettatore \t la figura centrale è la stessa: gilet rosso, cappello, aria sicura \\
A sinistra un vecchio, a destra una donna con un bambino \t rimanda alle vergini con bambino popolari di Caravaggio \\
Dietro una folla poco definita \t e poi un paesaggio, e una luce frontale che proietta la luce dietro i personaggi \\

\ss[Il quarto stato]
Si ritrovano gli stessi personaggi in primo piano, mentre la donna è a piedi nudi \t è sua moglie Teresa \\
La massa retrostante non è casuale come prima, ma distinta \t occupa la fascia centrale del dipinto completamente \\
I personaggi sono caratterizzati individualmente \t hanno dei gesti, parlano tra loro, alcuni hanno occhi chiusi per la luce del sole all'alba \\
Attenzione maggiore ai singoli personaggi \t utilizza gli abitanti di Volpendo come modello \\
Figura al centro \t è sull'asse di simmetria verticale \\
Poi 3 fasce orizzontali:
\bi
    \item la prima comprende i 3 personaggi
    \item la seconda la folla
    \item la terza il paesaggio
\ei
Il quarto stato sono gli operai e i braccianti \t i + poveri \\
Marciano in protesta, che però è lenta e sicura \t non come quella di DeLacroix, che era violenta e dirompente \\
Qua si legge la sicurezza di avere dei diritti \t e anche dell'ottimismo, nel fatto che si raggiungeranno gli obiettivi \\
È un'opera monumentale \t diventa infatti un'opera iconica \t diventa il simbolo del socialismo italiano, ma viene anche ripresa quando si vuole rappresentare la protesta \\
All'inizio in realta l'opera non ha successo \t e in aggiunta alla morte della moglie di parto, si suicida

\v

Il dipinto viene acquistato dal comune di milano \t durante il fascismo rimane nelle cantine \t poi dopo WW2 viene esposto a palazzo marino \t poi al museo del 900 \t e infine al GAM

\ss[La rivoluzione siamo noi]
Beuys \t è artista concettuale tedesco e leader dei verdi \t fa questa fotografia, che è un autoritratto, a grandezza naturale \\
Chiaro riferimento alla figura centrale del "Quarto stato"

\ss[Things aren’t as bad as they could be]
Liu Xiadong, artista cinese, di indirizzo iperrealista \t realizza quello che viene considerato "Il quarto stato dei migranti" \\
Ha fatto a Milano una serie di bozzetti preparatori \t davanti a milano centrale ha fatto posare le persone \\
Le tre figure centrali, con la donna e il bambino \t riferimento molto chiaro \\
Ora i migranti devono lottare per i loro diritti 

\v

Film "900" di Bertolucci \t due protagonisti: il primo è figlio del proprietario terriero, altro del mezzario \t finisce con la prima guerra mondiale
\end{document}
