\documentclass[12pt]{article}

\usepackage[a4paper, total={6in, 8in}]{geometry}
\usepackage{textcomp}

\begin{document}
\setlength{\parindent}{0pt}

\def \t {\textrightarrow}
\def \v {\vspace{1em}}
\def \bi {\begin{itemize}}
\def \ei {\end{itemize}}
\def \s[#1] {\section*{#1}}
\def \ss[#1] {\subsection*{#1}}
\def \sss[#1] {\subsubsection*{#1}}

\s[Quotes from Mary Shelley]
\bi
\item 3: Realism claim: The work isn't merely supernatural terrors but based on a plausible scientific premise exempt from typical ghost story disadvantages. Coleridge comparison: Yes, similar attitude—both blend imagination with verisimilitude requiring "willing suspension of disbelief."
\item 4-9: Pict, sublime, sublime, sublime, sublime, pict
\item 11-16: To gods with unlimitd powers, who can controll nature. The analogy is with Adam: the creature compares himself to him.
\item 17-19: Show hubris, overreaching human limits, defying natural order, Promethean theft of divine prerogatives (creation, conquering death) for glory.
\item 21-23: Convergence: Both seek glory, knowledge, have ardent curiosity, risk everything. Divergence: Walton seeks geographic discovery; Victor seeks scientific creation. Both seek forbidden knowledge and learn dangers of overreaching too late.
\item 24-27: Reveals tabula rasa (blank slate), sensory confusion, existential crisis about identity/origin, gradual learning through experience, shaping by culture/books.
\item 29-30: initial desire for a connection and a companionship. Final revenge and hate
\item 31: they are indissoulubely bounded to eachother, and only the annihilation of one of them can end this connection.
\item 33: inital denial of reality that turns in a gradual aceptance, and the realization of the presence of grief in the world.
\ei

\s[Teacher's notes]
Ask note of first two questions

\ss[Question 3]
Partly it is a gothic novel (which was however becoming less and less popular), while there is a blending of supernatural and realism \\
For Coleridge fancy and immagination were different, while are generally considered synonyms \\
There are some gothic elements, but it is more realistic than the typical gothic novel \\
Coleridge requires a "suspension of disbelief" in Biograpia literaria \t but here she makes clear that Frankenstein has a realistic frame \t there a focus on the importance of realism

\ss[Question 13]
“I became myself capable of bestowing animation upon lifeless matter.” \t reference to Prometheus \t see both myths \\
1:His liver get eaten every day from an eagle \\
2: He creates humans from clay \t and Victor impersonificates both: he creates life but goes beyond human limits

\ss[Question 15]
Biblical reference \t the creature compares himself to Adam, but finds himself completely different \t while Adam was beatiful, he is horrible and isolated just like his creator

\ss[Question 17]
Titanism = act of ribellion against an unjust power \t Prometheus \\
Victor wanted to achive glory and benefit humanity, by eliminate disease

\ss[Question 18]
Reference to Prometheus

\ss[Question 19]
Victor is tinanic, but also Walton \t he wanted to explore new places for humanity \\
The third figure is the cretaure, that rebells and seeks revenge

\ss[Question 21]
Victor also wants to achieve glory \t not only to overcome human limits \t Victor with knowledege, Walton with exploration

\ss[Question 23]
Victor warns Walton \t this is what is makes it a cautionary tale \\

\ss[Question 24]
The creature is talking about its first months of life \t it emerges the absense of a family, and he is like a tabula rasa \t he had to learn everything by himself \\
Outside he was a grown man, inside just a baby \t and his cognitive state was \\
He was haunted by questions like "who was I?" \t he seeked his identity

\ss[Question 26]
He initially had only perception \t and he could understand this stimuli and sensations only by refining his intellect \\
He developed from a congitive point of view, and developed the ability to memorize \\
He leared by reading and taught it to himself

\ss[Question 29]
At the beginning he wanted to enter society, but then decided to remain isolated

\ss[Question 30]
It is the creature who has power, and has become the master of its creator \t he defines Victor as a slave

\ss[Question 31]
Life is an essential theme \t but also death, a lot of characters lose their life \\
The creature believes that their mutual misery (creator and creature) will only end when one of them dies

\ss[Question 33]
One of the most touching quotations \\
Specific theme and universal one \t Victor is reflecting on the death of his mother \t but then the reflection becomes universal \\
It is death: it happens to everyone and there is a reflection on mourning and its stages \\
The different stages of loss are a state of shock at first, then little by little you achieve acceptance of the death of this person and then you keep living

\ss[Question 36]
Progressive (actually radical) statement \t there were only two democratic countries: usa and the swiss republics \t here Victor says that it is better to live in a republic than in a monarchy \\
Here the lower classes are free and not slaves like in monarchies

\ss[Question 39]
"the sanguinary laws of men" this phrase is Russonian because it is related to the "Good savage myth" \t man is born good, but gets corrupted by civilization

\ss[Question 40]
It is racist because Victor seemed fine, because he was white \t he calls the locals as savages

\ss[Question 43]
Reflection on language, typical of the second generation of romantics \t it reflects on the limitations of words, that cannot express everything

\s[Main points about the novel]
The genre is hybrid \t combination of epistual and gothic novel (even if she wanted to overcome this genre) and it is an anticipation of sci-fi \\
It is also a chinese box \t characters tell stories \\
We know that we are in the 18th century \t there are no chronological references, and the letters are missing the century \\
There are different places in which it is set:
\bi
    \item Geneva: main setting, when it comes to Victor's childhood (which was a place of happiness) and adulthood (but then becomes a city where he met tragedy as well: his parents died and his brother was strangulated here)
    \item Ingodstadt: where he studies and works on his plan of creating life artificially \t most scenes are at night here \t place associated with darkness (physical and metaphorical: he was obsessed and isolated)
    \item North Pole: here he meets Walton and where he (and creature perhaps) die
    \item minor settings: UK, scotland, the Lake District and Northern Ireland (here he was accused of killing Henry, and remained in prison for 3 months)
\ei

\v

Other themes are sublime and pictoresque nature, then science \t through science the impossible is achieved \t but it is also delt ethically \t scientists cannot research everything they want \\
There have to be ethical boundaries \\
The creation of life is another topic \t but it is a perveted creation of life \t because brought up by a scientist and not god \\
He also tries to remove the right to women to procreate (according to some feminist critiques) \\
Another theme is titanism \\
The persuit of glory and knowledge \t both Walton and Victor search \t they were driven by the overcoming of human limits but also by glory \\
The creature is a metaphorical figure \t because he rebells against his creator \t and his legitimate pursuit of knowledge, because he wanted to aunderstand who he was \\
This is also a tribute to Locke's theory of understanding \t the creature had to learn through experience and a cognitive evolution is developed \\
William Godwin was also an anarchist \\
Walton is a double of Frankenstein (see page 278-79) \\
Reasons for inspiration: Galvani's experiment, personal anxiety and love for ghost stories \\
The creature is the like the good savage of Rousseau \\
"The rebel" page 283
\end{document}
