\documentclass[12pt]{article}

\usepackage[a4paper, total={6in, 8in}]{geometry}
\usepackage{textcomp}

\begin{document}
\setlength{\parindent}{0pt}

\def \t {\textrightarrow}
\def \v {\vspace{1em}}
\def \bi {\begin{itemize}}
\def \ei {\end{itemize}}
\def \s[#1] {\section*{#1}}
\def \ss[#1] {\subsection*{#1}}
\def \sss[#1] {\subsubsection*{#1}}

\s[Dickens - Hard times]
Coke = burnt coal \t coal was the fossil fuel used and in northern england there were many coal mines \\
Gradgrind is a philantropy \t he founds a school, of which he was principle and teacher \t he wanted to teach his students only facts, things that are aknowledged to be true \\
He also had two children and was a widower \t he brought up his children only with the importance of facts, and not by developing their imagination \t he failed at being a parent \\
These two children became unhappy \t the daughter married a man with whome she was not in love him, and eventaully separated as he bankrupted \\
The brother was bad an robbed the husband bank, and he escaped the law because he hid in a circus \\
Sissy Jupe the year before, had been abandoned by his father (a circus man) \t Gradgrind adopted her and forced the same education of his children \t she pretended to follow his educational point of view \t but kept her sensitive side, and in fact she had a happy life \\
Gradgrind looked down at sissy and the circus (it was a too imaginative and creative reality) \t but by the end of the novel, he understands his mistakes as an educator and as a father \\
Usually dicken's characters are flat \t here Gradgrind is actually round, and is an iconic character \\
His phylosophy of education was materialistc \t emphasis on facts only and not on imagination and creativity

\v

Hard times is divided in three volumes \t each one divided in chapters \\
Bounderby was a banker and a factory holder \t he considers workers as objects \\
Utilitarism is a phylosophy that was born in england \t Benthan is the founder \t everything that is beneficial for the majority of the population, it is necessary good morally \\
It is materialistic \t what is useful is good
\end{document}
