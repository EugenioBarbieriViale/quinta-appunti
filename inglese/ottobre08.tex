\documentclass[12pt]{article}

\usepackage[a4paper, total={6in, 8in}]{geometry}
\usepackage{textcomp}

\begin{document}
\setlength{\parindent}{0pt}

\def \t {\textrightarrow}
\def \v {\vspace{1em}}
\def \bi {\begin{itemize}}
\def \ei {\end{itemize}}
\def \s[#1] {\section*{#1}}
\def \ss[#1] {\subsection*{#1}}
\def \sss[#1] {\subsubsection*{#1}}

\s[Coleridge - page 300]
Coleridge and wordsworth were friend, and also collaborated \t they created strong bonds \\
Friendship was a substitute of family \t some of them had difficult familiar situations (like Blake, Wordsworth, Keats \t they lost close relatives) \\
Shelley and Byron were shawnt by their families \t because they had an unconventional lifestyle, not approved by the aristocratic families they came from \\
At some point they started to disagree \t colerdige wanted to write more elaborated poetry, and considered wordsworth works simplistic \\
Christable \t is gothic poem (with gothic elements, not in the style)

\ss[Gothic novel]
\textbf{Gothic literature}: mainly made up of novels \t it is an expression of pre-romanticism \\
The first gothic novel was the "Castle of Otranto" \t written by H. Walpole (1764) \\
Walpole was the son of the first british prime minister \\
Other important gothic novels are:
\bi
    \item "The monk" \t Lewis
    \item "The mysteries of Udolpho" \t Radcliffe (female author)
\ei
The raise of the novel \t women were central in the development of the novel \t first time in history that women took part to the dev. of a literary genre \\
Women were frequent readers \t some of them even wrote them (like Bradly, Aphra Behns \t she wrote "The royal slave" at the end of the 17th century) \\
The royal slave is considered the first novel in english literature \\
Canon = the authors that are considered the most rapresentative of a nation \\
Behns was disappeard from the Canon for a long time \t she was a female and the novel was anti-colonial \\
The slave was an african prince who was captured \t but he lead a riot against the owner \t he is a positive character 

\v

Gothic novels were like horror stories \t dealing with mistery, crimes \t they tend to be set in south europe (in particular in Catholic countries) \t demonisation of catholic europe \t british authors were protestant \\
Crimes were committed by villains in dark places like abbeys \t the protagonists are mainly innocent people \\
Some of them have supernatural elements (like ghosts)

\ss[Interpretation of the acient mariner]
Other interpretation is a colonial one \t the albatross is a native of the place and helps the visitors \t but then gets killed \\
The last thursday of november the Thanks Giving day is celebrated in USA and Canada \t in 1621 was celebrated for the first time \\
The piligrim fathers wanted to celebrate their first harvest \t in the newly founded colony in new england \t they thanked God and the native americans \t they helped the piligrim fathers \\
But then they were persecuted by americans \\
Colerdge may have made a critique to the persecution of native americans by the colonizers \\
Piligrim fathers were puritans \t they were persecuted by the anglican king James I \t they left in year 1620
\end{document}
