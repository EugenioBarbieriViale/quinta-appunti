\documentclass[12pt]{article}

\usepackage[a4paper, total={6in, 8in}]{geometry}
\usepackage{textcomp}

\begin{document}
\setlength{\parindent}{0pt}

\def \t {\textrightarrow}
\def \v {\vspace{1em}}
\def \bi {\begin{itemize}}
\def \ei {\end{itemize}}
\def \s[#1] {\section*{#1}}
\def \ss[#1] {\subsection*{#1}}
\def \sss[#1] {\subsubsection*{#1}}

2 most relevant british romantic painters are Constable and Turner \\
Constable tended to put the focus on pictoresque landscaped, while Turner on sublime ones

\s[Second generation of british Romanticism]
They were:
\bi
    \item Byron
    \item P. B. Shelley
    \item Mary Shelley
    \item J. Keats
\ei
They are called members of the second generation because they were younger \t they were born 20 years after Wordsworth and Coleridge \\
They also died young, under tragic circumstances \\
Byron died for example while he was in Greece, as he prepared to fight against the Turkish Empire \t he died from a malaric feaver, during training \\
P. B. Shelley died during a storm, while sailing on a ship he owned, in Liguria \\
Keats died of tubercolosis \t he was burried in a protestant cemetery in Rome 

\v

Politically speaking \t they were radicals \t they didn't have the time to become conservative \\
They were also critical of Coleridge's and Wordsworth's conservatism as adults

\s[Lord Byron]
1788-1824 \\
He was the member of a powerful aristocratic family \t his family was so prestigious that he had a seat in the british parliament \\
Parliament was made of 2 houses:
\bi
    \item house of commons \t MPS (members of parlament) are elected through general election
    \item house of lords \t whose seats are inherited 
\ei
However his childhood was not happy \t his parents neglected him \t he had been raised by his calvinist nanny (she brought him up and influenced him from a religious point of view) \\
She was fundamental in his upbringing (not education)

\v

He then attended university \t then he became extremly famous in London \t he was like an idol
\bi
    \item literary fame \t his works were appreciated and were best sellers
    \item biographical fame \t his lifestyle was scandalous \t some critized him for this reason, others appreciated his freedom
\ei
He was married, but had several lovers \t he believed in free love \t he despised marriage as an istitution \\
It is also believed that he was bisexaul \\
At one point he became member of the parliament \t however a day a mob (infuriated crowd) attacked his house \t some rumors had spread about his homosexuallity \\
It was also believed that he had incestous relationship with his half sister \\
He wasn't hurt, but he was so shocked that in year 1816 he decied to leave England for good \\
For him english society was too narrow minded and moralistic \t he had a fall from grace 

\v

He moved to the continent (continental europe) \t he leaved in Geneva, where he owned Villa Diodati \\
Thi villa is important for romanticism \t in year 1816 one night he was hosting p. b. shelley and his wife (mary) \\
They decided to have a writing competition \t the winner had to right the most horrifying horror story \t the winner was Mary Shelley \t she wrote only a short tale \\
Then she expanded it in a full novel in the following 2 years \t it is Frankenstein \\
In the introduction she claims that the story comes only from her imagination \t but recognizes that Byron and his husband played an important role \\
She wrote the tale after a nightmare she had \t the dream was inspired by a conversation she had heard bewteen Byron and P. B. Shelley \\
The 2 were talking about an experiment of Galvani \t he tried to make dead frogs come to life with electricity 

\v

Byron travelled then in europe \t he lived in Venice, Rome, Ravenna \t while he was in Italy, he was in touch with italian indipendetists, who were against the austrian empire \\
The austrian police found him \t so he went to Greece \t he was very radical from a political point of view \t he was ready to risk his life to help indipendentists (both italian and greek)

\v

He was very athletic \t but he was a lame, a cripple \t malformation of a foot \\
He was psycologically touched by this physical disabilty \t he suffered from this \\
Even his view of nature depended on this disability he was born with \t he was fascinated by the sublime aspects, on the other hand he accused nature to be a evil stepmother \t she made his life difficult \\
Nowadays british scholars do not consider neoclassicism and romanticism as in stark opposition \t byrond embody this duality and the evolution of romanticism fron neoclassicism \\
He composed romantic and neoclassical works \t he was a romantic poet but also neoclassical production \\
Most famous for narrative production \\
Romantic works:
\bi
    \item lyrical poems \t collected under "Hours of Idleness (ozio)"
    \item verse tales \t narrative poems in verses \t most important ones under the collection "Oriental tales" \t most two relevants are "The Corsair" and "Lara" \t here the character feels strong and powerful emotions, and the atmosphere is gothic \t typical main character is the figure of the "Byronic hero"
    \item "Childe Harold's Piligrimage" \t semi-autobiographical poem \t not totally, because the events are loosely based on his life \t the protagonist is Harold, an aristocratic young man, who travells on a grand tour in Southern Europe (like Byron did)
    \item verse plays \t written in poetry (not prose) \t philosophical plays, focusing on political freedom \t they were rarely put on stage because are very complex to follow \t they are mainly meant to be read \t ex. "Manfred", "Marino Falliero", "The two Foscari"
\ei
He also wrote neoclassical works, in the form of satirical / mock-heroic poems \t written in "ottava rima" (eight-line stanzas) \\
They made fun of the hypocrisy of the british society \t ex. "Don Juan" \t he is an anti-hero \\
The verses followed a specific rime pattern \\
In writing this works, he was influenced by Augustian poetry, which was mocking and used poetic dicition \t he was also influenced by Swift (for the use of satire) and Sterne (for the use of digressions)

\end{document}
