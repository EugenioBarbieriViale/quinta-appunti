\documentclass[12pt]{article}

\usepackage[a4paper, total={6in, 8in}]{geometry}
\usepackage{textcomp}

\begin{document}
\setlength{\parindent}{0pt}

\def \t {\textrightarrow}
\def \v {\vspace{1em}}
\def \bi {\begin{itemize}}
\def \ei {\end{itemize}}
\def \s[#1] {\section*{#1}}
\def \ss[#1] {\subsection*{#1}}
\def \sss[#1] {\subsubsection*{#1}}

\s[Byron]
He is more complex from a stylistic point of view?? \t he is culturally elitist in his satirical poems \t he used poetic diction (lofty style characterzing augustian poetry = latinisms, figures of speech) \\
But also the use of satire, wit and irony \t important aspects of augustians poetry and Byron's one
\bi
    \item wit = the ability to relate things in a sophisticated and funny way
    \item irony = the use of words to express the opposite of what you really mean
    \item satire = is is an aritsitc technique aiming at exposing and discrediting people's vices by recurring to wit and irony
\ei
On the one hand Byron was fascinated by sublime natura \t but at the same time it felt resentemnt because of his physical disability \\
Byron is the embodyment of the overcoming of romanticism over neoclassicism \t they are not in contrast \\
Titanism = attitude of ribellion against authoriry that the subject perceives as dispotic and unjust \t term comes from the mythological figure of the titan (most famous one: Prometheus) \\
Prometheus has 2 versions:
\bi
    \item the titan creating humans out of clay
    \item the titan stealing fire from zeus to donate it to humans (more important version to romanticism) \t for this reason he was tied to a rock for eternity (an eagle ate his liver every day) \t he represents the punished rebel for a right cause
\ei
The tone in the romantic works is titanistic, idealistic, rebellious, fascinated by Nature, gothicism, nationalism \t he was a champion of political freedom (not of individuals, but of peoples) \\
The concept of the exocit has changed with the development of means of travel \t italy was an exotic place at the time \\
The neoclassical works are more realistic \t skeptical realism, because his pessimistic point of view on society

\v

The byronic hero is the typical character of byron's romantic poems

\ss["Lara" \t The Byronic Hero]
Proud, mysterious, dark, rebellious \\
Scoreful towards society \t for him common people tended to lack from sensibiltiy \\
He is mysterious \\
He helps to be admired, not for pity 

\v

Lara is the sequial to another oriental tale \t "The corsair" \\
The protagonist is a spanish count and Lara is the name of the possession he owned \\
He decides to leave his privileged position and travel the world \\
Then he returns to his lands \t in the meantime they were usurped by another aristocrat \\
Similar to the corsair \t why ??? \\
The byronic hero is considered by scholars as an alter-ego of byron himself \t but it is not an autobiagraphical figure (this would be too far fetchedl) \\
It is a romantic character, but also a controversial one

\ss[Analysis of the text]
He is controversional because he conjured up mixed feelings in people \\
Line 2 \t there was something to be loved and feard \t he is dangerously attractive \\
He was a taciturn \t he didn't talk much, and people tended to speculate on his secrets for this reason \\
He had got a mysterious past
\end{document}
