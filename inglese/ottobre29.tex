\documentclass[12pt]{article}

\usepackage[a4paper, total={6in, 8in}]{geometry}
\usepackage{textcomp}

\begin{document}
\setlength{\parindent}{0pt}

\def \t {\textrightarrow}
\def \v {\vspace{1em}}
\def \bi {\begin{itemize}}
\def \ei {\end{itemize}}
\def \s[#1] {\section*{#1}}
\def \ss[#1] {\subsection*{#1}}
\def \sss[#1] {\subsubsection*{#1}}

\s[Byron]
Life \t page 314 \\
He was an unconventional aristocrat with a scandalous life \t first people were attracted by this \\
Then he was sorrounded by scandals and debts \t so he leaves \\
He made negative comments of romantic poetry \t he defined it unelegant and boring, using even swear words (like against Keats) \\
He believed in indidual freedom and freedom of nations from empires \\

\ss[Childe Harold's Piligrimage]
Page 318 \\
A childe = a yuong man waiting for knighthood (minor british aristocricy, not the major one, like barons, earls and dukes) \\
Semi-autobiographical \t the adventures of Harold replicate the ones of Byron \\
Childe Harold is a sort of the representation of the byronic hero \\
There is a red thread in byron's works \t it is the importance of fighting against political oppression \\
Main themes:
\bi
    \item journey
    \item nature \t picturesque and exotic, but also sublime \t often it mirrors the character's mind and feelings
    \item freedom \t individual, or the one of whole populations from tyrans
\ei


\s[P. B. Shelley]
Was member of the second generation of the british romantics, along with Byron, Mary Shelley and John Keats \\
He died really young \t he was 30 years old \\
He belonged to a rich and aristocratic background \t just like Byron \\
He could have a great education \t he attended Eton (not a university, but a prestigious highschool, a prep school for university) \\
Shelley hated his experience there \t he did not enjoy his time there \t his classmates tended to be posh kids and they were also obsessed with sports \\
The whole institution was obsessed with sports \t but he hated sports, for this reason he was bullied \\
He was not into sports, but also had strong political ideas \t this being radical clashed against his aristocratic background \\
Then he enrolled to Cambridge university \t he liked the experience from an academic point of view, but on 1812 he published a controversial pamphlet \\
Its title was: "The necessity of Atheism" \t pamphlet = essay (work in prose) about a social/political topic \t a specific political idea is argued by the author \\
It was controversial becaus this essay was in favor of atheism, which was illegal in those years \t he published the pamphlet anounimously \t but his identity was uncovered \\
He wasn't arrested thanks to the connections of his family \t but got expelled from the university \\
His parents were disappointed and cut off their financial support \\
But shelley was supported by his two elder sisters, who were close to him

\v

In the pamphlet he claims that atheismm is necessary to achieve democracy in society \\
To him there is no rational proof of the existence of god \t so he doesn't believe in god \\
He also doesn't understand how god can be described as both the god of love (in the gospels) and a punisher god (in the bible) \t for him this is contradictory \\
This theological view became political \\
European monarchies are based on the "divine rights of kings" \t according to which a rightful monarch is one of god representatives on earth \\
Monarchy has to exist, according to this \t but if the existence of god is denied, then also the existence of monarchy is unjustified
\end{document}
