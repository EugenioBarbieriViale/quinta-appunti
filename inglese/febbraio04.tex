\documentclass[12pt]{article}

\usepackage[a4paper, total={6in, 8in}]{geometry}
\usepackage{textcomp}

\begin{document}
\setlength{\parindent}{0pt}

\def \t {\textrightarrow}
\def \v {\vspace{1em}}
\def \bi {\begin{itemize}}
\def \ei {\end{itemize}}
\def \s[#1] {\section*{#1}}
\def \ss[#1] {\subsection*{#1}}
\def \sss[#1] {\subsubsection*{#1}}

\s[Charles Dickens]
He was born 1812 and died in 1870 \\
He was born in the South of England \t moved to Kent as he was a child \t so he lived in the countryside for some years \\
These biographical experience had literary consequences \t his novels tend to be set in the city, to explore the social issues of the victorian age \\
However, some scenes are set in the countryside \t and they tend to be happy \t the countryside it is an idillic place \\
Finally he moves to London \t he spends the rest of his life here (and part of his childhood) \\
He came from a middle-class family, but his father ended up contracting a lot of debts \t as a result, Dickens had to dropout school and work in a factory of shoe-polish = traumatic experience \\
This influenced some of his novels \t some of his characters worked in factories \\
The financial situtation didn't improve and his father ended up in a debtors' prison \t so his family had to move to a workhouse \\
Then, they received an unexpected inheritance \t the father was able to pay off his debts \t so Dickens could stop working, they moved back to a house and Dickens continued attending school \\
He graduated from high school, but didn't attend university \t at first he was a clerk (impiegato) in a law firm \\
This experience was important also for his novels \t he talks about justice in england and how it is unjust \t Oliver Twist goes to court, and he wasn't treated rightfully by the judges \\
Then he became journalist \t covering news regarding trials and the parliament \t Court and Parliament reporter \\
His first literary work was a collection of humorous articles, published under the name of Boz \t "Scketches by Boz" \\
As he became successfull, he became a professional writer \t he kept being a journalist, but also wrote fictionary novels

\v

His first novel was "The Picwick Papers" \t a houmurus novel, aobut a group of rich middle-class men passionated about sport \\
They travelled around England for the sake of sport and lived a series of misadventures \\
His attitude towards victorianism: he shared the ideals of the age (industriesness, family, etc.) \t but he was aware, that there were tragic issues \t like prostituition, child labor, etc. \t he explores these themes \\
He had also 9 children, but then he fell in love with another woman (an actress) \t Dickens's main hobby was theatre \\
In the novels there are brilliant dialogues \t one can see the influence of theatre in novels for this aspect \\
Eventually he separated from his wife, and lived an official relationship with this woman \t but, being a proud victorian, he felt guilty \\
He was a celebrity in the UK \t he received publich backlash because of this decision regarding the separation \\
He was also famous in the USA \t here he gave public readings of his novels and univeristy lectures \\
He was also a workatholic ??? \t he worked really hard, and had a consuming life \t he died pretty young

\v

His novels can be clasified in early and late ones. \\
Early:
\bi 
\item The Picwick Papers
\itme Oliver Twist
\ei
\ss[] 
A Christmas Carol \t novel, whose protagonist is Scrooge (stingy and greedy old man) \t he accumulates money, not to be generous \t and hated christmas, and cosidered it a waste of money \t scrooge = party pooper \t but over the course of the novel, he changes  \\
His employee is Bob \t he is happy and has a numerous family \t but Tiny Tim is sick (one of his children, and is going to die, unless someone pays for his medical bills) \\
Scrooge has a profitable business, and doesn't want him to take some days off \\
On the night of christams eve, he is visited by 4 ghosts \t the first one is the one of his previous college, who is wearing chains \t he says, that if scrooge doesn't change he will end up like him \\
Then he is visited by the ghosts of Christmas Past, Christmas present and Christmas future \\
The first one shows him his childhood \t his parents died and he was left behind by his sister \\
It also shows him his previous fiance, who broke up with him because he only thought about money \\
And also shows his boss when he was an employee \t here he loved chrsitmas \\
The, in presence, he learns about the life of Tiny Tim and his family \t and also shows him other people having fun during chirstmas \\
Then the future one shows him his funeral \t where nobody was there and everybody was occupied in taking possession of his belongings \\
So in the end he changes, starts to love christmas and pays for Tiny Tim medications \\
Dickens contributed to the creation of the cult of chirstmas 

\bi
    \item david copperfield \t this is a novel of formatio \t he achieves happyness without compromising and without losing his good heart
    \item bleak house \t about the iniquities of victorian laws
    \item hard times
    \item a tale of two cities
    \item great expectations
\ei

There is usually a third person omniscient narrator \t and uses irony to pass judgement on his characters \\
Victorian novelists tend to privilege this type of narrator in general \\
The typical setting of Dickens' novels are english cities \t mainly london, but also, like Oliver, who was born in an unnamed industrial city \t but moves to London??? \\
They can be also ficitonal cities \t but parts of his novels are also set in the countryside, which is idealized and idillic \t there are no paesants starving \\
His characters belong to the middle and working class \\
His characters tend to be flat \t Scrooge makes an exception, he changes during the novel \\
Dickens' tone is melodramatic \t the characters had to go through tragic situations, but in the end they find happy endings, for which he uses plot twists \t critized for this \\
Melodramatic because his novels teach moral lessons ??????? \\
Also comic \ the uses humour irony and wit \\
Some characters were so memorable that their names entered the english vocabulary

\ss[Questions]
1) Describe the circumstances of Oliver’s birth. Oliver was born in a workhouse in a nameless town. His mother, a young, exhausted woman found lying in the street, died almost immediately after giving birth to him. She left no name or identification, leaving Oliver an orphan from his first moments. \\
2) Where was baby Oliver sent after his birth? What was his life like there? He was “farmed” out to a branch-workhouse run by a woman named Mrs. Mann. His life there was miserable; the children were subjected to systematic neglect and starvation. Mrs. Mann kept the majority of the parish funds intended for the children's food and clothing for herself. \\
3) Who was Mr Bumble? Mr. Bumble was the parish beadle, a minor church official with authority over the workhouse. He is portrayed as a pompous, self-important, and hypocritical man who delights in his petty power and mistreats the paupers under his care. \\
4) Where did Oliver end up when he turned nine? What was his life like there? Upon turning nine, Mr. Bumble removed Oliver from Mrs. Mann’s care and took him back to the main workhouse. Life there was harsh and monotonous; the inmates were slowly starved on a diet of thin gruel and forced to perform hard labor, such as picking oakum. \\
5) What happened to Oliver after the “Please, Sir, I want some more” accident? After Oliver famously asked for a second helping of gruel, the workhouse board was scandalized. He was immediately confined to a dark room and a notice was pasted on the gate offering five pounds to anyone who would take him off the parish's hands as an apprentice. \\
6) Who tried to obtain Oliver? Mr. Gamfield, a chimney sweep, tried to take Oliver as an apprentice. However, the magistrate refused to sign the indentures because he noticed Oliver’s genuine terror and realized that chimney sweeping was a dangerous trade often fatal to young boys. \\
7) Who was Oliver finally given to? He was apprenticed to Mr. Sowerberry, the parish undertaker. \\
8) How was Oliver welcomed there? He was not welcomed warmly. Mrs. Sowerberry commented that he was small and would cost money to feed. He was fed scraps (broken bits of meat) that the dog had refused and was forced to sleep among the coffins in the workshop. \\
9) Who did Oliver meet first thing in the morning after spending his first night at the undertaker’s? He met Noah Claypole, a charity-boy who worked for Mr. Sowerberry. Noah was older, larger, and immediately bullied Oliver, enjoying the chance to feel superior to someone else (since he himself was usually looked down upon). \\
10) What new mansion was Oliver assigned one day? What first funeral scene did he witness? Oliver was promoted to become a mute (a professional mourner) at children's funerals because of his melancholy expression. His first experience was the funeral of a young woman in a destitute slum, where he witnessed the raw grief of the mother and the indifference of the officials. \\
11) Why did Oliver and Noah fall out badly one day? What happened next? They fell out because Noah insulted Oliver’s dead mother, calling her a "regular right-down bad 'un." Enraged, Oliver attacked Noah, despite being much smaller. He beat Noah, leading Noah to scream for help. \\
12) Who else took part in Noah and Oliver’s fight? What was its outcome? Charlotte (the maid) and Mrs. Sowerberry rushed in to help Noah. They beat Oliver and locked him in the dust cellar. Mr. Bumble was called to discipline him, and eventually, Mr. Sowerberry beat Oliver as well to appease his wife. \\
13) What did Oliver decide to do the day after? What was the journey like? Who did he meet right before leaving? Oliver decided to run away to London. The journey was grueling; he had to walk the entire way (70 miles), suffering from hunger, cold, and bleeding feet. Before leaving the town, he passed the workhouse and saw his old friend Dick, a dying child who blessed him and wished him well. \\
14) Who did Oliver meet in a village north of London? What was he like? In the town of Barnet, Oliver met Jack Dawkins, better known as The Artful Dodger. He was a young boy who acted and dressed like a grown man, wearing a coat that was too large and a top hat. He had a swaggering, street-smart attitude. \\
15) What was Oliver’s first impression of London? Oliver found London to be dirty, smelly, and crowded. The area the Dodger led him through (near Saffron Hill) was a maze of narrow, muddy streets filled with filth and poverty, quite different from the grand city he had imagined. \\
16) Who was Oliver welcomed by once in London? Where? Describe him. He was welcomed by Fagin in a dirty, smoke-filled house. Fagin is described as a "shriveled Jew" with a villainous and repulsive face, matted red hair, and dressed in a greasy flannel gown. He was cooking sausages when Oliver arrived. \\
17) What happened the very first morning Oliver stayed at Fagin’s? What characteristics of Fagin’s personality emerge? Oliver woke up to see Fagin admiring a box of hidden watches and jewelry. When Fagin realized Oliver was awake, he grabbed a bread knife and looked at Oliver with murderous suspicion, asking what he had seen. This revealed Fagin's paranoia, greed, and hidden menace. \\
18) What did the children living at Fagin’s turn out to do for a living? Did Oliver realize it straight away? How did Fagin train them? The children were pickpockets. Oliver was innocent and did not realize this immediately; he thought they made wallets and handkerchiefs. Fagin trained them by playing a "game" where he would walk around the room with items in his pockets, and the boys had to remove them without him feeling it. \\
19) What would have happened if the kids had returned home having stolen nothing? They would have been punished, usually by being sent to bed without dinner or beaten. \\
20) How did the two gentlemen react when they realized they were being pickpocketed? Mr. Brownlow (the gentleman reading at a bookstall) realized his handkerchief was gone. When he looked up, he saw Oliver running away (Oliver ran because he was terrified after seeing the Dodger steal the handkerchief). Brownlow shouted "Stop thief!", prompting a chaotic chase. \\
21) What happened to Oliver in the court of justice? Oliver was dragged before the magistrate, delirious with fever and unable to stand properly or speak in his defense. He was locked in a cell briefly before the hearing. \\
22) Who was Mr Fang? Describe him. Mr. Fang was the presiding magistrate. He is described as a disagreeable, bullying, and hot-tempered man who abused everyone in his court and seemed more interested in rushing through cases than dispensing justice. \\
23) What was Oliver sentenced to at first? Mr. Fang sentenced Oliver to three months of hard labor mostly because Oliver was too ill to answer his questions, which Fang interpreted as disrespect. \\
24) Who unexpectedly turned up at the court of justice? What happened next? The bookseller who owned the stall where the theft occurred arrived just in time. He testified that he saw the other two boys (the Dodger and Charley Bates) steal the handkerchief, not Oliver. Consequently, Mr. Fang abruptly cleared Oliver of the charges, and Mr. Brownlow took the sick boy home to care for him.

\end{document}
