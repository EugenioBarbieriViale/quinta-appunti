\documentclass[12pt]{article}

\usepackage[a4paper, total={6in, 8in}]{geometry}
\usepackage{textcomp}

\begin{document}
\setlength{\parindent}{0pt}

\def \t {\textrightarrow}
\def \v {\vspace{1em}}
\def \bi {\begin{itemize}}
\def \ei {\end{itemize}}
\def \s[#1] {\section*{#1}}
\def \ss[#1] {\subsection*{#1}}
\def \sss[#1] {\subsubsection*{#1}}

\s[P. B. Shelley]

\ss[Life]
He was expelled from Cambridge after publishing the pamphlet \\
He married really young with Harriet Westbrook \t they lived a nomadic life and moved around the UK \\
He spent some months in Ireland (which so not indipendent) \t the Republic of Ireland (Eire) has existed since 1921 \t before it was a british colony (partially) \\
The northern part (Ulster) was and still is part of the UK \\
In Ireland Shelley supported irish indipendentists and made political propraganda that supported the rebellion against UK \t it was strange for a british citizen and a member of the upper class \\
He shared the radical and anti-colonial ideas of Byron \\
Harriet and Shelley had two children \t when they returned to London, they eventually divorced \t Shelley fell in love and eloped (run away for love) with Mary Godwin (who will become Mary Shelley) \\
They went to Switzerland and then Italy, where Shelley died \\
Mary was the daughter of two relevant british intellectuals of the time: 
\bi
    \item William Godwin \t radical philosopher known for elaborating the doctrine "radical idealism" = tyranny and monarchy were going to disappear and be replaced by democracy \t unstoppable process started with the French Revolution
    \item Mary Wollstonecraft \t proto-feminist (proto because she anticipated feminism, which became a social movement in the 20th century with suffragettes) \t she wrote "A vindication of the rights of woman" and the main points are:
        \bi
            \item women lived in a situation of social inferiority and marginality due to their lack of proper education \t and her daughter was educated by the father
            \item in order to achieve the same rights of men, they had to be given access to education
        \ei
\ei
Shelley was into philosophy and started to visit Godwin's house \t he was one of his disciples \t and here he meets Mary \\
Shelley's wife committed suicide because of the abandonment of the husband \t he also wanted the custody of the children, but the judge denied it \\
His children were brought up by Shelley's parents \\
After Harriet's death he was a widower and was legally allowed to marry another woman \t for this reason he married Mary \\
They had 3 children \t 2 died \\
Shelley died too, during a storm while he was sailing in Liguria \t Mary became a widow and return to England \\
He was burried in the protestant cemetery of Rome 

\ss[Works]
Most famous: philosophical and lyrical poems (while Byron narrative)
\bi
    \item "Necessity of atheism" \t see previous notes
    \item philosophical/meditative poems \t "Mont Blanc" = meditative poem on nature and its sublime aspects + relationship between mankind and nature
    \item verse drama (like Byron) \t plays written in poetry and philosophical \t "The Cenci" and "Prometheys Unbound" \t themes: political freedom, self sacrifice, fight against tyranny
    \bi
        \item Prometheus is the titan who rebelled against Zeus and stole the fire for humans 
        \item Cenci is set in renaissance Italy \t the protagonist is a young woman named Beatrice Cenci (powerful aristocratic roman family) and takes place in Rome \t she had a violent and abusive father, so she killed him \t she knew she was going to be senteced to death, but did it to free his family
        \item both characters as symbols of titanism (rebellion against an unjust authority)
    \ei
    \item lyrical poems \t "Adonais", "Ozymandias", "Ode to the West Wind" \t and reflect on freedom and how it is going to triumph over tyranny (radical idealism) \t are all political execpt "Adonais":
    \bi
        \item it is an elegy = poem revolving around the theme of death, inteded as both a personal loss of the speaker and a universal theme of mankind
        \item it is about the death of John Keats (another meber of the second gen. of british romantics) \t he died at 25 and was Shelley's friend \t he caught tubercolosis
        \item Adonais is a mythological character, a young man killed by a wild boar \t the gods turned the meadow into a bed of roses \t Keats and Adonais both died of a premature death and are compared for this reason
    \ei
    \item "A defence of poetry" \t published in 1821 \t it is an essay and written in prose
\ei

\ss[A defence of poetry]
Peacock had written the essay "The four ages of poetry" \t main point: in such an age of progress as the 19th century, poetry had become useless \\
Shelley disagreed and wrote this essay \t here he:
\bi
    \item stressed the everlasting importance of poetry \t it has to do with the sphere of emotions, which is something that characterizes humans
    \item defines poetry: "Poetry is indeed something divine. It is at once the centre and circumference of knowledge; it is that comprehends all science."
    \item "It [poetry] exalts the beauty of that which is most beautiful, and it adds beauty to that which is most deformed…it strips the veil of familiarity from the world" \t same point of view of Wordsworth, "coloring effect of imagination" \t here it is beauty
    \item define poets: "Poets…are the unakcnowledged legislators of the world." \t different from the one of Wordsworth, who says they are men among men \t while Blake said that poets where prophets
\ei

\ss[Main themes]
\bi
    \item religion \t he was an atheist and God was for him a social construction \t the king was considerated legitimate because of God
    \item nature \t mainly in its sublime aspects and its relationship with mankind
    \item politics (largely) \t political freedom, showing titanism
    \item language and its limitations \t typical of the romantics of 2nd gen.
    \item love \t a force moving the world and believed in free love \t he had different relationships 
    \item was into science and had a passion for philosophy \t he embranced Godwin's radical idealism, but he also embraced neoplatonism \t he belevies in two worlds: the material and the one of ideas \t but he says that the material one is inferior but can be interesting anyway
    \item material reality is like a veil, like a "many-coloured glass" staining true reality \t the material world is not doll and is interesting nevertheless
\ei
\end{document}
