\documentclass[12pt]{article}

\usepackage[a4paper, total={6in, 8in}]{geometry}
\usepackage{textcomp}

\begin{document}
\setlength{\parindent}{0pt}

\def \t {\textrightarrow}
\def \v {\vspace{1em}}
\def \bi {\begin{itemize}}
\def \ei {\end{itemize}}
\def \s[#1] {\section*{#1}}
\def \ss[#1] {\subsection*{#1}}
\def \sss[#1] {\subsubsection*{#1}}

\s[Introduction Lyrical Ballads - page 191]
"The Rhyme of the acient mariner" is a narrative poem \t not a lyrical one, like the other
Lyrical poems that result in intimate accounts of \\
In the counryside people where under no social constrains \t so they were more pure \t moreover they lived in nature \t their emotions were more true \\
The language is a simple one, but it is purifed from grammar errors and curse words 

\v

The poet in order to write needs to be in a situation of tranquillity \t the poet never writes about powerful experiences while they are taking place \\
He writes after, in a situation of tranquillity

\s[Biographia literaria]
It is a long pilosophical autobiographical (he reflects on his own biografy) essay, written in 1817 by Coleridge \\
Typical subjectivity expressed during romanticism \\
Colerdige ans Wordsworth (after writing the Lyrical Ballads) their friendship cooled down and Coleridge became critical of Wordsworth literature \\
For him it was too sentimental and simplistic (wordsworth's writings) 

\v

In this extract they talk about the birth of the lyrcal ballads \\
They moved together to write them \\
They both reflected on the dual nature of poetry:
\bi
    \item it was supposed to be realistic
    \item at the same time, it was supposed to be interesting \t through the use of imagination, which sometimes broke away from realism
\ei
They managed to combine these to contradictory aspects of poetry in the lyrical ballads \\
They made an agreement \t coleridge was going to concentrare on the supernatural \t in the rhyme of the ancient mariner there will be supernatural characters (like ghosts) \\
Suspension of disbelief \t the reader was required to apply it \t col. wanted the reader not to warry about the realism of what they were reading, just enjoy they the reading \\
In order to achieve that \t in colerdige's works the frame is realistic, but with some supernatural aspects \\
Wordsworth had to concentrare on the everyday life of paesents in contact with nature \\

\ss[We are seven]
Typical aspects of the lyrical ballads:
\bi
    \item simple language, used in everyday life, but purifed from grammar errors 
    \item children represent the purity of the life far away from societal conventions and of emotions in the countryside
    \item the little girl is described as the typical paesant
    \item it is set the country side (the cottage is a typical house) and natural elements, such as the sea, the tree and the sun-set
\ei
She is a so innocent girl that she hasn't comprehended the meaning of death \t in fact, she plays by their grave \\
The narrator has a more realistic approach \\
She talks about her everyday life \\
Emotions felt by the girl and the lyrical I / narrator are relevant 

\v

\sss[Prof notes]
It is made of quatrains \t a four-line stanza \t but the last one is of 5 verses \\
Quatrains is the typical structure of a ballad \t which is a narrative poem \\
A story is being told, but there is a focus on the emotions \t hybrid nature of the "Lyricall ballads" \\
The veres/lines are written in either \t iamb \t the first one is with accent, the second one no:
\bi
    \item iambic trimeters: what should | it know | of death? \t 3 poetic feet (called iambs) \t each iamb has 6 syllabals
    \item iambic tetrameters: her hair was thick with many a curl 
\ei
Iambic trimeters  is made of 3 iambs \t first one being stressed, the second one is unstressed \\
It is written in alternate/intelocking rhyme 

\sss[Stanza 1]
And feels its life in every limb \t she loves life even if she is in tragic conditions (for ex. her fahter is not mentioned) \\
What should it know of death? \t rethorical question \t a child should know nothing of death \\
Line 3 \t allitterarion of L and a consonance in F \\
Allitteration = consonant is repeated at the beginning of each word, in consonance not necessary 

\sss[Stanza 2]
Line 3 \t consonance in H

\sss[Stanza 3]
Physicall description, that lets us understand that she is poor \\
Allitteration of W at line 2 \\
However, she is still beatiful and almost angelic

\sss[Stanza 4]
The dialog begins \t typical of ballads \t in middle ages they were learned in memory \t easier to remeber with dialogs 

\sss[Stanza 5]
And two is an anafora \\
2 of her siblings are sailors
\end{document}
