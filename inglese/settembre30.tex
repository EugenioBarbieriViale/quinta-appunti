\documentclass[12pt]{article}

\usepackage[a4paper, total={6in, 8in}]{geometry}
\usepackage{textcomp}

\begin{document}
\setlength{\parindent}{0pt}

\def \t {\textrightarrow}
\def \v {\vspace{1em}}
\def \bi {\begin{itemize}}
\def \ei {\end{itemize}}
\def \s[#1] {\section*{#1}}
\def \ss[#1] {\subsection*{#1}}
\def \sss[#1] {\subsubsection*{#1}}

\s[Daffodils]
Wordsworth's most famous poem, published in 1807, but not part of the "Lyricall ballads" \\
It about a beatiful natural landscape \t daffodils are near a lake and describes it poetically \\
The verbs of the first 3 stanzas are in the past, while the last one is in the present \t he remembers the experience in tranquillity \\
Poetry is a natural overflow of feelings, that happens after the experience is finished. The author remembers his ecounter with the flowers as he sits on a couch 

\v

He had a sister (Dorothy) who was a best friend and a secretary \t she also wrote journal about his relationship with wordsworth \\
In the journal it is written that the two went on a walk and saw the field of daffoldills \t in 1803, 4 years after \\
Time passes, and he writes about nature afterwards, in a state of tranquillity \t he has to feel an emotion similar to the one he felt = 2 conditions two write about the experience 

\ss[Stanza 1]
I \t subjectivity of romantic poetry \\
Line 1 \t simile \\
Line 3/4 \t there is a personification \t daffoldils are described as people (a crowd, a host) \\
Line 3 \t assonance in O, line 4 still in O \\
Line 5 \t allitteration in B \\
Line 6 \t again personification, because the daffoldills dance \\
The verb tenses are past tenses \\
The description is of a pictoresque natura (not sublime in this case) 

\ss[Stanza 2]
Simile \t "continuos as" \\
Line 2 \t consonance in K \\
Line 5 \t hyperbole (tenthousand flowers, exaggeration) and allitteration in T \\
Line 6 \t personification of dacing flowers and consonance \\

\ss[Stanza 3]
Personification of the waves of the lake \t they dance \\
Diacopy of "gaze" \t to look fixatilly \\
At the same time he was unaware of the powerful feelings he was experiencing \\

\ss[Stanza 4]
Inward eye = imagination \\
Solitude = situation of tranquillity \\
Last line \t consonance in D and in A

\v

The sensation produces the emotions, which are than stored in the poet's memory, a similar emotion reproduces these emotions and can than write \\
He wrote "Lyricall ballads", "Daffoldills", "The prelude" (long meditative poem) \t here he reflects on his own life and his relationship with nature \\
It was published before his death \t he worked on it all his life \\
What is nature to ww? \t it is explained in the "Prelude". It is a source of beauty, consolation, literary inspiration and an educator \\
WW claims to have been educated by nature throughout his life \\
In the "Prelude" he tells about an event which occured when he was 8 years \t he had just stolen a little boat (a mischievious action) \\
He was sainling, but then he saw that a mountain was following him \t for this reason he decide to bring the boat back \t example of sublime nature \\
He felt like he was been educated by the mountain

\v

WW wrote 7 poems of the "Lyrical ballads" \t but 5 of them are called the "Lucy poems" \t "We are seven" is not part of the "Lucy poems" \\
The Lucy poems are about lucy, who has died young after living a pure life at contact with nature \t she was pure and innocent \\
Her presence can still be perceived in the countryside area, where she lived \\
Jamaica Kincaid wrote a novel "Lucy", published in 1990 \t the author is from the west indies (carabbeans) \t but she writes in english, because she was born in an english colony \\
She mentions Wordsworth \t but the settlement is contemporary \\
Lucy here is an 19y ou pair girl, who decides to migrate to USA \\
It is a post-colonial novel \t 2 requirments:
\bi
    \item it has to be written by a writer coming from a previous colony of an european country (there are also of france, germany, ...) 
    \item it explores the relationshipt between colonizers and colonized people
\ei

\end{document}
