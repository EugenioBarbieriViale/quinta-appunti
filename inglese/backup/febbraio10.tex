\documentclass[12pt]{article}

\usepackage[a4paper, total={6in, 8in}]{geometry}
\usepackage{textcomp}

\begin{document}
\setlength{\parindent}{0pt}

\def \t {\textrightarrow}
\def \v {\vspace{1em}}
\def \bi {\begin{itemize}}
\def \ei {\end{itemize}}
\def \s[#1] {\section*{#1}}
\def \ss[#1] {\subsection*{#1}}
\def \sss[#1] {\subsubsection*{#1}}

\s[Dickens - Oliver Twist]
Oliver is often misunderstood by adults, but he keeps his nature throughout the story \\
1: he was born in a workhouse and two professionals were present: a surgeon and a nurse \t but they didn't have a professional attitude \t the nurse was drunk and the doctor wasn't compassionate and thught she was a prostitute \\
2: he was sent to a branch of the workhouse, that worked as a sort of orphanage \t here he was neglected and underfed, and Mrs. Mann kept all the money ment for the kids to herself and took care of the kids only the day before an inspection \t during these inspections she also pretended to be nice with them \\
3: Mr. Mumble is a memorable character \t he was the beadle of the workhouse appointed by the board \t he was arrogant, cruel, not smart, insensetive
4: he left that branch to join the workhouse itself \t he had to learn a trade (as in practical profession) \t he had to untwist ropes for the navy 
5: they ate gruel (sort of porrige, made of unknown low quality ingredients) \t and the portions were tiny \t one day one of his friends threatened to eat one of them if he wasn't given enough gruel \t so one of them had to ask for more gruel, and the one who had to do it was Oliver \t the reaction of the workers was exaggerated \t so he got locked up in a room for days and Mr. Bumble hit him and was humiliated in front of other children \t he was also donated with 5 pounds
6: first a chimney sweeper tried to obtain him \t he was cruel, violent (he describes the way he treats children and the donkey he rides) \t but he is desparate for money \t the old magistrate decides not to give Oliver to the sweeper \\
7: he was then donated to Mr. Sowerberry (a good person all in all) and he is an undertaker \t he has a wife, who is not nice, and there is also a maid \t and also has an apprentice, named Noah (slightly older then Oliver and bullies him) \\
8: he was given the remains of dog food and he had to sleep in the basement surrounded by coffins \t and he was afraid of the dark, and this first night was tragic for him \\
9: Noah woke him up by kicking the door of the basement \t Noah was a charity boy: he did not live in a workhouse, but in a regular house, and was not an orphan, but was still considered an underdog \t the parish supported his family (his father was a veteran, who didn't work, and his mother had a humile job) \t so he could relate to a person inferior to him (Oliver) \\
10: then Mr. Sowerberry assings a special job to Oliver, and becomes a mute (he is a beautiful boy with innocent features)\t a mute is present to funeral and remains sad an serious \t Noah gets jelous for this \t in the first funeral a poor woman died of starvation \t her husband was desparate, her children were not particularly aware and her mother was actually happy, because she could participate in the funeral's banquet and was given a cloak \\
11: Oliver was patient, until Noah called his dead mother a prostitute \t so Oliver attacked him, and Mrs. Sowerberry and the maid fought him \t so Mr. Sowerberry had to punish Oliver, even though he knew it wasn't right \\
13: in the end Oliver decided to flee and went to London (he was born in a fictional city, called Mudfog) \t but before leaving, he wanted to say goodbey to his mate of the workhouse, who was to the point of dying, but blessed him \t melodramatic because everything bad happens to him \t his journey was terrible, and only an old couple gave him food \\
14: In a village north of london he met a child, called the Artful Dodger, who took him to London to a man who could host him \\
15: the first thing he saw in london was a slum \\
16: Fagin was an old man and a Jew, who welcomed Oliver \t he was hosting children and trained them as pickpockets \t he developed a strategy using handkerchief \t and Oliver doesn't realize \\
17: he saw Fagin putting jewelery in a box and attacks Oliver with a knife \t he is an extremely greedy person \t racist depiction \\
Mr. Brownlow gets robbed by the children, and Oliver gets accused by the angry mob and he gets arrested, and is taken to court \t in the trial scene, we understand Dickens' critiques to Victorian justice \t Mr. Fand, in charge of the trial, is rude and insensitive and unprofessional \t but in the end he is not sentenced to prison, thanks to the librarian who saw the scene (was a witness) and testefies, Oliver was didn't commit the crime

\v

The plot of Oliver Twist gets complicated \\
Oliver gets adopted by Mr. Bronwlow \t who lives in a beatufil house in the countryside \t here he is given food and is educated \\
However, Fagin and Bob Sikes (an adult too) decide to kidnapp Oliver \t they are afraid that he will report them to the police \t they kidnap him thanks to the help of a little girl Nancy \t and he returns to London in Fagins house \\
Here he is obligated by them to help them to burgle a house (buglary, buglar) of a rich family, made of a young woman (Miss Rose) and her guardian \\
During the bulgary, Oliver is wounded but again taken back to Fagin's house
\end{document}
