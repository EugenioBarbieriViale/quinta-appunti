\documentclass[12pt]{article}

\usepackage[a4paper, total={6in, 8in}]{geometry}
\usepackage{textcomp}

\begin{document}
\setlength{\parindent}{0pt}

\def \t {\textrightarrow}
\def \v {\vspace{1em}}
\def \bi {\begin{itemize}}
\def \ei {\end{itemize}}
\def \s[#1] {\section*{#1}}
\def \ss[#1] {\subsection*{#1}}
\def \sss[#1] {\subsubsection*{#1}}

\s[Lyrical Ballads]
They wanted to achieve a cultural revolution \t non violent, because they were disappointed in the french revolution \\
It is revolutionary because the characters are common people \\
Moreover the language was simple \t so that more people could understand it

\v

It is also a peculiar title \t it is oximoric, contradictory: poetry tends to be of 2 kinds:
\bi
	\item lyrical \t feelings and emotions of the speaker (the lyrical eye) \t examples are sonnets (14 lines), oaths (no fixed number of lines) and elegies (no fixed numer of lines, main topic is death)
	\item narrative \t relates a story in verses \t tend to be longer
\ei
It is oximoric because they are intimate accounts of life in the english country side \\
They are about stories of common people in the country side but expressed in a lyrical way, with a focus on the feelings \\
It is oximoric and also hybrid \\

\ss[Question]
Poems are about ordinary lives about ordinary people coming from the country side \t they are lucky, because they live near nature \\
The language used is similar to the one people actually used \t it is more emphatic and purified from his difects \\
He uses the couloring effect of imagination to write in an unsual way \\
Poets have sharper sensibility and imagination + better understanding of human nature \\


\end{document}
