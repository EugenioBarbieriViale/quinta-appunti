\documentclass[12pt]{article}

\usepackage[a4paper, total={6in, 8in}]{geometry}
\usepackage{textcomp}

\begin{document}
\setlength{\parindent}{0pt}

\def \t {\textrightarrow}
\def \v {\vspace{1em}}
\def \bi {\begin{itemize}}
\def \ei {\end{itemize}}
\def \s[#1] {\section*{#1}}
\def \ss[#1] {\subsection*{#1}}
\def \sss[#1] {\subsubsection*{#1}}

\s[Non-romantic literature]
Romanticism in Britain is know for being mainly poetic \t but there were also prose works, like:
\bi
    \item the gothic novel
    \item "Frankenstein" by Mery Shelley
    \item the historical novel, invented by Walter Scott \t who also influenced Manzoni in his writing "The bethroded"
\ei

\ss[The novel of manners]
Jane Austen is another important author of this period, but is not romantic \t she wrote novels of manners, like "Pride and prejudice", "Emma", "Sense and sensibility", and "Northanger abbey" (novel and parody of the gothic novel) \\
The gothic novels began to be written in the 1760s \t but then became less trendy (Frankenstein is gothic only partly) \t readers got tired of this genre, so she mocks it \\
The characters are not round, but flat \t they are not complex \t these attributes of characters were invented by Forster in "Aspects of the novel" \\
Flat characters cannot undergo an evolution and their psychology doesn't change \t while round characters have more complex psychology 

\v

The protagonists of Jane Austen's novels are young women of the middleclass living in the country side \t they are usually smart and moral, and their goal is to find the right husband \\
These novels revolve on the themes of marriage, love, money (these girls have to have a proprer dowry) \t but these women are unconventional and independent: they want to marry the right person \\
Third person omniscient narrator is frequent in novels of manners \\
Irony is the most typical ingredient \t and these novels are still successful nowadays \\
Richardson wrote epistolar novels (like Frankenstein partially is), but also wrote some novels of manners

\ss[The historical novel]
Romantics are interested in the past \t in the middle ages and the greek world, as a sort of escapism \\
Walter Scott's more significant novels are: 
\bi
    \item "Waverley" \t regarding the history of scotland \t Jacobite ragings around 1715/1745 \t in 1707 the Act of Union and the UK was established \t and scotland lost its independence and became part of the UK \t Scotland and England had been enemies since ever \t Jacop Ragings wanted to make Scotland independent again with Stuarts
    \item "Ivanhoe" \t regarding the history of england \t it is set in england following the year 1066, year of the battle of Hastings, when the anglosaxons were defeated by the normans of France \t they invaded and started ruling \t the novel is about the tensions (complex social political relations) between these two ethnic groups
\ei
Scott was interested in historical crisis, and the effect of historical events on commmon people \\
It is interesting also linguistically: the novel are written in standard english, but also uses dialects to achieve realism \\
Normans and anglosaxons also spoke different languages \t and french became the official language of the country

\s[Victorian age]
It is a long phase of british history going on from 1837 (Victoria's accession to the throne) - 1901 (Victoria's death) \\
Her reing was the second longest in the history of UK \t the first one was Elisabeth the 2nd (1952-2022) \\
During this period the country experienced imperial and economic growth and uk became the most powerful country in the world \t but common people lived in poverty = social problems \\
There are three parts:
\bi
\item early victorian period = until 1850
\item mid victorian period = 1850-70
\item late victorian period = 1870-1900
\ei

\ss[Early period]
The toughest years were the ones of the early period \t the 1840s are know as the "hungry forties" = decage of unemployment (industrial crisis) and bad crops in the countryside \\
The situation was particularly bad in ireland \t in these years there were the "irish famines" \t mainly in the non-UK ireland, which was a colony \t northern was part of uk because the majority was anglican \\
The rest of ireland was a colony \t they were catholics \\
There was probably an illness striking potatoes \t the population was reduced to one third \t many died, many migrated to North America and Australia \\
There was also problems with factories in the uk \t the industrialization was unregulated, owners of fabrics could do everything they wanted and factoryworkers were exploited \\
An important political movement that emerged was the "Chartist Movement" \t it was a political movement founded in 1839, with the publication of the "People's Charter" \t the chartists demanded universal suffrage, fewer working hours, higher wages \\
They were not successfull, but their ideas started to be put into practice in the following decades

\ss[Mid period]
The situation improvedd \t more favorable economic situation worldwide and the state start interveing by passing important reforms, which improved the living conditions of the lower classess and created less sociali comflicts (chartists were sometimes violent) \\
Some reforms were:
\bi
    \item Poor laws \t controversial: institutions (mentioned in Oliver Twists) like the workhouses were created \t they were meant to host poor people, where people had to work \t there were for men and for women and children \t so families were separated \t and the living and working conditions were terrible \t but there was a social phylosophy, coming from the portestant work ethic (poor people were lazy and didn't have god in favor): people had to live a miserable life in a workhouse, to find the motivation to earn more
    \item Reform acts \t lowered the income threshold in order to be allowed to vote 
    \item Education acts \t started obliging children under a certain age to go to school rather than working \t elementary education was compulsory until the age of 9, then the age was increased \t but victorian schools were terrible as well: corporal punishment was common
\ei
\end{document}
