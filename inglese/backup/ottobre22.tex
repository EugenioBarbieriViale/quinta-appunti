\documentclass[12pt]{article}

\usepackage[a4paper, total={6in, 8in}]{geometry}
\usepackage{textcomp}

\begin{document}
\setlength{\parindent}{0pt}

\def \t {\textrightarrow}
\def \v {\vspace{1em}}
\def \bi {\begin{itemize}}
\def \ei {\end{itemize}}
\def \s[#1] {\section*{#1}}
\def \ss[#1] {\subsection*{#1}}
\def \sss[#1] {\subsubsection*{#1}}

\s[Byron]
Satirial neoclassical poems 

\ss[Beppo - an extract]
Is set in venice \t is about an aristocrat, who is believed to have died \t he returns to Venice and has to reconquer the love of his previous lover \\
Here Beppo makes a digression on language \t Byron was influenced by Sterne in this \t he wrote an experimental novel, called "Tristan Shandy" \t so experimental, that he inspired experimental novelists of 20th century \\
"Tristan Shandy" is full of digressions 

\v

The point made is that neo-latin languages are superior in comparsion with germanic languages \t because they sound better \\
Here he use a specific stanza \t called "ottava rima": \\
Foot = a syllable
\bi
    \item 8 lines \t first 6 iambic pentameters (10 feet) + last 2 hexameters (6 feet), also called "alexandrines"
    \item rhyme AB AB AB CC
\ei

\v

Bastard because latin developed from greeks \\
"Melts like kisses" \t figure of speech: simile \\
"from a female moth" \t alitteration of F \\
"sounds / should / satin (raso)" \t alitteration of S \t creates musicality \\
Line 4: personification of syllables \t they are breathing \t + alitteration in S \\
Pat in = smooth \\
Uncouth = (rozzo) \\
"Like our" \t simile \\
Hiss = sound of the snake \\
Sputter = spit

\ss[Don quan - another extract]
Marriage is love being spoiled \t marriage and love don't mix, marriage ruins love \\
Love is pure, superior \t gets spoiled by marriage, just like wine turns into vinegar \\
It is a digression on the theme of love \\
In Byron's opinion, it is a hypocritical institution \\
Line 2 \t allitteration in F \\
Line 3 \t allitteration in C \\
"Like vinegar from wine" \t simile \\
Love is superior to marriage \t they only rarely combine \\
Don Juan is written in ottava-rima
\end{document}
