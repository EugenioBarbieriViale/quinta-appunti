\documentclass[12pt]{article}

\usepackage[a4paper, total={6in, 8in}]{geometry}
\usepackage{textcomp}

\begin{document}
\setlength{\parindent}{0pt}

\def \t {\textrightarrow}
\def \v {\vspace{1em}}
\def \bi {\begin{itemize}}
\def \ei {\end{itemize}}
\def \s[#1] {\section*{#1}}
\def \ss[#1] {\subsection*{#1}}
\def \sss[#1] {\subsubsection*{#1}}

George III was an Hannover \t this dinasty began when the last Stuart Monarch (Anne I) died without children (1714) \\
The Hannover came from Germany \t the first one was George I (he couldn't speak english, only german) \\
He was helped by the prime minister \t then this position remained \\
In 1780 he suffered a mental confusion know as the royal madness \\
In the last years of his kingdom he was still the king, but his son ruled \\
After the french revolution, England had to fight multiple times the Napoleonic Wars \\
Nelson was in charge of the navy \t Wellington was in charge of the army \\
The press coined the name "Peterloo" \t it critized the government because it was fighting Napoleon in name of justice and democracy, but then opend fire on civilians

\v

There was a strong migration towards the cities \t the rural areas had less inhabitants and this zones were called the Rotten Boroughs \\
They still had many seats in the parliament, which was unfair because they were less \\
This was fixed with the Great Reform Act \\
Starting from 1836, Victoria ruled for 64 years 

\v

Augustian poets = neoclassical poets \\
They wrote in the first half of the 18th century in Britain \\
They were intrested in the greek and roman poetry \\
Their poetry was complex from a conceptual and linguistic point of view \t 
\bi
    \item conceptual: it was rich of allusions to classical poetry and lots of figureso of speech
    \item linguistic: they used a lofty style \t there was a prevalence of latinized syntax and high sounding vocabulary
\ei
Romantic poetry differs \t usually the language is more accessible and the poetry was more subjective \t it was more about feelings and emotions \\
Subjectivity is the main feature of romantic poetry \\
Augustian poetry: example "The rape of the lock" by A. Pople \\
Romantic poets were more interested in the middle age \t augustian poets absolutely not \\
In Augustian poetry nature is presented as something controllable by humans thanks to their reason \\
In romanticism nature is presented as an incontrollable force 

\v

Burke was an irish philosopher and thought that there existed 2 types of nature
\bi
    \item picturesque or beatiful \t cosy
    \item sublime nature \t destructive nature, that causes fear and ammiration
\ei


\end{document}
