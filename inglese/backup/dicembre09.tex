\documentclass[12pt]{article}

\usepackage[a4paper, total={6in, 8in}]{geometry}
\usepackage{textcomp}

\begin{document}
\setlength{\parindent}{0pt}

\def \t {\textrightarrow}
\def \v {\vspace{1em}}
\def \bi {\begin{itemize}}
\def \ei {\end{itemize}}
\def \s[#1] {\section*{#1}}
\def \ss[#1] {\subsection*{#1}}
\def \sss[#1] {\subsubsection*{#1}}

\s[Ozymandias - P. B. Shelley]
Remains of the ruins of a colossial statue \t it portrays Ozymandias \\
He owns a large empire and says that someone should be afraid of him \t but this clashes with the sorroundings, which is a desert only \t his empire is reduced to a desert \\
It is a sonnet (lyrical poem) with 14 lines with iambic pentameters \\
Ozymandias is the greek name of the pharaoh Ramses II \t this other name sounds more exotic \\
Sonnet consists of a dialog \t there is a chinese-box (and a dialogical structure): the speaker is talking to a traveller, who is reading the inscription on Ozymandias's colossal statue \\
The inscription is composed of Ozymandias's words \\
It is really subjective (because it is romantic) and in fact starts with "I" \t it reflects Shelley's political ideas (he believed in democracy \t radical idealism) \\
This poem is a fierce critique of imperialism \t O. created an empire (= etnicity with rules over other etnicities) \t but what remains of this empire? nothing

\ss[Analysis]
Antique land = Egypt \\
The statue is in ruins \t and parts of it are missing \t there are only two legs without a trunk \\
Three dots = cesura \\
Visage = face (french origin) \t which is shattered \t still you cant recognize an arrogant expression, typical of a tyran \\
Malicious simle full of disgust \\
The sculpture was good at mocking the expression of Ramses \t mocking in the sense of imitating or in the sense of making fun \t ambiguos and ambivalent passage \\
It is like Ozymandias is talking \t "ye Mighty" \t very ambigous: it might be referring to god or other emeperors, who chose despair \t proabibly because he is going to conquer their empires \\
"Boundless and bare" \t allitteration in B \\
Of Ozymandias's empire remains only his statue \t everything that humans create is ephimeral \\
Radical idealism \t empires are doomed to finish \\
A lot of signs of punctuation \t rythm is fragmenteted, like the statue is \\
It is also a critique of imperialism

\s[Ode to the west wind - P. B. Shelley]
\ss[Introduction]
It is an ode = long lyrical poem \t without fixed number of lines \\
There are 5 sections, and eachone is a sonnet (made of 14 lines):
\bi
    \item 1-2-3: the west wind is presented as a sublime natural element
        \bi
            \item 1: described the effect of the wind on the earth \t it is both a destroyer (it causes the shadding of trees in autumn, so leaves are carried away by the wind \t comepared to people following an enchanter) and a preserver (it carries seeds in the soil covered as if they were corpses)
            \item 2: described the effect of the wind on the sky \t it causes the moving of clouds and storms \t clouds are compared to the locks of an approaching storm 
            \item 3: described the effect of the wind on the sea \t the Atlantic ocean and the Mediterrenean sea are mentioned \t the second one is picturesque and adapted to humans (along his coast different civilisations had developed and have left landmarks) \t the first one is sublime: rough, full of abysses \t at the bottom of the sea there is vegetation and are described as shaking of fear because they perceive the coming of the west wind (personification)
        \ei
    \item 4-5: presented as a political symbol
\ei
Section for starts at line 43

\ss[Analysis of section 4]
Lyrial I \t romantic \\
Dead leaf is referring to the first section \t and cloud to section 2 \t and wave to the third \\
There is a cesura with an exclamation mark \t and an apostrophe (referring to the wind) \\
In these five lines \t the author is willing of the fusion with the west wind \\
Line 43 and 44 \t there is the anaphora of "if I were" \\
After a cesura usually there is a change of topic \t in fact there is a nostalic recollection of the speaker youth \t while he was young, he was active and quick as the wind \\
Line 54: contrast between his active and happy boyhood and the present as and adult man \t "thorns of life" is a metaphor for the setbacks of life \\
On too like thee \t the speaker like a young man \t now a "heavy weight of hours" \\
Politically speaking Shelley is an idealist and has optimistique views \t but in his life he was pessimistic \\

\ss[Analysis of section 5]
He is asking the wind to make him a lyre (musical instrument, symbol of poetry) \\
The poet is longing for a fusion with the wind, which is defined as spirit \\
Now he wants the wind to carry around ideas of freedom \t to quicken their diffusion \\
His political ideas are like a prophecy \\
Last line \t there is a rethorical question (it is obious that spring will follow winter) \\
Thanks to the wind he will be able to change the world
\end{document}
