\documentclass[12pt]{article}

\usepackage[a4paper, total={6in, 8in}]{geometry}
\usepackage{textcomp}

\begin{document}
\setlength{\parindent}{0pt}

\def \t {\textrightarrow}
\def \v {\vspace{1em}}
\def \bi {\begin{itemize}}
\def \ei {\end{itemize}}
\def \s[#1] {\section*{#1}}
\def \ss[#1] {\subsection*{#1}}
\def \sss[#1] {\subsubsection*{#1}}

\s[Recuperare 1 ottobre]
\s[Lucy]
She has relocated in New York \t she has a conversation with her boss, who was talking about spring and daffoldills \t this makes her rember something about her childhood \\
She was a pupil \\
The was complimented on her pronunciation \t one can understand that she was no native speaker \\
It is common for post colonial works to have marginalized characters \t for example, among the colonized, women, memebers of ethical minorities or children, whose view on reality is more sincere and pure (in a sense) \\
In her dream she gets "punished" for her act of rebellion \t the flowers submerge her \\
She tried to forget this as a defensive mechanism \\
Anger \t typical emotion of post-colonized people 

\v

Post colonial literature can deal with the relationship between colonizer and colonized and historical events such as:
\bi
    \item colonialism \t (formally started in 1492, with the "discovery of america")
    \item imperialism \t (1) the political phylosophy behind colonialism: an empire, by definition, is made by different ethnicities and a government that rules \t (2) other consider it an evolution of colonialism and a much worse version (1900)
    \item neocolnialism \t newer form of col. starting at the beginning of the 20th century \t typically brought by the usa, and after WW2 also UK, France, Spain (previous colonial empires on the process of collapsing \t they influence elections of colonized countries and control economically by buying raw materials at low prices = indirect but deep political and economical influence)
\ei
USA wanted to control the American continent \t according to the "Monroe doctrine", usa had the right to do it \\
France, ... \t regarded mainly africa, but also asia

\v

Usa was bringing neocolonialism in the carabbieans \t for this reason Lucy is settled in USA

\s[Coleridge]
He lived less than Wordsworth \t he had been plyed all his life with chronic diseases \\
He treated himself with opium (it was considered as a drug) \\
He became an addict \t and this had consequences on his literature too \\
Coleridge in his youth had opium-induced visions, about which he wrote in his poems \t but opium ruined his imagination as time passed

\ss[Kubla Khan]
It was published in 1816 \t more mature phase of Coleridge's literal production than the lyrical ballads \\
An example is "Kubla Khan", which is a poetic fragment wrote after an opium-induce vision \\
Fragment because it remaind unfinished \\
It is about a vision about a beautiful asian palace, owned by the mongul emperor Kubla Khan \\
The palace was huge and elegant, sorrounded by a lush garden with bushes and trees of any kind \\
There was also a spring, from which a river originated \t there were also an abyss \t the spring has been interpreted as a symbol of poetic inspiration \\
The rives is the poetry the poet can produce \\
It is a metapoetic poem = it also deals with poetry \\
But this palace is also an iverly tower \t because it was sorrounded by huge walls, that isolated the palace from the rest of the world \t you could here ecos of voices coming from the outside \\
He thought that the poet should lock himself in an iverly tower \t the poet had to be less concerned by the real world \\
In a passage of "Biographia literaria" Coleridge talks about the writing of Kubla Khan \t before he had taken Laudanum (a mixture of wine and opium) \\
He had the vision and then started writing \t but then he was interrupted by a friend who visitied him \t when he started to write the poem again, the inspiration was gone and he couldn't finish it 

\v

As a young person he was a radical \t he supported the french revolution \\
With Robert Southey he realaborated a political project \t they wanted to create "Pantisocracy", a new comunity founded in america by 12 couples (and then start the population) \\
In Pantisocracy women were supposed to have the same right as men, and private proprety had to be abolished too \\
This never took place \\
Southey was considered a major romantic \t now it is considered a minor \\
After his radical youth, he became a conservative later
\bi
    \item "Lyrical balldas"
    \item "Christable and other poems" (which includes Kubla Khan) \t christable is a gothic poem, whose protagonist is the witch Christable
    \item "Biographia literaria" \t phylosophical and autobiographical long essay \t here he elaborates his theory on imagination
\ei

\s[Rhyme of the ancient mariner]
It is a long narrative poem (not lyrical) \t specifically, it is a ballad, but also a cautionary tale in verse \\
The first ballads were written in the middle ages \t they were very popular, because they were meant for common people \t they couldn't read, so minstrels or storytellers recited and sang ballads in streets and squares
\bi
    \item they were written in quatrains
    \item the language was simple
    \item full of repetitions (to be learned by heart and be comprehended more easly)
    \item they had a dialogical structure (dialogs were frequent)
    \item about tragic events \t they reflected the life of the people of middle ages
\ei
During Reinassance they lost their central role \t theatre was the most famous entartaiment method \\
Ballads were transmitted orally \t only in 18th they got written down \t and also poets wrote them (called literary ballads), but were more complex and refined as the ones of the middle ages 

\v

It is cautionary because it teaches a moral story \t it teaches the reader to respect nature (or you will get terribly punished) \\
The mariner was disrispectful towards nature \t now he has to spend his life by warning others no to commit their mistakes \\
October 7: \\
Every aspect of nature is a manifestation of God \t so it have to be respected \\
There are 7 sections and there are 2 narrators:
\bi
    \item main external 3rd person omniscent one
    \item the marin (first person)
\ei
The poem presents both realistic and supernatural aspects \t the supernatural aspect requires a suspension of disbelief by the reader \\

\ss[Plot]
Plot \t page 301 ex. 1 \\
After the violent storm an albatross appeared, and the ship started moving again \t it was like a messanger sent by God to help the crew \\
The mariner was not grateful \t he kills the albatross \\
The mariner for this reason get punished \t the ship stops moving (in the equator area) and they run out of water \\
There are only water snakes that swim around their ship and no other signs of life \\
Then they are approached by a phantom ship \t there are two ghosts (Death and Life-in-Death), and they cast dices with the lives of the mariners \\
Death wins the life of the mariners, and Life-in-Death wins the one of the ancient mariner \t he will live but be dead and will have to tell his story to others \\
After the death of the mariners, the ship is still not moving \t it moves only when he blesses the snakes and understands the importance of nature \\
Open ending \t but it implies that the mariner will spend his life as a wanderer and his sense of guilt will end with his death 

\ss[Killing of the albatross]
There are some prose parts, that anticipate the content of the stanzas \\
Ship is personificated (she) \\
'th = ancient way of saying 's \\
Stanza 1  \t the external narrator speaks \\
Stanza 2  \t next of kin = close relative of the bridegroom (the male spouse) \\
Stanza 3  \t first and last line: consonance in H, and dialogial structure \\
Stanza 4  \t the mariner mesmorizes him \t stood still = alliteration in S \t like a 3 years child = simile \\
Stanza 5  \t line 2 alliteration in c \t line 3: inversion \\
Stanza 6  \t the mariner becomes the second narrator \t kirk = archaic word for chirch \t diacope of below \\
Stanza 7  \t line 2: inversion \t sun referret to with "he": personification \t line 3: consonance \\
Stanza 8  \t general narrator is talking \\
Stanza 9  \t line 2: simile and inversion \t merry minstresly: alliteration \\
Stanza 10 \t he hypnotizes with his eye \\
Stanza 11 \t the narration continues, by the equator there was a storm (he and tyrannous, personification) \t storm = sublime nature \t then storm with animalisation (has wings) \\
Stanza 12 \t sloping = inclined \t line 2: consonance in W \t and long double simily: the storm is someone chasing their foe (enemy, the ship here) and is so close that can walk on its shadow \\
Stanza 13 \t at the south pole they find a sublime landscape \t anafora of "and" \t ice is green as emerald (simile) \t fearful: sublime nature creates fear \\
Stanza 14 \t no creatures, only ice \\
Stanza 15 \t they feel overwelmed by ice (sense of oppresion) \t diacopy of ice \t animalisation of ice (growled, roared, howled) \t last line: simile, like a dream \\
Stanza 17 \t diacopy of round \\
Stanza 18 \t the albatross becomes a constant presence \\
Stanza 19 \t moon in the nigth: exampole of picturesque nature
Stanza 20 \t anticipation of destiny of the mariner \t he will have to live like a dead man \\

\ss[A sadder and wiser man]
Moral of the whole poem \\
The mariner is talking to the guest \t there is and Apostrophe \\
Even beasts have a soult \t panteist vision of the world, that Coleridge will abaond in his later life \\
Later he becaome more conservative (politically) and also more ortodox religiously (abandoned the panteistic view) \\
The very conclusion is relevant to the coutionary tale \\
The guest grows morally \t he becomes wiser
\end{document}
