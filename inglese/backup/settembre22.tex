\documentclass[12pt]{article}

\usepackage[a4paper, total={6in, 8in}]{geometry}
\usepackage{textcomp}

\begin{document}
\setlength{\parindent}{0pt}

\def \t {\textrightarrow}
\def \v {\vspace{1em}}
\def \bi {\begin{itemize}}
\def \ei {\end{itemize}}
\def \s[#1] {\section*{#1}}
\def \ss[#1] {\subsection*{#1}}
\def \sss[#1] {\subsubsection*{#1}}

\s[Romanticism]

In Neoclassicism (Augustian age) nature was perceived as something controllable \t humans are endowed with reason \\
In Romanticism it is not possible \\
For Augustans the most powerful human faculty is reason \\
For Romantics it is imagination \t it it what makes us humans

\v

Coleridge (coauthor of the Lyrical Ballads (first edition 1789)) \t he travelled to Germany and studied philosophy \\
He believed that there were 2 types of imagination:
\bi
    \item primary: allows humans to understand reality
    \item secondary: more powerful faculty, allowing humans to recreate a new reality
\ei

\v

Before Romaticism a good poet was a good copietor?? \t in romanticism originality played an important role \\
Childhood was considered as the most important stage during life \t children have the sharpest imagination \\
Romantics were also agaisnt child labor because it was dangerous and because in this way children skipped their childhood 

\v

Romanticism emphazied individuality \t there was interest in the outcast and rebel \\
Romanticism followed in Rousseau philoshopy \t the myth of the noble savage??? \t man is naturally good, than civilazition corrupts 

\v

English romantic poems were setteld either in the past (often in the middleages) and in exotic places (far away from Britain) \\
Escapism = desire to escape the time one lives in, that is considered unjust \\
Reality was also considered an expression of God = pantheism \t other were atheists 

\v

Augustian poetry was caracterized by poetic diction (complex language and figure of speechs) \\
Romantic poetry is more readable \t syntax was not latinized and the words used were not so high sounding \\
Romanticism comes from romance, the languages such as italian french etc... \\
In the beginning it had a negative meaning (extravagant, unreal) \t but in the 18th century it became netrual and referred to pictoresque nature 

\v

Relationship between the feelings of the speaker and the nature sourrinding \t his feelings reflect into reality \\
Wordsworth said that nature was a soruce of consolation for mankind 

\s[Wordsworth]
First generation of romantic poets \t the members of the second died young in tragic circumstances \\
He was born in 1770 in a lake district, area in the north of England \t near the border with scotland \t caracterized by lakes, woods, and rivers \t but was impacted by the industrial rev. \\
He had a tragic life \t he lost his parents in his childhood \t he was than raised by relatives \t he was close to his sister Dorothy \t togheter they had long walks and he also wrote poetry to her \\
He was a brilliant student \t he attended Cambridge University \\
As a young man he was a radical \t he supported the french revolution \t he even decided to go there and witness \\
There he met a woman, with whom he had a daughter but not married \t he then became disappointed of the revolution, which was deranging into pure violence \\
He than decided to leave france \t and prepared all the paperwork necessary for his lover and daughter to travel to England \\
But the borders closed and he was extremly disappointed \t this changed also his political view \t he became more conservative and critical of revolutions \\
He was a peculiar conservative: he was also an environmentalist, he was against the poor laws (1834, established workhouses)  \\
They were institutions that had to host homeless, but there they had to work hard in exchange with room and board \\
The living conditions were also terrifing \\
Workhouses were expected to be terrible, because they had to convice the poor people they had to improve their social condition \\
This is in line with the Protestant work ethic \t god approves hard work \\
He also believed that people without an education shouldn't vote 

\v

In England he married another woman, with which he had 5 children \\
His daughter never married and lived with his family \\
Wordsworth became a very famous poet, he was also appointed a Poet Laureate (the official poet of the british empire) \\
The Poet Laureate was also expected to write poems for public occasions \\
Wordsworth is know for the Lyrical Ballads (a collection of 23 poems), some written by him, others by Coleridge (coauthorship) \\
There were 3 editions:
\bi
    \item 1798 anonymous
    \item 1800 with preface
    \item 1802 enlarged
\ei
The 2 most notorius ones are Tinter Abbey (Wordsworth) and the Rime of the ting??? 

\ss[Tinter Abbey]
It has to do with a walking tour wordsworth and doroty they had in wales \t here they visited the ruins of a monument (tinter abbey) \\
It was an abbey destroyed in 1534 following "The act of supremacy" \t it was passed by king Henry the VIII (Tutor), which made England an anglican country \t henry broke away from the church of rome \\
Monasteries and abbeys were destroyed \\
The two visited the ruins and wordsworth wrote a poem about that \\
In tinter abbey wordsworth reflects on the passing of time and on the fact that nature has taken control of the ruins \\
Nature is ethernal, meanwhile everyting humans create is ephimeral \\


The lyrical ballads had to make up for political feilures (such as french revolution) and wanted to achive their own revolution, a cultural revolution \\
It is democratical because 
\bi
    \item it is a ballad about common people (not heros, nor aristocrats) \t commonworth wrote mainly about paesants
    \item really simple language
\ei
\end{document}
