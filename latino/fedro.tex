\documentclass[12pt]{article}

\usepackage[a4paper, total={6in, 8in}]{geometry}
\usepackage{textcomp}

\begin{document}
\setlength{\parindent}{0pt}

\def \t {\textrightarrow}
\def \v {\vspace{1em}}
\def \bi {\begin{itemize}}
\def \ei {\end{itemize}}
\def \s[#1] {\section*{#1}}
\def \ss[#1] {\subsection*{#1}}
\def \sss[#1] {\subsubsection*{#1}}

\s[Fedro]
In questo periodo dell'epoca imperiale l'intellettuale non è libero \t termine intellettuale è generico, comprende il filosofo, lo scrittore, il poeta \\
Intellettuale comunica il suo giudizio sulla realta

\v

Fedro sperimenta un genere nuovo \t in questa epoca ci sono diversi generi "eruditi" \t le favole di fedro non hanno carattere erudito, ma morale \\
Fedro è vissuto in un periodo post-imperiale \t intorno al 20 a. C., sotto augusto, poi sara attivo anche sotto Caligola, fino a Claudio \\
Capisce che non puo dire direttamente quello che vuole esprimere \t deve camuffare attraverso la finzione letteraria \\
Finzione letteraria modifica anche i tempi della narrazione, ed allunga o abbrevia la narrazione \\
Attraverso il genere della favola e le voci degli animali contesta l'impero e i potenti \\
Umili e potenti \t richiama a Manzoni

\v

La vera novita pero sono gli animali, che vengono umanizzati \t si comportano come delle persone \\
È l'archetipo per il macrotema degli animali nella letteratura (poi anche in Leopardi sarà importante) \\
Le favole di Fedro hanno una finalità morale \\
Qua il vero è camuffato dalla finzione \t l'utile e l'interessante \t fa passare dei contenuti morale attraverso delle storie quasi per bambini \\
Contengono anche provocazioni \\
Ha anche il modello di Esopo \t la tradizione greca si tramanda \t Fedro di differenzia per la finalita politica \\
Fedro viene perseguito dalla giustizia \t ma non si smosse mai dalle sue idee \t coerenza idiologica \\
Abolisce ogni tipo di aberrazione e asimmetria 

\v

Il lupo e l'agnello \t due figure che nell'aspetto allegorico \t come Machiavelli: leone e volpe \\
Lupo incarna il potente, l'agnello un umile \\
Il lupo accusa l'agnello di aver sporcato l'acqua del fiume \t linguaggio della prepotenza e della calunnia \\
Poi lo sbrana \t emblema di una favola semplice che riflette tematiche più profonde \\
Nei totalitarsmi si puo sperimentare l'annullazione del singolo \t individuo esiste solo in funzione dello stato \t la "Ragion di stato" di Botero 

\v

Favole sono anche caratterizzate da una "brevidas" e da una struttura rigida \\
Animali presenti anche in Saba (mia moglie è una capra), Pascoli, Montale chiama la moglie un insetto (mosca) \\
Anche Cesare Pavese \t "Prima che il gallo canti", e poi anche nei dialoghi con Leuco?? \\
Anche Apuleio \t "Asinus aurus" \t anche la Fattoria degli animali di Orwell \\
Parte introduttiva e fortemente narrativa \t corpus centrale, in cui la vicenda si snoda \t e la parte finale che ha un forte impatto morale \\
Favole hanno struttura rigida e struttura quasi circolare \\
Animali parlano solo con altri animali e attraverso il discorso diretto \t non con uomini \\
Per discorso diretto \t sono anche vicini al teatro \\
Favole sono in versi \t 90 divise in 5 libri, ora alcune sono andate perse \\
Fedro ha preso le favole e le ha piegate alla mentalità romana \\
Le morali di fedro sono la voce degli emarginati \t spesso l'autore commenta i sopprusi \t come Verga \\
In Fedro gli emarginati sono quelli vittima di potere \\
Usa un linguaggio molto semplice, che fa cogliere un adesione alle classi umili

\v

Accenti polemici sono camuffatti attraverso gli animali \\
La pericolosita degli attacci dei vertici del potere \\
Seneca non si rispoarmia dal fare riferimento agli animali \t nella "Consolatio ad Polybium" fa un riferimento ad un pubblico dotto \\
In quest'opera Seneca parla anche dell'esilio \t fa una riflessione sull'esilio \t dice che l'esilio non è altro che un cambiamento di luogo \\
Il sapiente (in de brevitate) come un timoniere sta su un'imbarcazione un po rovinata e impreda a una mareggiata \t ma il timoniere tiene la rotta: le avversita della vita non ci devono distruggere \\
Boccaccio \t il corvo che mangia il cuore
\end{document}
