\documentclass[12pt]{article}

\usepackage[a4paper, total={6in, 8in}]{geometry}
\usepackage{textcomp}

\begin{document}
\setlength{\parindent}{0pt}

\def \t {\textrightarrow}
\def \v {\vspace{1em}}
\def \bi {\begin{itemize}}
\def \ei {\end{itemize}}
\def \s[#1] {\section*{#1}}
\def \ss[#1] {\subsection*{#1}}
\def \sss[#1] {\subsubsection*{#1}}

\s[Quintilliano]
\bi
    \item rapporto con il potere
    \item la formazione della classe dirigente
    \item la scuola e il valore dell'insegnante \t il docente viene con lui riconosciuto e stipendiato
\ei
Nella roma imperiale, la scuola non era pubblica \t c'era solo istruzione domiciliare: i precettori si recavano presso le famiglie (agiate) e formavano i bambini \\
Quindi era una Roma preclusiva dell'istruzione \t se l'insegnamento è 1 a 1, porta a una conoscenza diretta ed è subordinata ad un pagamento \\
Rivoluzione della scuola come istituzione e come occasione di socialita \t figura dell insegnante rivalutata \\
Favoritismi \t l'insegnate cerca famiglia + prestigiosa e + ricca \\
La scuola deve essere pubblica \t deve offrire la convivialita, e questo insegna Quintilliano \t deve essere l'anticamera della vita

\v

Nasce in spagna nel 35 d. C. \t il padre era maestro di retorica \t frequenta scuole di grammatica, e è uno dei filgi di Palomone e studente di Milizio Afro ???? \\
Ha diverse espreienze forensi \t mettono insieme la vita del foro, e oratoria \t ma poi abbandona il foro \\
Un insengnante deve essere un oratore \\
Vuole sdoganare la figura del precettore ad insegnante pubblico \t la scuola ha un ruolo fondamentale \\
Inizia l'attività di maestro, ed insegna \t tra i suoi discepoli: Plinio il Giovane, Tacito, e nel 78 Vespasiano gli affida la prima cattedra statale (di eloquenza) \t con una retribuzione stabilita dallo stato \\
Dal 78, l'insegnante viene riconsciuto come dipendente dello stato \\
Finisce di insegnare nell' 88 e muore nel 95 \\
Si è dovuto misurare con un grande tema, che si ricollega a Cicerone, che è l'eloquenza

\v

C'è un trattato perduto: "De causis corruptio eloquentiae" \t "Sulle cause della corruzione dell'eloquenza" \t quindi è corrotto anche la macchina della giustizia, che è legata all'eloquenza \\
Cosa manca che invece c'era con Cicerone? \t manca la possibilita di dialogo nel foro tra optimates e popoluares \t ma con l'impero non c'è dibattito \\
Ma scrive altre opere sull'eloquenza \\
Durante l'imepero c'è forte impianto di regista???
\end{document}
