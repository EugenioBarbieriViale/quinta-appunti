\documentclass[12pt]{article}

\usepackage[a4paper, total={6in, 8in}]{geometry}
\usepackage{textcomp}

\begin{document}
\setlength{\parindent}{0pt}

\def \t {\textrightarrow}
\def \v {\vspace{1em}}
\def \bi {\begin{itemize}}
\def \ei {\end{itemize}}
\def \s[#1] {\section*{#1}}
\def \ss[#1] {\subsection*{#1}}
\def \sss[#1] {\subsubsection*{#1}}

\s[Somnium scipionis]
Ferroni, Garbarino, Giambiagio Conte, Pontiggia \\
Struttura cosmologica del tempo, struttura pitagorica, gestione uomo-spazio, relativismo del punto di osservazione, tema politico con la gloria di roma \\
Poi anche la finzione del sogno, il tema del sogno e l'anticipazione freudiana del sogno, popolazioni sorde perchè private del suono \\
3 blocchi: 1-8, 8-15, 15-21 \t dopo ci sono i capitoletti \\
54-51 \t anni della composizione \\
L'anno del racconto è 129 \\
Ricordare il numero a lato (capito + paragrafo) nell'esposizione \\
L'opera inizia focalizzando i protagonisti \t il tema del merito nella vita terrena è centrale \t merito è definito da un comportamento giusto \t come deve essere il politico corretto? aristotele e aritosfane, che discutono sulle forme di stato \\
Cicerone predilige sistema misto politico, repubblica \\
Poi mappatura dell'universto \t snodo importante \\
Blocco politico, scientifico, filosofico \t discussione filosofica ciceroniana in merito alla disposizione degli astri \\
L'opera viene diffusa \t solo nel ? grazie ad Angelo Mai \t viene fatta l'edizione critica del De repubblica di Platone \\
Anastrofe e iperbati \t nella prima frase \t potius quam è una comparativa \\
Sotto questi tetti \t metonimia \\
Deinde ego illum ... ille me \t chiasmo \\
Verbis \t metonimia per discorsi \\
Ille dies consumptus est nobis \t oratorio \\
Familiarita profonda tra i due \\
Forte carica emotiva \t misto di dialogo e misto di discorso indiretto, che definisce il valore dell'amicizia nel mondo classico \t il ricordo che torna, porta a illusione confortante \t Leopardi \\
Brevitas è di sallustio \t mentre qua sintassi in linea con cicerone \t concinnitas \\
Paragrafo 10 \t esempio di concinnitas: non solum ... sed etiam \\
Qui vigilassem \t variato, relativa impura, causale \\
Macrotema del sogno \\
Forma = sembianze in questo caso \\
Corruesco = avere un brivido \\
Trade, ades, omitte \t congiuntivo esortativo / imperativo
\end{document}
