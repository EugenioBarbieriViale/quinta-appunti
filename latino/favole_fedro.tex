\documentclass[12pt]{article}

\usepackage[a4paper, total={6in, 8in}]{geometry}
\usepackage{textcomp}

\begin{document}
\setlength{\parindent}{0pt}

\def \t {\textrightarrow}
\def \v {\vspace{1em}}
\def \bi {\begin{itemize}}
\def \ei {\end{itemize}}
\def \s[#1] {\section*{#1}}
\def \ss[#1] {\subsection*{#1}}
\def \sss[#1] {\subsubsection*{#1}}

\s[Il lupo e l'agnello]
Rapporto tra intellettuale e potere, prevaricazione e arroganza, insubordinazione e prepotenza \\
Siti compulsi \t participio congiunto \\
Gestione dello spazio \t il lupo stava più alto, mentre l'agnello in basso \t ellissi del verbo ? \\
Cur inquit \t domanda diretta, senza le caratteristiche proprie del discorso diretto \t discorso libero, senza introduzione vera e propria della domanda \\
"Perche hai reso torpida codesta (dispregiativo) acqua?" \t laniger è metonimia, è la lanosità dell'agnello per l'agnello \\
Interrogativa che sembra anticipare il pensiero libero, lo "stream of consciusness", flusso di pensiero che inizialmente non ha neanche la sintassi \t caratteristica tipica di Joyce \\
Dialogo concitato ma non regolare \t domande si alternano \t molto teatrale, perchè gli animali comunicano direttamente tra loro \\
Mentre in Machiavelli la volpe e il leone erano solo metafore \\
Bibentis \t participio presenti da bibere \\
Laniger... \t risposta senza introduzione \t timens, participio presente \\
Continuo rimbalzo di botta e risposta senza alcun riferimento della prassi delle interrogative indirette \\
Maledicere = dire male di me, verbo del dativo \t vogliono dativo di vantaggio/svantaggio \t anche malevelle e benevelle (che per Catullo è l'emblema) \\
Da benevelle viene benevolenza (ciceroniana, nel "De Amicitia") \\
Iniusta nece \t vuole sottolineare la prepotenza sui deboli, ingiusta \t è pleonastico, nessuna morte è giusta \t ma questa morte simboleggia l'ingiustizia dei potenti sugli umili \\
Ultima frase: iperbato \t spiega la morale \t i potenti sopprimono gli umili con motivazioni fittizie \\
Figura antropologica dell'uomo non deve essere schiacciata dalla prepotenza \t cosa messa in luce dalla filosofia illuminista \\
Fictis = fittizio, falsato \t è qualcosa di arbitrario, che nel caso della violenza non ha mai ragione di essere 

\v

Questo periodo imperiale non può essere descritto se non con l'assoluta arbitrarietà del potere \t con caligola e nerone \\
Testo consete di riflettere sui regimi \t peso del potere sul singolo \t durante il 900, e l'intellettuale subisce questa realta 

\s[La volpe e l'uva]
Calliditas = astuzia \\
Messaggio del'uva acerba corrisponde all'occasione non ancora propizia \\
Nella brevitas la morale arriva subito \t non bisogna disprezzare ciò che non si può avere e chi lo riesce a raggiungere \\
Volpe è emblema della calliditas 
\end{document}
