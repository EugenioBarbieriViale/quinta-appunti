\documentclass[12pt]{article}

\usepackage[a4paper, total={6in, 8in}]{geometry}
\usepackage{textcomp}

\begin{document}
\setlength{\parindent}{0pt}

\def \t {\textrightarrow}
\def \v {\vspace{1em}}
\def \bi {\begin{itemize}}
\def \ei {\end{itemize}}
\def \s[#1] {\section*{#1}}
\def \ss[#1] {\subsection*{#1}}
\def \sss[#1] {\subsubsection*{#1}}

\s[Seneca]
Capostipite del pensiero stoico a roma, durante il periodo imperiale \\
L'animo è al centro della sua indagine \\
Spazia con collegamenti e riflessioni attuali \t è un interprete della vita dell'animo \\
Agostino e seneca sono paralleli \t agostino prosegue il pensiero di seneca, che proietta all'esterno il linguaggio dell'anima \\
Il linguaggio senechiano non può essere considerato come traducibile \t alcune espressioni sono solo sue \\
Interiotita è emotivamente conflittuale \t anche con il potere \\
Muore tagliandosi le vene e ricucendole \\
Seneca è antesignano della cultura cristiana \t sperimenta cosa rappresenti il cristianesimo in una roma imperiale, dominata dalla prepotenza di linea politica

\ss[Vita]
Nasce nel 4 a.C. \t educazione retorica e filosofica \t che sono il fondamento per l'intellettuale \\
Nel 26 si reca in Egitto \t qua conosce la tradizione e la cultura, che è incentrata nella vita oltre la vita \\
Poi torna in italia, dove viene accusato di essere stato coautore di uno scandalo nel 41 \\
L'imperatore era Claudio \t che lo accusa di una congiura antiimperiale \t questo si tradusse nell'esperienza dell'esilio in Corsica \\
Qui conduce una vita isolata \t ma qua matura la concezione di esilio come un semplice spostamento fisico \\
Vive in un ambiente desolato, ostico, impervio, pericoloso \t ambiente abbastanza consono per lui, che si impediva di cadere nelle passioni e si dava all'astinenza alimentare

\v

Viveva una vita quasi da saggio e da sapiente \t due termini che sono fondamentali per descrivere l'autore \\
Mira a raggiungere l'impertubabilità del saggio \\
Un mare in tempesta, un imbarcazione descritta come "quassatas naves" (distrutta) e un timoniere che tiene il timone, che sa destreggiarsi senza mai affondare \t questo è il sapiente \\
L'ira è perdita di controllo \t e il sapiente non deve perdere il controllo \\
Crea genere nuovo: le "Consolationes": "Consolation ad Martiam", "ad Polybium", "ad Helviam Matrem" \t nuovo tema \\
Genere consolatorio inizia da cicerone \t si genera dal pensiero autodifensivo \t che seneca chiama consolazione \\
A Polibio scrive che l'esilio non è un cambiamento cosi importante 

\v

Seneca è il portavoce dell'eta imperiale \t e vede diversi imperatori \t vede una personalita compromessa dalla vita a corte \\
Seneca è un filosofo, ed è uno stoico \\
Probabilmente viene dalla spagna \t fa carriera politica a roma \t poi permanenza a corte, dopo esperienza di oratore e retorico \\
Seneca filosofo \t bisogna parlare di Claudio e Nerone \\
Imperatore Claudio lo manda all'estero, in un posto inospitale \t conflitto con l'impero \t corsica era un luogo ostile ed impervio \t fa riflessioni filosofiche sull'esilio

\v

Il genere delle consolazioni rimandano a Cicerone dopo la morte della figlia \t sono delle lettere a diversi recipienti \\
In corsica matura il senso della distanza fisica \t che non implica la distanza interiore \\
Lettere anche alla madre \t questa donna viene privata del figlio, lui le scrive che si è un dolore immenso, ma le chiede di ricordarsi della roma in cui vivono, in cui i valori del mos maiorum sono persi \\
Ad marcian \t lettera rivolta al dolore della privazione \t marcia vede morire il nipote \t lei era imparetata con Cremuzio Cordo, che era filo repubblicano \\
Seneca era educatore di nerone \t ma non si avvicina mai alla sua linea (nerone era un folle, era un pericolo per i romani) \\
Il suo essere nella cerchia imperiale non lo risparmia della accusa di essere fautore della congiura anti-neroniana \t quella Pisoniana \\
Lui non fu mai artefice di questa congiura \t ma questo lo porta costretto al suicidio

\v

Dottrina del distacco \t lo stoicismo \\
Seneca è stoico

\v

"Naturales questiones" \t primo esempio di scienza moderna a roma, di seneca \t lucrezio riprendeva la scienza dal mondo greco \\
Ha argomenti poliprospettici \t ogni sua opera è dedicata a una sfumatura filosofica \t ha amore per la speculazione \\
Filosofia stoica \t al tempo era confusa come quella epicurea \t parte da quello \\
Sincretismo senechiano \\
Nel De Ira \t impianto dialogico \t il sincretismo stoico si manifesta come controllo delle emozioni \\
Nel De Providetia \t parla di una controversa definizione di "necessitas" = è quello che per manzoni è la providdenza divina, un determinarsi degli eventi secondo un ordine predeterminato all'interno del quale l'uomo realizza la sua vita \\
Per necessitas \t è ancora piu forte del fatum di virgilio \t il fatum li era un ente superiore \t ma necessitas senechiana supera il fatus \\
Insieme al "De Otio" e le "Consolationes" \t è inserita nei "Dialogi" \t platone ed aristotele \t dialogo platonico lascia aperte le soluzione, non giunge a conclusioni decisive \\
Esiste uno stoicismo solo senechiano, stoicisimo senechiano = sincretismo stoico \\
Non c'è ancora platonismo in seneca (di Marsilio Ficino) \t seneca lascia le tracce di un pensiero cristiano, ma la conversione non avviene \\
Seneca vede l'incendio di roma, attribuito ai cristiani \t conosce il pensiero cristiano, ma non si converte \\
Parla spesso di Paolino \t si intuisce relazione con San Paolo \t ci sono dei riferimenti agostiniani \t non è lontano dal mondo cristiano \\
Presenza del dialogo anche in opere più vicine alla riflessione morale e politica \t ad esempio: "De clementia" \\
Ma anche "De beneficiis" \t i benefici sono delle concessioni dei potenti agli umili (manzoni) \\
Importanza della "liberalitas" != liberta, ma = generosità, il concedere qualcosa a qualcuno, una flessibilita, un dovere da un beneficiante a un beneficiato \\
Seneca analizza il vantaggio che il beneficiante ha rispetto al beneficiato ???? \\
Pensiero evangelico cristiano simile \\
Volontaria benevolenza \\
Lui conduce uno studio storico \t i ricchissimi, ricchi e classe benestante si muovono inversamente proporzionale ai loro averi (il ricchissimo è avaro) \t a volte la generosita parte da una situazione di difficolta \\
Marx legato già legato a questo discorso 

\ss[De Beneficiis]
In Kant \t individuo viene considerato nelle sue pari possibilita \\
Nell'antica roma secondo Seneca si aveva una lettura dei rapporti di potere \t quelli dei benefici \\
Fedro: "non è mai consigliata l'amicizia con i potenti (sodalitas cum potentis)" \\
Come potevano i singoli sopravvivere senza qualche concessione tra gli stra potenti dell'antica roma? \\
La prospettiva dell'imperatore era di trovare seguaci, proseliti (proselitismo) \\
"Io ho donato per il gusto di donare" \t frase topica \\
"Dai il tuo aiuto ora dondando ..." \t momento coercitivo/imperativo \t il beneficio è correlato ad aver fatto il proprio dovere \\
Se ti è dovuto devi chiedere, se hai un diritto lo devi pretendere \\
Alcune elargizioni hanno avuto un senso etico \t per esempio durante mussolini furono bonificate diverse paludi \\
L'elargizione è sempre fine a se stessa \t come Lucia quando da le noci al frate \\
Anche nella germania di tasso \t si parla di usura \\
"Cedono a un beneficio offerto con ferma costanza" \t l'interazione avviene tra il beneficiante e il beneficiato \\
Ci si proietta nell'ambito politico con la corruzione \\
Qua si ha il dare per avere \t il recepire è quello che consente ha chi da di guadagnare l'autostima \t il recipiente puo non essere interessato \\
Seneca mette a fuoco che nell'impero nulla era fatto per nulla \t fa riflessione lucida \\
Egli incontra anche l'argomentazione espressa nella frase "mai cercare mai rifiutare" \t dal punto di vista etico in realta ci sono delle cose da rifiutare \\
Benevoletia di cicerone nel de amicitia \t e il catullo che nega il benevelle a Lesbia \\
Il benevelle comprende pero tutto il bene \t chi dona si sente arricchito, e chi riceve non pretende \\
Il beneficiario deve rifiutare l'opportunismo

\v

Valore del dono e significato del dono, fine a se stesso \\
Chi riceve un dono dal potente? \t lo schiavo \\
Forti aspettative in assenza di ganci a roma \t gli schiavi erano stranieri, e spesso venivano reclutati nelle corti perchè alcuni erano colti (per esempio dalla grecia, e costituivano una risora) \t per esempio Fedro \\
Seneca riflette su questo \t fino a che punto si possono unificare morale e schiavitù \t ne parla nel de beneficiis, che verte sulle concessioni \\
Un dono è una concessione, ma in questo caso non libera: in cambio si chiede un serivizio \t ma schiavo non ha diritti \\
Macroclimax ascendente sugli esuli \t che devono scappare \t ne parla nel Consolatio ad Polybium \\
Rapporto schiavitù / beneficio \t nella schiavitu deve essere conservata la dignita della persona, lo afferma \\
La virtu è aprte a tutti \\
"Se lo schiavo non reca .. assoluto" \t ultime 7 righe \\
Fare del bene al padone va al di la del fare il proprio dovere \t lo schiavo puo anche avere un valore oltre, che lo porta a dedicarsi al padrone \t questo puo implicare che il padrone conceda un beneficio allo schiavo \\
Qual è l unico tratto distintivo dello schiavo è che ha valore morale di per se, ma una concessione / aiuto fornito volontariamente non implica che si richieda qualcosa in cambio \\
Si ricorda Lucia che dona le noci \t ma si aspetta qualcosa in cambio \\
Seneca esprime la totale liberta del donare e del ricevere \\
Riconoscimento umano dello schiavo \t si aggancia alla problematica dei diritti umani \\
Alfieri, tema della liberta, schiavitu della maschera di Pirandello (ruolo nella societa) \\
Il dovere del padrone nei confronti dello schiavo è di consideralo un uomo

\ss[Contro il perdono indiscriminato]
L'autore insiste sul concetto che il perdono possa intendere un recupero \t lo dice nel rapporto tra tolleranza e perdono \\
Guicciardini \t focalizza il singolare e non il generale \t infatti ha visione segmentata e parcellizzata \\
"Rigoroso" cozza con perdonare \t perdonare in modo rigoroso significa attribuire il giusto peso al recupero che si vuole offrire \\
Questo pero coinvolge anche l'aspetto del potere \t che non viene in realta specificato qua \\
Per seneca ci sono modalita diverse di governo \t potere assoluto e non paternalistico, come quello di Nerone \\
Ci sono altre forme, che non rientrano in quella dell'imperatore \t che è una figura assoluta \t re e sovrano è potere + paternalistico, rivolto ai singoli \\
Potere assoluto invece è egoriferito, non rivolto al singolo \t visione aristotelica di potere misto, che cicerone accoglie \\
Cicerone è nel momento della crisi della repubblica \t c'è ancora un governo pluralistico, perchè ci sono diversi opinioni \t ma con il passaggio all'impero, il pluralismo cessa di esistere \\
La figura di sovrano e re perdono di valore, il potere è nell'imperatore \t governo autocratico, autarchico e assoluto = accentratore ed egoriferito \\
In governo assoluto \t manca un confronto \t il dialogo non si ha \\
Il saggio sa scegliere dove il perdono è un beneficio, oppure è inutile \t li perdono è dannoso \\
Seneca è stato visto come il presecutore del cristianesimo \t contatto con san paolo e insegue pauline? \t ma non ha visione cristiana del perdono, quindi seneca non puo avere preceduto il cristianesimo

\ss[Naturales questiones - Fenomeni celesti]
Opera scientifica \t seneca è sull orma del cicerone del somnium (nelle parti degli astri) e di lucrezio (de rerum natura) \\
Si occupa delle maree, dei vulcani \t ma non ha risposte certe \t scienza è un processo mai interrotto \\
Popper \t il sapere non è mai perfetto \\
Quale questione apre la mancanza di un linguaggio scientifico? \t bisogna trovare un nome, una definizione adatta (sole) \\
Lucrezio aveva adattato gli esameri alla musicalita greca \\
Manca pero una nomenclatura, uno strumento espressivo \t e di una comunita scientifica \t Plinio ne parlera ne "Historia ??" \\
Somnium scipionis è centrato \t movimento degli astri \t inoltre porzioni della cosmologia dantesca di matrice tolemaica, nella terza cantica della divina commedia \\
Le domande sono un aspetto fondamentale della scienza \t lo scienziato si muove con le domande, che non sono mai definitive \\
Questo coinvolge il leopardi che pone le domande, nelle sue opere
\end{document}
