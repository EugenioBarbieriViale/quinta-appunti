\documentclass[12pt]{article}

\usepackage[a4paper, total={6in, 8in}]{geometry}
\usepackage{textcomp}

\begin{document}
\setlength{\parindent}{0pt}

\def \t {\textrightarrow}
\def \v {\vspace{1em}}
\def \bi {\begin{itemize}}
\def \ei {\end{itemize}}
\def \s[#1] {\section*{#1}}
\def \ss[#1] {\subsection*{#1}}
\def \sss[#1] {\subsubsection*{#1}}

\s[Seneca]
Capostipite del pensiero stoico a roma, durante il periodo imperiale \\
L'animo è al centro della sua indagine \\
Spazia con collegamenti e riflessioni attuali \t è un interprete della vita dell'animo \\
Agostino e seneca sono paralleli \t agostino prosegue il pensiero di seneca, che proietta all'esterno il linguaggio dell'anima \\
Il linguaggio senechiano non può essere considerato come traducibile \t alcune espressioni sono solo sue \\
Interiotita è emotivamente conflittuale \t anche con il potere \\
Muore tagliandosi le vene e ricucendole \\
Seneca è antesignano della cultura cristiana \t sperimenta cosa rappresenti il cristianesimo in una roma imperiale, dominata dalla prepotenza di linea politica

\ss[Vita]
Nasce nel 4 a.C. \t educazione retorica e filosofica \t che sono il fondamento per l'intellettuale \\
Nel 26 si reca in Egitto \t qua conosce la tradizione e la cultura, che è incentrata nella vita oltre la vita \\
Poi torna in italia, dove viene accusato di essere stato coautore di uno scandalo nel 41 \\
L'imperatore era Claudio \t che lo accusa di una congiura antiimperiale \t questo si tradusse nell'esperienza dell'esilio in Corsica \\
Qui conduce una vita isolata \t ma qua matura la concezione di esilio come un semplice spostamento fisico \\
Vive in un ambiente desolato, ostico, impervio, pericoloso \t ambiente abbastanza consono per lui, che si impediva di cadere nelle passioni e si dava all'astinenza alimentare

\v

Viveva una vita quasi da saggio e da sapiente \t due termini che sono fondamentali per descrivere l'autore \\
Mira a raggiungere l'impertubabilità del saggio \\
Un mare in tempesta, un imbarcazione descritta come "quassatas naves" (distrutta) e un timoniere che tiene il timone, che sa destreggiarsi senza mai affondare \t questo è il sapiente \\
L'ira è perdita di controllo \t e il sapiente non deve perdere il controllo \\
Crea genere nuovo: le "Consolationes": "Consolation ad Martiam", "ad Polybium", "ad Helviam Matrem" \t nuovo tema \\
Genere consolatorio inizia da cicerone \t si genera dal pensiero autodifensivo \t che seneca chiama consolazione \\
A Polibio scrive che l'esilio non è un cambiamento cosi importante 

\v

Seneca è il portavoce dell'eta imperiale \t e vede diversi imperatori \t vede una personalita compromessa dalla vita a corte \\
Seneca è un filosofo, ed è uno stoico \\
Probabilmente viene dalla spagna \t fa carriera politica a roma \t poi permanenza a corte, dopo esperienza di oratore e retorico \\
Seneca filosofo \t bisogna parlare di Claudio e Nerone \\
Imperatore Claudio lo manda all'estero, in un posto inospitale \t conflitto con l'impero \t corsica era un luogo ostile ed impervio \t fa riflessioni filosofiche sull'esilio

\v

Il genere delle consolazioni rimandano a Cicerone dopo la morte della figlia \t sono delle lettere a diversi recipienti \\
In corsica matura il senso della distanza fisica \t che non implica la distanza interiore \\
Lettere anche alla madre \t questa donna viene privata del figlio, lui le scrive che si è un dolore immenso, ma le chiede di ricordarsi della roma in cui vivono, in cui i valori del mos maiorum sono persi \\
Ad marcian \t lettera rivolta al dolore della privazione \t marcia vede morire il nipote \t lei era imparetata con Cremuzio Cordo, che era filo repubblicano \\
Seneca era educatore di nerone \t ma non si avvicina mai alla sua linea (nerone era un folle, era un pericolo per i romani) \\
Il suo essere nella cerchia imperiale non lo risparmia della accusa di essere fautore della congiura anti-neroniana \t quella Pisoniana \\
Lui non fu mai artefice di questa congiura \t ma questo lo porta costretto al suicidio

\v

Dottrina del distacco \t lo stoicismo \\
Seneca è stoico

\v

"Naturales questiones" \t primo esempio di scienza moderna a roma, di seneca \t lucrezio riprendeva la scienza dal mondo greco \\
Ha argomenti poliprospettici \t ogni sua opera è dedicata a una sfumatura filosofica \t ha amore per la speculazione \\
Filosofia stoica \t al tempo era confusa come quella epicurea \t parte da quello \\
Sincretismo senechiano \\
Nel De Ira \t impianto dialogico \t il sincretismo stoico si manifesta come controllo delle emozioni \\
Nel De Providetia \t parla di una controversa definizione di "necessitas" = è quello che per manzoni è la providdenza divina, un determinarsi degli eventi secondo un ordine predeterminato all'interno del quale l'uomo realizza la sua vita \\
Per necessitas \t è ancora piu forte del fatum di virgilio \t il fatum li era un ente superiore \t ma necessitas senechiana supera il fatus \\
Insieme al "De Otio" e le "Consolationes" \t è inserita nei "Dialogi" \t platone ed aristotele \t dialogo platonico lascia aperte le soluzione, non giunge a conclusioni decisive \\
Esiste uno stoicismo solo senechiano, stoicisimo senechiano = sincretismo stoico \\
Non c'è ancora platonismo in seneca (di Marsilio Ficino) \t seneca lascia le tracce di un pensiero cristiano, ma la conversione non avviene \\
Seneca vede l'incendio di roma, attribuito ai cristiani \t conosce il pensiero cristiano, ma non si converte \\
Parla spesso di Paolino \t si intuisce relazione con San Paolo \t ci sono dei riferimenti agostiniani \t non è lontano dal mondo cristiano \\
Presenza del dialogo anche in opere più vicine alla riflessione morale e politica \t ad esempio: "De clementia" \\
Ma anche "De beneficiis" \t i benefici sono delle concessioni dei potenti agli umili (manzoni) \\
Importanza della "liberalitas" != liberta, ma = generosità, il concedere qualcosa a qualcuno, una flessibilita, un dovere da un beneficiante a un beneficiato \\
Seneca analizza il vantaggio che il beneficiante ha rispetto al beneficiato ???? \\
Pensiero evangelico cristiano simile \\
Volontaria benevolenza \\
Lui conduce uno studio storico \t i ricchissimi, ricchi e classe benestante si muovono inversamente proporzionale ai loro averi (il ricchissimo è avaro) \t a volte la generosita parte da una situazione di difficolta \\
Marx legato già legato a questo discorso \\

\end{document}
