\documentclass[12pt]{article}

\usepackage[a4paper, total={6in, 8in}]{geometry}
\usepackage{textcomp}

\begin{document}
\setlength{\parindent}{0pt}

\def \t {\textrightarrow}
\def \v {\vspace{1em}}
\def \bi {\begin{itemize}}
\def \ei {\end{itemize}}
\def \s[#1] {\section*{#1}}
\def \ss[#1] {\subsection*{#1}}
\def \sss[#1] {\subsubsection*{#1}}

\s[Seneca]
Capostipite del pensiero stoico a roma, durante il periodo imperiale \\
L'animo è al centro della sua indagine \\
Spazia con collegamenti e riflessioni attuali \t è un interprete della vita dell'animo \\
Agostino e seneca sono paralleli \t agostino prosegue il pensiero di seneca, che proietta all'esterno il linguaggio dell'anima \\
Il linguaggio senechiano non può essere considerato come traducibile \t alcune espressioni sono solo sue \\
Interiotita è emotivamente conflittuale \t anche con il potere \\
Muore tagliandosi le vene e ricucendole \\
Seneca è antesignano della cultura cristiana \t sperimenta cosa rappresenti il cristianesimo in una roma imperiale, dominata dalla prepotenza di linea politica

\ss[Vita]
Nasce nel 4 a.C. \t educazione retorica e filosofica \t che sono il fondamento per l'intellettuale \\
Nel 26 si reca in Egitto \t qua conosce la tradizione e la cultura, che è incentrata nella vita oltre la vita \\
Poi torna in italia, dove viene accusato di essere stato coautore di uno scandalo nel 41 \\
L'imperatore era Claudio \t che lo accusa di una congiura antiimperiale \t questo si tradusse nell'esperienza dell'esilio in Corsica \\
Qui conduce una vita isolata \t ma qua matura la concezione di esilio come un semplice spostamento fisico \\
Vive in un ambiente desolato, ostico, impervio, pericoloso \t ambiente abbastanza consono per lui, che si impediva di cadere nelle passioni e si dava all'astinenza alimentare

\v

Viveva una vita quasi da saggio e da sapiente \t due termini che sono fondamentali per descrivere l'autore \\
Mira a raggiungere l'impertubabilità del saggio \\
Un mare in tempesta, un imbarcazione descritta come "quassatas naves" (distrutta) e un timoniere che tiene il timone, che sa destreggiarsi senza mai affondare \t questo è il sapiente \\
L'ira è perdita di controllo \t e il sapiente non deve perdere il controllo \\
Crea genere nuovo: le "Consolationes": "Consolation ad Martiam", "ad Polybium", ad "Helviam Matrem" \t nuovo tema \\
Genere consolatorio inizia da cicerone \t si genera dal pensiero autodifensivo \t che seneca chiama consolazione \\
A Polibio scrive che l'esilio non è un cambiamento cosi importante 
\end{document}
